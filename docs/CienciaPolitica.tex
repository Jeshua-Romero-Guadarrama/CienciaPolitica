% Options for packages loaded elsewhere
\PassOptionsToPackage{unicode}{hyperref}
\PassOptionsToPackage{hyphens}{url}
%
\documentclass[
]{book}
\usepackage{amsmath,amssymb}
\usepackage{lmodern}
\usepackage{ifxetex,ifluatex}
\ifnum 0\ifxetex 1\fi\ifluatex 1\fi=0 % if pdftex
  \usepackage[T1]{fontenc}
  \usepackage[utf8]{inputenc}
  \usepackage{textcomp} % provide euro and other symbols
\else % if luatex or xetex
  \usepackage{unicode-math}
  \defaultfontfeatures{Scale=MatchLowercase}
  \defaultfontfeatures[\rmfamily]{Ligatures=TeX,Scale=1}
\fi
% Use upquote if available, for straight quotes in verbatim environments
\IfFileExists{upquote.sty}{\usepackage{upquote}}{}
\IfFileExists{microtype.sty}{% use microtype if available
  \usepackage[]{microtype}
  \UseMicrotypeSet[protrusion]{basicmath} % disable protrusion for tt fonts
}{}
\makeatletter
\@ifundefined{KOMAClassName}{% if non-KOMA class
  \IfFileExists{parskip.sty}{%
    \usepackage{parskip}
  }{% else
    \setlength{\parindent}{0pt}
    \setlength{\parskip}{6pt plus 2pt minus 1pt}}
}{% if KOMA class
  \KOMAoptions{parskip=half}}
\makeatother
\usepackage{xcolor}
\IfFileExists{xurl.sty}{\usepackage{xurl}}{} % add URL line breaks if available
\IfFileExists{bookmark.sty}{\usepackage{bookmark}}{\usepackage{hyperref}}
\hypersetup{
  pdftitle={Ciencia Política: Teoría y Práctica},
  pdfauthor={Jeshua Romero Guadarrama},
  hidelinks,
  pdfcreator={LaTeX via pandoc}}
\urlstyle{same} % disable monospaced font for URLs
\usepackage{longtable,booktabs,array}
\usepackage{calc} % for calculating minipage widths
% Correct order of tables after \paragraph or \subparagraph
\usepackage{etoolbox}
\makeatletter
\patchcmd\longtable{\par}{\if@noskipsec\mbox{}\fi\par}{}{}
\makeatother
% Allow footnotes in longtable head/foot
\IfFileExists{footnotehyper.sty}{\usepackage{footnotehyper}}{\usepackage{footnote}}
\makesavenoteenv{longtable}
\usepackage{graphicx}
\makeatletter
\def\maxwidth{\ifdim\Gin@nat@width>\linewidth\linewidth\else\Gin@nat@width\fi}
\def\maxheight{\ifdim\Gin@nat@height>\textheight\textheight\else\Gin@nat@height\fi}
\makeatother
% Scale images if necessary, so that they will not overflow the page
% margins by default, and it is still possible to overwrite the defaults
% using explicit options in \includegraphics[width, height, ...]{}
\setkeys{Gin}{width=\maxwidth,height=\maxheight,keepaspectratio}
% Set default figure placement to htbp
\makeatletter
\def\fps@figure{htbp}
\makeatother
\setlength{\emergencystretch}{3em} % prevent overfull lines
\providecommand{\tightlist}{%
  \setlength{\itemsep}{0pt}\setlength{\parskip}{0pt}}
\setcounter{secnumdepth}{5}
\usepackage{amsthm}
\usepackage{float}
\usepackage{rotating, graphicx}
\usepackage{multirow}
\usepackage{tabularx}

% new command for pretty oversets with \sim
\newcommand\simcal[1]{\stackrel{\sim}{\smash{\mathcal{#1}}\rule{0pt}{0.5ex}}}

\newcommand{\comma}{,\,}

\floatplacement{figure}{H}

\PassOptionsToPackage{table}{xcolor}

\usepackage{tcolorbox}

\definecolor{kcblue}{HTML}{D7DDEF}
\definecolor{kcdarkblue}{HTML}{2B4E70}

\makeatletter
\def\thm@space@setup{%
  \thm@preskip=8pt plus 2pt minus 4pt
  \thm@postskip=\thm@preskip
}
\makeatother

% \makeatletter % undo the wrong changes made by mathspec
% \let\RequirePackage\original@RequirePackage
% \let\usepackage\RequirePackage
% \makeatother

\newenvironment{rmdknit}
    {\begin{center}
    \begin{tabular}{|p{0.9\textwidth}|}
    \hline\\
    }
    {
    \\\\\hline
    \end{tabular}
    \end{center}
    }

\newenvironment{rmdnote}
    {\begin{center}
    \begin{tabular}{|p{0.9\textwidth}|}
    \hline\\
    }
    {
    \\\\\hline
    \end{tabular}
    \end{center}
    }

\newtcolorbox[auto counter, number within=section]{keyconcepts}[2][]{%
colback=kcblue,colframe=kcdarkblue,fonttitle=\bfseries, title=Key Concept~#2, after title={\newline #1}, beforeafter skip=15pt}
\ifluatex
  \usepackage{selnolig}  % disable illegal ligatures
\fi
\usepackage[]{natbib}
\bibliographystyle{apalike}

\title{Ciencia Política: Teoría y Práctica}
\author{Jeshua Romero Guadarrama}
\date{2021-07-25}

\begin{document}
\maketitle

{
\setcounter{tocdepth}{1}
\tableofcontents
}
\hypertarget{prefacio}{%
\chapter*{Prefacio}\label{prefacio}}
\addcontentsline{toc}{chapter}{Prefacio}

Publicado por Jeshua Romero Guadarrama en colaboración con JeshuaNomics:

{ Git Hub}
{ Facebook}
{ Twitter}
{ Linkedin}
{ Vkontakte}
{ Tumblr}
{ YouTube}
{ Instagram}

Jeshua Romero Guadarrama es economista y actuario por la Universidad Nacional Autónoma de México, quien ha construido el presente proyecto en colaboración con JeshuaNomics, ubicado en la Ciudad de México, se puede contactar mediante el siguiente correo electrónico: \href{mailto:jeshuanomics@gmail.com}{\nolinkurl{jeshuanomics@gmail.com}}.
Última actualización el domingo 25 del 07 de 2021

El presente texto nace al calor de las exigencias pedagógicas de todos los ciudadanos mexicanos interesados en la ciencia política. A partir de los años que he pasado detrás de innumerables libros (con el objetivo de gestar, buscar y probar nuevos conocimientos), ve la luz pública este trabajo, que fué creciendo y cambiando lentamente, empezando como apuntes de la universidad. Creo que ha alcanzado la madurez suficiente para ser compartido con el mundo. Con independencia de su valor intrínseco, tengo entendido que hace mucho tiempo que no se hacía una obra de este tipo (lo que ciertamente le corresponde al lector juzgar). En la bibliografía especializada disponible en castellano, el antecedente más inmediato que conozco es \textbf{Teorías políticas contemporáneas: Una introducción}, de Klaus von Beyme. La primera edición alemana es de 1972, cuya edición en castellano (difícil de hallar), es de 1977. Esa obra fué mi primera orientación; en consecuencia, mantengo en lo fundamental su esquema y algo de su terminología (tengo una deuda intelectual con el formidable profesor de Heildelberg).

\hypertarget{las-convenciones-usadas-en-el-presente-curso}{%
\subsubsection*{Las convenciones usadas en el presente curso}\label{las-convenciones-usadas-en-el-presente-curso}}
\addcontentsline{toc}{subsubsection}{Las convenciones usadas en el presente curso}

\begin{itemize}
\item
  El texto \emph{en cursiva} indica nuevos términos, nombres y similares.
\item
  El texto \textbf{en negrita} se usa generalmente en párrafos para referirse a conceptos que se recomienda memorizar.
\item
  Texto de ancho constante sobre fondo gris indica un enfoque teórico o metodológico comúnmente utilizado en la práctica por los politólogos.
\end{itemize}

\hypertarget{reconocimiento}{%
\subsubsection*{Reconocimiento}\label{reconocimiento}}
\addcontentsline{toc}{subsubsection}{Reconocimiento}

A mi alma máter: Universidad Nacional Autónoma de México (Facultad de Economía y Facultad de Ciencias). Por brindarme valiosas oportunidades que coadyuvaron a mi formación.

\hypertarget{contenido}{%
\section*{Contenido}\label{contenido}}
\addcontentsline{toc}{section}{Contenido}

Parte I Introducción a la teoría política

Capítulos:

\begin{itemize}
\tightlist
\item
  La teoría política
\item
  Las teorías normativas
\item
  Las teorías empirico-analiticas
\item
  Las teorías critico-dialécticas
\end{itemize}

Parte II Conceptos fundamentales de teoría política

Capítulos:

\begin{itemize}
\tightlist
\item
  En busca de modelos de la sociedad y la política
\item
  Los modelos de integración y orden
\item
  Los modelos de conflicto
\item
  Algunos enfoques teóricos
\end{itemize}

Parte III La teoría política ante el panorama mundial

Capítulos:

\begin{itemize}
\tightlist
\item
  Teorías del primer mundo para el análisis del segundo y del tercer mundo
\item
  Teorías del desarrollo político
\item
  Teorías del imperialismo y de la dependencia
\item
  La teoría política ante América latina. análisis y perspectivas
\end{itemize}

Parte IV Derecho constitucional mexicano
Capítulos:

\begin{itemize}
\tightlist
\item
  Teoría de la Constitución
\item
  Los poderes Ejecutivo y Legislativo
\item
  El Poder Judicial
\item
  El federalismo mexicano actual
\end{itemize}

\hypertarget{uxedndice-de-contenido}{%
\section*{Índice de contenido}\label{uxedndice-de-contenido}}
\addcontentsline{toc}{section}{Índice de contenido}

Parte I Introducción a la teoría política

\begin{enumerate}
\def\labelenumi{\arabic{enumi}.}
\tightlist
\item
  La teoría política
\end{enumerate}

\begin{itemize}
\tightlist
\item
  Consideraciones generales:

  \begin{itemize}
  \tightlist
  \item
    La teoría científica social.
  \item
    Cuestiones metodológicas.
  \item
    Principios actuales.
  \item
    Críticas a la ciencia.
  \end{itemize}
\item
  Fases de la actividad científica:

  \begin{itemize}
  \tightlist
  \item
    Teorías representativas y normativas.
  \item
    Descripción.
  \item
    Explicación.
  \item
    Generalización.
  \item
    Teoría.
  \item
    Cuasi-teorías: Clasificaciones, dicotomías y analogías.
  \end{itemize}
\item
  La evaluación del fenómeno político:

  \begin{itemize}
  \tightlist
  \item
    Ciencia y valoración.
  \item
    Los componentes del juicio normativo: descripción.
  \item
    Evaluación técnica.
  \item
    Juicio normativo.
  \item
    Justificación del juicio normativo.
  \end{itemize}
\item
  El concepto teórico político. Comparaciones con los de otras ciencias:

  \begin{itemize}
  \tightlist
  \item
    Teoría y Filosofía Política
  \item
    Ciencia Política como disciplina autónoma
  \item
    Teoría Política e Historia de las Ideas
  \item
    Teorías generales y de alcance medio
  \item
    Dificultades para la elaboración teórica.
  \end{itemize}
\end{itemize}

\begin{enumerate}
\def\labelenumi{\arabic{enumi}.}
\setcounter{enumi}{1}
\tightlist
\item
  Las teorías normativas
\end{enumerate}

\begin{itemize}
\tightlist
\item
  Rasgos generales:

  \begin{itemize}
  \tightlist
  \item
    Condiciones históricas y trasfondos ideológicos.
  \item
    Clasificación de las teorías normativas.
  \item
    Raíces intelectuales.
  \item
    Fundamentos.
  \item
    Finalidad.
  \item
    Relaciones.
  \item
    Metodología.
  \end{itemize}
\item
  Teorías políticas normativas clásicas:

  \begin{itemize}
  \tightlist
  \item
    Chinas, hindúes, judías, islámicas, griegas, romanas, medievales y modernas.
  \end{itemize}
\item
  Teorías políticas normativas contemporáneas:

  \begin{itemize}
  \tightlist
  \item
    El asalto al absolutismo.
  \item
    Las consecuencias de la Revolución Francesa.
  \item
    Socialismos y nacionalismos.
  \item
    Las teorías normativas actuales.
  \end{itemize}
\item
  Enfoques metodológicos usuales:

  \begin{itemize}
  \tightlist
  \item
    Métodos: histórico, analógico, práctico, tópico, pedagógico.
  \item
    El pragmatismo metodológico.
  \end{itemize}
\end{itemize}

\begin{enumerate}
\def\labelenumi{\arabic{enumi}.}
\setcounter{enumi}{2}
\tightlist
\item
  Las teorías empírico-analíticas
\end{enumerate}

\begin{itemize}
\tightlist
\item
  Rasgos generales:

  \begin{itemize}
  \tightlist
  \item
    El positivismo, el empirismo y sus derivados.
  \item
    El objeto y el método.
  \item
    Problemas actuales.
  \end{itemize}
\item
  Behaviorismo, estructural-funcionalismo y enfoque sistémico. El enfoque comparatista:

  \begin{itemize}
  \tightlist
  \item
    Descripción de los enfoques.
  \item
    Síntesis de obras teóricas de estas corrientes.
  \end{itemize}
\item
  Las explicaciones de base psicológica individual:

  \begin{itemize}
  \tightlist
  \item
    La Psicología del estímulo-respuesta.
  \item
    La Psicología de la Gestalt.
  \item
    La Teoría del Campo.
  \item
    El freudismo ortodoxo.
  \item
    El neofreudismo.
  \end{itemize}
\item
  El Formalismo:

  \begin{itemize}
  \tightlist
  \item
    La Teoría de los Juegos.
  \item
    La Teoría de la Información y la Cibernética.
  \item
    Modelos y simulaciones.
  \end{itemize}
\item
  Enfoques metodológicos usuales:

  \begin{itemize}
  \tightlist
  \item
    Puntos en común.
  \item
    Particularidades metodológicas.
  \item
    Reflexiones sobre el lenguaje y la elaboración conceptual.
  \end{itemize}
\end{itemize}

\begin{enumerate}
\def\labelenumi{\arabic{enumi}.}
\setcounter{enumi}{3}
\tightlist
\item
  Las teorías crítico-dialécticas
\end{enumerate}

\begin{itemize}
\tightlist
\item
  Introducción general:

  \begin{itemize}
  \tightlist
  \item
    Repercusiones del tema.
  \item
    Aportes perdurables del marxismo.
  \end{itemize}
\item
  El marxismo clásico - Rasgos generales:

  \begin{itemize}
  \tightlist
  \item
    Marx y Engels.
  \item
    Contenidos del marxismo.
  \item
    Primera y segunda generación de sucesores.
  \end{itemize}
\item
  El marxismo occidental:

  \begin{itemize}
  \tightlist
  \item
    La Escuela de Frankfurt
  \item
    Otros intelectuales europeos marxistas.
  \item
    Intelectuales norteamericanos marxistas.
  \item
    La Nueva Izquierda.
  \item
    La labor teórica en los países socialistas europeos.
  \end{itemize}
\item
  Las teorías crítico-dialécticas en los países del tercer mundo:

  \begin{itemize}
  \tightlist
  \item
    El maoísmo y sus derivados asiáticos.
  \item
    El socialismo africanoEl marxismo latinoamericano: Justo, Mariátegui y Haya de la Torre.
  \item
    La Nueva Izquierda latinoamericana, el castrismo, el sandinismo y el allendismo chileno.
  \item
    Relaciones de estas teorías con la Teología de la Liberación.
  \end{itemize}
\item
  Enfoques metodológicos usuales:

  \begin{itemize}
  \tightlist
  \item
    Materialismo dialéctico y materialismo histórico.
  \item
    Teoría y praxis.
  \item
    Otros aportes metodológicos.
  \end{itemize}
\end{itemize}

Parte II Conceptos fundamentales de teoría política

\begin{enumerate}
\def\labelenumi{\arabic{enumi}.}
\setcounter{enumi}{4}
\tightlist
\item
  En busca de modelos de la sociedad y la política
\end{enumerate}

\begin{itemize}
\tightlist
\item
  Los modelos como instrumentos del pensamiento. Los modelos clásicos:

  \begin{itemize}
  \tightlist
  \item
    Estadios en la historia de la ciencia.
  \item
    Funciones de los modelos.
  \item
    Concepto de modelo.
  \item
    Criterios de selección de modelos.
  \item
    Algunos modelos clásicos: Sociedad, Alfarero, Ciudad, Pirámide, Rueda, Flecha, Espiral, Balanza, Hebra, Tejido.
  \end{itemize}
\item
  Los modelos modernos y contemporáneos:

  \begin{itemize}
  \tightlist
  \item
    El mecanismo.
  \item
    El organismo.
  \item
    Modelos derivados de la historia.
  \item
    Modelos matemáticos.
  \item
    Los ``tipos ideales''.
  \end{itemize}
\item
  Los modelos cibernéticos, de comunicación y control:

  \begin{itemize}
  \tightlist
  \item
    Información y comunicación.
  \item
    Memoria.
  \item
    Realimentación.
  \item
    Decisión y voluntad.
  \item
    Autodeterminación.
  \end{itemize}
\item
  Modelos cibernéticos de comunicación y sistemas de decisión política:

  \begin{itemize}
  \tightlist
  \item
    Información y cohesión social.
  \item
    Comunicación y legitimidad .
  \item
    Liderazgo.
  \item
    Mímesis y obediencia.
  \item
    Aprendizaje.
  \item
    Promoción política.
  \item
    Autoridad.
  \item
    Conducción.
  \item
    Decisión.
  \item
    Autonomía.
  \item
    Soberanía
  \item
    Valor funcional de las virtudes.
  \item
    Desarrollo.
  \end{itemize}
\item
  Consideraciones generales sobre modelos de integración y modelos de conflicto:

  \begin{itemize}
  \tightlist
  \item
    La actividad científica y sus trasfondos cosmovisionales e ideológicos.
  \item
    Afinidad de los modelos con esos trasfondos.
  \end{itemize}
\end{itemize}

\begin{enumerate}
\def\labelenumi{\arabic{enumi}.}
\setcounter{enumi}{5}
\tightlist
\item
  Los modelos de integración y orden
\end{enumerate}

\begin{itemize}
\tightlist
\item
  Rasgos generales y conceptos centrales:

  \begin{itemize}
  \tightlist
  \item
    Rasgos.
  \item
    Ennumeración de los conceptos usuales.
  \item
    Relación con teorías normativas y empírico-analíticas.
  \end{itemize}
\item
  El Estado, el poder y el sistema político:

  \begin{itemize}
  \tightlist
  \item
    Descripción de los conceptos.
  \item
    Contenidos y variables.
  \item
    Investigaciones.
  \end{itemize}
\item
  La cultura, el estilo y la socialización política. El cambio y el desarrollo político:

  \begin{itemize}
  \tightlist
  \item
    Descripción de los conceptos.
  \item
    Contenidos y variables.
  \item
    Investigaciones.
  \end{itemize}
\item
  La democracia. Modelos estáticos y dinámicos:

  \begin{itemize}
  \tightlist
  \item
    Descripción de los conceptos.
  \item
    Contenidos y variables.
  \item
    Investigaciones.
  \end{itemize}
\end{itemize}

\begin{enumerate}
\def\labelenumi{\arabic{enumi}.}
\setcounter{enumi}{6}
\tightlist
\item
  Los modelos de conflicto
\end{enumerate}

\begin{itemize}
\tightlist
\item
  Rasgos generales y conceptos centrales:

  \begin{itemize}
  \tightlist
  \item
    Rasgos.
  \item
    Ennumeración de los conceptos usuales.
  \item
    Relación con teorías.
  \end{itemize}
\item
  El pluralismo y conflicto de grupos:

  \begin{itemize}
  \tightlist
  \item
    Descripción del concepto.
  \item
    Contenidos y variables.
  \item
    Investigaciones.
  \end{itemize}
\item
  La lucha de clases:

  \begin{itemize}
  \tightlist
  \item
    Teorías marxistas de clases.
  \item
    Teorías no marxistas de clases.
  \end{itemize}
\item
  La confrontación élite-masa:

  \begin{itemize}
  \tightlist
  \item
    Descripción del concepto.
  \item
    Contenidos y variables.
  \item
    Investigaciones.
  \end{itemize}
\end{itemize}

\begin{enumerate}
\def\labelenumi{\arabic{enumi}.}
\setcounter{enumi}{7}
\tightlist
\item
  Algunos enfoques teóricos
\end{enumerate}

\begin{itemize}
\tightlist
\item
  Representación y participación:

  \begin{itemize}
  \tightlist
  \item
    Concepto de representación.
  \item
    Crisis de la representación.
  \item
    La participación, en sentido amplio y en sentido estricto.
  \end{itemize}
\item
  Legalidad y legitimidad:

  \begin{itemize}
  \tightlist
  \item
    Concepto de legalidad.
  \item
    Límites de la legalidad.
  \item
    Legitimidad y consenso.
  \item
    La tergiversación del consenso.
  \end{itemize}
\item
  La transición democrática en el campo de la cultura:

  \begin{itemize}
  \tightlist
  \item
    Congruencia entre cultura y régimen político.
  \item
    Comparación entre cultura autoritaria y cultura democrática.
  \item
    La coexistencia de valores democráticos y autoritarios: características que produce.
  \end{itemize}
\item
  La ideología política:

  \begin{itemize}
  \tightlist
  \item
    Significado fuerte de la ideología.
  \item
    Significado débil de la ideología.
  \item
    ¿``Declinación'' y ``fin'' de las ideologías?.
  \end{itemize}
\item
  El mito político. Reflexiones para su recuperación como concepto analítico en el estudio de la política:

  \begin{itemize}
  \tightlist
  \item
    El mito político.
  \item
    Una experiencia de límite y de pasaje.
  \item
    Condenación y exaltación del mito.
  \item
    Valor del concepto de mito para el análisis de las situaciones políticas.
  \end{itemize}
\item
  La utopía y la ucronía. La utopía y el mito:

  \begin{itemize}
  \tightlist
  \item
    La utopía.
  \item
    La ucronía.
  \item
    Características generales de las utopías sociales.
  \item
    Relación entre utopía y mito.
  \end{itemize}
\end{itemize}

Parte III La teoría política ante el panorama mundial

\begin{enumerate}
\def\labelenumi{\arabic{enumi}.}
\setcounter{enumi}{8}
\tightlist
\item
  Teorías del primer mundo para el análisis del segundo y del tercer mundo
\end{enumerate}

\begin{itemize}
\tightlist
\item
  Introducción:

  \begin{itemize}
  \tightlist
  \item
    Análisis de las denominaciones.
  \end{itemize}
\item
  Teorías sobre Totalitarismo:

  \begin{itemize}
  \tightlist
  \item
    Origen del concepto.
  \item
    El totalitarismo según Hanna Arendt.
  \item
    El totalitarismo según Friedrich y Brzezinski.
  \item
    Análisis comparativo.
  \item
    Revisiones de las teorías clásicas.
  \item
    Resumen.
  \end{itemize}
\item
  Teorías de convergencia:

  \begin{itemize}
  \tightlist
  \item
    Caracteres generales.
  \item
    La teoría de P.A. Sorokin.
  \item
    Aportes de R. Aron.
  \item
    Convergencia en el pragmatismo.
  \item
    Supuestos básicos.
  \end{itemize}
\item
  Las comparaciones funcionales:

  \begin{itemize}
  \tightlist
  \item
    Estudios comparados.
  \item
    Papel de las ideologías.
  \item
    Estratificación social.
  \item
    Actitudes y alienación.
  \item
    Papel de las élites.
  \item
    Pluralismo de intereses.
  \end{itemize}
\item
  Los modelos evolucionistas decursivos:

  \begin{itemize}
  \tightlist
  \item
    Los países subdesarrollados.
  \item
    El ``camino único''.
  \item
    Caracteres del subdesarrollo.
  \item
    Enfoques teóricos del desarrollo.
  \end{itemize}
\end{itemize}

\begin{enumerate}
\def\labelenumi{\arabic{enumi}.}
\setcounter{enumi}{9}
\tightlist
\item
  Teorías del desarrollo político
\end{enumerate}

\begin{itemize}
\tightlist
\item
  Introducción:

  \begin{itemize}
  \tightlist
  \item
    Antecedentes y relación con desarrollo económico.
  \item
    Desarrollo y subdesarrollo.
  \end{itemize}
\item
  El desarrollo político como modernización:

  \begin{itemize}
  \tightlist
  \item
    El proceso de modernización y sus manifestaciones objetivas.
  \item
    Autores asociados a este enfoque.
  \item
    El planteo de G. Almond.
  \item
    Estructura y cultura.
  \item
    Etapas y condiciones del desarrollo.
  \end{itemize}
\item
  El desarrollo político como institucionalización:

  \begin{itemize}
  \tightlist
  \item
    Las variables de participación.
  \item
    Movilización social.
  \item
    El enfoque de S. Huntington: desarrollo y movilización.
  \item
    Etapas del desarrollo político.
  \item
    Estrategia para el desarrollo.
  \end{itemize}
\item
  El desarrollo político como incremento de la capacidad del sistema político:

  \begin{itemize}
  \tightlist
  \item
    Capacidad y modernización.
  \item
    Construcción de la democracia.
  \item
    El enfoque de Diamant y Organski: el desarrollo y sus etapas.
  \end{itemize}
\item
  El desarrollo político como modernización más institucionalización:

  \begin{itemize}
  \tightlist
  \item
    Los enfoques de Weiner y Horowitz.
  \item
    El planteo de H. Jaguaribe.
  \item
    Variables y dirección del desarrollo.
  \item
    Aspectos del desarrollo.
  \item
    Desarrollo generalizado o especializado.
  \item
    Etapas reales y funcionales.
  \end{itemize}
\item
  Evaluación crítica de las teorías del desarrollo político. Las teorías del desarrollo político y la crisis de la modernidad:

  \begin{itemize}
  \tightlist
  \item
    Fases de la ``cultura del desarrollo''.
  \item
    Teorías del desarrollo y cultura de la incertidumbre.
  \item
    Nudos problemáticos.
  \end{itemize}
\end{itemize}

\begin{enumerate}
\def\labelenumi{\arabic{enumi}.}
\setcounter{enumi}{10}
\tightlist
\item
  Teorías del imperialismo y de la dependencia
\end{enumerate}

\begin{itemize}
\tightlist
\item
  Algunas precisiones conceptuales:

  \begin{itemize}
  \tightlist
  \item
    Imperialismo y colonialismo.
  \end{itemize}
\item
  Teorías marxistas del imperialismo:

  \begin{itemize}
  \tightlist
  \item
    Bases teóricas.
  \item
    Rosa de Luxemburgo.
  \item
    Lenin.
  \item
    Baran y Sweezy.
  \end{itemize}
\item
  Teorías no marxistas del imperialismo:

  \begin{itemize}
  \tightlist
  \item
    Teorías social demócratas.
  \item
    Teorías del capitalismo anti-imperialista.
  \item
    Teorías del ``Estado-potencia''.
  \item
    Teorías de la soberanía nacional absoluta.
  \item
    Teorías de la repartición desigual de los réditos.
  \item
    Teorías de las estructuras sociales atrasadas.
  \item
    Teorías de las crisis económicas y del orden social amenazado.
  \item
    Teorías de las rivalidades entre potencias.
  \item
    Teorías de los determinantes extraeuropeos.
  \item
    Colonialismo y neocolonialismo.
  \end{itemize}
\item
  El imperialismo en las actuales teorías de las relaciones internacionales:

  \begin{itemize}
  \tightlist
  \item
    La escuela realista de H. Morgenthau.
  \item
    El paradigma trasnacional.
  \item
    El paradigma organizativo.
  \item
    La escuela behaviorista.
  \item
    La reacción antibehaviorista.
  \item
    El estudio del imperialismo.
  \end{itemize}
\item
  Las teorías del neoimperialismo y de la dependencia:

  \begin{itemize}
  \tightlist
  \item
    Teorías neomarxistas del imperialismo contemporáneo.
  \item
    Teorías neomarxistas de la dependencia.
  \item
    Teorías de la dependencia.
  \item
    Teorías radical-burguesas de la dependencia.
  \item
    Teoría del desarrollo ``interdependiente''.
  \item
    El enfoque de H. Jaguaribe.
  \end{itemize}
\end{itemize}

\begin{enumerate}
\def\labelenumi{\arabic{enumi}.}
\setcounter{enumi}{11}
\tightlist
\item
  La teoría política ante América Latina. Análisis y perpespectiva
\end{enumerate}

\begin{itemize}
\tightlist
\item
  Introducción:

  \begin{itemize}
  \tightlist
  \item
    Planteo general del tema.
  \end{itemize}
\item
  Principales características estructurales de América Latina:

  \begin{itemize}
  \tightlist
  \item
    Esquema centroperiferia.
  \item
    Dependencia internalizada.
  \item
    Extraversión económica.
  \item
    Cultura política de las apariencias.
  \item
    Violencia.
  \item
    Diferencia entre cultura de masas y de élites.
  \end{itemize}
\item
  Causas del subdesarrollo de los países latinoamericanos:

  \begin{itemize}
  \tightlist
  \item
    Subdesarrollo y dependencia.
  \item
    Estancamiento, marginalidad, desnacionalización.
  \item
    Cambios en la clase política.
  \end{itemize}
\item
  Dependencia o autonomía: Situación actual y perspectivas a mediano plazo:

  \begin{itemize}
  \tightlist
  \item
    Crisis del modelo satelital.
  \item
    Autonomía impensable.
  \item
    Transición a la dependencia provincial.
  \end{itemize}
\item
  Tendencias a largo plazo:

  \begin{itemize}
  \tightlist
  \item
    Descripción de un escenario probable.
  \end{itemize}
\end{itemize}

Parte IV Derecho constitucional mexicano

\begin{enumerate}
\def\labelenumi{\arabic{enumi}.}
\setcounter{enumi}{12}
\item
  Teoría de la Constitución
\item
  Los poderes Ejecutivo y Legislativo
\item
  El Poder Judicial
\item
  El federalismo mexicano actual
\end{enumerate}

\hypertarget{Introducciuxf3n}{%
\chapter{Introducción}\label{Introducciuxf3n}}

Algunas consideraciones generales sobre las Ciencias Sociales y, en particular, sobre la Ciencia Política.

En el texto ``Teoría de la Organización'', Giorgio Freddi\footnote{Note: Ver DICCIONARIO DE POLITICA de N. Bobbio et al.~- México -Ed. Siglo XXI - 1986 - pg. 1150.} construye un argumento de suma reelevancia que puede ser aplicado en todo campo científico social:

\begin{quote}
``(\ldots) en el presente contexto entendemos más bien por teoría un esquema conceptual o, mejor aún, un conjunto de esquemas conceptuales (que pueden ser complementarios o si no alternativos) cuyo objetivo (no necesariamente conseguido) es el de permitirnos describir, interpretar, posiblemente prever y eventualmente controlar fenómenos organizativos (\ldots)''.
\end{quote}

Giorgio deja en claro la imprecisión con que estamos acostumbrados a usar la palabra ``Teoría'' en el campo de las ciencias sociales: en sentido estricto o en sentido amplio; como principio abstracto o como ``lección de la historia''; como meta o como etapa del camino científico; o como ese mismo camino, cualquiera sea el modo y medida en que se lo haya recorrido. Al mencionar el ``camino científico'' estamos aludiendo a un tema fundamental: las cuestiones metodológicas. Respecto de ellas creemos valioso exponer un resumen y comentar las ideas de Eugène J. Meehan\footnote{Note: Ver ``La Ciencia - Minotauro o Mesías'' en Eugène J. Meehan: PEN- SAMIENTO POLITICO CONTEMPORANEO - Madrid - Rev.~de Occidente - 1973 pg. 55 y ss.}.

Todo estudioso serio tiene que interesarse en metodología. Introducirse en el estudio de la Teoría Política, por ejemplo, es introducirse en los problemas metodológicos de la Ciencia Política y en la discusión sobre cómo han sido resueltos. El problema de los métodos siempre ha motivado diferencias de opinión. La etapa contemporánea de esta discusión comenzó con la fragmentación de la gran ``mater scientia'', la Filosofía, a principios de la Edad Moderna.

La expansión del conocimiento científico se produjo tras una ruptura revolucionaria: el razonamiento deductivo a partir de principios a priori y la apelación a la autoridad fueron reemplazados por el razonamiento inductivo a partir de observaciones empíricas y por el cuestionamiento de toda autoridad. Dentro de este contexto, se pretendió que las ciencias sociales y hasta las humanidades imitaran a las ciencias físicas. A decir verdad, los resultados fueron deplorables. Finalmente, las ciencias físicas experimentaron una revolución conceptual profunda a principios del siglo XX, con la aparición de la Teoría de la Relatividad y la Mecánica Cuántica.Aún en ese campo, de las ``ciencias duras'', las relaciones absolutas fueron reemplazadas por relaciones probabilísticas (semejantes a las obtenibles en ciencias sociales) y se dejó de creer que la ciencia física produjera un conocimiento objetivo del mundo. El gran científico inglés Eddington concluyó su clásica obra sobre Física preguntándose si al mundo físico lo descubrimos o lo inventamos\ldots{}

La actividad científica newtoniana era algo así como explorar una compleja máquina, para descubrir lo que ya estaba allí. Esta idea es en buena parte responsable de las simplificaciones y de los dogmatismos de los positivistas del siglo XIX. Hasta un sociólogo de la talla de Emile Durkheim sostenía que no eran necesarios los estudios comparados porque creía que una investigación bien planteada descubriría el ``mecanismo básico'', de vigencia universal. La búsqueda de regularidades sociales que pudieran expresarse como leyes se abordó en términos newtonianos,pero esta concepción no resistió los embates teóricos, en especial los de la Antropología Comparada; se acumularon demasiadas anomalías y falló finalmente la analogía mecánica. Se produjo así la gran revolución intelectual moderna, cuyas consecuencias aún hoy continúan.

Dos principios han emergido de este proceso, de especial interés para las ciencias sociales:

\begin{itemize}
\item
  La ciencia no se ocupa de la naturaleza de la realidad. Nada esencial puede decir de la realidad. La ciencia es un procedimiento para ordenar y relacionar sistemáticamente los elementos de la experiencia humana, para anticipar experiencias ulteriores a la luz de las relaciones establecidas. La ciencia es una tarea específicamente humana: es lo que el hombre puede hacer por el hombre guiándose por la experiencia del hombre.
\item
  Todas las proposiciones científicas son relativas, condicionales y no absolutas. Son enunciados de probabilidad y no relaciones invariantes. Aunque el lenguaje científico no siempre lo exprese con claridad, las leyes científicas son siempre condicionales porque son inducciones, y no puede haber generalizaciones inductivas absolutas.
\end{itemize}

Con respecto a los supuestos metodológicos de la ciencia, cabe decir, en primer lugar, que la medición es súmamente deseable. En el campo social la medición es muy difícil, a veces casi imposible, sobre todo por la falta de verdaderas unidades de medida. Hay que admitir esa dificultad; hay que aceptar (transitoriamente, al menos) esa imposibilidad. También es cierto que la medición no es todo: hay aspectos del fenómeno que la medición no capta, pero no hay que confundir lo posible con lo deseable: poder medir sigue siendo deseable, aunque no podamos hacerlo.

En segundo lugar, pero no menos importante como actitud básica para la investigación, está el principio de que el conocimiento científico se define en términos de percepción y experiencia humana; no en términos de ``realidad'', ``verdad'' o ``absoluto''. A su vez, el concepto de EXPERIENCIA debe ser precisado. En este campo no se trata de la experiencia personal, subjetiva, única e irrepetible. La experiencia científica ha de ser pública, plural, abierta a la verificación o falsación por otros.

En principio, la finalidad de la ciencia es la explicación de los fenómenos observados. También abarca la organización de las observaciones y experiencias en generalizaciones y teorías que permitan predecir acontecimientos futuros. Cabe hacer notar, sin embargo, que la predicción no es un requisito indispensable del conocimiento científico, y que actualmente tiende a ser abandonada como actividad científica para ser vista más bien como una aplicación técnica.

La ciencia requiere que sus afirmaciones sean confirmadas por cotejo con los hechos, por observación sistemática y experiencia, si es posible hacerla, pero cabe aquí hacer notar que a medida que las estructuras explicativas se hacen más complejas, aparecen muchos niveles de generalización como intermediadores entre los observables y la teoría, con lo cual la comprobación (o falsación) empírica se hace cada vez más difícil. Esto es particularmente cierto en el caso de las llamadas TEORIAS GENERALES, que transitan por niveles de abstracción muy elevados, muy por encima de los hechos que serían en última instancia su base empírica.

En el campo científico, como es sabido, no corresponde enunciar valores de tipo moral. La ciencia es axiológicamente neutra, lo que no significa que los valores no existan, o que el científico, en cuanto hombre, no los tenga. Simplemente significa que esos valores caen fuera de la esfera de acción propia de la ciencia.

La ciencia se apoya en el empirismo; su camino es la observación, medición, conceptualización y generalización. No le basta la coherencia interna del pensamiento: necesita verificar la conexión entre los conceptos y los fenómenos concretos. La ciencia, por ejemplo, no puede negar validez al postulado idealista de la posibilidad afirmativa de captar las esencias, o a la afirmación cristiana que ve la mano de Dios en la historia del hombre. Sencillamente, no tiene lugar en sí misma para tales afirmaciones, que no pueden verificarse empíricamente, en los términos de la verificación empírica científica.

Tales son algunas de las grandezas y limitaciones de la ciencia como construcción del espíritu humano. Es importante hoy despojarla de los mitos consagratorios y de las condenaciones fulminantes para verla con serenidad en su real dimensión humana. El siglo XIX fué un siglo ilusionado con la ciencia. El siglo XX es un siglo decepcionado. Para una comprensión más real del tema, puede ser útil repasar las actitudes críticas que ha inspirado la ciencia moderna. Según Eugène Meehan hay tres grupos críticos principales:

\begin{itemize}
\tightlist
\item
  Los esencialistas y teleologistas.
\item
  Los dualistas.
\item
  Los cultores de la ``Verstehende Soziologie''.
\end{itemize}

Los críticos más duros de la actual metodología científica son los neoplatónicos, los aristotélico-tomistas y los idealistas hegelianos. Otra fuente de anticientificismo es el existencialismo, tanto en su versión teísta como en su versión secular, a causa de su vinculación con la Fenomenología de Husserl y el vitalismo de Bergson. Ambos niegan sentido a la realidad objetiva y postulan la existencia de un ``sentido interno esencial'' que se manifiesta únicamente en el proceso de ``experimentar la existencia''.

El anticientificismo más extremo, sin embargo, no se encuentra en la Filosofía sino en la literatura moderna, que en muchos casos tiende a valorar casi exclusivamente los aspectos subjetivos de la existencia y a aproximarse al nihilismo. Desde Franz Kafka hasta el movimiento dadaísta (con su renuncia a toda comunicación racional) o hasta los místicos del tipo de Simone Weil; a Ernest Hemingway, con su rechazo al pensamiento abstracto y su preferencia por la acción, hasta el mismo Sartre, que propone la ``acción sobre el medio'' como escape a la absurdidad de la existencia, son todos ejemplos de este anticientificismo, en un contexto en el que ``saber'' significa ``hacer''.

Los anticientíficos más adversos ven en la ciencia, no una estrategia equivocada sino un auténtico mal, un desvío, un peligro moral. La ciencia es vista como una práctica que priva a la vida de su misterio, de su pasión y su grandeza. Sören Kierkegaard, por ejemplo, afirma que todo conocimiento esencial gira en torno a la existencia, y que es verdad lo que el hombre cree apasionadamente. En esa visión, la ciencia es una distracción. Gabriel Marcel escribió páginas amargamente críticas (y lúcidas) contra la sociedad de masas, producto directo del cientificismo y la tecnocracia. Leo Strauss afirmó que el intento de crear una ciencia social ``científica'' ha llevado a una crisis filosófica total, porque hecho y valor constituyen una unidad que la ciencia, desgraciadamente, ha roto.

El segundo grupo mencionado por Eugène Meehan es el de los dualistas. Son los críticos más moderados de la ciencia, porque reconocen su valor en ciertos campos pero consideran que otros le son inaccesibles. Karl Jaspers, por ejemplo, concibe a la existencia dividida en tres sectores: la existencia empírica, la conciencia y el espíritu. Cada uno de ellos tiene su propia verdad. De tal modo, ciencia y filosofía ocupan esferas separadas,si bien mantienen contactos entre sí. Jacques Maritain, eminente tomista, también da una solución dualista al problema de la investigación científica, porque acepta el valor de la ciencia positiva en su propio encuadre, pero considera que en el campo ontológico el conocimiento se obtiene por percepción interna, no sujeta a la observación y verificación científicas. Similares opiniones sustentan autores como Ortega y Gasset, Reinhold, etc. En general, la posición dualista no se opone frontalmente a la ciencia sino que intenta sustraerle un amplio sector de fenómenos naturales y, sobre todo, culturales y sociales.

El tercer grupo mencionado por Meehan es el de los partidarios de la ``Verstehende Soziologie''. Estos constituyen un grupo muy diferente de los anteriores: los esencialistas y teleologistas rechazan por entero la ciencia tal como la entendemos; los dualistas tratan de restringirla a determinados campos; mientras que estos partidarios de la ``sociología de la comprensión (o simpatía)'' rechazan el método científico como inapropiado para el estudio de los fenómenos sociales, y proponen una vía alternativa: la ``comprensión''. Este concepto fué usado por Wilhem Dilthey en historia y por Max Weber en sociología y economía. Dilthey sostenía que las relaciones humanas contienen una ``cualidad significativa'', y que para captarla el investigador debe necesariamente hacer referencia a su propia experiencia humana, a su propia ``humanidad''.

Todo hecho o acto humano va siempre acompañado de una representación interna de su valor. La intencionalidad y el significado más profundo del acto emerge de esa representación, que no es observable desde el exterior y que solo puede captarse por simpatía o comprensión, en un contexto de interacción humana y de compromiso en la acción. La COMPRENSION es, pues, consecuencia de una visión interna de la condición humana, común a todos los hombres más allá de las pautas culturales particulares. Dilthey veía en la comprensión el fin mismo de toda investigación. Para Weber, ello no bastaba: la comprensión tenía que ser sometida a comprobación empírica.

El concepto de ``Verstehen'', que traducimos aproximadamente por ``comprensión'' es muy difícil de definir: visión en profundidad de las relaciones sociales; percepción afectiva de los motivos de la conducta humana; conocimiento interno, logrado por participación en los acontecimientos, etc. De todos modos, es siempre una forma de conocimiento lograda mediante la acción.

La ``Verstehende Soziologie'' plantea sus objeciones a las prácticas científicas corrientes por medio de postulados que, en nuestra opinión, contienen su parte de verdad, pero que llevados al extremo merecen a su vez serios reparos. Por ejemplo, su oposición a todo intento de generalización en la explicación de los hechos humanos. Sostiene ésto, en primer lugar, en base a la singularidad de los hechos. En toda la Historia -dicen- no hay dos hechos iguales, de modo que no pueden explicarse hechos mediante generalizaciones, que serían relaciones entre dos o más clases de eventos. Ahora bien,las clases de eventos se establecen, no en base a un criterio de igualdad sino a un criterio de semejanza, y todos sabemos que no existen hechos iguales pero sí hechos semejantes, por lo que esta objeción no nos parece válida. Otra razón que esgrimen se basa en la individualidad de los hechos. Sostienen algo muy cierto: los fenómenos sociales son totalidades, entidades indivisibles, cuyas partes no pueden analizarse sin alterar cualidades esenciales del todo. Esto es cierto, pero es necesario diferenciar las partes de un todo de sus rasgos, que sí pueden analizarse sin que el todo pierda sus cualidades propias.

Otra objeción de la ``Verstehende Soziologie'' se basa en la dosis de subjetivismo y de libre voluntad que contienen las acciones humanas. Toda acción humana -dicen- consta de dos partes: una subjetiva (no observable) y otra objetiva (observable). Una explicación adecuada de la conducta debe incluir ambos aspectos, lo que plantea el problema de los motivos de la conducta objetiva, que efectivamente son muy difíciles de determinar en forma certera. Esto es cierto, pero cabe observar que una buena parte de la conducta humana puede explicarse sin referencia a motivos subjetivos, o infiriéndolos hipotéticamente, en especial si esa conducta se produce en el contexto de situaciones muy estructuradas, como ocurre en el campo po- lítico. Por su parte, el argumento de la libre voluntad pierde buena parte de su eficacia si se recuerda que las generalizaciones (y más ampliamente, los enunciados científicos) tienen un significado probabilístico, tendencial, que deja márgen para comportamientos individuales fuera de norma. La experiencia ha evidenciado una notable regularidad de las conductas humanas, aún en períodos de cambio; y el carácter marginal de los comportamientos inesperados, si se trabaja con grandes números de relaciones.

A ésto se podría agregar la crítica que hace Habermas en su teoría de los intereses constitutivos de saberes, a las ciencias hermenéuticas, interpretativas, basadas en métodos de Verstehen. Estas ciencias están inspiradas en un interés práctico; producen un saber de entendimiento significativo, capaz de guiar el juicio práctico. Pero no son, según Habermas, una base adecuada para las ciencias sociales porque, si bien captan el significado subjetivo de los hechos objetivos, no descubren el modo en que ese significado subjetivo está condicionado o distorsionado por las condiciones sociales, culturales y políticas imperantes. Ese logro está, según Habermas, reservado para la ciencia social crítica, inspirada en un interés emancipatorio. Esto a su vez ha sido criticado, porque Habermas no proporciona claramente la base epistemológica, los criterios de racionalidad, que le permitan convalidar el ``saber emancipador''que surgiría de ella.

Ante este panorama cuestionador de la ciencia, Eugène Meehan concluye diciendo que la idea de que las reglas de la investigación científica no son aplicables al campo humano y social es ciertamente exagerada y no puede aceptarse, pero hay que tomar en cuenta la parte de verdad que contiene: la investigación científica social afronta problemas muy específicos, sobre todo en relación con las significaciones atribuíbles a los hechos.

La aplicación del método científico al campo social fué recibida con hostilidad por las tradiciones y los intereses establecidos, pero también fué cuestionada por los críticos sociales, que en general la vieron como una estrategia inadecuada para el conocimiento y la solución de los problemas sociales. Las principales influencias intelectuales reconocidas por los críticos sociales son:

\begin{itemize}
\tightlist
\item
  El marxismo, en especial el denominado ``humanismo marxista'' derivado de los escritos del Marx jóven sobre la alienación del trabajador respecto de su producto, etc.
\item
  La teoría psicoanalítica,en especial esa derivación llamada (bastante impropiamente) ``neofreudismo'', que quizás por influencia del ideario socialista de Alfred Adler, prácticamente invierte las concepciones sociales de Freud.
\item
  La filosofía de Hegel, en particular su enfoque metodológico.
\end{itemize}

Los críticos sociales en general cuestionan la ciencia,tienden al relativismo y opinan que el hombre ha de estar comprometido en la acción, y que el conocimiento se alcanza por participación. En IDEOLOGIA Y UTOPIA, de Karl Mannheim\footnote{Note: Ver Karl Mannheim: IDEOLOGIA Y UTOPIA - Madrid - Aguilar - 1973.}, encontramos un buen resumen de los cuestionamientos metodológicos de este grupo, en el que aproximadamente pueden incluírse autores como Barrington Moore, Irving Louis Horowitz, Maurice Stein, Arthur Vidich, Erich Fromm, Harry Stack Sullivan, Karen Horney, David Riesman, Norman Brown, Herbert Marcuse, etc. Entre sus argumentos básicos se destacan los tres puntos siguientes:

\begin{itemize}
\tightlist
\item
  La afirmación de que todo conocimiento es relativo a la situación social, y especialmente a la situación de clase;
\item
  La tendencia a concentrarse en la fuente del conocimiento o en los medios para adquirirlo más que en los procedimientos de verificación;
\item
  La estrecha relación que -se supone- existe entre crítica social y participación.
\end{itemize}

Para Mannheim (que lo toma de Marx) el observador social es un partícipe necesario de los procesos que observa. La teoría surge de un impulso social y clarifica la situación en que el impulso surgió. En ese proceso de clarificación,la teoría sirve para modificar la situación, y de ese cambio surge la exigencia de una nueva teoría. Es un interesante empleo de la dialéctica hegeliana, que niega la existencia de ``teorías puras'' y afirma que ``toda forma de pensamiento histórico y político está esencialmente condicionada por la situación vital del pensador y su grupo''. La teoría, pues, no puede separarse de la acción. Este enfoque podría llevar a un relativismo total y a un activismo devoto, situación que Mannheim intenta evitar, destacando por una parte que el pensamiento se ilumina ``no solo mediante la acción sino también mediante la reflexión que ha de acompañarla; y recurriendo, por otra parte, como tipo ideal de investigador al''intelectual desarraigado", de tenues vínculos de clase, poco condicionado por la ideología de su grupo y por ende, menos parcial.

Aún así, la posición de Mannheim permanece demasiado adscripta a un subjetivismo activista, que abre las puertas a interminables discusiones sobre la parcialidad subyacente en las explicaciones científicas de la política. En opinión de Meehan ``\ldots desde el punto de vista metodológico, la doctrina (de Mannheim) simplemente no funciona''.

Meehan concluye su tratamiento del tema recordando unas palabras de Anatol Rapoport\footnote{Note: Anatol Rapoport:``The Scientific Relevance of C. Wright Mills'' en Horowitz I.: THE NEW SOCIOLOGY pg. 107.}: ``La ciencia, con su actitud de desinterés, es el único modo de conocimiento de que disponemos que permite hacer productivos los choques entre opiniones incompatibles y que permite poner de manifiesto el grado de incompatibilidad entre opiniones distintas. De aquí que no se pueda prescindir del análisis lógico, la extensión de los conceptos, la comprobación de las hipótesis y todo lo demás si deseamos que el choque entre pensadores serios engendre luz además de calor''.

Nos ha parecido necesario hacer estas consideraciones introductorias al tema de la Teoría Política, para que se comprenda claramente el panorama que presenta en la actualidad el campo científico social, y particularmente el político, que es dentro del cual se van a inscribir todos los desarrollos posteriores. Comenzamos haciendo notar la amplitud de uso del vocablo TEORIA en este campo, y su relación con los problemas metodológicos.

Tres ideas emergen de allí con claridad:

\begin{itemize}
\tightlist
\item
  La independencia de la ciencia respecto del problema de la ``verdad'', en sentido religioso o filosófico;
\item
  Su sentido y valor como ordenador de la experiencia humana concreta en el mundo;
\item
  Su carácter relativo y condicional, por estar construída con generalizaciones inductivas.
\end{itemize}

En una expresión aún más sintética, podemos decir que la ciencia es una tarea humana que construye un ``sistema abierto de conocimientos''. Esta tarea ha recibido críticas. Algunas de ellas, en nuestra opinión, deben ser desechadas porque la critican o la niegan queriendo que la ciencia sea lo que no es. Otras sí deben ser tenidas en cuenta porque expresan dimensiones que pueden mejorarse en la actual y futura construcción y reconstrucción de la ciencia. En particular, dos enfoques aparecen claramente como valiosos:

\begin{itemize}
\tightlist
\item
  La comprensión (``Verstehen'') de la representación interior del valor de los actos humanos como complemento insoslayable de la observación sistemática de su manifestación interna.
\item
  El compromiso con la acción, superadora de la situación social que la teoría clarifica, pero sin perder la ``actitud de desinterés'' que diferencia a la ciencia de la ideología.
\end{itemize}

Este enfoque sobre las características y el valor humano de la Ciencia y la Teoría intenta ser amplio y realista a la vez. En nuestra opinión, él explica el criterio que ha presidido la construcción del ``panorama general de la Teoría Política'' que pretendemos presentar en los próximos capítulos. Hemos delimitado un vasto campo: colinda por una parte con la Filosofía Política y por otra con la política práctica; tiene otro límite en la ideología y el restante en las ciencias del hombre. Tiene, además, amplias franjas de interacción en todos esos rumbos.

Dentro de él hay lugar para muchas ``lecturas científicas'' de la realidad política; en él se han levantado muchos edificios teóricos sobre diferentes fundamentos metodológicos y cosmovisionales.

Esa ``ciudad de la Política pensada'', construida pacientemente por los hombres de muchos lugares a lo largo de muchas centurias, es lo que intentaremos describir aquí, posando una mirada comprensiva y -si se nos permite- afectuosa sobre el esfuerzo pensante de tantas generaciones. Por eso aquí están los planteos estructural-funcionalistas y sistémicos de la Teoría Política occidental; los enfoques crítico-dialécticos, en la amplia gama de sus manifestaciones teórico-prácticas; y los estudios normativos, desde Sun Zi hasta Platón, desde Aristóteles hasta Maquiavelo, desde Santo Tomás hasta Bertrand de Jouvenel.

Todos tienen algo que decirnos, algo que enseñarnos, y merecen nuestro respeto aunque nos parezcan equivocados.

La Ciencia,en su sentido más amplio y profundo, no es solo saber sino también comprender; no es solo conocimiento sino también sabiduría y aunque no es una mera receta técnica ni su finalidad se agota en la aplicación práctica, también ilumina el camino y orienta las acciones en la ``ciudad de la política vivida''.

\hypertarget{Lateoruxedapoluxedtica}{%
\chapter{La teoría política}\label{Lateoruxedapoluxedtica}}

Primera parte:

\begin{itemize}
\tightlist
\item
  Fases de la actividad científica: Teorías representativas y normativas
\item
  Descripción
\item
  Explicación
\item
  Generalización
\item
  Teoría
\item
  Cuasi-teorías: clasificaciones, dicotomías y analogías.
\end{itemize}

Segunda parte:

\begin{itemize}
\tightlist
\item
  La evaluación del fenómeno político: Ciencia y valoración
\item
  Los componentes del juicio normativo: descripción, evaluación técnica, juicio normativo, justificación del juicio normativo.
\end{itemize}

Tercera parte:

\begin{itemize}
\tightlist
\item
  El concepto teórico político
\item
  Comparaciones con otras ciencias: Teoría y Filosofía Política
\item
  Ciencia Política como disciplina autónoma
\item
  Teoría Política e Historia de las Ideas
\item
  Teorías generales y de alcance medio
\item
  Dificultades para la elaboración teórica.
\end{itemize}

\hypertarget{primera-parte}{%
\section*{Primera parte}\label{primera-parte}}
\addcontentsline{toc}{section}{Primera parte}

\hypertarget{fases-de-la-actividad-cientuxedfica}{%
\subsection*{Fases de la actividad científica}\label{fases-de-la-actividad-cientuxedfica}}
\addcontentsline{toc}{subsection}{Fases de la actividad científica}

Bertrand de Jouvenel,en su libro TEORIA PURA DE LA POLITICA, cuando habla sobre ``teoría'' en general, hace notar que las observaciones en sí mismas carecen de significado. Para darles sentido se debe formular una hipótesis que sea capaz de explicarlas.

Esto significa elegir conceptos, establecer relaciones entre ellos para elaborar un ``modelo'' que interprete adecuadamente la realidad. Esta compleja actividad de la mente humana se designa habitualmente como TEORIZAR; los modelos así elaborados tienen una función representativa-explicativa y carecen de valor normativo.

Bertrand de Jouvenel también menciona que en la Ciencia Política clásica, la llamada Teoría Política también ofrecía modelos, pero de otro tipo: eran modelos ideales o normativos, expresivos de un ``deber ser'' de los fenómenos aludidos, animados de una intención preceptiva.

Por respeto al pluralismo filosófico y porque forman indudablemente parte del pensamiento político sistemático, vamos a incluir en este libro el estudio de las teorías normativas, pero hacemos notar que en la Ciencia Política actual predomina netamente la actitud descriptiva-explicativa, estrictamente no-normativa.

La actividad científica cuyo producto final son las teorías, y a la que en su conjunto hemos llamado teorizar, consta de varias fases, que se encadenan en una sucesión ordenada. Esas fases reciben los nombres de: * Descripción; * Explicación; * Generalización; * Teoría o Cuasi-teoría.

\hypertarget{la-descripciuxf3n}{%
\subsection*{La descripción}\label{la-descripciuxf3n}}
\addcontentsline{toc}{subsection}{La descripción}

Las descripciones proporcionan el punto de partida al pensamiento; precisan aquéllo que luego hay que intentar explicar. Una descripción es válida, desde el punto de vista científico, si es producto de la observación sistemática y puede ser verificada mediante otras observaciones. Para describir hay que tener bien clara la diferencia entre ``hecho'' y ``concepto''. Un hecho es un conjunto de propiedades observadas, a las que se les ha puesto nombre. Un concepto es un artificio intelectual, un principio de abstracción, que permite operar con esas observaciones. La validez de los conceptos depende de la relación que guarden con los hechos de observación empírica en el mundo concreto.

Una descripción es más o menos fiable según la calidad y tipo de las observaciones que hayan servido para construirla. En general conviene tener en cuenta los siguientes principios:

\begin{itemize}
\tightlist
\item
  La observación de aspectos objetivos es más fiable que la observación de estados subjetivos.
\item
  Los datos controlados son más precisos que los obtenidos por simple observación.
\item
  Los datos medibles son más fiables que los no medibles, pero éstos suelen ser más importantes.
\end{itemize}

Para describir no basta con disponer de un cúmulo de observaciones. Es necesario tener, además, un esquema conceptual. En principio, este esquema configura una hipótesis e influye mucho en la descripción, y en el significado atribuíble a los hechos involucrados, por lo que es importante que no contenga prejuicios valorativos que puedan afectar la fiabilidad de la descripción.

\hypertarget{la-explicaciuxf3n}{%
\subsection*{La explicación}\label{la-explicaciuxf3n}}
\addcontentsline{toc}{subsection}{La explicación}

Un cúmulo de observaciones de hechos aislados no tiene en si mismo significado; la descripción le da un principio de significación, pero la plenitud de su significado y utilidad la alcanza cuando se lograr relacionar sistemáticamente los hechos. Ese proceso de conexión coherente de hechos diferentes se llama EXPLICACION y se hace a partir de descripciones.

En el contexto científico, explicar no significa ``captar la esencia'' ni nada por el estilo. Todo lo que podemos afirmar es que las cosas ocurren ``como si'' actuaran de determinada manera, y que podemos usar con razonable seguridad ese conocimiento, aunque no podamos ``explicar'' (en un sentido más profundo) porqué ese comportamiento es efectivamente así. La explicación vincula dos o más acontecimientos y a la vez crea un conjunto de espectativas hacia el futuro sobre la base de la experiencia del pasado.

En Ciencia Política -como en las ciencias del hombre en general- la inmensa mayoría de las explicaciones son inducciones probabilísticas. Muy rara vez es posible enunciar explicaciones deductivas. Encontramos explicaciones de hechos que probablemente van a ocurrir, pero con un considerable márgen de incertidumbre. Se usan, pues, expresiones tales como ``tiende a'', o ``generalmente'', o ``en la mayoría de los casos'', o a lo sumo ``en el n\% de los casos'' , en lugar de expresiones tales como ``siempre'' o ``nunca''.

La búsqueda de una ``explicación de la explicación'' es el paso a las fases siguientes, de la generalización y la teoría.

\hypertarget{la-generalizaciuxf3n}{%
\subsection*{La generalización}\label{la-generalizaciuxf3n}}
\addcontentsline{toc}{subsection}{La generalización}

Las generalizaciones se construyen a partir de explicaciones. Formalmente pueden ser definidas como ``proposiciones que relacionan dos o más clases de acontecimientos, de modo que todos o algunos de los acontecimientos de una clase lo son también de la otra u otras''. Hay tres tipos básicos de generalizaciones: * Las generalizaciones universales, que responden a la forma ``todo A es B''. Esta relación no es reversible: no todo B es A.

\begin{itemize}
\item
  Las generalizaciones probabilísticas, cuya forma es ``el n\% de A es B''. Este tipo de generalizaciones solo puede aplicarse a clases enteras, no a los miembros de una clase en forma aislada.
\item
  Los enunciados de tendencia, cuya forma es ``algunos A son B'' o ``A tiende a ser B, a menos que algo lo impida''. Se diferencian de los anteriores en que no especifican una relación numérica o porcentual entre A y B. También son aplicables a clases, no a individuos aislados.
\end{itemize}

Actualmente la Ciencia Política está compuesta casi totalmente por generalizaciones probabilísticas y enunciados de tendencia.

Las generalizaciones no son tautológicas porque añaden un conocimiento nuevo al vincular clases de acontecimientos. Son afirmaciones que van más allá de las descripciones y las explicaciones que les sirven de base. Dicen cosas sobre clases de acontecimientos no observadas en su totalidad, razón por la cual ninguna generalización es totalmente cierta, pero sí lo es en la medida de su alcance relativo y contingente.

Una generalización -y en general, toda proposición inductiva-nunca puede ``probarse'' mediante su cumplimiento en casos particulares, aunque así aumenta evidentemente su márgen de credibilidad. En cambio sí puede ``falsearse'' mediante la verificación de los casos en los que no se cumple, los cuales, de producirse, invalidan la proposición. En esencia, ésta es la posición epistemológica de Popper.

\hypertarget{las-teoruxedas-y-cuasi-teoruxedas}{%
\subsection*{Las teorías y cuasi-teorías}\label{las-teoruxedas-y-cuasi-teoruxedas}}
\addcontentsline{toc}{subsection}{Las teorías y cuasi-teorías}

Formalmente, una teoría es ``un conjunto de generalizaciones deductivamente vinculadas, que sirve para explicar otras generalizaciones''. Fundamentalmente, una teoría debe tener potencia explicativa sobre un determinado orden de fenómenos. También suele tener capacidad predictiva; indica áreas cuyo estudio debe profundizarse y sugiere los probables efectos de cambios producidos o promovidos en las variables que configuran una situación.

Las cuasi-teorías son estructuras conceptuales de tipo teórico, pero no deductivamente vinculadas. Algunas cuasi-teorías explican pero no predicen; otras predicen pero no explican; otras no explican ni predicen pero son muy sugerentes o aportan claridad al ordenamiento de las ideas.

En un planteo lógico-formal estricto, ``teoría deductiva'' es una jerarquía de proposiciones universales formalmente deducidas de un conjunto de primeros axiomas. En las ciencias del hombre no hay este tipo de teorías. Forzosamente hay que tener un criterio más amplio. Según A. Kaplan, cuando las generalizaciones están conectadas entre sí por medio del fenómeno que han de explicar (que es el caso más frecuente en las ciencias sociales) tenemos las llamadas ``teorías concatenadas''. Un ejemplo de ellas lo proporcionan las llamadas ``teorías de factores'', que explican fenómenos determinando las condiciones necesarias, o las suficientes, o ambas, para que el fenómeno se produzca.

Las teorías, pues, pueden ser deductivas (si cumplen las condiciones formales) o concatenadas, las cuales a su vez pueden ser: * causales: se refieren a las condiciones de aparición de los fenómenos; * genéticas: se refieren a los estadios de desarrollo de los fenómenos; * teleológicas: se refieren a su finalidad.

Hasta ahora, la mayor parte de las estructuras conceptuales de la Ciencia Política son cuasi-teorías, excepto algunas teorías factoriales. En Ciencia Política las generalizaciones realmente adecuadas para construir teorías son escasas; hay amplias zonas aún no exploradas en profundidad; la medición es difícil y muchas veces imposible; son muy pocas las posibilidades de realizar experimentos controlados, y la terminología es imprecisa. Por consiguiente, las teorías son débiles y los desarrollos científicos se basan sobre todo en cuasi-teorías, especialmente en dicotomías y analogías.

Para construir cuasi-teorías se supone que un conjunto de fenómenos se comporta de acuerdo a ellas. Se opera con los datos -por ejemplo- como si la analogía o la dicotomía fueran una teoría sólidamente establecida. Estas estructuras explicativas son valiosas; constituyen una estrategia de investigación positiva; son a menudo fuentes de futuras teorías, pero entrañan un riesgo grande: forzar los hechos para acomodarlos a una estructura previa, lo que produce resultados científicamente cuestionables. Los principales tipos de cuasi-teorías son las clasificaciones, las dicotomías y las analogías.

\hypertarget{las-clasificaciones}{%
\subsection*{Las clasificaciones}\label{las-clasificaciones}}
\addcontentsline{toc}{subsection}{Las clasificaciones}

Son las formas más simples de estructuras conceptuales teóricas. Son conjuntos de categorías a priori, usados para ordenar los datos provenientes de la observación. Un sistema de clasificación afirma que todos los miembros de una clase particular comparten -por definición- ciertas propiedades. Un buen sistema de este tipo clarifica y puede sugerir muchas cosas, pero no es en sí mismo una explicación ni añade nada nuevo a nuestros conocimientos. Su utilidad reside en el servicio que presta para la recolección ordenada de datos; y en las sugerencias con que puede orientar una investigación, especialmente en áreas poco exploradas. Aunque el ordenamiento propuesto luego resulte incorrecto y haya que reelaborarlo, lo mismo tiene valor porque siempre es más fácil manejar datos ordenados que datos distribuídos al azar. No existe un paradigma clasificatorio que sirva para todo. Cada clasificación responde a un propósito y su única condición de validez es que sea útil.

\hypertarget{las-dicotomuxedas}{%
\subsection*{Las dicotomías}\label{las-dicotomuxedas}}
\addcontentsline{toc}{subsection}{Las dicotomías}

Hay dos formas de dicotomías: una, más simple, está compuesta por dos polos opuestos, sin términos medios (son, por ejemplo, del tipo blanco/negro, día/noche, etc.). Otra. más compleja, toma la forma de un ``continuum'' entre dos polos extremos, con un centro o término medio y ciertos intervalos (medidos o no medidos) formando una escala o gradación entre los extremos. Una dicotomía compara y ubica, pero no explica. Enfoca la observación y sugiere estudios posteriores, pero tiene el inconveniente de que degrada fácilmente en un sistema de valoración. Técnicamente, puede decirse que una dicotomía es una forma particular de esquema clasificatorio. La utilidad explicativa de la dicotomía es heurística: plantea distinciones que requieren explicación y llevan al desarrollo de teorías factoriales. La principal objeción metodológica que puede hacersele es que compara cosas sin saber realmente si son comparables.

\hypertarget{las-analoguxedas}{%
\subsection*{Las analogías}\label{las-analoguxedas}}
\addcontentsline{toc}{subsection}{Las analogías}

Este tipo de cuasi-teoría es muy interesante y complejo. Tiene una larga tradición en el campo de la Ciencia Política. En general se reconoce la existencia de una relación de analogía cuando dos o más fenómenos pueden interpretarse como manifestaciones de un mismo principio regulador, en distintos planos.

En el campo de la Ciencia Política se utilizan principalmente analogías mecánicas u orgánicas. Se supone -por ejemplo- que la política en general o algún aspecto de ella es análogo en todo o en parte a alguna estructura mecánica o a algún organismo vivo, cuyo conocimiento puede servir para explorar, explicar o predecir algo respecto de los fenómenos estudiados.

El uso de analogías es útil mientras no se olvide que es solamente una comparación que sirve para dar una primera idea de la cosa, mientras se busca una enunciación más precisa. Por ello su valor es más didáctico y heurístico que investigativo propiamente dicho. Su principal problema es demostrar la real existencia de una relación de analogía entre el fenómeno y su presunto análogo. En la mente del investigador debe estar siempre presente el recuerdo de los peligros que entraña el uso indiscriminado de analogías o metáforas: * Atribuir a la realidad propiedades que son solo de su análogo.

\begin{itemize}
\item
  Pasar del análogo a la realidad y de ésta al análogo, creando falsas espectativas.
\item
  No precisar la congruencia entre el análogo y la realidad.
\item
  No tener clara conciencia de la utilidad solo parcial de estos instrumentos teóricos\footnote{Sobre el tema de este apartado en general, ver Eugène J. Meehan: PENSAMIENTO POLITICO CONTEMPORANEO - Madrid - Rev.~de Occidente - 1973 - pg. 19 y ss.}.
\end{itemize}

\hypertarget{segunda-parte}{%
\section*{Segunda parte}\label{segunda-parte}}
\addcontentsline{toc}{section}{Segunda parte}

\hypertarget{la-evaluaciuxf3n-del-fenuxf3meno-poluxedtico}{%
\subsection*{La evaluación del fenómeno político}\label{la-evaluaciuxf3n-del-fenuxf3meno-poluxedtico}}
\addcontentsline{toc}{subsection}{La evaluación del fenómeno político}

Como la intención general de esta obra apunta no solo a reseñar el estado actual de la investigación científica en el campo político sino también a aportar elementos para la práctica del análisis político por parte de los lectores, resulta pertinente incluir aquí algunas consideraciones sobre la evaluación del fenómeno político.

En el pensamiento de Eugène Meehan\footnote{Eugène J. Meehan: op. cit., pg. 41 y ss.} hay un intento muy claro y serio de incluir la evaluación entre las tareas de la Ciencia Política. En general, dice Meehan, los científicos huyen de la valoración y es sorprendente ver lo poco que se ha hecho en el ámbito de la Ciencia Política para desarrollar criterios y métodos adecuados para el análisis y evaluación de los fenómenos políticos. Su conclusión es que ese ámbito, abandonado por los politólogos, ha sido finalmente ocupado por otros, con resultados en general lamentables por su subjetivismo, tendenciosidad y condicionamiento ideológico. No hay razón, en su opinión, para que el desarrollo de juicios normativos no se lleve a cabo con el mismo espíritu, con los mismos instrumentos y por las mismas personas, que la explicación científica política.

Hay que producir, pues -según este criterio- un esquema analítico que clarifique la estructura de los juicios normativos en sus aspectos más significativos; y pautas valorativas que les puedan ser aplicadas. Según Meehan, los juicios de valor han de basarse en conocimientos sustantivos de Ciencia Política. El juicio normativo ha de referirse a una realidad, y desarrollarse en forma paralela al proceso de descripción-explicación -generalización que acabamos de ver. Meehan sostiene que es un grosero error pensar que, por la oposición que existe entre enunciados de hecho y de valor, no es posible sostener una discusión razonada sobre las argumentaciones normativas.

Un juicio de valor, o juicio normativo, consta de cuatro elementos:

\begin{itemize}
\tightlist
\item
  Una situación, o sea un conjunto de hechos relacionados, que va a ser objeto de la evaluación, tal como lo provee la descripción, tema que tratamos en el apartado anterior;
\item
  Un análisis de la relación medios/fines, y un análisis de las consecuencias probables de las acciones, o sea lo que se denomina precisamente evaluación técnica;
\item
  La reacción o respuesta del evaluador frente a la situación, de acuerdo a su sistema de valores, o sea un juicio normativo;
\item
  La fundamentación o justificación del juicio normativo, o sea el conjunto de razones de más o menos generalizada aceptación que lo avalan.
\end{itemize}

La situación (descripción): Es el punto de partida de todo el proceso de evaluación del fenómeno político. La conexión entre observaciones de hechos (obtención de datos) y la definición de la situación está dada por un esquema conceptual.

Para superar en todo lo posible el subjetivismo de estos esquemas, se pueden dar los siguientes pasos:

\begin{itemize}
\item
  Ver si la definición de la situación resulta aceptable a la luz del conocimiento científico de los fenómenos, empleando los mismos criterios utilizados para evaluar descripciones o explicaciones;
\item
  Estimar en qué magnitud la definición de la situación incluye orientaciones normativas o condicionamientos ideológicos. Ideológicamente, por ejemplo, se suelen disfrazar las evaluaciones de ``hechos evidentes por sí mismos''.
\item
  Ver qué aspectos de la situación son enfatizados en su definición. Se enfatizan las consecuencias para la sociedad o para el individuo? Se destacan los aspectos subjetivos o los objetivos?
\item
  Ver qué esquema conceptual se ha utilizado para construir la definición de la situación.
\end{itemize}

Hay que examinar, pues, cuatro puntos fundamentales: La situación está definida en términos científicamente aceptables? El esquema conceptual contiene alguna orientación normativa? La evaluación parte del individuo o de la comunidad? La evaluación parte de aspectos subjetivos u objetivos? La evaluación técnica: La aparición de la evaluación técnica se debe a que todo juicio normativo en el campo político consta de dos elementos:

\begin{itemize}
\tightlist
\item
  Enunciados sobre la relación entre acciones y objetivos, o sea la relación entre medios y fines de la acción política (que es el objetivo específico de la evaluación técnica);
\item
  Juicios de valor propiamente dichos (enunciados sobre bondad, conveniencia, justicia, etc., de tales acciones).
\end{itemize}

La forma general de la evaluación técnica suele ser: ``Para conseguir A, hágase B''. Una vez definidos los fines de la acción, la elección entre caminos alternativos para realizarlos es un problema de evaluación técnica. Se trata de lograr los ``mejores'' medios para lograr el fin propuesto (Cuáles? Los más seguros? Los más rápidos? Los más económicos? Los más éticos?). Las evaluaciones técnicas requieren explicaciones potentes, capaces de predecir el probable curso de los acontecimientos, y de determinar qué combinación de variables fundamentan esa predicción.

El juicio normativo: Es la reacción o respuesta de un evaluador frente a una situación. Generalmente se expresa en proposiciones que incluyen expresiones tales como ``bueno/malo'', ``justo/injusto'', etc. El significado de tales expresiones es relativo a cada orbe cultural. No está cerrada, ni mucho menos, la discusión filosófica sobre su contenido. Qué son? Reflejos condicionados? Respuestas emocionales? Percepciones personales de cualidades intrínsecas de las situaciones? Lo concreto es que tal significado difiere según las personas y los ámbitos culturales, y que tales juicios ``se hacen'': las personas los hacen al percibir las situaciones desde el complejo sistema formado por su estructura psicológica, su experiencia existencial, los valores que asimilaron o rechazaron de su sociedad y su cultura, sus emociones y sentimientos, sus intereses y racionalizaciones.

En sí mismos, los juicios normativos son enunciados de hecho sobre la reacción del individuo que los formula ante una situación. Hasta allí no hay nada que decir. Los cuestionamientos pueden surgir cuando se intenta fundamentar o justificar tales juicios.

La justificación del juicio normativo: Para justificar científicamente un juicio normativo tendríamos que disponer de criterios de los que la ciencia, al menos hasta ahora, carece. Los juicios normativos, mientras permanecen en un nivel personal no requieren justificación. Pero los razonamientos morales casi siempre tienden a salir de ese nivel y hacerse prescriptivos. Lo que ``es bueno para mí'' tiende a convertirse en lo que ``los demás deben aceptar como bueno'', o, más aún, en lo que ``es bueno en sí mismo''. En ese paso desde lo personal hacia lo social prescriptivo aparece el problema de la justificación del juicio normativo.

Dónde encontrar esos principios que resulten científicamente aceptables como fundamento de los juicios normativos? Cómo escapar a la crítica científica de los principios filosóficos, religiosos y éticos, cuestionados desde el punto de vista científico por considerar que incurren en subjetivismo, relativismo cultural, etnocentrismo, etc.? Recordamos dos intentos de este tipo: uno vinculado al nombre de Immanuel Kant; otro, al de Alfred Stern.

Dice el imperativo categórico de Kant: ``Hay que actuar como si la máxima que inspira tu acción hubiera de convertirse por tu voluntad en una ley natural universal''. Este célebre enunciado es, sin duda, una de las cumbres del pensamiento filosófico, pero aparecen no pocos obstáculos cuando se intenta instrumentarlo en la práctica, o sea utilizarlo como fundamento de juicios normativos concretos. Kant mismo intentó aportar los criterios necesarios para ello, pero sin llegar a una solución plenamente satisfactoria:

\begin{itemize}
\tightlist
\item
  Hay que tratar a los hombres como fines y no como medios;
\item
  No hay que eximirse a sí mismo de las normas morales;
\item
  Hay que aceptar la buena voluntad como único bien intrínseco.
\end{itemize}

Tales normas morales son muy valiosas, sin duda, pero no son decisivas. Kant desembocó finalmente en una especie de utilitarismo, y el utilitarismo por sí solo no puede habilitar una elección de pleno sentido ético entre líneas alternativas de acción.

Otro pensamiento de Kant , de similar orientación aunque más limitado en sus alcances, si bien alude directamente a un problema claramente político (que es el de la conflictiva relación entre el poder visible y el poder invisible), se encuentra en el Apéndice de su ``Paz perpetua'', en el que Kant enunció e ilustró el principio fundamental según el cual ``\ldots todas las acciones relativas al derecho de otros hombres, cuya máxima no es susceptible de tornarse pública, son injustas''.

Norberto Bobbio\footnote{Norberto Bobbio: IL FUTURO DELLA DEMOCRAZIA. UNA DIFESA DELLE REGOLE DEL GIOCO - Torino - Einaudi Ed. - 1984 .- (3) Alfred Stern: LA FILOSOFIA DE LA HISTORIA Y EL PROBLEMA DE LOS VALORES - Bs. As. - Eudeba - 1965.} la comenta diciendo que una acción que me veo obligado a mantener secreta es ciertamente no solo una acción injusta sino sobre todo una acción que, si se volviera pública, suscitaría una reacción tán grande que tornaría imposible su ejecución. Para usar el ejemplo dado por el propio Kant: Qué Estado podría declarar públicamente, en el mismo momento en que firma un tratado internacional, que no lo cumplirá? Qué funcionario público podría afirmar en público que usará el dinero público para fines privados? De este planteo del problema resulta que la exigencia de publicidad de los actos de gobierno es importante no solo (como se acostumbra decir) para permitir al ciudadano conocer los actos de quien detenta el poder y así controlarlos, sino también porque la publicidad es en sí misma una forma de control, un recurso para diferenciar lo lícito de lo ilícito.

Alfred Stern, en su libro LA FILOSOFÍA DE LA HISTORIA Y EL PROBLEMA DE LOS VALORES (3), después de hacer amplias referencias al carácter relativo, contingente, cultural, histórico de los valores en general, afirma haber encontrado un valor trans-histórico, válido para todo tiempo, lugar y cultura: ``Todos los hombres le han atribuido siempre un valor positivo a la vida y a la salud y un valor negativo a la enfermedad y a la muerte''. El enunciado es interesante, y el autor lo fundamenta en numerosas observaciones históricas (``no hubo suicidios masivos en los campos de concentración'', por ejemplo), pero cabrían algunas consideraciones para matizarlo, sobre la importancia de las condiciones de esa vida y el rol de la esperanza en la superación de condiciones-límite.

En síntesis, todo intento de justificación científica de razonamientos normativos conduce al enunciado de ``primeros principios'' que científicamente no se pueden fundamentar ni rechazar.No ocurre lo mismo en otros planos (moral, filosófico, religioso) de acuerdo a cuyas normas sí es posible formular evaluaciones normativas de fenómenos políticos. Cuál es, entonces, en definitiva, el aporte posible del enfoque científico en la formulación y el análisis de los juicios normativos? En nuestra opinión, ese aporte -muy importante, porque es un punto de partida- consiste en un más preciso esquema descriptivo-explicativo del fenómeno en sí, y en la correcta formulación de una evaluación técnica, sobre la adecuación de medios a fines. Ese es el límite del enfoque científico puro. Más allá se entra en un terreno donde lo científico colinda y se superpone con lo filosófico y lo religioso.-

\hypertarget{tercera-parte}{%
\section*{Tercera parte}\label{tercera-parte}}
\addcontentsline{toc}{section}{Tercera parte}

\hypertarget{el-concepto-teuxf3rico-poluxedtico.-comparaciones-con-los-de-otras-ciencias}{%
\subsection*{El concepto teórico político. Comparaciones con los de otras ciencias}\label{el-concepto-teuxf3rico-poluxedtico.-comparaciones-con-los-de-otras-ciencias}}
\addcontentsline{toc}{subsection}{El concepto teórico político. Comparaciones con los de otras ciencias}

Klaus von Beyme, en su obra TEORÍAS POLÍTICAS CONTEMPORÁNEAS-UNA INTRODUCCIÓN\footnote{Klaus von Beyme: TEORIAS POLITICAS CONTEMPORANEAS - UNA INTRODUCCION - Instituto de Estudios Políticos - Madrid - 1977.}, recuerda que en el contexto de las ciencias sociales, el desarrollo autónomo de la Ciencia Política moderna ha sido relativamente tardío. Hoy se entiende a la Ciencia Política como una ciencia diferenciada, en el ámbito de las ciencias sociales, que ha logrado un grado apreciable de acuerdo sobre su objeto y sus métodos.

La clásica separación de la Ciencia Política-teoría de las instituciones y Ciencia Política-teoría de los procesos políticos es cada vez menos sostenible. Mientras tanto, en todo el ámbito de las ciencias sociales se incrementa la exigencia de una colaboración interdisciplinaria. Esto se debe a dos razones: el riesgo que supone para las ciencias sociales la excesiva atomización de sus objetos; y el hecho ampliamente comprobado de que cada ciencia se basta a sí misma para describir los fenómenos de que se ocupa pero necesita del apoyo de otras ciencias para explicarlos.

La Teoría Política es, sin duda, un caso bastante particular, porque durante dos milenios la Filosofía Política ha proporcionado la contribución más importante a la teoría de la política. Pese a ello, hoy la Ciencia Política está reconocida como disciplina científica autónoma, al menos en todas las democracias occidentales, pero hay que hacer notar que, a diferencia de otras ramas filosóficas, la Filosofía Política se caracterizó siempre, al margen de su preocupación normativa, por su fuerte contenido empírico.

En su proceso formativo como ciencia social, la Ciencia Política tuvo que afrontar dos reproches principales: arrancar a otras disciplinas ``las plumas para adornarse con ellas'' (compartir parcialmente su objeto de estudio con otras disciplinas); y ser la responsable de la decadencia de la teoría política en el siglo XX porque los valores morales ya no tienen cabida en ella y la dominan técnicos y especialistas.

Una inseguridad adicional para la Ciencia Política -continúa comentando von Beyme- surgió del hecho de que a los cultores de esta disciplina no les correspondía ningún papel fijo que desempeñar dentro del cuadro de los roles profesionales establecidos en la sociedad burguesa. Con el tiempo, dice von Beyme, los graduados en Ciencia Política en los países desarrollados han ido consiguiendo puestos de trabajo en los siguientes campos: * Tareas docentes (profesores de ciencia social, formación de adultos); * Medios de comunicación de masas; * Actividades organizativas en la economía, la política y sus asociaciones; y en la administración pública, debido al desarrollo de una ciencia administrativa orientada cada vez menos en sentido jurídico y cada vez más como ciencia social.

A nuestro entender, desde que von Beyme anotó estas reflexiones a principios de la década de los setenta hasta hoy, el panorama de los roles profesionales de los politólogos se ha ampliado y esclarecido pero siempre en esa misma dirección básica. Creemos que hoy el conjunto de funciones sociales accesibles al politólogo puede describirse como sigue:

\begin{itemize}
\tightlist
\item
  Investigación científica (pura y aplicada);
\item
  Análisis político (asesoramiento específico o formación de opinión pública a través de los medios de comunicación social);
\item
  Docencia en ciencias sociales (secundaria, terciaria, universitaria, promoción cultural de la tercera edad, capacitación empresarial);
\item
  Gestión de políticas (diseño, planificación, coordinación de procesos de toma de decisión, coordinación de la ejecución, evaluación de políticas, análisis-aprendizaje);
\item
  Coordinación de equipos interdisciplinarios para la resolución de problemas públicos;
\item
  Político profesional;
\item
  Servicio exterior de la Nación u organismos internacionales;
\item
  Función pública jerarquizada.
\end{itemize}

Volviendo a la historia de nuestra ciencia, encontramos que, dentro del conjunto de las ciencias sociales, la Ciencia Política fue reconocida como disciplina independiente primero en los EE.UU., bajo fuerte influencia europea. La primera cátedra norteamericana de la especialidad fue creada en la Universidad de Harvard hacia fines de la década de 1850, y confiada a Francis Lieber, un profesor emigrado de Alemania, de tendencia liberal. Los pioneros americanos en este campo fueron J.W. Burgess y A.P. Bentley, que realizaron estudios de especialización en Alemania.

En Francia, en la década de 1870, encontramos la ``Ecole Libre des Sciences Politiques'', fundada en 1872 por Emile Boutmi, la cual es aún hoy el principal centro francés de estudio de las ciencias políticas.

En Inglaterra, un rol similar fue cumplido por la ``London School of Economics and Political Science'', institución que incluso alcanzó mucha influencia política práctica debido, por ejemplo, a la labor de Harold Laski.

En Alemania, recién después de la Primera Guerra Mundial se creó en Berlin un organismo investigador y docente (la ``Hochschule für Politik'') que fue el origen del mayor instituto alemán actual de Ciencia Política, el ``Otto Suhr - Institut''.

En España, el ``Instituto de Estudios Políticos'' de Madrid nació como institución de propaganda de la Falange, pero con la tendencia, que luego se desarrollaría ampliamente, hacia estudios políticos autónomos.

En Italia, los gloriosos antecedentes históricos que remontan a Maquiavelo y reconocen en Mosca y Pareto a los fundadores de la escuela italiana de Ciencia Política, sobrevivían solamente en el ``Instituto Cesaro Alhieri'' de Florencia, que fue suprimido por el fascismo, que fundó luego otras escuelas (Pavía, Padua, Perugia y Roma) que fueron la base de esa magnífica floración de la Ciencia Política italiana actual, que reconoce en B. Leoni, N. Bobbio y G. Sartori a tres grandes formadores de las nuevas generaciones de politólogos italianos.

En general, en sus manifestaciones académico-institucionales, la Ciencia Política ha cumplido un doble rol, como ``ciencia auxiliar de los gobernantes'' (afirmación que muchas veces se formula como un reproche) y como ciencia crítica y sobre todo esclarecedora respecto de la política práctica.

No hay en Ciencia Política una teoría general o unitaria predominante, de generalizada aceptación, como la que podemos encontrar, por ejemplo, en Economía. La actitud científica dominante en el mundo académico anglosajón -el neopositivismo- se pronuncia abiertamente en favor del pluralismo teórico, y aunque ya quedó atrás la postura del behaviorismo extremo, que equiparaba la Teoría Política con la Historia de las Ideas, y se le reconoce un lugar propio y autónomo en el ámbito de las ciencias sociales, aún se afirma, como dice H. Albert\footnote{H. Albert: TRAKTAT ÜBER KRITISCHE VERNUNFT, 1968, pg. 49; citado por K. von Beyme, op. cit.} que ``\ldots nunca se puede estar seguro de que determinada teoría sea cierta, aún cuando parezca resolver los problemas que plantea''.

Por nuestra parte, recordamos aquí que las teorías generales transitan por un nivel muy elevado de abstracción , muy alejado del nivel empírico donde podrían hallar verificación o falsación.

La producción teórica en Ciencia Política se inscribe en su mayor parte en las que Robert Merton denomina ``teorías de alcance medio'': teorías descriptivas-explicativas de modesto alcance, con algunos intentos de elevación hacia mayores niveles de abstracción.

En Ciencia Política, al igual que en otras ciencias sociales, se pueden encontrar los siguientes tipos de teorías: Teorías descriptivas: Son conjuntos de generalizaciones (relaciones entre clases de acontecimientos) basadas en conceptualizaciones y relaciones de origen empírico, ocasionalmente cuantitativas.

Teorías sistemáticas: Son sistematizaciones de base empírica, construidas en el marco de supuestos genéricos, de cierto nivel de abstracción.

Teorías deductivas: Formulan patrones de conducta hipotéticos, deducidos a partir de algunos axiomas básicos.

Teorías funcionales: Son interpretaciones de fenómenos que son parte de conjuntos mayores, construidas a partir del análisis de la función que tales fenómenos cumplen para el mantenimiento del conjunto en un determinado estado (o para cambiar de estado).

Teorías genéticas: Formulan hipótesis sobre el origen y el desarrollo inicial de fenómenos, estableciendo relaciones de causalidad o implicancia.

C.J. Friedrich\footnote{ C. J. Friedrich: PROLEGOMENA DER POLITIK. ERFAHRUNG UND IHRE THEORIE, Berlín, 1967, pg. 9; citado por K. von Beyme, op. cit.} plantea una tipología de las teorías más simple:

\begin{itemize}
\tightlist
\item
  Teorías morfológicas (tipo Copérnico);
\item
  Teorías genéticas(tipo Darwin);
\item
  Teorías funcionales (tipo Newton).
\end{itemize}

El prestigio académico y social de la Teoría Política ha variado mucho a lo largo del tiempo. Klaus von Beyme hace notar que en la historia de las ciencias sociales se alternan períodos de rechazo a la teoría (como la década de los '50) y períodos de gran auge teórico (como la década de los '60). Parece lógico pensar, como dice K. Deutsch, que en toda investigación importante la creación teórica, la metodología y los resultados empíricos se equilibran; pero desde el punto de vista del sentido final de la labor científica pensamos que pueden suscribirse las palabras de Dahrendorf cuando dice: ``La intención de la ciencia empírica es siempre teórica. La investigación experimental tiene justificación lógica únicamente como medio de control de las hipótesis derivadas de las teorías\ldots{}''.

Veamos, entonces, cuales son las características principales de las teorías políticas. En Ciencia Política -a semejanza de otras ciencias sociales- las teorías contienen tres elementos:

\begin{itemize}
\tightlist
\item
  Un sistema de proposiciones estructuradas, referentes a partes de la realidad política;
\item
  Una especificación de las condiciones bajo las cuales son válidas tales proposiciones;
\item
  La posibilidad de formular hipótesis predictivas sobre desarrollos futuros, en forma de enunciados de tendencia o de probabilidad, o sea proposiciones condicionales.
\end{itemize}

Cuando una teoría ha sido confirmada muchas veces, cuando ha demostrado ampliamente su operatividad, se la denomina ley. Cuando aún necesita verificaciones posteriores, se la llama hipótesis.

El cuerpo teórico de la Ciencia Política está compuesto por elementos de diverso grado de abstracción:

\begin{itemize}
\tightlist
\item
  Generalizaciones (relaciones entre clases de acontecimientos) que constituyen la mayor parte de la Ciencia Política;
\item
  Teorías sobre temas parciales (semejantes a las teorías de alcance medio, de R. Merton);
\item
  Intentos de plantear una teoría general (no aceptados en forma generalizada) como la teoría sistémica política de D. Easton.
\end{itemize}

En muchos casos, la política (lo mismo que la sociedad) es estudiada en sus posibilidades de ser manipulada, buscando, no una comprensión de sus procesos, sino soluciones prácticas, inmediatas, a problemas políticos concretos. Esto lleva frecuentemente a un exagerado auge de los procedimientos analíticos y de los conceptos que resulten operativos en la práctica, sin que preocupen mayormente su veracidad, su sentido histórico, etc. Priman en estos casos las exigencias de su aplicación en una tecnología social determinada.

La Ciencia Política encuentra numerosas dificultades en su elaboración teórica. Hemos de tener cuidado, en un repaso como el que vamos a hacer en los próximos capítulos, para no ser demasiado exigentes, porque muchas obras no satisfacen, o satisfacen a duras penas, las exigencias formales de una teoría científica.

Una dificultad principal en la elaboración teórica de la Ciencia Política se origina en la ubicación de las fuentes; no tanto de las fuentes de los procesos sociales como las fuentes individuales dispersas: los poderosos, los que realmente toman las decisiones o hacen que otros las tomen por ellos. Allí, frecuentemente el poder se protege a sí mismo, en el ocultamiento de los ``arcana imperii'', todavía vigentes, pese al torbellino de mensajes con que nos bombardean los medios, o gracias a ellos.

Hay muchos trabajos valiosos en Ciencia Política, que más que teorías acabadas son interpretaciones o esquemas analíticos. Tienen valor como acumulación de materiales; como manual divulgatorio o introductorio; como recensión del ``estado actual de la cuestión'' o ensayo provisional. Sirvan estas líneas como explicación de la presencia, en un Manual de Teoría Política, de muchos trabajos que un criterio más estricto hubiera desechado.

\hypertarget{Lasteoruxedasnormativas}{%
\chapter{Las teorías normativas}\label{Lasteoruxedasnormativas}}

Primera parte:

\begin{itemize}
\tightlist
\item
  Rasgos generales: Condiciones históricas y trasfondos ideológicos
\item
  Clasificación de las teorías normativas
\item
  Raíces intelectuales
\item
  Fundamentos
\item
  Finalidad
\item
  Relaciones
\item
  Metodología.
\end{itemize}

Segunda parte:

\begin{itemize}
\tightlist
\item
  Teorías políticas normativas clásicas: chinas, hindúes, judías, islámicas, griegas, romanas, medievales y modernas.
\end{itemize}

Tercera parte:

\begin{itemize}
\tightlist
\item
  Teorías políticas normativas contemporáneas: El asalto al absolutismo
\item
  Las consecuencias de la Revolución Francesa
\item
  Socialismos y nacionalismos
\item
  Las teorías normativas actuales.
\end{itemize}

Cuarta parte:

\begin{itemize}
\tightlist
\item
  Enfoques metodológicos usuales: Métodos: histórico, analógico, práctico, tópico, pedagógico.
\item
  El pragmatismo metodológico.
\end{itemize}

\hypertarget{primera-parte-1}{%
\section*{Primera parte}\label{primera-parte-1}}
\addcontentsline{toc}{section}{Primera parte}

\hypertarget{rasgos-generales}{%
\subsection*{Rasgos generales}\label{rasgos-generales}}
\addcontentsline{toc}{subsection}{Rasgos generales}

En general puede decirse que las obras de la gran corriente teórica normativa intentan, como toda teoría, describir y explicar los fenómenos de la vida política, pero ellas lo hacen poniendo el acento en lo que la política puede o debe ser, razón por la cual se aproximan fuertemente a la Filosofía Política, hasta confundirse con ella en algunas ocasiones. En todo teoría de esta corriente siempre subyacen preguntas tales como: Cuál es el mejor régimen político? o Cuál es el mejor régimen político posible? Estas teorías están siempre en relación con lo que se piensa que puede esperarse de la convivencia humana; y con el sentido de la vida que tenga cada autor y cada época según su particular cosmovisión. Transitamos, como puede verse, por un ámbito de fuerte vocación filosófica.

Las teorías de todo tipo son siempre producto del trabajo intelectual humano, en el marco de condiciones históricas objetivas y de trasfondos cosmovisionales de naturaleza fundamentalmente ideológica. Esto es particularmente visible en el caso de las teorías normativas, a tal punto que su mejor clasificación la proporciona la Historia de las Ideas Políticas. Podemos hablar así de teorías políticas normativas clásicas y de teorías contemporáneas. Las clásicas abarcan la producción de la Antigüedad (Grecia, Roma y Edad Media, en Occidente) y de la Modernidad (siglos XV a XVIII). Las contemporáneas son las originadas a partir del siglo XVIII. Todo esto se refiere al marco de la cultura occidental. Algo similar, con algunas diferencias, encontramos en el pensamiento político chino e hindú, como veremos más adelante.

Las teorías políticas clásicas antiguas abarcan el período mencionado porque en el pensamiento político hay continuidad y no ruptura entre el mundo greco-romano y el medieval. En cambio, sí hay marcadas diferencias entre aquellas obras y las que se producen en Occidente en la Edad Moderna, o sea desde el surgimiento de las naciones-estado (siglo XV).

Otro cambio importante encontramos en las obras de los siglos XVIII a XX, tras ese profundo cambio del principio de legitimidad que trajo consigo la difusión del ideario antiabsolutista.

En definitiva, creemos que podemos esquematizar el siguiente cuadro de clasificación de las teorías normativas: CLÁSICAS ANTIGUAS MODERNAS CONTEMPORÁNEAS ASALTO AL ABSOLUTISMO CONSECUENCIA DE LA REVOLUCIÓN FRANCESA SOCIALISMOS Y NACIONALISMOS ACTUALES Las teorías políticas antiguas se presentan como expresiones de filosofía práctica, en las que se entrecruzan las especulaciones racionales con las observaciones de la experiencia histórica y del devenir cotidiano. Procuran configurar doctrinas de la vida justa y buena, muy vinculadas a la Ética. En general entienden que la Ética es la visión estática y la Política es la visión dinámica del mismo objeto. Estas teorías se refieren a fenómenos que no son del ``episteme'', o sea de los determinismos naturales, sino del campo de las opciones conscientes de los hombres, en las que lo esencial es lograr la ``phronesis'', es decir, la cabal comprensión de la situación para actuar con lucidez y mesura, algo que también expresa el significado latino originario de la ``prudentia''.

Entre los saberes humanos, la Política ocupa el lugar más prominente en el pensamiento clásico, como ciencia práctica, ciencia del hacer (``prattein''), no de la especulación teórica (``theorein'') como la Lógica y la Matemática, ni de la creación (``poiëin'') como la Retórica, la Música o la Poesía.

El objetivo del saber político clásico no es solo el logro de la supervivencia sino la búsqueda de la seguridad de una vida buena,en libertad y virtud. No la hacían extensiva a todos, por supuesto (consideraban, por ejemplo, que la esclavitud era algo natural) pero ello no debe extrañarnos: siempre los hombres han racionalizado sus necesidades\ldots{}

Con el agregado del mensaje escatológico cristiano, esta tendencia se prolonga en el pensamiento político medieval, para el que el objetivo final de la comunidad política es permitir la marcha de la vida tras la virtud; en definitiva, es una larga meditación sobre las condiciones del bien común, entendido como conjunto de las condiciones socio-políticas que coadyuvan a la realización de la finalidad transpolítica del hombre: la salvación de su alma. Este esquema, con variantes individuales, es una constante en el pensamiento político antiguo.

A fines de la Edad Media y principios de la Edad Moderna se produjo una variación fundamental. La emergencia de los estados-naciones estuvo signada por cruentas guerras civiles, y en el pensamiento político el sistema de fines suprapolíticos fue sustituido por un sistema de supervivencia. El máximo objetivo político pensable parecía ser la simple seguridad de la existencia. Se produjo entonces una marcada separación entre Política y Ética, y se realiza con Maquiavelo una acabada exploración de las posibilidades técnicas de mantener una comunidad política, proceso que culmina en la formulación de la teoría de la razón de Estado, respaldo poderoso del absolutismo.

La etapa de las teorías políticas contemporáneas comienza con el asalto ideológico al absolutismo, obra principalmente del pensamiento político racionalista liberal. Es común denominador de estas primeras obras la reflexión sobre el equilibrio del poder y la libertad, y sobre el encauzamiento de la participación política acrecentada. El hecho culminante originado en este pensamiento fué la Revolución Francesa que al cumplirse originó obras de ampliación y esclarecimiento, y también obras de reacción crítica.

La segunda mitad del siglo XIX y los primeros años del siglo XX se caracterizan por obras que marcan la emergencia de los socialismos y los nacionalismos, en una atmósfera ideológica en general opuesta, por diversos motivos, a las ideas de 1789.

La experiencia socio-política emergente de la crisis económica de 1929, el surgimiento de los totalitarismos de derecha e izquierda y la Segunda Guerra Mundial configuran el marco fáctico originario de las obras normativas ``actuales''. Son éstas las que más nos interesan aquí, por su vigencia y por reflejar las condiciones de nuestro tiempo. Siguiendo en esto a von Beyme\footnote{Klaus von Beyme: op. cit.} vamos a sintetizar así sus principales características comunes: * Raíces intelectuales: La mayoría intenta restaurar la clásica teoría aristotélica de la política, en una nueva lectura influida por el relativismo de los valores, la quiebra de las antiguas democracias y la aparición de las dictaduras totalitarias del siglo XX. Tienen un fuerte interés en los estudios de historia de las ideas políticas. Destacan los valores supratemporales de las antiguas teorías políticas y procuran basarse en ellas. Están evidentemente dominadas por el ``realismo conceptual'' y la pasión hermenéutica, y revelan un cierto conservadurismo en su apego al significado originario de los conceptos y su rechazo a los neologismos.

\begin{itemize}
\tightlist
\item
  Fundamentos filosóficos: Son sumamente variados. Van desde el tomismo hasta el conservadurismo escéptico. Después de la Segunda Guerra Mundial no han aparecido teorías normativas con fundamento religioso.
\end{itemize}

La mayor parte de estas teorías basan sus desarrollos en alguna ontología. Avanzan por medio de conceptos hacia la construcción de una visión sistematizada, basándose en alguna ontología deductiva, de inspiración humanística teocéntrica o antropocéntrica. En general aceptan la hipótesis de la ``verdad objetiva'', aunque discrepen en los métodos para acercarse a ella o reconocerla.

\begin{itemize}
\tightlist
\item
  Finalidad: Su finalidad cognoscitiva es la acción, no el conocimiento en sí mismo. La Teoría Política Normativa, como ciencia práctica, apunta a perfeccionar la gestión política. Los autores que militan en esta corriente se oponen a la separación positivista y neokantiana entre el ser y el deber ser de la Política. Atribuyen a esa separación la falta de educación política y la generalización de la inmadurez política de gobernantes y gobernados.
\end{itemize}

Estas teorías acentúan la importancia de las teorías del gobierno y de la administración, en detrimento de los temas relacionados con la participación pública. A veces manifiestan una tendencia a la evasión hacia el esteticismo, tendencia que, por otra parte, comparten con muchos teóricos dialécticos de izquierda, desde Adorno hasta Marcuse.

\begin{itemize}
\tightlist
\item
  Relación con otros enfoques: Muchos teóricos normativistas conciben a la Teoría Política clásica como un medio para liberarse ``del rigor de los juristas, la brutalidad de los técnicos, la vaguedad de los visionarios y la vulgaridad de los oportunistas'' (Leo Strauss).
\end{itemize}

Estas teorías en general alientan un fuerte escepticismo sobre el valor real que pueda tener la acumulación de datos pormenorizados, al estilo positivista o empirista. Tienen, en cambio, algunos puntos en común con los enfoques críticos de la nueva izquierda: la oposición al neopositivismo, la finalidad del conocimiento orientada a la acción, etc. Por su parte, suelen recibir desde la izquierda el reproche de que pretenden construir una teoría finalista pero que no define su finalidad, y que termina adhiriendo en la práctica al sistema vigente y al statu-quo.

\begin{itemize}
\tightlist
\item
  Metodología: Las teorías normativas han aportado poco a la investigación empírica. Su enfoque metodológico no es semejante al de las ciencias naturales (medición, explicación causal, generalización) sino similar al de las ciencias prácticas, como la jurisprudencia, la terapéutica o la educación, que parten de problemas individualizados, o sea de la casuística, y tratan de resolverlos en base a reglas generales y precedentes. Son muy escépticas respecto del valor de los modelos abstractos y las teorías de alcance medio, y en especial de la teoría sistémica. Prefieren las teorías históricas (genésicas), los estudios de casos y las monografías prescriptivas. Frente a los intentos de reducción de los procesos políticos a otros tipos de variables, tales como las clases sociales, las condiciones tecnológicas o de producción, etc., son decididas partidarias de la autonomía de la política y de la ``política pura''.
\end{itemize}

En cuanto al lenguaje, los autores de esta corriente mantienen una relación estético-normativa con el idioma. En general escriben con un estilo depurado, elegante, consumado; y rechazan el vocabulario tecnicista de los neopositivistas.

En síntesis, podemos decir que las teorías normativas han promovido el estudio de las ideas políticas; que han hecho sugerencias valiosas sobre temas significativos para la investigación empírica; y que su aporte es muy importante para neutralizar la irracionalidad en los planteos del deber ser.

Pese a sus limitaciones, aún en medio de la polémica con los empiristas, la originalidad y erudición de los normativistas es siempre digna de respeto, ya que en ocasiones alcanzan niveles de ``sabiduría política'', de innegable valor.

No disponemos en esta obra de espacio para un tratamiento exhaustivo del tema. El lector interesado puede consultar la buena bibliografía existente sobre Historia de las Ideas Políticas\footnote{Ver, por ejemplo: G.H. Sabine: HISTORIA DE LA TEORIA POLITICA - México - FCE- 1984 J.J. Chevalier: LOS GRANDES TEXTOS POLITICOS - DESDE MAQUIAVELO HASTA NUESTROS DIAS - Madrid - Aguilar - 1979; y muy especialmente: F.Chatelet, O.Duhamel y E.Pisier: DICTIONNAIRE DES OEUVRES POLITIQUES - Paris - PUF - 1989.}.

\hypertarget{teoruxedas-poluxedticas-normativas-cluxe1sicas}{%
\subsection*{Teorías políticas normativas clásicas}\label{teoruxedas-poluxedticas-normativas-cluxe1sicas}}
\addcontentsline{toc}{subsection}{Teorías políticas normativas clásicas}

El pensamiento político clásico se caracterizó siempre por una intensa combinación de elementos de orígen filosófico especulativo y elementos de observación empírica provenientes de la experiencia vivida por los pueblos a través de la historia. De allí proviene el tono sorprendentemente moderno y hasta científico, en el sentido actual del término,que encontramos en tantas obras del pensamiento político, en el que también podemos abrevar algo que muchas veces echamos de menos en las creaciones del genio científico contemporáneo: la sabiduría y la comprensión de la política.

Este no es, desde luego, un libro de historia del pensamiento político. Las exigencias del espacio nos obligan, pues, a un programa suscinto: una ennumeración de las obras principales y el comentario más detallado de algunas obras especialmente representativas de los diversos períodos. Vamos a comentar, eso sí, algunas obras poco citadas en la bibliografía especializada, para hacer un aporte que no sea reiterativo. Asímismo, vamos a tratar de no incurrir en esa centración en Occidente de la que suelen adolecer muchas obras sobre la historia del pensamiento político; incluiremos, pues, consideraciones y referencias al pensamiento político no occidental.

Un listado de obras del pensamiento político universal que responda a dicho programa, debe mencionar al menos las siguientes:

\begin{enumerate}
\def\labelenumi{\alph{enumi})}
\item
  Pensamiento político chino: Confucio: TRATADOS MORALES Y POLITICOS (s. V aC); Sun Zi: EL ARTE DE LA GUERRA (s. V aC).
\item
  Pensamiento político hindú: (Atribuído a Manú): MANAVA DHARMASASTRA (?); Kautilya: ARTHASASHA (?).
\item
  Pensamiento político judío clásico: (Atribuído a Moisés): PENTATEUCO (?) Maimónides: GUIA DE LOS EXTRAVIADOS -- COMENTARIO SOBRE LA MISHNAH -- MISHNEH TORAH -- (1200 dC) d) Pensamiento político islámico clásico: Mahoma: CORAN (610-632); Ibn Taymiqya: TRATADO DE POLITICA JURIDICA (1311-1315); Ibn Khaldun: PROLEGOMENOS A LA HISTORIA UNIVERSAL (1375-1379).
\item
  Pensamiento político griego clásico: Tucídides: HISTORIA DE LA GUERRA DEL PELOPONESO (s. V aC); Platón: LA REPUBLICA - LAS LEYES- EL POLITICO (s. IV aC); Aristóteles: POLITICA (s. IV aC).
\item
  Pensamiento político romano clásico: Cicerón: DE LA REPUBLICA (55aC); Séneca: CARTAS A LUCILIUS (65aC).
\item
  Pensamiento político medieval: San Pablo: EPISTOLAS (65dC); San Agustín: LA CIUDAD DE DIOS (413-426dC); Santo Tomás de Aquino: SUMA TEOLOGICA (1266-1273); Dante Alighieri: DE MONARQUIA (1310); Marsilio de Padua: EL DEFENSOR DE LA PAZ (1324); Guillermo de Ockham: LA MONARQUIA DEL SACRO IMPERIO ROMANO (1349); Jan Hus: DE ECCLESIA (1415).
\item
  Pensamiento político moderno: N.Maquiavelo: EL PRINCIPE (1513); DISCURSOS SOBRE LA PRIMERA DECADA DE TITO LIVIO (1513-1519); T. Moro: UTOPIA (1516); M. Lutero: A LA NOBLEZA CRISTIANA DE LA NACION ALEMANA SOBRE LA ENMIENDA DEL ESTADO CRISTIANO (1520); J. Calvino: INSTITUCION DE LA RELIGION CRISTIANA (1536); E. de la Boetie: DISCURSO SOBRE LA SERVIDUMBRE VOLUNTARIA (1548); San Ignacio de Loyola: LAS CONSTITUCIONES DE LA COMPAÑIA DE JESUS (1556); T. de Bèze: DEL DERECHO DE LOS MAGISTRADOS (1574); J.Bodin: LOS SEIS LIBROS DE LA REPUBLICA (1576); H. Languet: REIVINDICACIONES CONTRA LOS TIRANOS (1579); T. Campanella: LA CIUDAD DEL SOL (1602); F. Suarez: DEFENSIO FIDEI (1613); Grotius: DERECHO DE LA GUERRA Y DE LA PAZ (1625); A-J. du Plessis, cardenal de Richelieu: TESTAMENTO POLITICO (1632-1639); R.Descartes: CARTAS A LA PRINCESA ISABEL (1643-1649); B.Pascal: PENSAMIENTOS (1662); S. Pufendorf: DERECHO NATURAL Y DE GENTES (1672); G. Leibniz: DEL DERECHO DE SOBERANIA Y DE EMBAJADA DE LOS PRINCIPES DEL IMPERIO (1677); J. Bossuet: LA POLITICA SACADA DE LA SANTA ESCRITURA (1677-1709); F. de Salignac de la Mothe (Fenelon): TELEMACO (1699); G. Vico: EL METODO DE ESTUDIOS DE NUESTRO TIEMPO (1709); Ch. I. Castel, abad de Saint-Pierre: PROYECTO DE PAZ PERPETUA (1713); F. Voltaire: CARTAS FILOSOFICAS (1734); Chr. Wolff: PRINCIPIOS DE DERECHO NATURAL Y DE GENTES (1758).
\end{enumerate}

Como puede verse, el criterio amplio utilizado en esta selección ha hecho incluir en ella obras que admiten más de una lectura. La Torah y el Corán, por ejemplo, tienen contenidos de teoría política normativa, pero son ante todo libros religiosos fundamentales; los libros sobre Derecho Natural son obras filosófico-jurídicas con contenido político, etc. Vayamos ahora a la descripción más detallada de estas corrientes de pensamiento y de sus obras más representativas.

\hypertarget{segunda-parte-1}{%
\section*{Segunda parte}\label{segunda-parte-1}}
\addcontentsline{toc}{section}{Segunda parte}

\hypertarget{el-pensamiento-poluxedtico-chino}{%
\subsection*{El pensamiento político chino}\label{el-pensamiento-poluxedtico-chino}}
\addcontentsline{toc}{subsection}{El pensamiento político chino}

No se trata de satisfacer aquí un gusto erudito por la erudición misma. Se trata, por un lado, de romper el esquema intelectual euro-céntrico (algo muy necesario en esta época de comunicación planetaria); y por otro de allegar información necesaria: no se puede, por ejemplo, comprender el marxismo maoísta y sus posteriores evoluciones sin conocer el sustrato cultural sobre el que está construído.

La organización política china clásica estuvo muy influída por el pensamiento filosófico, así como la filosofía china estuvo muy acotada por preocupaciones sociales y políticas, en sus fines y problemas. En el pensamiento político chino clásico encontramos dos corrientes principales y muy diferentes entre sí: el confucianismo (JU-CHIA) y el legalismo (FA-CHIA), que en la praxis política luego se unieron en una curiosa convergencia\footnote{Luigi Pareti et al.: HISTORIA DE LA HUMANIDAD - DESARROLLO CULTURAL Y CIENTIFICO - Tomo II - (Unesco) - Bs.As. - Editorial Sudamericana - 1969.}.

Confucio (551-479 aC) se basó en el modelo de la sociedad de su tiempo, de estructura feudal, planteando para ella una política basada en altos principios morales: el ``entendimiento de lo justo'' y una escala graduada de afecto y respeto que está formada por las ``cinco relaciones'': afecto entre padre e hijo, respeto entre gobernante y gobernado, amor entre marido y mujer, afecto entre hermano mayor y menor, lealtad entre amigos. Esa escala es la base del Estado, concebido esencialmente como un ente moral.

La elevada conducta moral del gobernante -sostiene Confucio- obliga a los gobernados a comportarse del mismo modo. Un Estado realmente bien organizado no necesita leyes ni policía ni tribunales. Si prevalecen la violencia y el crimen, la culpa es del gobernante que no da un ejemplo elevado. Esa es la diferencia entre el soberano legítimo (WANG) y el tirano (PA). El tirano, en la concepción confuciana, pierde moralmente su derecho a gobernar y el pueblo adquiere el derecho de rebelarse y derrocarlo.

El ideal político confuciano busca su fundamento remontándose míticamente al más remoto y venerable pasado, pero no es una teoría conservadora sino revolucionaria, que rechaza las precariedades y violencias del presente y del pasado próximo y evoca una ``edad de oro'' idealmente reconstruída y proyectada hacia el futuro.

Estos elevados principios chocaron muy frecuentemente con la dura realidad de las convulsiones sociales y la violencia de los estados feudales guerreros. El confucianismo intentó entonces ciertas formas de adaptación. Esa fué la obra de Hsün-Tzu (s.IIIaC) quien partió de la idea de la maldad intrínseca de la naturaleza humana para afirmar la necesidad de formular normas de conducta (LI), las que no son, de todos modos, leyes positivas coactivas sino un código de conducta, de cumplimiento obligado por el conformismo social pero sin sanción penal.

En el siglo IIIaC, por obra de Han-Fei-Tzu, surgió otra escuela de pensamiento político: el legalismo (FA-CHIA), muy opuesta a la anterior. Considera que la naturaleza humana es mala y que el hombre actúa bien solo bajo el acicate de la recompensa y la amenaza del castigo. Por su parte, afirma que las tradiciones del pasado carecen de valor porque ``a medida que las condiciones del mundo cambian se practican principios diferentes''.

El Estado -sostiene Han-Fei-Tzu- debe ser gobernado por medio de un claro y preciso conjunto de leyes (FA) que explique lo que se debe hacer y el premio y el castigo por hacerlo o no. El gobernante tiene autoridad (SHIH) para premiar y castigar. No necesita ser sobrehumano: solo precisa conocer el arte del gobierno (SHU) para encontrar y dirigir un personal eficiente, que cumpla sus órdenes.

Aplicando las teorías legalistas se creó un Estado autoritario-militar en el noroeste de China, que pronto dominó al resto del país: fué el estado CH'IN. El exceso produjo un gobierno de hierro, de exasperado centralismo. La rebelión generalizada de la población barrió con la dinastía CH'IN; los doctrinarios del legalismo fueron muertos y sus libros fueron quemados.

La dinastía emergente (HANG), invocando el nombre del confucianismo, en realidad combinó ambas escuelas: fué un aparato estatal legalista manejado por confucianos. El Estado fué gobernado por funcionarios de carrera, que estructuraron un imperio burocrático-centralizado, manejado por personas de alta cultura literaria tradicional. La receta fué tán eficaz que duró dos mil años, hasta nuestro siglo, sobreviviendo en su aplicación bajo diversas dinastías y a traves de las más variadas vicisitudes históricas.

Durante esa larga historia, la guerra fué la principal ocupación de la nobleza china. En ese contexto nació una obra notable, que tuvo y tiene una gran influencia: EL ARTE DE LA GUERRA, de Sun-Zi (S. V-IV aC).

Nuestra cultura occidental -ya lo hemos señalado- es excesivamente eurocéntrica: Grecia, Roma, Edad Media\ldots Pocas obras de otras culturas han logrado ejercer una influencia considerable en nuestro ámbito, y entre ellas se encuentra ésta, la más antigua obra de estrategia militar conocida, y sin duda una de las más notables. Los trece breves capítulos que la componen ocupan poco más de cien páginas, pero contienen, según autorizadas opiniones, como la de B. H. Liddel Hart, ``la quintaesencia de la sabiduría sobre la conducción de la guerra''.

Nada sabemos de su autor, Sun Zi, quien vivió bajo la dinastía HAN. En China y en Japón fué siempre tenido en alta estima, como puede verse por la cantidad y calidad de sus comentadores. A Occidente fué traído y traducido por el jesuíta francés J.J.M. Amiot, y publicado por primera vez en 1772. Tuvo luego una amplia difusión, multiplicándose las ediciones en francés, inglés, alemán y ruso.

Leyendo esta obra, enseguida surge el paralelo con Clausewitz, quizás el único teórico moderno que se le pueda comparar. Sin embargo, lo que Sun Zi escribió hace más de dos mil cuatrocientos años aparece hoy más claro, más profundo, más fresco. Tienen, por cierto, mucho en común: por ejemplo, ambos entienden a la guerra como emergente del orden político. ``La guerra es asunto de importancia vital para el Estado -dice Sun Zi- fuente de vida y de muerte, camino que lleva a la sobrevivencia o a la aniquilación. Es indispensable estudiarla a fondo''. Así comienza este tratado. Antes de pensar en la conducción de la guerra, Sun Zi establece su principio fundamental: la paz dicta su sentido a la guerra.

Antes que preocuparse por los problemas de técnica militar, que son epocales, Sun Zi se esfuerza por expresar la esencia de la estrategia militar en su relación con la política del Estado, que es lo permanente. Para Sun Zi, la guerra es una realidad inevitable, y aconseja limitar lo más posible su duración. Su tratado se refiere a la inteligencia de las relaciones de fuerza y al uso más racional (quiere decir, más económico) de las tropas. Busca conseguir la victoria por una combinación de astucia, sorpresa y desmoralización del adversario. Este último factor tiene la mayor importancia. Pocos teóricos de la guerra han enfatizado más la importancia de la guerra psicológica: el rumor, la intoxicación mental, la quintacolumna; sembrar la discordia entre el enemigo. corromper a sus cuadros jerárquicos, especialmente si son tropas mercenarias o generales de lealtad poco segura, etc.

Sun Zi considera que las guerras más mortales son las guerras de religión, las guerras civiles y las ``guerras nacionales''. Su idea de la guerra ``política'' se refiere principalmente a una guerra practicada en el seno de la misma sociedad, con medios y objetivos relativamente limitados, en el cuadro de reglas generalmente aceptadas: algo similar a los conflictos medievales europeos.

En sus principios generales para la conducción de la guerra, Sun Zi enfatiza la importancia de la moral y la cohesión de las tropas, y sobre todo de ``la armonía del pueblo con sus dirigentes''. Su estrategia se basa en el conocimiento del adversario, de sus concepciones y modos de obrar. ``Es de la más alta importancia -dice- combatir la estrategia del enemigo''. Aconseja tomar ventaja de los defectos de preparación del enemigo, evitar su fuerza y golpear su inconsistencia, hasta lograr un golpe decisivo. La guerra, cuanto más breve mejor, so pena de agotar también al vencedor. Es claro el eco que de estas concepciones pueden encontrarse, por ejemplo, en las obras de Mao sobre la guerra revolucionaria, como DE LA GUERRA REVOLUCIONARIA DE CHINA (1936) o DE LA GUERRA PROLONGADA (1938).

Sun Zi es un teórico no dogmático, consciente de la capacidad de adaptación a circunstancias imprevistas. ``Así como el agua no tiene una forma estable, no existen en la guerra condiciones permanentes'' -dice, y añade: ``no hay que temer quebrantar las órdenes del soberano si la situación sobre el terreno lo exige''. El coraje y el talento del jefe de la guerra se miden también por la capacidad de infringir las órdenes cuando se tiene la íntima convicción de poseer la llave táctica de una situación.

Lejos de alabar la guerra en sí, Sun Zi desea limitarla en el tiempo y hacerla menos costosa en medios y en hombres gracias al factor moral. Por ello desaconseja las guerras de sitio y aconseja las de movimiento, que juegan con el factor sorpresa y el punto débil del enemigo.

En esencia, el ``Arte de la Guerra'' es un tratado militar, que toma como postulados básicos una política prudente, un empleo mesurado de la fuerza, el uso de la inteligencia y de la astucia, combinadas con la firmeza de espíritu y la tenacidad. La obra de Sun Zi es una conceptualización genial de los conflictos militares. La guerra no es considerada en ella bajo su ángulo moral ni como un hecho accidental. Para Sun Zi, el problema de la guerra es central para el Estado, un acto consciente que puede ser analizado rigurosamente y cuyo sentido es dictado por la paz.

\hypertarget{el-pensamiento-poluxedtico-hinduxfa}{%
\subsection*{El pensamiento político hindú}\label{el-pensamiento-poluxedtico-hinduxfa}}
\addcontentsline{toc}{subsection}{El pensamiento político hindú}

La ley religiosa-social, o sea el DHARMA, que es algo distinto de la administración y la política, es el tema de una abundante literatura en la India. La obra más importante, al parecer, es el MANAVA DHARMASASTRA, atribuído a Manú, el primer hombre, la cual ejerció una enorme influencia jurídica, política y social en la vida del pueblo hindú. Se la ha conocido en Occidente con el nombre de CODIGO DE LAS LEYES DE MANU.

Según el MANAVA DHARMASASTRA hay cuatro fuentes de la ley: las Sagradas Escrituras, los libros legales, las costumbres de los hombres santos y el sentir íntimo del hombre sobre lo justo y lo injusto. La garantía de la ley es el castigo, graduado según la falta y según la casta del infractor.

Este es el libro que consagra el sistema de castas en la India. Los brahamanes ocupan todos los puestos dotados de ascendiente social y de poder político: sacerdote, maestro, juez, ministro, miembro de la Comisión Legislativa Permanente (DHARMA-PARISHAT). Sus delitos en general tienen penas más leves y nunca son condenados a muerte.

Los ksattriyas tienen el privilegio y el deber de hacer la guerra, con el carácter de una obligación religiosa. La guerra asumió un carácter ceremonial, con complejas reglas rituales, aunque la presencia de invasores extranjeros (que no respetaban las reglas) impidió que se transformara completamente en un rito. De todos modos, ese estilo ``tradicional'' y conservador de hacer la guerra aseguró el triunfo de todos los invasores que a lo largo de los siglos penetraron en el territorio hindú.

Los sudras son tratados duramente por las leyes de Manú, y se les reservan los trabajos y posiciones inferiores, pero no las actividades consideradas degradantes e ``impuras'', que están reservadas a los parias o ``intocables'', que están fuera del sistema de castas.

La India careció de una tradición unitaria y de una burocracia centralizada. Cada reinado tenía su propia organización, dentro de un modelo tradicional, del que en realidad poco se sabe. El Rey era jefe titular del Estado y también jefe del Gobierno. Era el centro de una vasta corte. Su gobierno se basaba en la sospecha sistemática, que daba trabajo a un ejército de espías y contra-espías, y hasta a una guardia de mujeres armadas, que controlaban el acceso a las habitaciones privadas. Los ministros formaban un cuerpo de consejeros y asesores que elegían a los funcionarios inferiores.

Una obra hindú que puede ser considerada de teoría política secular es el ARTHASASHA, atribuído a Kautilya, el ministro de quien se dice que fué el verdadero fundador del imperio Mauria. La forma de gobierno que allí se describe es una monarquía absoluta, en la que el poder real no está limitado por la costumbre, aunque el Rey está aconsejado por un conjunto de altos funcionarios, cabezas de la administración pública. El contacto con la opinión pública se mantenía por medio de un bien organizado sistema de espías y agentes secretos. Es un esquema político típico de pueblos dominados por invasores externos: el Estado no es una unidad sino un elemento de un conjunto, en cuyo centro está el conquistador, con su círculo de aliados ocasionales y de enemigos reales y potenciales. En ese contexto signado por la deslegitimación y la deslealtad, la política es un arte práctico, despojado de su dimensión moral.

\hypertarget{el-pensamiento-poluxedtico-juduxedo-cluxe1sico}{%
\subsection*{El pensamiento político judío clásico}\label{el-pensamiento-poluxedtico-juduxedo-cluxe1sico}}
\addcontentsline{toc}{subsection}{El pensamiento político judío clásico}

El pensamiento político judío clásico está raigalmente vinculado al ``libro'' por antonomasia - la BIBLIA; y en particular a sus cinco primeros libros - la TORAH, como es nombrada por judíos y musulmanes, o el PENTATEUCO, según la denominación cristiana. Todas las tradiciones atribuyen su inspiración al Dios único, y su autoría material a un personaje algo histórico y algo legendario: Moisés ben Amram. El primer libro de la Torah -GENESIS- narra la creación del mundo y la genealogía de las familias humanas después de Adam y Eva, hasta la llegada de los hijos de Jacob a Egipto. Los otros cuatro libros -ÉXODO, LEVÍTICO, NÚMEROS y DEUTERONOMIO- relatan la actividad política de Moisés como profeta: organizador de la huída de Egipto, legislador de inspiración divina, jefe del campamento israelita durante los cuarenta años de la ``travesía del desierto'', creador de las bases ideales de la ``ciudad de Dios'' en la Tierra Prometida.

Las leyes de Moisés han constituido la referencia esencial de tres universos espirituales: judaísmo, cristianismo e islamismo; y la base ideal de los más diversos sistemas políticos. En el caso judío, ellos abarcan desde el gobierno militar de Josué, el régimen de los Jueces, los reinos de Saúl, David y Salomón, la conducción del retorno del cautiverio en Babilonia, el reino de los Macabeos, etc. Las leyes de Moisés son también el tema mayor de la exégesis de los Sabios, los Doctores de la Ley, luego Rabinos, en esa inmensa literatura omnicomprensiva de lo humano (y por consiguiente también política) que es el TALMUD, de Jerusalem y de Babilonia. La Torah ha servido también de motivación y bandera a todos los cuestionamientos sectarios, cismáticos o heréticos que se han alzado frente al poder ortodoxo de los Rabinos. Algo similar ha ocurrido en el ámbito cristiano y en el musulmán, de modo que a través de lecturas sucesivas y de etapas de interpretación, la Ley de Moisés, considerada como Palabra de Dios que utiliza a Moisés como portavoz, ha sido y es el fundamento ideal al cual se refieren los partidarios ortodoxos de las tres religiones monoteístas, así como también los cuestionadores de la autoridad temporal o espiritual de los cleros en el seno de cada una de las tres grandes familias religiosas.

El segundo libro de la Torah -EXODO- es el que contiene la Ley fundamental, los Diez Mandamientos; pero desde el punto de vista puramente político, los libros más densos son LEVITICO y NUMEROS, que enuncian en todos sus detalles las leyes, reglamentos, mandamientos y observancias revelados por mediación de Moisés a los israelitas. De allí surge la descripción de un sistema de gobierno y de organización social complejo y coherente, de un tipo relativamente único en esa región y en esa época: un estado sacerdotal y militar que emerge sobre un orden tribal que aún subsiste. Ese Estado-Ley legisla, prohíbe y reprime, pero oculta su monopolio de la violencia. Es Dios quien aparece castigando y exterminando a los rebeldes, y es la comunidad quien ejecuta por lapidación a los delincuentes, como contrapartida de la igualdad de todos ante el juicio de la Ley. La base de la ciudadanía no es la igualdad de condición sino la sumisión a la Ley y la participación en el consenso social.

Este poder de la Ley no reposa únicamente sobre el peculiar sistema de control social militar-policial de diseño cuadriculado (los jefes de mil, los jefes de cien, los jefes de diez) instaurado por Moisés, sino que se basa también en la existencia de una tribu-casta ``consagrada al servicio de la Tienda'', o sea del Arca de la Alianza, versión nómada del Templo.

Por medio del monopolio de los sacrificios (que implica también el control del consumo de carnes) y de la administración de justicia según la Ley, esa tribu-casta configura un régimen singular, fundado en una burocracia sagrada, que realiza una concepción del poder sacerdotal sobre bases religiosas. Ella opera como contrapeso de los poderes monárquicos o aristocrático-militares. El cuadro se completa con la acción de los Profetas, personas iluminadas, que hablan en nombre de la Divinidad, trasmitiendo sus mensajes en forma directa, sin intermediación de las instituciones sacerdotales establecidas; mensajes que con frecuencia presentan contenidos fuertemente críticos hacia el accionar de los gobernantes y del mismo pueblo. Se configura así una particularísima ``división de los poderes'' que frena las tentativas hegemónicas.

Entre los pensadores judíos importantes para la historia de las ideas políticas, quizás el más significativo sea Moisés Maimónides, cuyos escritos dejaron una impronta profunda en todo el pensamiento político posterior.

Moisés Maimónides nació en Córdoba (España) en 1135 o 1138. Estudió la Ley hebrea con su padre, y Filosofía y Ciencias Naturales con sabios musulmanes, en un período de feliz convivencia inter-religiosa, que pronto tuvo fin. Maimónides debió emigrar por la persecución religiosa desatada por los almohades, y vivió sucesivamente en Marruecos, Acre, Jerusalem y finalmente en El Cairo, donde fue el médico del visir de Saladino. Murió en dicha ciudad en 1204.

Como pensador político, Maimónides escribió:

\begin{itemize}
\item
  COMENTARIO SOBRE LA MISHNAH (1168): Escrito en árabe, es una explicación del gran código de derecho rabínico (la ``Mishnah'') que fué elaborado en el siglo III dC e incorporado al Talmud.
\item
  MISHNEH TORAH (1180): Escrita en hebreo, es igualmente una tentativa de exponer las leyes talmúdicas de una manera clara y sistemática.
\item
  GUIA DE LOS EXTRAVIADOS (1185-1190?): Obra magistral de Maimónides, escrita en árabe, examina el problema planteado por la filosofía griega a aquellos que creen en la Verdad Revelada.
\end{itemize}

Maimónides utilizó categorías conceptuales tradicionales, pero reinterpretadas de manera no tradicional, como puede verse en su redefinición de PROFETA, de la ERA DEL MESIAS y del OTRO MUNDO. Sostiene, por ejemplo, que solo un individuo intelectualmente perfecto (un filósofo, en definitiva) puede ser profeta, lo que en una óptica tradicional entrañaría una limitación a Dios en cuanto a la elección de quien desee como profeta. Otro ejemplo es el MESIAS, tradicionalmente percibido como una figura apocalíptica, propia del fin de los tiempos, y que es transfigurado por Maimónides en un jefe político que, sin cambiar nada en las leyes naturales, logrará la independencia política y la soberanía para los judíos en la tierra de Israel, lo que lo convierte en un remoto precursor del sionismo moderno. La vida en el OTRO MUNDO es vista por Maimónides como la unión del alma teorético-racional con el intelecto activo, realizable en forma individual, al margen de la redención colectiva, que era la concepción hebrea tradicional.

Maimónides utiliza fuentes religiosas tradicionales de una manera nueva. Pasa por alto las fuentes que no concuerdan con su punto de vista y acentúa la importancia de aquellas que refuerzan su posición, es decir, su propia comprensión filosófica del judaísmo. Por otra parte, desarrolla su pensamiento filosófico utilizando elementos filosóficos de origen no judío, sobre todo griegos e islámicos. Admira especialmente a Aristóteles, de quien decía que ``su inteligencia representa el extremo de la inteligencia humana, excepto la de quienes han recibido inspiración divina''; y a Al-Farabi, cuyos AFORISMOS DEL POLÍTICO le hicieron afirmar que ``todos sus escritos son irreprochablemente excelentes'' y que ``se los debe estudiar y comprender, porque es un gran hombre''.

Maimónides procura siempre interpretar las informaciones bíblicas y post-bíblicas según razones naturales, consideraciones prácticas y explicaciones racionales, antes de apelar a lo milagroso. Intenta encontrar explicaciones racionales a todas las leyes del código judío, y dar razones educativas a casi todos los acontecimientos de la historia humana y natural. Por otra parte, a diferencia de pensadores árabes como Al-Farabi o Ibn-Ruchd (Averroes), que procuran elaborar sus teorías políticas en términos teóricos aplicables a todas las naciones y religiones, Maimónides mantiene su teoría política dentro del contexto del judaísmo.

El pensamiento político de Maimónides puede sintetizarse en dos áreas complementarias: una referida a la vida práctica, a la estructura político-social que recomienda para que sea adoptada por las comunidades judías; y otra referida a la estructura teórica de su pensamiento, donde se evidencian las influencias filosóficas no judías que experimentó.

Respecto del primer punto, Maimónides destaca la importancia y las responsabilidades que gravitan sobre los jefes comunitarios de todo tipo y nivel, cuyas cualidades para esas tareas se centran en la adquisición de la prudencia; y cuyas diferentes misiones o cometidos deben asegurar una división de poderes personales, que impida la emergencia de tentativas hegemónicas que irían en contra de la soberanía última de la Ley, entendida como expresión de la Voluntad Divina. En ese sentido, cabe considerar a Maimónides un lejano precursor de la ``división de poderes'' que varios siglos después postulara Montesquieu. El pensador judío que comentamos la fundamenta en la necesidad de preservar la primacía de la Ley contra la propensión arbitraria de los poderosos, por medio de una adecuada división de las funciones de conducción política.

En cuanto a la estructura teórica de su filosofía política, cabe mencionar los siguientes elementos: - La distinción entre la ELITE, o sea el pequeño número de los que realmente poseen la ``virtud intelectual'', y la MULTITUD de la gente ordinaria, que es vista como ``enferma del alma'', con características animalescas, cuyo sentido de vida es servir y acompañar a los sabios. Es notoria la influencia de Platón y de Al-Farabi en esta concepción de la sociedad.

\begin{itemize}
\item
  El análisis del conflicto entre el compromiso comunitario y la contemplación metafísica solitaria. Sus planteos no carecen de ambigüedad en este aspecto, pero en definitiva sus escritos y su ejemplo personal reconocen que el compromiso comunitario es parte importante de la actividad del individuo virtuoso y perfeccionado. Aún así, en páginas de cálida y espontánea humanidad, lamenta que sus múltiples ocupaciones, sus diarias tareas de médico de la Corte y sus tareas vespertinas de consejero de la comunidad judía de El Cairo, le dejen tán poco tiempo para sus escritos y sus meditaciones\ldots{}
\item
  El estudio de la fuerza y la debilidad de la Ley religiosa como encuadre de las acciones del pueblo judío. Es la Ley quien organiza la vida política del pueblo judío como un conjunto. Solo unos pocos individuos en cada generación pueden vivir según los principios de la Razón. Para todos los demás, la Religión brinda una guía irreemplazable. La soberanía última de la Ley debe ser defendida y preferida, aún en contra de la soberanía del más sabio de los gobernantes.
\end{itemize}

En síntesis, podemos percibir una ``continuidad en el cambio'', una actualización de la misma esencia, entre las concepciones políticas de la tradición hebrea antigua, raigalmente basadas en las palabras sagradas de la Torah, y las concepciones extrañamente modernas (en pleno siglo XIII) de este profundo pensador político judío, nutrido de cultura griega e islámica, que alza la primacía de la Ley contra la arbitrariedad de los gobernantes y propone una división de funciones de gobierno, de sabor cuasi-constitucional, para evitar las hegemonías personales de los hombres, siempre propensos a desbordar los marcos de la prudencia\ldots{}

\hypertarget{el-pensamiento-poluxedtico-isluxe1mico-cluxe1sico}{%
\subsection*{El pensamiento político islámico clásico}\label{el-pensamiento-poluxedtico-isluxe1mico-cluxe1sico}}
\addcontentsline{toc}{subsection}{El pensamiento político islámico clásico}

El Corán es una obra de pensamiento político normativo\ldots y es también mucho más que eso. El Corán recoge las revelaciones que Alah hizo al profeta Mahoma, principalmente por intermedio del Arcángel Gabriel, en las ciudades de La Meca y Medina, en Arabia, entre los años 610 y 632 dC según nuestro calendario.

A los ojos de los creyentes en el Islam, este mensaje cierra el ciclo de la profecía monoteísta, que en un arco ascendente va desde Adam a Noé, a Abraham, a Moisés, a David, a Jesús, para culminar en Mahoma, a partir del cual una línea recta (que a veces se corta porque los hombres son aún atraídos por el Mal) impulsa a la Historia hacia la Parusía como meta final del devenir del hombre.

La estructuración del Corán en capítulos, suras, etc., data verosímilmente del siglo X de nuestra Era, y no se corresponde con el orden en que las suras fueron reveladas. La sura 96 es considerada la primera según la tradición, y fué revelada a Mahoma cuando meditaba en la gruta del monte Hira. La tradición musulmana ha indicado al comienzo de cada sura si ella fue revelada en La Meca o en Medina.

A diferencia de la Torah hebrea, o del Antiguo y Nuevo Testamento cristianos, el Corán no es una crónica de acontecimientos, ni una recopilación de jurisprudencia, sino un conjunto integral de normas de vida (política, social, familiar, religiosa, etc.) para los musulmanes. La lucha del Profeta Mahoma por imponerse y por imponer el mensaje de Alah en el mundo árabe hizo del Corán un texto político, vale decir, le dió énfasis a la dimensión política de una concepción religiosa que tiene una vocación omniabarcativa respecto de la existencia humana, en todas sus dimensiones físicas, anímicas y espirituales.

En esa lucha por conquistar a los árabes ``contra ellos mismos'' la Profecía se convirtió en Código. La expansión vertiginosa del Islam sobre diversos territorios y pueblos transformó el proyecto escatológico en sistema político-jurídico. A diferencia del Cristianismo, el Islam no es ``mahometanismo'' sino ``coranismo''. El Corán no tiene, como la Torah o los Evangelios, un status ambiguo en el plano político. En el caso del Islam, su rol es bien claro: se trata de generar una ``praxis'', o sea de configurar actitudes mentales y sociales coherentes a partir de un texto inmodificable, cuyo carácter totalizador es indispensable a los fines de su comprensión y aceptación, y que produce muy rápidamente instituciones uniformes, basadas en prescripciones intangibles, sobre los más diversos medios geográficos y sustratos culturales.

Los occidentales en general entendemos mal al Islam, porque tendemos a ``separar lo que está unido'' (como nos dicen los musulmanes) y a sobrentender la autonomía relativa de lo político. El Corán no es socialista, ni democrático ni reaccionario. Es el vector espiritual a traves del cual el creyente cumple su propia ascensión en un mundo que tiene un orden y un sentido, es decir, un FIN, en su doble significado de meta u objetivo y de cierre o conclusión. Ante sus propios ojos, los pueblos islámicos forman la comunidad (``UMMA'') depositaria y portadora de la última y definitiva expresión de la Voluntad Divina, comunidad que debe mostrar a la Humanidad entera el horizonte de la Salvación.

En esa comunidad, la misión de los sabios (``ulama'') es instruir y guiar al pueblo: asumir la enseñanza y la dirección político-religiosa de la sociedad. Los intelectuales realizan esa misión, a veces hasta el extremo del martirio por la defensa de la estricta ortodoxia, y a veces se apartan de ella, o la interpretan de un modo muy personal, hasta llegar a ``traicionarla'' (al menos desde el punto de vista de esa misma ortodoxia). En el mundo cultural musulmán, Ibn Taymiyya e Ibn Khaldun son considerados arquetipos históricos de esas dos actitudes.

Ibn Taymiyya nació en Harran (Siria) en el año 1263 dC, en el seno de una familia de teólogos de la escuela hanbalita, o sea una de las cuatro escuelas que integran la ortodoxia musulmana (el ``sunnismo''), la cual fué fundada por Ibn Hanbal en 855 dC. El padre de Ibn Taymiyya dirigía una ``madrasa'' (escuela religiosa) en Damas, cuya dirección heredó nuestro autor. Desde muy joven fue éste un teólogo y jurisconsulto notorio. Ibn Taymiyya se caracterizó por su intransigencia en materia de derecho musulmán y su constante resistencia a las autoridades que se marginaban de la ortodoxia musulmana. Como cabal hanbalita que era, su pensamiento y su acción estuvieron marcados por el respeto extremo a la tradición coránica y profética, en la que la ciencia del derecho y la ciencia teológica convergen en lo concreto de la existencia animada por una fe vivida, basada en el mantenimiento monolítico de la tradición y el respeto incondicional al texto escrito, pero también abierta a las aspiraciones del espíritu y del corazón, a los valores de la justicia, la sinceridad, la rectitud en la acción, privilegiando en definitiva el espíritu del texto frente a las interpretaciones interesadas o forzadas.

Ibn Taymiyya fue un luchador de la ``gran jihad'', la lucha interna contra los defectos y fallas que separan entre sí a los musulmanes. Por sus denuncias y críticas fué puesto en prisión varias veces (cinco, según sus biógrafos). Murió en prisión, en Damas, en 1328. Se comprende que esta figura sea hoy el modelo que, por su pensamiento y acción, inspira a los movimientos fundamentalistas, a los islamistas militantes, a los combatientes radicalizados que llamamos ``integristas'', y que sus obras no perdidas, especialmente las ``Fatawa'', hayan sido reiteradamente editadas después de 1970 por la Arabia Saudita.

La actualidad de Ibn Khaldun es de otra naturaleza: él es el centro de una polémica entre teólogos modernistas y tradicionales, porque este autor aparece como un singular precursor del pensamiento moderno, de la dialéctica, del positivismo (mucho antes que Hegel y Comte), del materialismo (varios siglos antes que Feuerbach y Marx) y de la Sociología moderna; autor, entre otras cosas, de una visión holística de la historia, cual si fuera un Spengler o un Toynbee extraviado por una máquina del tiempo en el siglo XIV\ldots{}

Ibn Khaldun nació en Tunes, en 1332 dC, en el seno de una familia de origen sevillano. Recibió una esmerada educación por parte de grandes maestros musulmanes. Viajó largamente por el mundo musulmán de su tiempo: Fez, Granada, Biskra, El Cairo, donde murió en 1406.

Su gran obra fué indudablemente su ``Historia Universal'' (``Kitab al-'Ibar'' - 1379) cuya Introducción, conocida en Occidente como ``Prolegómenos'' (``Al-Muqaddima'') contiene lo esencial de su pensamiento político.

El esfuerzo intelectual de Ibn Khaldun, testigo presencial y casi premonitorio del comienzo de la decadencia árabe, luego de su vertiginosa expansión, apuntó a descifrar el sentido de la historia. El eje principal de sus observaciones es lo que podríamos llamar ``la etiología de las decadencias'': el estudio comparativo (porque entre muchas otras cosas, Ibn Khaldun fué un precursor del método comparado) de los síntomas y de la naturaleza de los males que ocasionan la muerte de las civilizaciones.

La detención de la expansión imperial y el inicio de la decadencia significó para muchos musulmanes de aquel tiempo una inquietud teológica, porque habían interpretado los rápidos triunfos iniciales como expresión de la ayuda que Alah presta a los verdaderos creyentes. No fué este el caso del sagaz Ibn Khaldun, verdadero precursor de la Sociología moderna, quien proponía otra explicación: ``Cuando dos bandos son iguales en número y fuerza -escribía- el más familiarizado con la vida nómade obtiene la victoria''. Esa sigue siendo, hasta la época actual, la gran explicación tradicional: la superioridad militar de los nómades sobre los sedentarios\footnote{León Poliakov: HISTORIA DEL ANTISEMITISMO - Tomo II:``De Mahoma a los marranos'' - Bs. As. - Proyectos Editoriales - 1988 - pg. 50.}.

Ibn Khaldun superó los procedimientos tradicionales del pensamiento árabe -analógico y racional- y llegó a una concepción dinámica del desarrollo dialéctico del destino del hombre, y a plantear sobre esa base una historia retrospectivamente inteligible, racional y necesario\footnote{F. Chatelet et al.: DICTIONNAIRE DES OEUVRES POLITIQUES - Paris-PUF- 1989.}.

Ibn Khaldun tenía conciencia de haber creado una ciencia nueva -la ciencia de la sociedad como totalidad (" 'Ilm al-'Umran``)- para la que había utilizado todas las ciencias conocidas en su época, desde la Matemática hasta la Economía y la Psicología, pero le confiere un sesgo completamente personal. Por ejemplo, utiliza una teoría cíclica, muy propia de la tradición árabe, que permite configurar una visión del mundo directamente inspirada en la teoría platónica de las esferas, pero Ibn Khaldun seculariza, laiciza, hasta cierto punto,''materializa" esos ciclos: dice, por ejemplo, que la ciudad, la vida urbana, pervierte a los hombres, los hace egoístas y débiles, mientras que los nómades (los ``lobos'', los llama) que merodean en la periferia, practican la solidaridad (``acabiyya'') y son fuertes; cuando la ciudad está podrida no solamente la asaltan sino que la quieren regenerar\ldots hasta que se pervierten a su turno y el ciclo recomienza, porque siempre hay nómades que merodean en la periferia de la civilización\ldots{}

Para construir su teoría, Ibn Khaldun forjó varios conceptos: el más conocido es el ya mencionado de ``acabiyya'' que puede traducirse aproximadamente como ``espíritu de cuerpo'' o ``solidaridad''. También son importantes los conceptos de ``umran badawi'' (la civilización, que en su lenguaje siempre es urbana) y de ``umran hadari'' (la ruralidad o beduinidad).

Ibn Khaldun comprendió -mucho antes que Weber- la diferencia entre lo que en lenguaje weberiano se conoce como ``Veraine'' y ``Anstalt'' -o sea entre la asociación comunitaria y el establecimiento urbano- y definió a la civilización como primordialmente urbana: hay campo porque hay una ciudad, por lo menos pensada. Su esquema básico de la socialización puede quizás resumirse así: la civilización es la cohabitación equilibrada, en las metrópolis o en lugares más apartados, con la finalidad de humanizarse, agrupándose para poder satisfacer esas necesidades que, por naturaleza, exigen la cooperación para ser atendidas. Los hombres viven ese proceso -según Ibn Khaldun- en respuesta a un ``llamado'' (``da'wa''), concepto éste de neto origen teológico.

La estructura de los ``Prolegómenos a la Historia Universal'' es la siguiente: en el prefacio define a la historia como quehacer humano en el tiempo (``La historia comienza cuando los hombres advierten que no están regidos sólo por la Providencia\ldots{}'' decía) y echa las bases de la crítica histórica: ella debe basarse en la adecuación a lo real. En el resto de los ``Prolegómenos'' desarrolla sus ideas sobre esa ciencia nueva ``de la sociedad como totalidad'' que preconiza: el capítulo 1 trata de la sociedad humana y de la influencia del medio sobre la naturaleza humana (en un enfoque comparable con el que siglos después desarrollaría Montesquieu en su ``Espíritu de las leyes'') y esboza también una Etnología y una Antropología. El capítulo 2 trata de las sociedades rurales. El capítulo 3 trata de las diferentes formas de estado, de gobierno y de instituciones. El capítulo 4 trata de las sociedades urbanas, o sea de la civilización propiamente dicha. El capítulo 5 trata del conjunto de los hechos económicos (en una visión que podría calificarse de ``macroeconómica'') y el capítulo 6, finalmente, trata del conjunto de las manifestaciones culturales.

Ibn Khaldun sostiene, en esencia, que el ``nervio secreto'' de la vida humana en sociedad es la ``acabiyya'' , es decir, el agrupamiento solidario, beduino, tribal, no necesariamente urbano desde un comienzo. La política no empieza con la polis, sino que se extiende a formas muy variadas y frecuentemente muy anteriores a la polis.

En contra de la tesis tradicional musulmana de la necesidad de un sentido escatológico del poder político, de una raíz metafísica trascendente para el orden político, Ibn Khaldun sostiene que el poder político es únicamente inseparable de la sociabilidad, porque es sólo un hecho humano contingente, carente de una referencia necesaria a la religión. Unicamente la solidaridad y su vinculación consciente con la sociabilidad son el fundamento real de toda forma política organizada, cualquiera sea la forma que asuma. El resto es sólo una cuestión de control y represión.

Esta concepción, innegablemente materialista y racionalista, llevó a Ibn Khaldun a decir, en el siglo XIV y en el mundo musulmán, como ya vimos, que ``la historia comienza cuando los pueblos advierten que no están regidos sólo por la Providencia\ldots y que las diferencias que se advierten entre los modos de ser de las generaciones expresan las diferencias que separan sus modos de vida económica\ldots{}''

\hypertarget{el-pensamiento-poluxedtico-griego-cluxe1sico}{%
\subsection*{El pensamiento político griego clásico}\label{el-pensamiento-poluxedtico-griego-cluxe1sico}}
\addcontentsline{toc}{subsection}{El pensamiento político griego clásico}

Ya mencionamos antes nuestro ``eurocentrismo cultural''.Creemos, sin embargo, que no hay ningún eurocentrismo en reconocer que, en su forma más plena y sistemática, la Filosofía Política, la Ciencia Política y con ellas las primeras teorías políticas normativas puras, nacieron en la Grecia clásica. En todo lo que hemos visto hasta ahora es evidente que hay pensamiento político e incluso sabiduría política, pero también es notorio que hay mucho magma religioso-teológico en esas obras, magma del cual hay que separar el pensamiento político como se separa el metal de la roca que lo contiene, para analizarlo, y luego restituirlo a él, porque sin ese sustento carece de sentido y no resulta incluso comprensible.

En la Grecia clásica, por primera vez primó el pensamiento secular, es decir, una cierta separación de la religión y la política. No es que los griegos no fueran religiosos: tenían una gran cantidad de dioses y muchos rituales, pero sus dioses eran sólo algo más que hombres, y su culto se parecía más a un ministerio de relaciones exteriores que a una adoración estática y temerosa. ``En Grecia, la Religión y la Política estaban relacionadas en una forma desconocida en otras partes -dice Hearnshaw\footnote{F.J.C. Hearnshaw: HISTORIA DE LAS IDEAS POLITICAS - Santiago de Chile - Empresa Letras - ?}- la Política dominaba y la Religión era secundaria''.

Los primeros intentos de reflexión política secular estuvieron muy influidos por esa versión de la matemática cargada de significación metafísica que caracterizó a Pitágoras y sus discípulos, que en este campo verdaderamente no obtuvieron resultados dignos de destacar.

Los primeros filósofos políticos propiamente dichos fueron los sofistas, en el siglo V aC. Fueron los intelectuales de su tiempo, altaneros y engreídos, que se enorgullecían de su emancipación respecto de la religión tradicional y de la moral convencional. Rechazaban el patriotismo y los deberes de la ciudadanía, y planteaban una libertad individual sin trabas y un libre pensamiento. Mucho antes que Maquiavelo, plantearon una completa separación de la conducta pública y la moral privada.

Los sofistas enseñaban que el Estado es de origen convencional y contractual; que las leyes expresan una relación de fuerzas desprovista de toda sacralidad, y que el derecho se identifica con el poder. Su imagen individual, de intelectuales desencantados, ciertamente lúcidos en muchas observaciones y hasta simpáticos en su individualismo anárquico y un tánto cínico, se eclipsaba ante las consecuencias prácticas graves que podía tener la generalización de sus teorías, que cuestionaban las bases implícitas de la ciudad misma y el conformismo social de la mayoría de sus habitantes.

Sus ideas, potencialmente subversivas, convocaron al campo de la controversia a un pensador incomparablemente más valioso y profundo que ellos: Sócrates (469-399 aC) quien, con su inimitable dialéctica mostró la falsedad de sus argumentos y enseñó el carácter natural y necesario del Estado, el fundamento inmutable y sagrado de la Ley, la necesaria sujeción del Poder al Derecho, la primacía de la Sociedad sobre el Individuo y el derecho social a exigir los servicios del hombre más sabio y mejor para su gobierno.

Como una cruel ironía, este hombre sabio y prudente (pero molesto en su punzante crítica a la mediocridad y corrupción de los poderosos) fue acusado de impiedad y condenado a muerte! por el ignorante y fanático ``demos'' de Atenas, mientras los sofistas seguían difundiendo sus ideas disolventes, en muchos casos ya convertidas en técnicas apropiadas para el éxito político momentáneo.

El asesinato de Sócrates fue una escandalosa injusticia, el prototipo del acto inicuo, contra el que debe luchar todo filósofo. Tal convicción animó la obra de Platón (427-347 aC), que fue su discípulo durante los últimos ocho años de la vida de Sócrates, y que dio a conocer y desarrolló en sus ``Diálogos'' las ideas de su Maestro, aunque quizás nunca sabremos realmente cuál fue el aporte de uno y otro a la construcción de esa verdadera columna vertebral de la filosofía occidental.

Los principios fundamentales de la filosofía platónica son: que el fin supremo de la existencia es la virtud, que la virtud es sinónimo de conocimiento, y que el intelecto, órgano del conocimiento, es el factor dominante en el hombre. Platón aplicó tales principios en sus tres diálogos políticos: ``La República'', ``El Político'' y ``Las Leyes''.

El objeto de ``La República'' es combatir las ideas políticas de los sofistas, y criticar las costumbres políticas de los gobiernos griegos de su tiempo -democracias o monarquías- por su falta de virtud cívica. Plantea en esta obra un ideal político demasiado abstracto y deshumanizado. En ``El Político'' formula un sistema más compatible con la naturaleza humana corriente: en este diálogo se inclina a pensar que el mejor gobierno posible es el del ``Rey-Filósofo'', que gobierna de acuerdo con las leyes. Finalmente, en ``Las Leyes'', Platón abandona la idea de alcanzar un ideal metafísico y concluye diciendo que en este mundo imperfecto (donde los Reyes-Filósofos son muy escasos) un Estado con división y separación de los poderes es lo mejor que prácticamente puede realizarse.

Aristóteles (384-322 aC) fue un discípulo rebelde y cuestionador (y el más capaz) de Platón, y tras la muerte de su maestro y muchos viajes, fundó en Atenas su propia escuela, el Liceo.

Su principal obra de pensamiento político, ``La Política'', no tiene el encanto literario de los diálogos platónicos, y al parecer proviene de apuntes de conferencias recopilados por discípulos. Esta obra continúa y acentúa decididamente la tendencia, que ya se insinuaba en el último Platón, de abandonar la vía puramente especulativa y fortalecer la participación del material empírico en la reflexión política, al punto de que Aristóteles puede ser considerado ``el padre fundador de la Ciencia Política clásica''.

Es difícil sintetizar la obra política de Aristóteles, pero en principio podemos decir que sus ideas básicas son: que las verdaderas bases del Estado son la Familia y la Propiedad privada; que el Estado es producto de una evolución desde la Familia, a través de la Comunidad tribal, hasta culminar en la Ciudad autónoma, de la que Atenas es el ejemplo supremo. Luego expone los rasgos más característicos de esa Ciudad-estado, y de los otros tipos de Estado existentes en su tiempo, de los que ofrece varias clasificaciones, de las cuales la más conocida es la basada en la pregunta: quién gobierna? Monarquías, aristocracias, repúblicas, cada una de las cuales tiene una forma corrupta (que se da cuando el gobernante atiende su interés particular en lugar del interés general): tiranías, oligarquías, democracias (nosotros hoy diríamos demagogias). Trata también muchos detalles de la actividad del Estado y de sus funciones. ``Como Platón -dice Hearnshaw- Aristóteles ve en la educación el principal preventivo contra las revoluciones''.

No creemos necesario extendernos más aquí porque todas las obras de Historia del Pensamiento Político contienen amplias referencias a estos aportes fundamentales al pensamiento político universal, y a ellas remitimos al lector interesado en profundizar el tema, no sin recomendar el invalorable contacto directo con las obras originales. Sí agregaremos aquí un comentario sobre otro trabajo, menos conocido pero a nuestro entender de gran valor como expresión del pensamiento político griego clásico, especialmente en su dimensión ``internacional''. No es la obra de un filósofo sino la de un historiador: se trata de la ``Historia de la Guerra del Peloponeso'' de Tucídides (460?-395 aC).

La constancia que ponen de manifiesto las sociedades humanas -cualquiera sea la forma de su organización política- en hacerse la guerra, asegura la actualidad permanente de la obra de Tucídides, que supo distinguir con claridad lo esencial de lo accesorio en la historia humana -especialmente en la historia de la guerra- y expresarlo en términos válidos para todos los tiempos. Dice Tucídides en las páginas introductorias de su obra: ``Yo me consideraría muy satisfecho si esta obra fuera considerada útil por aquellos que quieran ver claro, tanto en los acontecimientos del pasado como en aquellos, parecidos o similares, que la naturaleza humana nos reserva en el porvenir. Más bien que una pieza literaria compuesta para el auditorio de un momento, es un capital imperecedero lo que se encontrará aquí''. Esta certeza, que Tucídides tenía, del carácter imperecedero de su obra, ha encontrado su confirmación a través de los tiempos. Muchos autores célebres posteriores lo citan, desde Hobbes y Hume, pasando por Hegel y Clausewitz, hasta Erik Weil y Raymond Aron en nuestro tiempo. Siempre se consideró, y se sigue considerando, que la lectura meditada de la ``Guerra del Peloponeso'' es una introducción formativa totalmente válida para la reflexión política\footnote{F. Chatelet et al.: DICTIONNAIRE DES OEUVRES POLITIQUES, Paris, PUF, 1989.}.

Dos razones tiene Tucídides para pensar en el carácter perdurable de su obra: la primera es la naturaleza del conflicto de que trata, sin duda una gran guerra, por la potencia adquirida por los antagonistas y por su objetivo: la futura hegemonía sobre el mundo civilizado. La segunda es que tal guerra, por su violencia sin piedad, lleva a su más alto punto, en estado de brutal pureza, a la naturaleza esencial del hombre, su agresivo aspecto dominante, que se revela a su propia conciencia por la dureza misma de las pruebas a que se ve sometido.

El objetivo ``político'' de la obra de Tucídides es muy claro: se trata de aportar a quienes quieren practicar seriamente su oficio de ciudadanos, los recursos de conocimiento que les permitan ubicar con acierto su reflexión y su acción, vale decir, disponer de las categorías que les permitan conocer lo esencial de la realidad del medio en el cual deberán luchar y actuar.

En el análisis de los hechos históricos que marcaron los principales procesos de la Guerra del Peloponeso, Tucídides descubrió un concepto clave para entender todo procesos político de confrontación entre entidades estatales: el concepto de IMPERIALISMO, en su acepción puramente política. La dinámica de la formación de un centro imperial y de una periferia dominada -advirtió Tucídides- tiene una lógica interna que es independiente de las intenciones de los actores. Si hay dos centros (si el sistema es bipolar, diríamos en el lenguaje de hoy) fatalmente el mutuo temor los llevará a enfrentarse sin que sea posible volver atrás ni encontrar otra salida: ``\ldots si la muy oligárquica Esparta se hubiera encontrado en la posición de la muy democrática Atenas, hubiera actuado sin duda de la misma manera y con las mismas consecuencias'', dice Tucídides.

Esta ``teoría del imperialismo'' se apoya en una concepción realista y ``sombría'' de la naturaleza humana. La guerra es para Tucídides un poderoso develador, que manifiesta en los actos colectivos algunas tendencias primordiales de nuestra naturaleza como individuos y como Humanidad: ``\ldots nuestra conducta no tiene nada que pueda sorprender\ldots nada que no esté en el orden de las cosas humanas\ldots{}'' dicen los plenipotenciarios atenienses ante la Asamblea espartana en la última negociación antes del estallido de las hostilidades.

El discurso analítico de Tucídides sobre la historia de esta guerra se caracteriza por un racionalismo riguroso y totalizador. Su análisis de los hechos históricos vincula permanentemente las acciones militares con las reacciones de las Asambleas y del ánimo de los pueblos. Se entrecruzan allí las polémicas sobre estrategia, los acuerdos entre aliados y los enfrentamientos de los negociadores hostiles. La complejidad de las situaciones y la dificultad que entrañan las opciones a hacer son acertadamente expresadas recurriendo a un método que ya había sido usado por los sofistas: la yuxtaposición en una misma escena de dos discursos, que expresan las opciones extremas a que da lugar cada situación. Las acciones militares y las deliberaciones políticas se confrontan y se refuerzan en una descripción vivísima de las situaciones, en un diálogo tenso y conflictivo. El discurso del historiador conceptualiza el conflicto pero no lo resuelve ni busca reabsorberlo imaginariamente en algún ``estado de equilibrio'' nuevo y no conflictivo. Quizás todos estos elementos de la visión de Tucídides son lo que le da a su obra ese aire de ``permanente actualidad'', de modernidad, que nos sorprende a cada lectura\ldots{}

\hypertarget{el-pensamiento-poluxedtico-romano-cluxe1sico}{%
\subsection*{El pensamiento político romano clásico}\label{el-pensamiento-poluxedtico-romano-cluxe1sico}}
\addcontentsline{toc}{subsection}{El pensamiento político romano clásico}

Aunque Roma conquistó y dominó a Grecia, como a todo el resto del mundo mediterráneo, en lo cultural fué muy grande la dependencia de Roma respecto de Grecia. Esto se aprecia en muchos campos, en el arte, la literatura, la religión, la filosofía. En el campo de la Ciencia Política también se ve claramente. El primer teórico político romano fué un griego, Polibio, quien vivió en Roma entre los años 167 y 151 aC.\footnote{J.F.C. Hearnshaw: op. cit.}.

Polibio (210-125 aC) fue un historiador griego, hijo del estratega aqueo Licortas. Luego de la derrota griega en la batalla de Perseo fue enviado a Roma como rehén. Allí fue pronto valorado e introducido en la mejor sociedad, llegando a desempeñarse nada menos que como consejero de Escipión el Africano durante el sitio de Cartago, interviniendo en diversas circunstancias como mediador. Su condición de testigo presencial de muchos hechos importantes de la vida romana de su tiempo estimuló sin duda su interés por la historia y la política romanas. Gran admirador de Roma, su preocupación intelectual era, al parecer, explicar el éxito imperial de Roma (originariamente una ciudad-estado en todo semejante a Esparta o Atenas) frente al lamentable fracaso de las ciudades griegas.

Estudió minuciosamente la historia romana, desde el comienzo de las Guerras Púnicas (264 aC) hasta sus días. En ese monumental trabajo dedica un notable capítulo al análisis de los principios que le dieron a la constitución romana su estabilidad y eficacia. Polibio se basó en la clásica clasificación aristotélica de los regímenes políticos: monarquías, aristocracias y repúblicas; y afirmó que las diferencias entre ellas son externas e institucionales, no de principios; y que las tres son diversos modos de resolución de conflictos de fuerzas. Basado en una buena cantidad de estudios de casos, llegó a la conclusión de que estas tres formas, en estado puro, son inestables a causa del antagonismo de las otras dos, y que tienden inclusive a sucederse en forma cíclica.

Explica el poder y la estabilidad de Roma y el éxito de su expansión imperial en base a las características estructurales de la constitución romana, que combina y armoniza las tres formas puras: el principio monárquico está representado por los Cónsules, el principio aristocrático por el Senado y el democrático por las Asambleas populares.

También Polibio expuso la primera teoría sobre lo que luego la ciencia del Derecho Constitucional llamaría ``frenos y contrapesos'', es decir, los mecanismos constitucionales de transacción entre fuerzas antagónicas, como es el caso del ``ius agendi'' y del ``ius impediendi'', o sea el derecho o el poder de actuar y de impedir que detentaban respectivamente los patricios y los plebeyos en la República romana.

Polibio alcanzó a ver, antes de su muerte, cómo esa estabilidad y armonía comenzaban a resquebrajarse, y se insinuaban conflictos y perturbaciones que, al no ser adecuadamente resueltos, con el paso del tiempo culminarían en la caída de la República y la instauración del Imperio.

Aproximadamente cien años después de Polibio apareció en Roma otro gran teórico político: Marco Tulio Cicerón (106-43 aC). Cicerón escribió en los tiempos en que Julio César, sobre las armas de su ejército victorioso, establecía un imperio dictatorial en Roma. Cicerón era un ardiente republicano, detestaba a César y quería restaurar el antiguo equilibrio de las instituciones. En sus obras, analiza las causas de la triste decadencia de la República. Partiendo de la teoría del equilibrio de las formas de gobierno que había diseñado Polibio, Cicerón atribuyó la crisis de su tiempo al excesivo poder alcanzado por el elemento democrático, del que lograron apropiarse demagogos como Mario y César. La obra política principal de Cicerón es ``De la República''(55 aC). Este tratado político ha llegado a nosotros por extraños caminos. Fue citado por San Agustín, pero luego cayó en el olvido durante toda la Edad Media y Moderna; se extraviaron los ejemplares que probablemente habría (salvo el fragmento llamado ``El sueño de Escipión'', que había sido trascripto por un copista a principios de la Edad Media. Figuró, entre otras tantas, como obra perdida, hasta que reapareció en 1819 por el hallazgo de un erudito italiano, Angelo Maï, quien encontró en la Biblioteca Vaticana un palimpsesto con comentarios de los Salmos de San Agustín, que al ser raspado reveló haber sido escrito sobre una copia del texto de Cicerón\ldots{}

La obra es fundamentalmente una reflexión sobre cuál es el mejor régimen político, reflexión hecha con la intención de actualizar ``La República'' de Platón, pero cambiando el enfoque: Platón parte de los grandes principios, como el Bien y la Justicia; Cicerón aborda la cuestión desde la técnica política, para llegar finalmente a la fundamentación metafísica del tema. Por otra parte, Cicerón sigue en buena medida el criterio de Polibio, verdadero puente entre el pensamiento griego y el romano: la forma de gobierno es vista como el factor determinante del Estado y, más allá, del mismo pueblo\footnote{Chatelet, Duhamel y Pisier, op. cit.}.

La estructura de la obra es clara: su primer tema es la forma política adecuada al Estado romano, cuya respuesta es la ``solución mixta'' de Polibio, que ya vimos; el segundo tema es el análisis de la experiencia histórica del pueblo romano, porque la Constitución ideal sólo es válida si tiene referencias en la vivencia concreta del pueblo. La forma de gobierno debe ser expresión adecuada de esa vivencia. Recién a esta altura de su discurso, Cicerón plantea los grandes temas platónicos: el fundamento del gobierno y de la ley: se pregunta si ese fundamento es una ``ley natural'' o simplemente la fuerza. Esto lo lleva a analizar la organización específica del Estado de la Roma republicana, al que considera lo más próximo posible al ideal político de la filosofía estoica. Finalmente, alcanza una culminación metafísica, al vincular las exigencias del bien público con la realización del Bien como categoría trascedente.

El punto de partida de Cicerón es una justificación de la práctica de la virtud política, presentada como una actividad digna del sabio: el ejercicio del gobierno es visto como un requisito para poner las potencialidades de la Sabiduría en acuerdo con el Mundo.

Para Cicerón, el objeto de la Ciencia Política es la ``cosa pública'', que se genera porque un pueblo es ``una reunión de hombres fundada en un pacto de justicia y una comunidad de intereses'', reunión basada en un ``espíritu de asociación'' que es natural, porque el hombre es un ``animal político''. A partir de allí, la cuestión que se plantea es una pregunta clásica en el pensamiento normativo: cuál es la mejor forma de gobierno. Gobierno de uno, de algunos, de la multitud? La respuesta de Cicerón, como la de Polibio, cien años antes, elige esa cuarta forma mixta, que surge de la mezcla equilibrada de las tres formas originarias.

Cicerón no se queda en la especulación teórica pura, y siguiendo una tradición ya sólidamente establecida, recurre a la experiencia. Reescribe la historia de Roma para configurar un esbozo de ``política experimental'': busca conocer los modos de marcha y las desviaciones de los Estados. Marca allí la crisis de su momento histórico afirmando que ``es falso que la cosa pública no pueda ser gobernada sin recurrir a la injusticia'' sino que, por el contrario, requiere ``una suprema justicia''.

El fundamento de lo político plantea un dilema: reposa sobre la Naturaleza o sobre una relación convencional de fuerzas? Por boca de Escipión, Cicerón se inclina por la ley natural: ``Hay una Ley verdadera, la recta razón, conforme a la Naturaleza, universal, inmutable, eterna\ldots en todas las naciones y en todos los tiempos\ldots Dios mismo le da nacimiento, la sanciona y la promulga\ldots y el hombre no puede desconocerla\ldots sin renegar de su naturaleza\ldots{}''dice.

Cicerón plantea como solución para su tiempo, de crisis profunda, un retorno a las costumbres y valores de la República primitiva, ya erigida en mito histórico. De aquí arranca la culminación de la obra: el famoso ``Sueño de Escipión'', único fragmento que fue conocido desde la Edad Media, por la trascripción que hizo el griego Macrobio en el siglo V dC.

La función de esta parábola, de este ``Sueño'', es describir el destino político como un ineluctable deber, ubicándolo en el orden cósmico de las cosas. A través de una poética evocación del Universo, la república política es incertada en una ``República Cósmica'', cadena universal en eterno movimiento, que vincula las grandes almas beneméritas de la Patria con la posteridad. Esta culminación poética no es una simple efusión sentimental: ``Erige a la Política en un reflejo del orden cósmico en el hombre, con lo que la Política se vuelve así la tarea por la cual el hombre ejerce su función de participación en el Cosmos'', dice P. Laurent Assoun\footnote{Chatelet, Duhamel y Pisier, op. cit.}.

Como trágico contraste existencial con sus elevadas ideas, la oposición de Cicerón a César y a Antonio (contra el que pronunció las llamadas ``Filípicas'', palabra que se ha incorporado al lenguaje común como discurso severamente admonitorio) le acarrearon su propia ruina y finalmente su proscripción y su muerte en Formia, donde le dieron alcance sus perseguidores. Allí hubiera podido quizás aún salvarse, pero acometido de un cansancio mortal, ante el derrumbe de sus ideales, hizo detener la litera y entregó su cuello a la espada del tribuno en medio del camino, entre el lamento de sus servidores, como un símbolo del fin de una época y del comienzo de otra.

Años después, durante el gobierno (o desgobierno) del emperador Nerón (del 54 al 68 dC), su preceptor y ministro Séneca, un filósofo estoico, encarna una nueva actitud, muy difundida luego: pese al inmenso contraste entre el ideal filosófico estoico y la realidad política de su tiempo, violenta y corrompida, Séneca y muchos otros como él apoyan al Imperio porque se sienten obligados a elegir entre dos calamidades: la tiranía o la anarquía, y entre los dos males prefieren el primero. Pero, como puede verse en sus ``Cartas a Lucilius'', el filósofo, ante el espectáculo de la desunión y la violencia,de la corrupción generalizada y la falta de esperanza de mejoramiento, intenta retirarse al refugio de su alma, a su ``ipseidad'', buscando la ``posesión de sí'' y esperando la muerte como emancipación, en una actitud de huída del presente, llamativamente similar a la de algunos post-modernos actuales. Pero ni su superficial adhesión al orden vigente, ni su huída al interior de sí mismo lo salvaron de verse involucrado, en el 65 dC, en la conjuración de Pisón, por lo que recibió de Nerón la orden de darse muerte. Murió, como Sócrates, acompañado de sus amigos, pero en el fastuoso ambiente que rodeó su vida, en franca contradicción con el ideario estoico que cultivaba.

\hypertarget{el-pensamiento-poluxedtico-medieval}{%
\subsection*{El pensamiento político medieval}\label{el-pensamiento-poluxedtico-medieval}}
\addcontentsline{toc}{subsection}{El pensamiento político medieval}

En los primeros siglos de nuestra Era, el pensamiento cristiano con implicancias políticas arranca de dos pilares evangélicos fundamentales: ``MI REINO NO ES DE ESTE MUNDO'' (San Juan, XVIII, 36) y ``DAD AL CESAR LO QUE ES DEL CESAR Y A DIOS LO QUE ES DE DIOS'' (San Mateo XXII, 21 y San Marcos XII,17).

Estos principios proclamaron la emancipación de la Religión respecto de la Política, separaron sus campos de acción y precisaron sus límites. ``Señalaron el asentamiento de una Iglesia distinta del Estado -dice Hearnshaw- el fin de esa subordinación del culto divino a la administración civil que había sido la notable característica de la Ciudad-estado griega y romana''\footnote{J.F.C. Hearnshaw: op. cit.}.

En el desarrollo inmediatamente posterior del pensamiento político cristiano, principalmente por obra de San Pablo, se consideró la complementación de tareas entre el Estado y la Iglesia: el primero mantiene la paz social y hace cumplir las leyes; la segunda se ocupa de la salvación de los hombres. Sobre esta base, la doctrina enseñó el orígen divino de la autoridad civil: ``LOS PODERES QUE EXISTEN SON ESTABLECIDOS POR DIOS'' (Rom. XIII,I); ``ROGAD POR LOS REYES Y POR TODOS LOS QUE POSEEN AUTORIDAD'' (I Tim. II,2); ``RECUERDENLES QUE SON SUBDITOS DE LA SOBERANIA Y DE LOS PODERES, PARA OBEDECER A LOS MAGISTRADOS Y PARA ESTAR PREPARADOS PARA TODA OBRA DIGNA'' (Titus III,1).

En los escritos de San Pablo es también posible encontrar conceptos muy acordes con los de la filosofía estoica, como el reconocimiento de la Ley Natural, inscripta en el interior del hombre, cualquiera sea su raza o circunstancias (Rom. II, 12-15), o como la afirmación de la igualdad de todos los hombres ante la Gracia Divina, cualquiera sea su condición o jerarquía en esta tierra (Philem. 10-17).

También encontramos conceptos similares en la llamada ``Primera epístola de San Pedro'': ``SOMETEOS A TODO MANDATO DEL HOMBRE POR AMOR A DIOS\ldots TEMED A DIOS, HONRAD AL REY'' (1 Pet. II, 13-17).

El Imperio Romano persiguió a los cristianos. Pese a su amplia capacidad para asimilar las religiones de los vencidos, se había alarmado mucho por el exclusivismo del culto cristiano (que se veía a sí mismo como ``la única y verdadera fé universal'') y por la consiguiente negativa de los cristianos a ofrecer sacrificios y desempeñar servicios incompatibles con sus principios. Se había alarmado mucho más aún por la creciente organización y poder de la Iglesia, su ascendiente sobre el pueblo bajo y su infiltración en círculos cercanos al poder.

Estas despiadadas persecuciones modificaron la óptica cristiana respecto del Estado romano. Ya no fue más visto como ``heraldo del Evangelio'' y cobraron relieve las palabras de la Revelación de San Juan: ``BABILONIA\ldots LA GRAN RAMERA\ldots LA MADRE DE LAS PROSTITUTAS Y DE LAS ABOMINACIONES DE LA TIERRA\ldots EBRIA DE SANGRE DE LOS SANTOS Y DE LOS MARTIRES'' (Rev.~XVII, 1,9).

Esas persecuciones cesaron en el año 311 dC, tras un completo fracaso en cuanto a frenar la difusión de la nueva religión, pero habiendo ocasionado entretanto sufrimientos sin cuento. En el año 313 dC, Constantino reconoce al Cristianismo como una de las religiones oficiales del Imperio, y ochenta años después, en el 392 dC, el emperador Teodosio I cerró los templos paganos y proclamó al Cristianismo como única religión oficial del Imperio.

Una curiosa consecuencia de este aparente triunfo fue la subordinación completa de la Iglesia al Imperio (o sea el llamado césaro-papismo) que eliminó temporariamente la separación entre Política y Religión. Ese movimiento de subordinación a lo secular de parte de la Iglesia fue resistido de varios modos: el monasticismo, el hermitañismo ascético, las revueltas heréticas (arianismo, donatismo, nestorianismo, etc.) y principalmente por la reflexión filosófica y la acción política de los obispos del Imperio Romano de Occidente, tras la muerte de Constantino. En el Imperio Romano de Oriente, en cambio, esa subordinación continuó durante largo tiempo.

En la Teoría Política, la consecuencia de esta situación en Occidente fue que, durante mil años, el eje de la controversia política pasó por la relación entre el soberano secular y la Iglesia dependiente o independiente de su poder, o queriendo subordinarlo al suyo.

En ese contexto emerge, como primera manifestación del debate, la formidable obra de San Agustín ``La Ciudad de Dios''. San Agustín reconoce la autoridad del Emperador romano, admite que ésta viene de Dios, prescribe a los súbditos el deber de obediencia y exhorta al Emperador a defender a la Iglesia contra los cismas y las herejías, pero no admite que, en cuanto Emperador, tenga alguna autoridad dentro de la Iglesia. La Fé y la Moral quedan reservadas a los Concilios y a los Obispos consagrados. Marca así nuevamente con claridad la diferencia entre la Ciudad de Dios y la ciudad terrenal.

En el pensamiento de San Agustín, estos dos conceptos tuvieron una notable evolución: al principio, el primero representa al cristianismo y el segundo al paganismo. En esta fase, San Agustín procura liberar al cristianismo de la acusación de ser responsable del saqueo de Roma por los visigodos de Alarico (410 dC) y mostrar que el paganismo no habría salvado a Roma del desastre ni aún en sus épocas de esplendor. Más tarde, la Ciudad de Dios representa a la Iglesia institucional y jerárquica, y la ciudad terrena, al mundo fuera de la Iglesia. Por último, la Ciudad de Dios designa a la ``comunidad de los santos'' mientras la ciudad terrena es ``la sociedad de los réprobos''\ldots{}

Es de hacer notar aquí que San Agustín, y otros Padres de la Iglesia de aquel tiempo, están ubicados, en forma similar a Séneca y los estoicos, ante un dualismo inquietante y aparentemente irreducible: lo espiritual y lo material, lo bueno y lo malo, la Iglesia y el Mundo, la autoridad espiritual y la autoridad secular. De allí en adelante, la historia de la Teoría Política medieval es la historia de las propuestas de resolución de este dualismo.

``La Ciudad de Dios'' (413-426 dC) ha ejercido una influencia política duradera, profunda y variada, sobre muchos autores, que van desde Bossuet a Comte y a los historiadores y comentaristas del siglo XX. El entendimiento de la doctrina política de esta obra debe buscarse en el contexto de la comprensión que San Agustín tenía del misterio cristiano.

Esa doctrina surge motivada por las luchas de San Agustín contra el dualismo de los maniqueos, contra el donatismo, contra el pelagianismo, contra la acusación hecha a los cristianos de haber contribuido por su misma religión al saqueo de Roma por las huestes de Alarico, pero no es una doctrina sólo para un tiempo, sino el producto de una reflexión permanente, con vocación de perdurabilidad, sobre la violencia y la guerra, la vida y la muerte y la ubicación de los cristianos en la prueba de la historia.

Surgido en un tiempo de crisis, el pensamiento de San Agustín se forjó en la confluencia de dos tradiciones: la cultura greco-romana y las Escrituras judeo-cristianas. De la cultura griega San Agustín valora principalmente la figura de Platón y su ``República''. Hay una filiación intelectual de idealismo platónico en el pensamiento agustiniano, lo que, entre otras cosas, lo ha convertido con el tiempo, en el involuntario inspirador de muchas corrientes heréticas, del mismo modo que las restauraciones de la ortodoxia generalmente se inspiran en Aristóteles\ldots Pero Agustín apela en su obra sobre todo a la cultura romana, de la que está impregnado. Conoce muy bien la historia de la ``Urbs'' por excelencia, y la utiliza para mostrar que los dioses paganos no podían servir al Estado, al contrario del Dios verdadero. San Agustín no le pide a Roma que renuncie a lo que la hizo grande sino que reciba finalmente los dones del Dios verdadero, tal como está prometido en las Escrituras.

En su esquema general, ``La Ciudad de Dios'' se presenta como un recorrido que parte de la crisis reciente (410 dC) para inducir al mundo romano a releer su historia política, para descubrir la vanidad de su ``teología civil'' y reconocer la necesidad de un mediador entre Dios y los hombres -Cristo- para que la ``ciudad terrestre'' se abra a ese camino de salvación y, al mismo tiempo, a una comprensión de su proceso histórico, que pueda esclarecer su destino político, al mismo tiempo que el destino último de los hombres y las naciones.

Según San Agustín, los hombres siempre forman parte de algún grupo, en una escala que va desde la familia hasta el Imperio, manteniendo en su seno una relación tan estrecha como ``la de una letra en una frase''. La existencia misma de grupos de diverso tipo supone la presencia de un acuerdo básico, una disposición social fundamental, propia del ser humano. Para San Agustín, PUEBLO es la reunión de una multitud de seres razonables, asociados ``por la participación armoniosa en aquéllo que aman''. Como toda sociedad, la ``Civitas'' requiere un consenso básico, un acuerdo que la induzca a perseguir ciertos objetivos antes que otros; un AMOR cuyo objeto (bueno o malo) evidencia la moralidad o perversidad del pueblo.

Una condición esencial de una verdadera ``Res publica'' es la JUSTICIA, cuyo objeto es el Derecho, el cual según San Agustín debe derivar de la Caridad. Esta idea de Justicia no está tomada sólo de la tradición latina: ella está transfigurada por la interpretación cristiana.

Dice San Agustín que ``la paz de la ciudad es la concordia bien ordenada de los ciudadanos en el gobierno y en la obediencia''. En su pensamiento, la PAZ es un valor central: ``La paz es tan esencial a los hombres que hasta los malvados la desean''. San Agustín sabe, por cierto, que hay paces injustas y admite la legitimidad de algunas guerras, pero denuncia sus atrocidades. En esos días turbulentos, el tema de la paz se plantea con fuerza, y también con el recuerdo cercano de la ``pax romana'', de los más bellos días del Imperio\ldots{}

Pero, heredero al fin de la tradición bíblica, San Agustín entiende que la vida política está marcada por una oposición fundamental: ``Dos amores han hecho dos ciudades: el amor de sí hasta el desprecio de dios, la ciudad terrestre; el amor de Dios hasta el desprecio de sí, la Ciudad Celeste. Una se glorifica en sí misma, la otra en el Señor\ldots{}''.

San Agustín considera que la Ciudad de Dios debe marcar con su impronta a la sociedad política, para que no triunfe en ella la ciudad terrena, la ``ciudad del Diablo''. Las leyes de la ciudad terrena deben ser observadas, pero en nombre de fines superiores. San Agustín reconoce que, en el mundo real, la ``ciudad del Diablo'' generalmente triunfa, al menos momentáneamente. La sociedad política no es neutra: después de la Caída, su campo es el campo de Lucifer. Ella subsiste, sin embargo, porque Dios, en su infinita paciencia y amor, le ofrece en forma permanente la oportunidad de convertirse en Ciudad de Dios. El pensamiento político de San Agustín desemboca así en una ``teología de la historia política'': Cristo, por su muerte redentora, ofrece a las ciudades terrestres la oportunidad de convertirse en ciudades de Dios.

La posteridad de la obra de San Agustín ha sido excepcional, pero su pensamiento, ha sido tergiversado o no? Hay o no una tercera ciudad, la ciudad del hombre, la ciudad de la política? El punto de vista de San Agustín sobre la relación entre lo temporal y lo espiritual, sobre la relación entre la Política y la Religión, parece rechazar todo intento de sacralizar el orden establecido. San Agustín es muy consciente de la precariedad de las cosas humanas, siempre próximas al caos, caos que la sociedad política debería, justamente, vencer.

La sociedad y la cultura: se sostienen sólo por el reconocimiento de su fin último? Cómo compatibilizar la precariedad de las construcciones políticas humanas con la vocación sobrenatural de la Humanidad? Temas actuales, preguntas profundas. La respuesta de San Agustín, generada en un tiempo de violencia y de decadencia, está signada por la esperanza cristiana y vislumbra, a través de las viscicitudes de los reinos terrestres, el advenimiento del ``Reino que no tendrá fin''\footnote{Chatelet, Duhamel y Pisier, op. cit.}.

Las invasiones de los bárbaros derrumbaron al Imperio Romano de Occidente, o lo que quedaba de él (recordamos aquí el pensamiento de Toynbee según el cual ningún Imperio cae por causas externas si no ha sido corroído previamente por causas internas, por sus propias contradicciones y conflictos) pero esos bárbaros se convirtieron al Cristianismo por obra de monjes y misioneros enviados por el Papa. La unidad política imperial fue reemplazada por la unidad de la Iglesia, por encima de la fragmentación política resultante de las invasiones. Por su parte, el Imperio Romano de Oriente subsistió durante casi un milenio, ejerciendo una sujección imaginaria del Occidente.

En realidad, las relaciones entre el Papa y el César bizantino fueron siempre malas, hasta que el Papa León III, a fines del siglo VIII decidió sacudirse el yugo: declaró ``destronada'' a la emperatriz Irene ``por sus enormes crímenes'' y ``trasladó'' la autoridad imperial a un representante más digno: Carlomagno, Rey de los francos, a quien coronó en las Navidades del año 800 dC., ratificando así una situación existente de hecho desde bastante tiempo atrás. Este movimiento político del Papa, opuesto incluso a la estrategia política que estaba intentando llevar adelante el mismo Carlomagno -por medio de una alianza matrimonial con la emperatriz Irene- planteó en el terreno de la Teoría Política, y también en el de la disputa ideológica y práctica, el problema de los dos poderes, en su forma más compleja.

La doctrina dominante durante no menos de cinco siglos (800-1300) fue la de la supremacía papal: el Papa era superior al Emperador y éste derivaba su autoridad real de aquél. En el campo teórico, los principales campeones de la supremacía papal fueron: - San Bernardo de Clairvaux (1091-1153); - Juan de Salisbury (1110-1180), quien escribió un tratado muy notable de Ciencia Política, el ``Policratus'', en el que desarrolló una teoría orgánica del Estado, basada en la analogía entre la constitución orgánica del hombre y la entidad política; - Santo Tomás de Aquino (1225-1274), sin duda el más notable de los filósofos medievales, aunque la amplitud y complejidad de su pensamiento nos hace vacilar al clasificarlo aquí. Más tarde comentaremos su obra y haremos algunas consideraciones al respecto; - Egidio Romanus (1247-1316), discípulo de Santo Tomás, quien hizo más bien una tarea de divulgación.

A partir del 1300, esa doctrina dominante comienza a ser crecientemente cuestionada. La causa de los Reyes nacionales contra las pretensiones papales estuvo también a cargo de escritores notables: - Juan de París (1300?) con su``Tratado de la Potestad Real y Papal''; - Pedro Dubois (1255?-1312?) con su ``Recuperación de la Tierra Santa''; - Juan Wycliffe (1320-1384) con su ``Del Dominio''.

Pero creemos que sobre todo hay que hacer mención de dos nombres, por ser precursores de líneas de pensamiento que serían dominantes en los tiempos modernos por venir: - Marsilio de Padua (1275?-1343?) por su obra ``El Defensor de la Paz''; - Dante Alighieri (1265-1321) por su obra ``De Monarquía''.

Vamos ahora a ver con más detalle algunas de las principales obras de este período.

Santo Tomás de Aquino reintrodujo, después de un olvido de mil años, la ``Política'' de Aristóteles en la teoría política occidental. Interpretó al filósofo griego en términos de teología cristiana y efectuó una magistral fusión de Aristóteles y San Agustín.

San Agustín se ocupaba de política pero su interés iba mucho más a la ``ciudad de Dios'' que a los reinos terrenales, a cuyos dirigentes a veces llamaba ``esos grandes bandoleros''. Por su parte, las escuelas monásticas de la alta Edad Media exaltaban los deberes de la piedad para los reyes y los deberes de la fidelidad para los vasallos, pero todo ello era expresión de una política absorbida por la moral religiosa, con eclipse de la Ciencia Política. Cuando en los reinos, los señoríos y las ciudades de la Cristiandad renació el orden político, fueron pensadores como Alberto Magno y Tomás de Aquino quienes iniciaron la restauración de la filosofía natural y de las ciencias, entre ellas la Política, que Aristóteles había compilado en la Grecia clásica.

Podemos considerar que cuando Tomás de Aquino comenzó a leer y comentar la ``Política'' de Aristóteles a sus alumnos, renació la Ciencia Política en Europa. A partir de allí ella va a rehacerse en torno a esa obra fundamental, ya sea con ella (como en Santo Tomás y tantos otros) o en contra de ella (como en Hobbes y muchos otros pensadores modernos).

El Comentario (prefacio o ``proemium'') que Santo Tomás hace de la ``Política'' de Aristóteles, y que todavía suele encabezar algunas ediciones, es de por sí una obra maestra: ubica a la Ciencia Política en el campo del saber y define su objeto, que en su opinion son las COMUNIDADES, en las que los conciudadanos acceden al ``buen vivir''. El mito (que luego se difundiría tánto) del ``estado de naturaleza'' es exorcizado de entrada: el hombre jamás vive sólo.

Realizar esas ``comunidades'' es el deber del hombre. Para hacerlo cuenta con la ciencia de la política, que es a la vez especulativa (observadora de lo real) y práctica (útil para la acción). La Ciencia Política no es nunca neutra. Los politólogos actuales harían bien en aprovechar esa lección del Comentario de Santo Tomás.

Hay una obra llamada ``De Regimine Principorum'', cuya autoría (al menos la de las primeras páginas) sería de Santo Tomás. En tal caso esta sería su obra más específicamente política. El problema es que tal autoría está cuestionada\footnote{Chatelet, Duhamel y Pisier, op. cit.}. De modo que vamos a buscar el pensamiento político de Santo Tomás en su obra más leída y más influyente: la ``Suma Teológica'', que no ofrece dudas en cuanto a su fuente de orígen. En ella, el tema político no tiene un lugar específico determinado. Está tratado en forma dispersa a lo largo de toda la obra. El lector interesado en este aspecto debe reunir los fragmentos por sí mismo y plantear las correspondientes cuestiones.

En la ``Suma Teológica'' la obra de Aristóteles es ampliamente comentada, pero Santo Tomás, según su costumbre, también la confronta con otros filósofos antiguos, con los Padres de la Iglesia y con las Santas Escrituras, y sus conclusiones tienen en cuenta todas esas consideraciones. Veamos algunos temas que presentan un interés actual.

En la ``Suma'', Santo Tomás no habla del Estado ni de los Derechos del Hombre, que son los conceptos omnipresentes en el pensamiento político moderno. En cambio, habla de ``comunidades'' que son de naturaleza relacional, y no han sido producidas por un pretendido ``contrato social'' sino por una relación entre ``sustancias primeras'': los individuos. Su orígen es muy claro: los bienes más importantes a que aspiran los individuos sólo pueden ser obtenidos y gozados ``en común''.

Así se constituyen grupos organizados, totalidades, tales como la ciudad. No se trata de un ``todo contínuo'' (como los organismos vivientes) ni tampoco de una fusión en un ser único. El pensamiento político de Santo Tomás no es organicista. La unidad política es otra cosa: una ``unidad de orden'', cuyas partes son distintas y autónomas, relacionadas sólo por la prosecución y disfrute de bienes que configuran un fin común.

El fundamento del poder es la necesidad de administrar, de dirigir, ese interés común. El bien común, el bien de todos, tiene neta preeminencia sobre los intereses particulares. Santo Tomás no tiene la menor estima por el desorden: asigna gran extensión al poder, exalta el valor de la virtud de la obediencia y considera a la sedición como uno de los pecados más graves. El oficio del Príncipe es regir, por medio de leyes, la conducta de los hombres asociados en pro del bien común. La ley positiva humana obliga a todos los ciudadanos desde su conciencia. La ley puede (en rigor, debe) castigar las trasgresiones, en forma acorde con la magnitud de las faltas, en casos extremos incluso con la muerte. El objeto de la ley es el ``buen vivir'': fomentar la virtud y reprimir el vicio.

Hasta aquí encontramos sólo razones en favor del ORDEN. Pero el pensamiento de Santo Tomás es complejo, dialéctico, y esas afirmaciones en favor del poder están muy matizadas: el deber de obediencia cesa frente al Príncipe injusto; la sedición deja de ser un pecado mortal y se convierte en una laudable virtud frente a los tiranos; si la ley ``no dice lo justo'' se desvanece su autoridad y no merece llamarse ley.

Una ley positiva, humana, es injusta si no es acorde con la Ley Eterna -ley natural- y con las Leyes divinas, expresadas en las Santas Escrituras. Esas fuentes metafísicas del Derecho y la Moral subordinan al Poder, que es esencialmente un poder legislativo.

La Ciudad es una ``comunidad perfecta'', última, autosuficiente: ella hace del hombre un ser ``civilizado''. Pero no es la única. También hay agrupamientos más extendidos, para los cuales Santo Tomás usa con frecuencia la expresión ``regnum'' en lugar de ``civitas'', como anunciando la extensión de la política a los grandes Estados modernos. En cambio, no considera ``comunidades'' a los Imperios, siempre hijos de la brutal fuerza militar.

La Ciudad es un agregado de familias, que son también comunidades naturales. En el pensamiento político de Santo Tomás, la familia tiene la carga del vivir, de la generación de niños, de su primera educación y de la subsistencia material. La economía, la riqueza, el bienestar, no son asunto de la Ciudad sino de las familias y de las asociaciones de las familias en el trabajo. La Ciudad tiene la carga de crear las condiciones generales donde puedan darse todas las actividades, incluso las económicas.

Esta concepción, en su conjunto, tiene desde luego un fundamento metafísico: la Comunidad más vasta y universal es la dirigida por Dios, que preside ``el Bien Común del Universo''. La pertenencia a esa comunidad suprema defiende al hombre de los excesos del poder público. La Iglesia Católica es, para Santo Tomás, la representante aquí abajo de esa Comunidad Global. De aquí puede quizás inferirse una posición favorable a la preeminencia papal, aunque cabe aclarar que Santo Tomás evitó siempre ``sacralizar'' la política (que es siempre una forma de sacralizar un statu quo determinado) o subordinar el orden secular al eclesiástico, como hicieron muchos de sus continuadores.

Fue Santo Tomás monárquico, como sostienen tantos tomistas? Cuál es para él el mejor régimen político? Respecto de la primera pregunta, Santo Tomás no aparece muy apasionado por este tema. Su temperamento lo inclinaba a respetar las instituciones establecidas y, de hecho, en la ``Suma'' encontramos argumentos a favor y en contra de la monarquía. El principio de unidad, el gobierno único de Dios sobre el Universo y las primeras páginas de ``De Regimine Principorum'' (si es que Santo Tomás las escribió -el resto sería de Ptolomeo de Lucques) abogan en favor de la monarquía. Pero también tiene -en páginas de autoría menos dudosa- argumentos en contra, que se sintetizan en la profunda idea de que los ``regímenes justos'' son variados y relativos a las circunstancias. En realidad, cada vez que Santo Tomás se plantea el tema del ``mejor régimen'', se pronuncia a favor del régimen mixto, donde uno solo reina, la élite tiene su parte en el gobierno y la elección de los gobernantes procede del pueblo.

Es en verdad difícil exagerar la importancia y la repercusión del pensamiento político de Santo Tomás. El solo hecho de retrasmitir a Occidente la ``Política'' de Aristóteles no sería pequeño mérito, pero Santo Tomás hizo mucho más que eso: la reelaboró en forma acorde con los valores de la civilización cristiana y la actualizó para los tiempos por venir\ldots{}

La grandeza de su obra -como la de Aristóteles- tiene mucho que ver con su método dialéctico, que lo lleva a confrontar las tesis de sus predecesores sobre cada cuestión. También tiene que ver con su modestia, que lo mantiene en el nivel de las ideas generales como filósofo y como hombre de ciencia, dejando a la prudencia de los hombres de acción la tarea de dar a la Ciudad sus leyes ``loco tempore convenientes'' -adaptadas a las contingencias históricas.

Es un pensamiento complejo el suyo, que va y viene entre los pro y los contra de cada cuestión, lo que motivó muchas lecturas e interpretaciones de sus obras. Acababa de restaurar la Ciencia Política en Occidente cuando ya Gilles de Roma se sirvió de ella para la causa política del Papa. Marsilio de Padua y el Dante para la del Emperador y Juan de Paris para la del Rey de Francia\ldots{}

Pasemos ahora al campo de los defensores de la autonomía del poder secular. Como ejemplos ilustrativos vamos a comentar las principales obras políticas de Marsilio de Padua y de Dante Alighieri.

El más notable de los últimos escritores políticos medievales (porque fue prematuramente moderno) probablemente fue Marsilio de Padua (1274-1343), hombre de compleja personalidad: médico, abogado, militar y político; eclesiástico, arzobispo de Milán, luego excomulgado y sus obras puestas en el Index, fue un hombre que se emancipó más que ningún otro de los moldes mentales de su tiempo. Enseñó, por ejemplo, la subordinación de la Iglesia al Estado, y del clero a los reyes. Enseñó también que los Pontífices y los Príncipes no poseían ninguna autoridad por derecho divino sino que todos la recibían por igual por delegación del pueblo soberano.

Su principal obra política fue ``El Defensor de la Paz'' (1324). Trata en ella tres temas: el Estado, la Iglesia y la relación entre ambos. Para Marsilio, el objeto del gobierno civil es la paz, y para lograrla considera que es mejor la monarquía que la república, pero también afirma que el Rey no tiene ninguna autoridad inmanente o metafísica: el poder le es conferido por el pueblo y lo debe ejercer sujeto al control popular y con las limitaciones de la ley, que procede del pueblo que lo eligió.

Por su parte, la Iglesia -sostiene Marsilio- no está compuesta sólo por el clero sino por todos los cristianos. Su autoridad no reside en los sínodos clericales ni menos en la curia papal sino en un concilio general, con representación de clero y laicos, donde los miembros más preparados (no necesariamente la mayoría) toman las decisiones. El clero debe limitarse a sus funciones espirituales y no mezclarse en asuntos temporales ni obstaculizar su actividad con riquezas mundanas. El Papa es una agente del concilio general, sin preeminencia inmanente alguna.

En cuanto a la relación entre Estado e Iglesia, Marsilio sostiene que ambos se componen de las mismas personas, aunque agrupadas de modo diferente. En el mundo venidero, el poder espiritual tendrá la preeminencia. En este mundo, el poder profano es el supremo.

Como puede verse, su pensamiento es fuertemente heterodoxo. Marsilio fue un pensador revolucionario, pero nació por lo menos dos siglos antes de tiempo. De todos modos, ``El Defensor de la Paz'' representa una etapa decisiva en la formación de la teoría sobre la que se edificó el Estado moderno: el principio de soberanía.

En este aspecto, Marsilio plantea dos elementos esenciales para el poder del Estado: la autonomía del poder político civil y el monismo estatal. La fundamentación de la autonomía del poder civil parte de Aristóteles: la Ciudad ``es creada para vivir, existe para vivir bien'', en el sentido secular del término. El bien extramundano, la vida eterna, etc., no cuentan como principio constitutivo de la Ciudad. El orígen de la Ciudad es subvenir a las necesidades materiales e intercambiar mutuamente los bienes capaces de satisfacerlas. De esta concepción, casi burguesa, de la dicha presente, se deduce el principio del gobierno. Quién debe gobernar? La autonomía de la sociedad civil tiene su correspondencia en la autonomía del poder político. El gobernante debe surgir de la sociedad misma, para coordinar las funciones que hacen al bien común terrestre. El clero no debe gobernar la ciudad terrestre, bajo grave riesgo de guerra civil.

Con respecto al monismo estatal, el razonamiento parte de afirmar la existencia de tres órdenes en la Ciudad: el Sacerdocio, encargado de la salvación eterna; la Producción y los Oficios, para satisfacer las necesidades; y la Coerción, para ejecutar las leyes y custodiar lo justo. La paz civil se logra si cada parte se limita a cumplir las tareas que le corresponden. Para evitar los conflictos, hay que considerar a esta totalidad compleja como una unidad. De la unidad del cuerpo social se deduce la unidad del mando: un solo jefe. Este es el principio del monismo estatal, que será desarrollado dos siglos y medio después por Jean Bodin. Ese jefe único debe gobernar según la ley, que tiene su causa eficiente en el pueblo, es decir, en la voluntad popular, en quien reside en última instancia, según Marsilio de Padua, la paz civil\footnote{Chatelet, Duhamel y Pisier, op. cit.}.

Pasemos ahora al caso de Dante Alighieri (1265-1321) y de su obra ``De Monarchia'' (1310?). Esta obra, escrita en latín, puede ser considerada como el tratado donde el pensamiento político del Dante se enuncia más explícita y completamente, más allá de las referencias ocasionales a la cosa política contenidas en ``De Convivio'' o en ``La Divina Comedia''.

``De la Monarquía'' desarrolla un planteo estratégico, directamente vinculado con los objetivos de una práctica política, que tiene a su vez un basamento teórico sustentado en una visión metafísica. Expresa el conflicto, la oposición entre el Estado monárquico moderno, en busca de su soberanía, y el poder espiritual de la Iglesia, pero pretende sustentar su estrategia en principios universales rigurosamente establecidos. En pocas palabras, es el trabajo de una racionalidad que busca los fundamentos metafísicos, filosóficos y jurídicos de la posición política asumida por el autor.

``De la Monarquía'', al igual que ``El Defensor de la Paz'' de Marsilio de Padua, respalda a la Monarquía en el conflicto que la engrenta con la Iglesia, y su trasfondo histórico es la lucha inmisericorde que libran los güelfos, fieles a la autoridad temporal del Papado, y los gibelinos, que afirman la primacía imperial.

La originalidad de la obra no reside tanto en su tema sino en la argumentación que desarrolla, en forma de tríptico.

En el primer libro, deduce ``la necesidad del principio imperial'' del principio último de ``unidad para la paz'', necesario para el bienestar del mundo en su faz secular.

El segundo libro plantea un problema de raíz histórica: si los romanos ejercieron o no ``de jure'' el dominio universal. Al resolver positivamente esta cuestión (lo que implica, dicho sea de paso, una revisión radical de la doctrina agustiniana planteada en ``La Ciudad de Dios'') Dante identifica al Derecho con la Voluntad de Dios y plantea el requerimiento de una ``santificación'' de la instancia imperial, creadora del orden terrestre. En resúmen, Dante concluye planteando un retorno al ``mito fundador'' de Roma.

El tercer libro refuta las objeciones que fueron hechas a la primacía del Emperador con argumentos sacados de las Santas Escrituras o de textos históricos. Dante niega a la Iglesia el derecho de otorgar autoridad al Emperador y funda la independencia de los poderes -el secular y el espiritual- en la dualidad propia de la naturaleza humana. El objetivo del campo secular es el bienestar terrestre, cuya obtención plantea la necesidad de un principio único dominante, para evitar las discordias ``inter partes'', con lo que volvemos a la idea expresada inicialmente.

El fundamento metafísico de su razonamiento es aristotélico. La Monarquía temporal es necesaria para el bienestar del mundo; la libertad de los sujetos sólo puede basarse en el poder de la instancia reguladora del conjunto social, que se hizo efectiva por primera vez en el mundo en el Imperio Romano, con Augusto y su ``pax romana''.

El Emperador, instancia portadora de la soberanía, es mucho más que una opción política de gobierno: es un requisito del mundo y de la naturaleza humana. El Emperador es un proveedor de paz, un modo de acceso a la prudencia y una expresión del vínculo ético del gobernante con los gobernados. Se trata de un vínculo indestructible entre la instancia soberana, que ejerce su poder dentro de los límites de su potencia, y los súbditos, que legitiman ese poder mediante su acatamiento y consenso, pero al mismo tiempo forman parte de la potencia imperial.

Entre los siglos XVI y XVIII emergerá en toda su fuerza la teoría moderna de la soberanía estatal. El Dante se anticipa a ella, pero al mismo tiempo se diferencia de ella, justamente por esa idea de una mediación ética en el vínculo entre gobernantes y gobernados. Si hemos de reconocer a la Etica algún lugar en la Política, ese lugar es justamente el vínculo necesario entre los súbditos, sujetos de la soberanía, y la instancia soberana. Se trata de una especie de necesaria ``substancialización'' antropológica del Bien Político. En ese sentido, la obra del Dante, aunque haya emergido como respuesta a determinadas circunstancias históricas concretas y hasta personales, es ciertamente mucho más que un ``escrito de circunstancias''\footnote{Chatelet, Duhamel y Pisier, op. cit.}.

\hypertarget{el-pensamiento-poluxedtico-moderno}{%
\subsection*{El pensamiento político moderno}\label{el-pensamiento-poluxedtico-moderno}}
\addcontentsline{toc}{subsection}{El pensamiento político moderno}

El tiempo que media entre Marsilio de Padua (1274-1343) y Nicolás Maquiavelo (1469-1527) es el tiempo de una gran transición; es el tiempo de ese Renacimiento que separa (o une) los tiempos medievales de los modernos. En su transcurso, el Imperio y el Papado declinaron en su importancia política, nacieron los Estados nacionales modernos y se establecieron fuertes monarquías en España, Francia e Inglaterra, mientras Italia y Alemania permanecían divididas en pequeños principados y ciudades-estados.

La pólvora originó un nuevo ``arte de la guerra''; la imprenta introdujo al mundo en lo que hoy nosotros (conscientes de su tremenda importancia a largo plazo) denominamos Galaxia Gutemberg; el descubrimiento de América y otras exploraciones ampliaron literalmente el horizonte de la visión europea del mundo; la teoría copernicana rompió los estrechos moldes mentales de la Cosmografía medieval, mientras la Reforma protestante y la Contrarreforma católica rompían por primera vez en siglos la unidad religiosa de Occidente. Estos cataclismos culturales tuvieron, por supuesto, su correlato político.

Podemos considerar a Maquiavelo como ``el padre fundador'' de la Ciencia Política moderna. Fue un agudo observador de las prácticas políticas habituales de su tiempo, y las consignó con precisión en sus escritos. Nada hubo en su vida que justifique la fama que ha hecho de su nombre sinónimo de inescrupuloso o inmoral. Maquiavelo era simplemente un patriota italiano que se dió cuenta de que su propio país se estaba quedando atrás de las emergentes potencias europeas, y de que en esas condiciones, su triste destino era la dependencia o la destrucción.

Cómo hacer para crear una Italia unida, capaz de resistir las agresiones externas y ocupar un lugar digno en el concierto de las naciones europeas? Este es el tema de fondo de sus tres obras políticas principales: ``El Arte de la Guerra'', ``Discursos sobre la Primera Década de Tito Livio'' y ``El Príncipe''.

Maquiavelo fue un estadista práctico, más que un teórico de la política, aunque tuvo una rara habilidad para expresar sus observaciones y experiencias en forma de principios generales de acción política. De todos modos, sus obras son tratados sobre el arte de gobernar y no teorías abstractas.

Para Maquiavelo, las causas del deplorables estado político de Italia eran la desunión, el desorden y el abandono; su primera consecuencia, la devastación por las tropas extranjeras. Cómo remediar ese estado de cosas? Según Maquiavelo, había dos medidas básicas a tomar: - la creación de un ejército nacional; - la formación de un Estado nacional.

Maquiavelo era republicano y pensaba que algún día Italia podría ser una república, pero esos grandes remedios sólo podían ser construidos por un monarca autocrático, un Príncipe, que actuara con gran libertad de medios, morales si puede e inmorales si debe.

Con Maquiavelo queda registrado en la teoría lo que venía dándose ampliamente en la práctica: la separación de la Etica y la Política, si la necesidad lo requiere. Ya no se habla de la ``buena vida'' como en los tiempos medievales sino de las condiciones de supervivencia y de las posibilidades de una construcción política relativamente estable en medio de la profunda crisis en que se debatía todo el Occidente en aquellos días. Como ya hemos visto, esas van a ser características perdurables del pensamiento político moderno.

En cualquier Historia del Pensamiento Político pueden encontrarse abundantes referencias a esta época. Aquí, por limitaciones de espacio y por ser otro el objetivo esencial de la obra, vamos a tomar como ejemplos ilustrativos sólo dos, poco conocidos y comentados en este ámbito. El primero es una propuesta de reacción positiva frente a la crisis: se trata de las ``Constituciones'' de San Ignacio de Loyola. El otro es un verdadero manual de arte política, comparable y a la vez diferente de las obras de Maquiavelo: se trata del ``Testamento Político'' del Cardenal Richelieu.

Veamos primero el caso de San Ignacio de Loyola (1491-1556) y de sus ``Constituciones de la Compañía de Jesús'' (1539-1556).

Si la Política, en un sentido amplio y profundo, es el arte de gobernar una sociedad humana, las ``Constituciones'' de San Ignacio pueden sin duda ser consideradas, al menos en una de sus dimensiones, como una obra política. En realidad, como todas las reglas monásticas, las ``Constituciones'' son una obra maestra del pensamiento político. Es necesario mucho genio político para trazar las condiciones de vida espiritual, material y administrativa de una comunidad en la perspectiva de una duración indefinida\footnote{Chatelet, Duhamel y Pisier, op. cit.}.

Las ``Constituciones'' fueron elaboradas a lo largo de 17 años, entre 1539 y 1556. San Ignacio aún trabajaba en ellas cinco meses antes de su muerte, y todo su ser está expresado en ellas. Quién era, pues, este hombre? Pocos fundadores de órdenes religiosas han sido objeto de visiones personales tan parciales, caricaturescas y malévolas: un puro militar, hábil intrigante, lo que hoy llamaríamos un pragmático total. Creemos que no vale la pena refutar hoy esos antiguos errores y calumnias. Es preferible re-descubrir al hombre leyendo los escritos que nos ha dejado.

Antes que nada, San Ignacio era un místico. Su política está impregnada de mística. Todas las etapas de su accionar están ``inspiradas'' a partir de esa experiencia primordial, acaecida en Manrese, en la que tuvo ``la inteligencia y conocimiento de numerosas cosas tanto espirituales como referentes a la fe y a la cultura profana''. En esa experiencia mística él ``comprendió'' cómo Dios había creado el mundo y percibió que el acto creador es un acto de amor, y que Dios sólo quiere que sus criaturas respondan a su amor y se dediquen a re-encontrarse con El en su gloria.

Esa es su intuición fundamental: la misión del hombre en la Tierra es cumplir la Voluntad de Dios: obrar para que todos los hombres amen a Dios y se hagan artesanos de su Gloria. El esquema ignaciano es, pues: el amor de Dios desciende hacia los hombres, y los hombres, por amor, remontan hacia Dios, no sin exhortar al mayor número posible de otros hombres a hacer lo mismo.

Esa visión define los objetivos esenciales de la ``política'' ignaciana: compartir con quienes quieran escucharlo su intuición primera, a fin de que ellos la propaguen, y que esa propagación sea continua e indefinida en sus alcances. Desde luego, no puede hacerse un ingenuo reduccionismo de la compleja política ignaciana a esa experiencia de una revelación personal, pero toda su actuación posterior encontró su inspiración y explicación profunda en la fuerza que emanó para él de la iluminación que recibió en Manrese.

Su primera tarea fue elaborar su visión, y ante el imperativo de ordenar su vida discernir cual es la voluntad de Dios respecto de él y adaptarse a ella. Ese es el objeto de los ``Ejercicios Espirituales'', que pronto se difundieron como práctica para quienes desearan ``ver claro en sus vidas y tomar un nuevo punto de partida'', más allá de ser una herramienta de la política ignaciana de reclutamiento.

La política corriente es esencialmente finalista: persigue objetivos concretos y predeterminados. Un rasgo extraño de esta política ignaciana impregnada de misticismo, es la indefinición del porvenir, reflejada en el concepto de ``indiferencia'' respecto del ``qué hacer''. La Psicología Religiosa ayuda a explicar esto: para San Ignacio y sus compañeros lo esencial es hacer la Voluntad de Dios, cualquiera sea ésta, y lo importante es ponerse en disposición de espíritu adecuada para percibirla. Toda actividad es buena, a condición de que Dios la inspire y ratifique. En caso de duda, siempre puede consultarse al Papa, Vicario de Dios en la Tierra. Esto explica la diversidad de tareas desempeñadas por la Compañía.

Dotada de consagración oficial en el seno de la Iglesia desde 1540, su política inicial consistió en no tener ninguna predeterminada sino satisfacer caso por caso las demandas que le fueran planteadas y que continuamente se acrecentaron más allá de sus posibilidades, porque estos hombres eran muy requeridos: eran letrados y conducían una vida ejemplar. Las grandes líneas de su heterogénea acción fueron: la misión evangelizadora, la reforma interna de la Iglesia (fueron los adalides de la llamada ``Contrarreforma'', como medio efectivo de enfrentar a los protestantes) y, en forma creciente, la educación, en una original forma mixta para novicios y laicos. En corto tiempo, como puede advertirse en la correspondencia ignaciana, la fundación y gestión de colegios se convirtió en una preocupación central de su política.

Otra línea política básica era el mantenimiento de relaciones con ``los grandes de este mundo''. Testimonio de ella es una abundante correspondencia con reyes y nobles, en una acción política que intenta servir a los intereses de la Iglesia y del Papado, y obtener apoyo para las obras de la Compañía. Esta acción se llevó a cabo con una clara comprensión de los beneficios que de la acción de la Compañía se derivan, o pueden derivarse, para el gobierno civil: por ejemplo, el efecto de la fundación de un Colegio el términos de desarrollo intelectual de una comunidad, de impacto sobre la opinión pública y sobre la concordia de los ciudadanos, etc.

Por supuesto, otra línea política fundamental se refería a la lucha contra los adversarios de la Iglesia: la Reforma Protestante y el Imperio Turco. Respecto de la primera, pronto se advirtió la conveniencia y la necesidad de enfrentarla en el terreno de la educación. Respecto del segundo, en cambio, San Ignacio diseñó una campaña militar que preanunció la que luego de su muerte puso fin al expansionismo turco en la batalla de Lepanto.

Las ``Constituciones'' de San Ignacio, políticas en cuanto se refieren al gobierno de personas, fueron y son la forja de los hombres que cumplieron y cumplen tareas en la Compañía ``a la mayor gloria de Dios''. Son una sabia arquitectura de disposiciones estructuradas en base a un principio fundamental, imperativo: la OBEDIENCIA. ``Perinde ac cadaver'' dice la fórmula latina ( a imitación del cuerpo de Cristo luego de su descendimiento de la Cruz?). Nuevamente encontramos aquí la raíz mística, que tanto diferencia la política ignaciana de otros enfoques ``seculares'' de la política. La obediencia al superior entronca en última instancia con la obediencia a la Voluntad de Dios: la desobediencia en cualquier escalón es una ofensa a Dios, pero esa obediencia está condicionada por principios éticos superiores y, por otra parte, el superior sabe que su orden debe ser lo más acorde posible con lo que cada hombre percibe como designio de Dios para él, aquello para lo cual es apto y sirve. Es fácil percibir la potencia política que puede generar una obediencia perfecta y voluntaria fundada en un absoluto de raíz metafísica y arraigada en una convicción interior sobre el sentido de la propia vida.

Quizás en esa extraña mezcla de disciplinada obediencia y de confiada delegación de funciones y responsabilidades en base a lo que cada uno siente como identidad propia y misión existencial, en ese enfoque participativo que por momentos parece posmoderno, se encuentre la explicación de la dimensión política de algunos extraños fenómenos históricos, como las misiones jesuíticas en América del Sur, en las que un puñado de hombres, sin posibilidad alguna de ejercer una coacción material efectiva, organizaron políticamente a varios miles de indios, en pueblos de vida y economía perfectamente articuladas sobre una enorme y dispersa extensión de territorios salvajes; estructura política que sobrevivió incluso a la expulsión de sus fundadores, ya que solo fueron abatidos por la violencia de una guerra cruel y despiadada.

Pasemos ahora al caso de Armand-Jean du Plessis, cardenal de Richelieu (1585-1642) y su ``Testamento Político'' (1632-1639 aprox.).

Richelieu, obispo de Lyon en 1606, en 1614 pasó a formar parte de los Estados Generales. Apoyó a la Regente María de Medici, lo que le valió integrar el Consejo Real en 1616. Acompañó en su destierro a la Regente y participó de las negociaciones de reconciliación de ésta con el Rey Luis XIII, lo que le valió el capelo cardenalicio y la reincorporación al Consejo (1624), del que asumió la presidencia, lo que terminó convirtiéndolo en árbitro de la política francesa en nombre del Rey. Participó con amplio sentido político en las guerras de religión y creó las bases de la centralización política y administrativa de Francia, fortaleciendo la autoridad monárquica en nombre de la razón de Estado. Su sucesor fue el cardenal Mazzarino.

De todas las obras atribuidas al cardenal Richelieu (``Memorias'', ``Máximas Estatales''), el ``Testamento Político'' es la más elaborada en cuanto a reflexiones sobre el gobierno del Estado. Aunque su autenticidad fue cuestionada casi desde su aparición, y es indudable que una gran parte fue redactada por colaboradores (como el célebre ``P. Joseph'') tampoco puede dudarse de que el trabajo de los secretarios fue dirigido por Richelieu y que el ``Testamento Político'' expresa fielmente su pensamiento.

En su dedicatoria al Rey, Richelieu explica sus intenciones al escribirlo: dejar al Rey un conjunto de consejos prácticos, en el que pudiera inspirarse para asegurar la continuidad de una política y una obra gubernamental que corría el riesgo de quedar inconclusa por causa de la crónica enfermedad del cardenal.

La obra presenta una forma muy estructurada: dos partes, de ocho y diez capítulos respectivamente, divididos a su vez en secciones. El tema mayor de la obra es el Estado.

La primera parte, luego de una introducción histórica (``una sucinta narración de las grandes acciones del Rey'') trata de la estructura del Estado, los órdenes que lo componen y los órganos que lo dirigen. La segunda parte trata de la manera de dirigir el Estado, los principios fundamentales que deben observarse en su gobierno. Es, pues, un manual de arte política, comparable (si bien con muchas diferencias de criterio) al ``Príncipe'' de Maquiavelo.

Con respecto a la estructura del Estado, Richelieu conserva esa concepción tripartita de la sociedad, de origen tradicional, que fue sistematizada por el jurista Charles Loyseau a principios del siglo XVII: los ``sujetos del Rey'' se agrupan en tres estados u órdenes; el clero, la nobleza y el ``tercer estado'', de desigual tamaño y de desigual (e inversa) importancia política. El clero es el primer orden del Reino, y Richelieu (en contra de lo que a veces suele creerse de él) se muestra en este aspecto como un ``hombre de Iglesia'', que busca preservarla de los excesos del poder estatal y al mismo tiempo regenerar al orden eclesiástico por medio de su adhesión a los principios de la Contrarreforma y de la restauración del poder episcopal. Aplica en esto un galicanismo moderado.

A la nobleza le dedica muchas alabanzas, como la de que constituye ``uno de los principales nervios del Estado, capaz de contribuir mucho a su conservación y su restablecimiento'', pero sin ocultar, por otra parte, su profunda desconfianza hacia un orden que produce peligrosos enemigos de la centralización del poder estatal: busca satisfacer sus demandas, pero a cambio de su estrecha sumisión al Estado. Richelieu fue quizás el más consciente propugnador de esa política centralizadora y unitaria que buscó fortalecer el poder real vinculándolo con la naciente burguesía y reduciendo a los señores feudales, a los nobles, a la condición de cortesanos, llenos de privilegios y placeres pero desprovistos de todo poder verdadero.

Al tercer estado le dedica un breve capítulo, referido sobre todo a sus estratos superiores: los oficiales de justicia y de finanzas, capítulo en el cual propone medidas para combatir la corrupción en esos niveles. Del pueblo, elemento residual del tercer estado, no hay en su obra más que breves referencias, impregnadas de cierto desprecio y dudas sobre su capacidad de sujetarse a la leyes por la razón, pero recomienda que los impuestos que gravitan sobre el pueblo sean moderados, en nombre de la justicia y del interés bien entendido del mismo Estado.

Su visión conservadora y organicista lleva a Richelieu a plantear un equilibrio entre los órdenes, fundado en una jerarquía de honores entre ellos. A la cabeza del Estado están el Rey y sus ministros, cuyo rol es exaltado. Da la impresión de que, en su concepción, la verdadera tarea del Rey es elegir buenos ministros, y que éstos son los que verdaderamente gobiernan. El Rey debe saber elegir como colaboradores a hombres probos, consagrados a los asuntos del Estado, que le sepan hablan con franqueza, indiferentes a la calumnia, desapegados de intereses y pasiones y sobre todo, de las mujeres. Recomienda para esas funciones a los eclesiásticos, ya que al carecer de esposa e hijos sienten menos que otros el deseo de hacer prevalecer sus intereses particulares. A sus consejeros competentes y devotos, el Rey ha de sostenerlos en su confianza contra las intrigas de los envidiosos y los descontentos. Su ``teoría del ministerio'' es en realidad una fundamentación racional del sistema que él mismo creó en la práctica: un Consejo de pocos miembros (cuatro, en su caso) uno de los cuales tenga total primacía para asegurar la unidad del mando ``porque nada es más peligroso en un Estado que diversas autoridades iguales en la administración de los negocios''.

El arte de conducir al Estado tiene reglas precisas, que Richelieu desarrolla largamente en la segunda parte de su ``Testamento'': * Respetar la Voluntad Divina, que es donde se encuentra el fundamento de la autoridad real. Cumplir sus deberes con la Iglesia, dar ejemplo de piedad, favorecer las conversiones voluntarias, no blasfemar; tales son los consejos que Richelieu da al Rey. Por otra parte, excluye el uso de la fuerza para obtener la abjuración de los protestantes; * En una actitud ``dividida'', típica del Humanismo, Richelieu sostiene que, una vez rendido a Dios y a su Iglesia el homenaje debido, se es libre de hacer política sólo con la guía de la filosofía antigua y del sentido común. El objetivo de su acción es asegurar la salud y fuerza de su Estado: es, en definitiva, la razón de Estado, que consiste antes que nada en dirigir al Estado por la razón: tener dominio de sí, firmeza, discreción, para aplicar la fuerza que sea necesaria para vencer las resistencias internas y externas a la acción ordenadora del Estado; * El arte de dirigir a los hombres necesita recurrir al uso de recompensas y castigos: para Richelieu son más importantes los segundos que las primeras. En política no hay lugar para la caridad o la piedad cristianas. El Poder es siempre el objeto y el medio del Estado y el Poder se debilita si se recurre a la conmiseración. El poder depende de la reputación del Príncipe en la opinión pública, de la fuerza de los ejércitos y la seguridad de las fronteras, y de la economía entendida como fundamento material del poder estatal, para lo cual aconseja el fomento del comercio exterior.

Esta obra fue publicada tardíamente, cuando el apogeo del absolutismo monárquico ya había producido una reacción pro-liberal. Es una obra que expresa, teórica y prácticamente, esa pasión casi mística por el Estado, que es el fundamento emocional del absolutismo y que lleva a concebir un Estado que trasciende en forma absoluta los intereses concretos de los grupos humanos que lo componen y expresa, o pretende expresar solamente el interés supremo de la Nación, al que todo ha de subordinarse. En ese sentido puede ser entendida como una visión precursora de las ideologías nacionalistas que en el siglo XX concibieron a la Nación, al Estado o a la Patria como una entelequia de naturaleza metafísica, desconectada de la concreta manifestación sociológica y antropológica de su encarnación histórica real\footnote{Chatelet, Duhamel y Pisier, op. cit.}.

\hypertarget{tercera-parte-1}{%
\section*{Tercera parte}\label{tercera-parte-1}}
\addcontentsline{toc}{section}{Tercera parte}

\hypertarget{teoruxedas-poluxedticas-normativas-contemporuxe1neas}{%
\subsection*{Teorías políticas normativas contemporáneas}\label{teoruxedas-poluxedticas-normativas-contemporuxe1neas}}
\addcontentsline{toc}{subsection}{Teorías políticas normativas contemporáneas}

Las obras políticas que vamos a intentar describir aquí abarcan un largo e intenso período de tiempo, que va desde fines del siglo XVII hasta nuestros días. Siguiendo en parte a J.J. Chevalier, hemos dividido ese tiempo en cuatro subperíodos, por razones de claridad expositiva y aceptando las limitaciones del esquematismo que tienen siempre tales divisiones: -el asalto al absolutismo (1690-1789); - las consecuencias de la revolución francesa (1789-1848); - los socialismos y los nacionalismos (1849-1927) - las teorías actuales (1928 en adelante).

El asalto al absolutismo.

El primer momento (1690-1789) expresa la reacción antiabsolutista, ideológicamente relacionada con la consolidación de la burguesía capitalista como clase dominante, que ya no se muestra dispuesta a actuar como aliado secundario de la monarquía en la conformación de un Estado centralizado, sino que, cumplido ese objetivo, aspira a un rol más protagónico y a poner en vigencia un ideario y una institucionalización política más acordes con su dinámica social. Esa reacción es fundamentalmente la obra del pensamiento racionalista liberal. Los grandes temas subyacentes en estas obras son, en nuestra opinión: - la búsqueda de un equilibrio entre el Poder y la Libertad; - el encauzamiento de la participación política acrecentada Sin pretender suministrar un listado exhaustivo de obras de este período, creemos sin embargo que entre las principales deben ser mencionadas al menos las siguientes:

\begin{itemize}
\tightlist
\item
  Cesare Beccaria: DE LOS DELITOS Y DE LAS PENAS (1764);
\item
  Jeremy Bentham: INTRODUCCIÓN A LOS PRINCIPIOS DE LA MORAL Y DE LA LEGISLACIÓN (1789);
\item
  Jean-Jacques Burlamaqui: PRINCIPIOS DE DERECHO POLÍTICO (1751);
\item
  David Hume: DEL CONTRATO ORIGINAL (1748) y DEL ORIGEN DEL GOBIERNO (1774);
\item
  Simon-Nicolas-Henry Linguet: TEORÍA DE LAS LEYES CIVILES O PRINCIPIOS FUNDAMENTALES DE LA SOCIEDAD (1767);
\item
  John Locke: DOS TRATADOS DEL GOBIERNO CIVIL (1690);
\item
  Jean-Louis Lolme: CONSTITUCIÓN DE LA INGLATERRA O ESTADO DEL GOBIERNO INGLES (1771);
\item
  Charles-Louis de Secondat, barón de Montesquieu: EL ESPÍRITU DE LAS LEYES (1748);
\item
  Thomas Paine: LOS DERECHOS DEL HOMBRE (1791-1792);
\item
  Jean-Jacques Rousseau: EL CONTRATO SOCIAL (1762);
\item
  Emmanuel Joseph Sièyes: QUE ES EL TERCER ESTADO (1789).
\end{itemize}

De este conjunto de obras vamos a ver con más detalle la que a nuestro juicio puede ser considerada la más completa y representativa del período, y quizás la que más persistente influencia ha ejercido en el pensamiento político europeo y americano: se trata de ``Dos Tratados sobre el Gobierno Civil'' de John Locke.

John Locke nació en 1632. Estudió en Oxford, donde alcanzó el grado de ``master'' en 1658. Se conserva memoria de su desagrado por el árido método escolástico imperante en su tiempo, pues ``le intersaban más los hechos reales que las abstracciones y las cuestiones sin utilidad''. En su carácter se destacaban dos notas: la simpatía por la libertad individual y un sosegado utilitarismo. Conoció el exilio y el retorno triunfante, tras la ``Glorius Revolution''. Murió en 1704.

Su obra es una de las más vigorosas críticas a la monarquía absoluta, cuyo rechazo está fundado sobre la idea de la necesaria subordinación de la actividad de los gobernantes al consentimiento popular.

Locke es un de los teóricos clásicos del liberalismo político. Propone una articulación rigurosa de los temas liberales fundamentales: la igualdad natural de los hombres, la defensa del sistema representativo, la exigencia de una limitación de la soberanía estatal, limitación requerida por la defensa de los derechos subjetivos de los individuos. Buscó un remedio a la tiranía en la división de los poderes del Estado, anticipándose en esto a Montesquieu.

De sus ``Dos Tratados\ldots{}'', el primero es de carácter polémico y puede decirse que no conserva mayor interés ni actualidad para nosotros, hoy. Se trata de una refutación de los argumentos desarrollados en otra obra, el ``Patriarcha'' de R. Filmer, quien pretendía demostrar el derecho de los príncipes al gobierno absoluto, asimilando la soberanía política al dominio primitivo de Adán sobre el mundo entero, dominio que, recibido directamente de manos de Dios, habría sido trasmitido a los monarcas a través de la Historia\ldots{}

El segundo tratado apunta, por el contrario, a establecer positivamente ``el origen, los límites y los fines verdaderos del poder civil''. Esta obra es la que hoy generalmente se publica\footnote{John Locke: ENSAYO SOBRE EL GOBIERNO CIVIL, Madrid, Aguilar, 1981.} y se lee, pero en el pensamiento de Locke las dos obras forman un todo deductivamente entrelazado. En una síntesis muy apretada, la filosofía política de Locke es la siguiente: El gobierno debe ejercerse con el consentimiento de los gobernados. El gobierno es una creación del pueblo, mantenida por el pueblo para asegurar su propio bien. Según Locke, esta teoría se basa en la vigencia de dos conceptos muy vinculados: la Ley de la Naturaleza y el Contrato Social.

En el ``estado de naturaleza'' los hombres eran libres, pero como ``libertad no es licencia'', no tenían derecho a hacer cualquier cosa sino a actuar en modo acorde con una ``ley'' de la Naturaleza: la RAZÓN, que indica que, si los hombres son libres e iguales, nadie puede dañar a otro, o convertirlo en instrumento de los propios fines. El estado de naturaleza no era un estado de guerra de todos contra todos -sostiene Locke, contrariando en esto a Hobbes- sino un estado que sería perfecto si los hombres se comportaran racionalmente, pero no sucede así. La guerra y la violencia son siempre posibles y plantean la necesidad de un gobierno, el cual se forma por el sometimiento voluntario de las libertades individuales a un poder superior, cuya tarea es protegerlas. Surge así el ``contrato social'', que se establece entre el pueblo y su gobernante.

El contrato social contiene dos ideas íntimamente unidas: el contrato de gobierno y el contrato de sociedad. Locke (al igual que Rousseau y que Hobbes) parte de este último. Cuando ya se ha organizado la comunidad, ésta decide confiar a un gobierno la protección y defensa de sus libertades y derechos, pero conservando la posibilidad de retirarle su confianza si su accionar no le conviene. En el fondo, lo que Locke busca es fundamentar filosóficamente un régimen de Monarquía constitucional, con un Parlamento que encarne la representación popular y que respete y haga respetar las libertades públicas.

\hypertarget{las-consecuencias-de-la-revoluciuxf3n-francesa}{%
\subsection*{Las consecuencias de la Revolución Francesa}\label{las-consecuencias-de-la-revoluciuxf3n-francesa}}
\addcontentsline{toc}{subsection}{Las consecuencias de la Revolución Francesa}

El segundo momento (1790-1848) es relacionado por J.J. Chevalier con las consecuencias de la Revolución Francesa porque, si bien el Absolutismo ha sido postrado y la Revolución se ha cumplido, no es un momento de plenitud sino de enfrentamiento con una realidad que en gran parte está aún por construir. Decía entonces Napoleón: ``Se ha destruido todo; se trata de recrear. Hay un gobierno, poderes; pero, qué es todo el resto de la Nación? Granos de arena''.

Precisamente porque la Revolución ha triunfado es concebible la emergencia de una pasión contrarevolucionaria. Su principal vocero fue Edmund Burke, con sus ``Reflexiones sobre la Revolución de Francia'' (1790). Es justamente porque ha triunfado el jacobinismo, con sus augustas abstracciones (la Nación, el Pueblo) que los pueblos vencidos responden con la emergencia de un nacionalismo concreto, apasionado y fuerte. Este sentimiento es cabalmente expresado por los ``Discursos a la Nación Alemana'' (1807-1808) de Johann G. Fichte. La Revolución estaba animada de un espíritu igualitario pero terminó siendo burguesa, como un momento de ese largo proceso en el que la pasión igualitaria se enfrenta con la pasión de libertad en el corazón del hombre. Ese es justamente el tema que, con singular maestría, afronta el jóven Alexis de Tocqueville en su ``Democracia en América'' (1835-1840).

Aparte de las obras mencionadas, creemos que en una lista de obras importantes de este período hay que mencionar por lo menos las siguientes:

\begin{itemize}
\tightlist
\item
  Gracchus Babeuf: COMPENDIO DEL GRAN MANIFIESTO\ldots PARA RESTABLECER LA IGUALDAD DE HECHO" (1793);
\item
  Pierre-Simon Ballanche: ENSAYO SOBRE LAS INSTITUCIONES SOCIALES EN SU RELACIÓN CON LAS IDEAS NUEVAS (1818);
\item
  Louis Blanc: ORGANIZACIÓN DEL TRABAJO (1840);
\item
  Louis de Bonald: LEGISLACIÓN POSITIVA CONSIDERADA EN LOS ÚLTIMOS TIEMPOS POR LAS SOLAS LUCES DE LA RAZON (1802);
\item
  Francois-René Chateaubriand: LA MONARQUÍA SEGÚN LA CARTA (1816);
\item
  Karl von Klausewitz: DE LA GUERRA (1816-1831);
\item
  Auguste Comte: PLAN DE TRABAJOS HISTÓRICOS NECESARIOS PARA REORGANIZAR LA SOCIEDAD (1822);
\item
  Benjamin Constant: LOS PRINCIPIOS POLÍTICOS APLICABLES A TODOS LOS GOBIERNOS (1806);
\item
  Johann Fichte: LOS FUNDAMENTOS DEL DERECHO NATURAL (1796);
\item
  Charles Fourier: TEORÍA DE LOS CUATRO MOVIMIENTOS Y DE LOS DESTINOS GENERALES (1808);
\item
  Francois Guizot: DE LOS MEDIOS DEL GOBIERNO Y DE LA OPOSICIÓN (1821);
\item
  Georg W.F. Hegel: PRINCIPIOS DE LA FILOSOFÍA DEL DERECHO (1821);
\item
  Felicite de Lamennais: PALABRAS DE UN CREYENTE (1834);
\item
  Pierre Leroux: DE LA HUMANIDAD (1840);
\item
  Jules Michelet: EL PUEBLO (1846);
\item
  Robert Owen: UNA NUEVA VISIÓN DE LA SOCIEDAD (1813-1814);
\item
  August Rehberg: INVESTIGACIONES SOBRE LA REVOLUCIÓN FRANCESA (1793);
\item
  Charles Renouvier: MANUAL REPUBLICANO DEL HOMBRE Y DEL CIUDADANO (1848);
\item
  Claude-Henri Saint-Simon: EL ORGANIZADOR (1819);
\item
  Max Stirner: EL ÚNICO Y SU PROPIEDAD (1845).
\end{itemize}

De este conjunto de obras vamos a repasar los principales contenidos de una obra que nos parece muy expresiva de los valores de esta época, signada por la herencia (positiva y negativa) de la Revolución Francesa y las condiciones históricas de la Modernidad. No tan conocida ni difundida como los trabajos de Comte o de Tocqueville, tiene a nuestro juicio un alto valor representativo del espíritu de su tiempo: se trata de ``El Organizador'' de Saint-Simon.

Claude-Henri de Rouvroy, conde de Saint-Simon (1760-1825) nació en Paris, en el seno de una familia noble, de poderosa influencia en la Corte de Luis XVI, y presunto descendiente de Carlomagno. Recibió una educación sumaria y descuidada, y se inició pronto (1776) en la carrera de las armas. Combatió en América, como Lafayette, bajo las órdenes de Washington y retornó a Francia en 1783. En 1790 renunció a sus títulos nobiliarios, pese a lo cual a duras penas logró salvar su vida durante la Revolución Francesa. Después de encarar diversas empresas, desde 1800 en adelante se consagró al estudio de las ciencias, gastando en ello su patrimonio. A lo largo de muchos años publicó gran número de obras sobre temas muy diversos, que van desde la pedagogía hasta la historia, pasando por la industria, la economía y la política. Con el tiempo, y después de muchas viscicitudes personales y de los grupos que formó, algunos de sus seguidores crearon en torno a su figura y sus ideas una atmósfera francamente mesiánica. Murió en 1825.

Mensajero del futuro, exegeta revelador del sentido del pasado, Saint-Simon ha sido considerado fundador del que luego los marxistas llamarían ``socialismo utópico''. Como un apóstol ateo de una religión nueva, de inspiración newtoniana, invención puramente humana dotada de una doble función de agregación y de mediación, Saint-Simon quiso mostrar que el progreso de la humanidad no se había detenido con los cambios producidos por la Revolución Francesa y que aún los hombres tenían frente de sí todo un mundo nuevo que conquistar\footnote{Saint-Simon: OEUVRES, Paris, Anthropos, 1966.}.

Ese ``deseo de otra revolución'', expresado bajo formas utópicas o científicas, inspirado en el ``deseo de dicha para los hombres'' (expresión de resonancias milenaristas) condujo a Saint-Simon a preguntarse sobre las condiciones del ser-en-sociedad del hombre en el contexto de la Modernidad. Su respuesta plantea la necesidad de superar el egoísmo y lograr una real vinculación entre los hombres para realizar el gran objetivo social: la producción, es decir, ``la satisfacción de las necesidades de todos''. Sobre la base de un planteo de neta inspiración positivista (``la capacidad científica positiva -decía- debe reemplazar al poder espiritual'', y también: ``debe lograrse la preponderancia de las capacidades sobre los poderes'') encuentra la gran respuesta en la producción de bienes por el trabajo.

``La verdadera sociedad cristiana -sostenía- es aquella donde cada uno produce alguna cosa que les falta a otros\ldots El interés de la unión es el interés de las alegrías de la vida; el medio de unión es el trabajo''.

En definitiva, propuso una nueva organización de la Humanidad fundada sobre la industria. La industria reúne a la sociedad en torno a un fin común y a una identidad práctica. En ese modelo, la ``política positiva'' se vuelve ``ciencia de la producción''. El dominio de los hombres debe ser mínimo: una mínima función de policía; las relaciones entre los hombres deben ser primordialmente relaciones de coordinación. El ``deseo de dominación'' que los hombres experimentan debe ser encauzado, no sobre los otros hombres, sino sobre la Naturaleza, para producir los bienes que permitan ``mejorar la suerte de la última clase social y volver dichosos a todos los hombres''.

\hypertarget{los-socialismos-y-los-nacionalismos}{%
\subsection*{Los socialismos y los nacionalismos}\label{los-socialismos-y-los-nacionalismos}}
\addcontentsline{toc}{subsection}{Los socialismos y los nacionalismos}

El tercer momento (1849-1927) es caracterizado por J.J. Chevalier por la emergencia de los socialismos y los nacionalismos. En este libro, el marxismo y todos sus derivados está ampliamente tratado en otro capítulo (ver Cap. 4) por lo que no mencionaremos aquí las obras de esa corriente. Aparte de ellas, una lista de obras importantes de este período debe a nuestro juicio mencionar al menos las siguientes:

\begin{itemize}
\tightlist
\item
  Emile Chartier(Alain):ELEMENTOS DE UNA DOCTRINA RADICAL (1925);
\item
  Maurice Barrès: LOS DESARRAIGADOS (1897);
\item
  Charles Darwin: EL ORIGEN DE LAS ESPECIES (1859);
\item
  Adolf Hitler: MI LUCHA (1925)
\item
  Jean Jaurès: HISTORIA SOCIALISTA DE LA REVOLUCIÓN FRANCESA (1901-1904);
\item
  Gustave Le Bon: PSICOLOGÍA DE LAS MULTITUDES (1895);
\item
  Paul Leroy-Beaulieu: EL ESTADO MODERNO Y SUS FUNCIONES (1889);
\item
  Charles Maurras: ENCUESTA SOBRE LA MONARQUÍA (1900-1909) y KIEL Y TANGER (1910);
\item
  Pierre-Joseph Proudhon: DE LA CAPACIDAD POLÍTICA DE LAS CLASES OBRERAS (1865);
  -Georges Sorel: REFLEXIONES SOBRE LA VIOLENCIA (1908) y MATERIALES PARA UNA TEORÍA DEL PROLETARIADO (1919);
\item
  Oswald Spengler: LA DECADENCIA DE OCCIDENTE (1918-1919).
\end{itemize}

De este período, tan abundante en obras fundamentales y polémicas, hemos elegido, para describir sus contenidos, una que en cierta medida sintetiza las dos ideas principales emergentes en este período: el socialismo y el nacionalismo. Se trata de ``Los desarraigados'' de Maurice Barrès.

Maurice Barrès (1862-1923) fué un escritor francés, proveniente de una familia de la alta burguesía. Su obra literaria de juventud (``Le culte du moi'', ``Un homme libre'') lo muestran en una posición egoísta y hedonista, políticamente orientada hacia un soberbio aristocratismo. Luego evolucionó hacia una toma de conciencia socialmente solidaria, de tipo fuertemente nacionalista, que se expresa cabalmente en la trilogía que, bajo el título general de ``La Novela de la Energía Nacional'', contiene a ``Los Desarraigados'', ``El Llamado al Soldado'' y ``Sus Figuras''. También publicó una recopilación de ensayos bajo el título ``Escenas y Doctrinas del Nacionalismo''. Con el tiempo, su nacionalismo se hizo acentuadamente conservador, y adoptó actitudes antiparlamentarias y antisemitas, que lo llevaron, por ejemplo, a mantener una posición reaccionaria frente al ``caso Dreyfus''.

``Los Desarraigados'' es una novela. No se presenta en absoluto como una obra de doctrina política. Al presentar la vida de siete jóvenes llegados a Paris desde su Lorena natal, en la búsqueda de grandes destinos, con el telón de fondo de los hechos nacionales ocurridos desde 1880, la obra sobrepasa ampliamente la intención novelística y desarrolla una visión de la historia nacional, de los principios y valores consagrados, referidos al orden y al devenir social, y al individuo en su papel respecto de la comunidad donde nació\footnote{Zeev Sternhell: MAURICE BARRES ET LE NATIONALISME FRANCAIS, Paris, A. Colin, 1972.}.

Barrès escribe esta obra en un momento singular de la historia francesa, en el que parece cobrar realidad el sueño de un ``socialismo nacional'' antiparlamentario y anticapitalista a la vez; un momento en el que parece posible la alianza política de los últimos tradicionalistas y monárquicos con socialistas proudhonianos y antiguos anarquistas.

En ese ambiente, Barrès es también sensible a las influencias del darwinismo social y de doctrinas inspiradas en un determinismo casi ``fisiológico'', que lo llevan a considerar socialmente viable un patriotismo fervoroso, nutrido con conceptos racionales y una cierta moral social.

Barrès define al nacionalismo como ``la aceptación de un determinismo''. Es el reconocimiento de la gravitación del pasado sobre el presente, la sumisión a la ley sagrada de las filiaciones, la obediencia a las grandes voces ``de la tierra y de los muertos''. Para Barrès, la expansión del individuo está vinculada al mantenimiento de la ``sustancia nacional'' que asegura las condiciones sociales de su desarrollo individual.

El nacionalismo de Barrès presenta dos fases: una faz contestataria, plebeya y socializante, animada de un cierto ``romanticismo de la acción'' antiburguesa y anticonformista; la otra faz es conservadora: apoya la preservación de la comunidad nacional apelando a todas las fuerzas del orden y de la jerarquía social, de la educación, la religión y las armas. En cuanto a la relación con la historia, Barrès piensa que todas las adquisiciones del pasado deben ser tomadas en cuenta: que se deben aceptar las cosas ``tal como están''. De allí su aceptación de las ideas de la Revolución Francesa y de la tradición republicana\footnote{Chatelet, Duhamel y Pisier, op. cit.}.

\hypertarget{las-teoruxedas-actuales}{%
\subsection*{Las teorías actuales}\label{las-teoruxedas-actuales}}
\addcontentsline{toc}{subsection}{Las teorías actuales}

El cuarto momento (desde 1928 hasta la actualidad) presenta muy variadas líneas de pensamiento, y es difícil conferirles un rasgo característico. Quizás puedan señalarse dos ejes dominantes en la preocupación de muchas de estas teorías normativas:

\begin{itemize}
\tightlist
\item
  la libertad individual y grupal frente al poder estatal;
\item
  la democracia frente al totalitarismo.
\end{itemize}

La producción es vastísima, y son bastante borrosos los límites entre teorías normativas y teorías que reconocen fundamentos metodológicos empírico-analíticos o crítico-dialécticos. Toda nómina de obras sería incompleta o cuestionable. De todos modos, creemos que no pueden dejar de mencionarse las siguientes:

\begin{itemize}
\tightlist
\item
  Theodor W. Adorno: MINIMA MORALIA (1944-1947);
\item
  Hanna Arendt: LOS ORÍGENES DEL TOTALITARISMO (1951);
\item
  Raymond Aron: DEMOCRACIA Y TOTALITARISMO (1958) y PAZ Y GUERRA ENTRE LAS NACIONES (1962);
\item
  Walter Benjamin: TESIS SOBRE EL CONCEPTO DE HISTORIA (1940);
\item
  Alain de Benoist: DEMOCRACIA: EL PROBLEMA (1985);
\item
  Leon Blum: A ESCALA HUMANA (1945);
\item
  Albert Camus: EL HOMBRE REBELDE (1951);
\item
  Pierre Clastres: LA SOCIEDAD CONTRA EL ESTADO (1974);
\item
  Frantz Fanon: LOS CONDENADOS DE LA TIERRA (1961);
\item
  Michel Foucault: VIGILAR Y CASTIGAR (1975);
\item
  Bertrand de Jouvenel: EL PODER (1945);
\item
  Herbert Marcuse: EROS Y CIVILIZACIÓN (1953) y LA NOCIÓN DE PROGRESO A LA LUZ DEL PSICOANÁLISIS (1968);
\item
  Jacques Maritain: EL HOMBRE Y EL ESTADO (1953);
\item
  Maurice Merleau-Ponty: HUMANISMO Y TERROR (1947);
\item
  José Ortega y Gasset: LA REBELIÓN DE LAS MASAS (1930);
\item
  Alfred Rosenberg: EL MITO DEL SIGLO XX;
\item
  Jean-Paul Sartre: CRITICA DE LA RAZÓN DIALÉCTICA (1960);
\item
  Karl Schmitt: TEORÍA DE LA CONSTITUCIÓN (1928);
\item
  Erik Weil: FILOSOFÍA POLÍTICA (1956).
\end{itemize}

Esta larga lista no impresiona tanto como la cantidad de autores importantes que han quedado afuera, desde Voegelin, Eucken y Hattich hasta Spiro y Dante Germino, etc. Realmente, estas últimas seis décadas, que han sido las más fecundas en obras empíricas y crítico-dialécticas, también lo han sido en obras normativas.

Cómo hacer para pintar un panorama sin mutilaciones? Imposible. De todos modos, varios de estos autores han sido mencionados en otras partes de esta obra: así, T. Adorno (pág. 181), Hanna Arendt (pág. 341), W. Benjamin (pág. 181), H. Marcuse (pág. 182). De los restantes elegimos aquí dos para desarrollar algo sus contenidos, por entender que son representativos de las líneas dominantes en el pensamiento político normativo actual: Alain de Benoist y Bertrand de Jouvenel.

Alain de Benoist es un joven pensador de la Nueva Derecha francesa. Su libro ``Democracia: el problema'' parte de un análisis histórico de la democracia desde los griegos y los escandinavos. Hace una defensa crítica de la democracia y un análisis de la contradicción existente entre soberanía popular y pluralismo, para desembocar en una visión de la crisis actual de la democracia y en una propuesta de ``democracia orgánica'', construída no sobre el valor LIBERTAD (como las democracias liberales) ni sobre el valor IGUALDAD (como las democracias populares) sino sobre el valor FRATERNIDAD, se entiende que sin excluir a los otros valores\footnote{Alain de Benoist: DEMOCRATIE: LE PROBLEME, Paris, Le Labyrin- the, 1985.}.

La esencia del pensamiento de Alain de Benoist sobre la democracia parece estar expresada, a nuestro juicio, en las ``Diez Tesis'' que, a modo de postfacio cierran la obra: -``La mejor aproximación al concepto de democracia es la histórica: saber en primer lugar qué significaba la democracia para los que la inventaron. La libertad de las democracias antiguas es una libertad-participación, en la que el interés común y el conformismo priman sobre los intereses particulares. La principal diferencia entre las democracias antiguas y las modernas está en que las primeras ignoran el individualismo igualitario que fundamenta a las segundas''.

\begin{itemize}
\item
  ``Liberalismo y democracia no son sinónimos. La democracia es una''cracia``, un gobierno, un poder; el liberalismo es una ideología de la limitación de todo poder político''.
\item
  ``La democracia no es antagonista de la idea de un poder fuerte, o de las nociones de autoridad, selección o élite. La regla de la mayoría no está destinada a decir la verdad; es sólo un medio para elegir entre posibles''.
\item
  ``La idoneidad política para gobernar no está en relación con el saber técnico o científico sino con la capacidad de decisión. El `gobierno de los expertos' generalmente produce resultados catastróficos''.
\item
  ``Los derechos políticos no derivan de `derechos inalienables de la persona humana' sino de la condición de ciudadano. El principio democrático fundamental es: un ciudadano, un voto''.
\item
  ``La noción clave del régimen democrático es la de participación. Es la participación del pueblo en las instituciones la que hace la democracia. El máximo de democracia es el máximo de participación''.
\item
  ``Se recurre al principio de mayoría porque el principio de unanimidad (supuesto teórico de la `voluntad general') es irrealizable. La mayoría es una técnica que permite reconocer el valor de la minoría (que puede ser mayoría mañana). El pluralismo tiene su límite en el bien común''.
\item
  ``Las actuales democracias, que son poliarquías electivas, son una decadencia del ideal democrático, corrompido por la prepotencia del dinero y el efecto de la masa. La información está condicionada y estandarizada, la opinión está formada por factores heterónomos, los programas y los discursos políticos tienden a hacerse homogéneos, lo que hace indistintas las opciones. El resultado es la apatía política, que se opone a la participación y, por lo tanto, a la democracia''.
\item
  ``La calidad de ciudadano no se agota en el acto de votar. Hay que explorar posibilidades que vinculen más directamente al pueblo con sus gobernantes y extiendan la participación. Una democracia orgánica puede desarrollarse en torno a la idea de fraternidad''.
\item
  ``La democracia es el poder del pueblo; donde no hay pueblo no puede haber democracia. Todo sistema que debilite la conciencia de pertenencia a esa entidad orgánica que es el pueblo, debe ser considerado como un sistema no democrático''.
\end{itemize}

Bertrand de Jouvenel (n.~1903 ) es un economista y ensayista francés, cuyos análisis se refieren principalmente a los orígenes y consecuencias del progreso tecnológico, investigando si nuestras sociedades hacen o no el mejor uso posible del aumento de volumen de consumo resultante de ese progreso. También ha incursionado con mucha penetración y ágil manejo de una gran erudición histórica, en el campo de la reflexión política normativa. Sus principales obras son: ``La Economía Dirigida'' (1929), ``La crisis del capitalismo americano'' (1933), ``El Poder'' (1945), ``Etica de la Redistribución'' (1955), ``Arcadia, Ensayo sobre el Vivir Mejor'' (1968), ``Teoría Pura de la Política'' (1963).

De esta amplia producción, vamos a ver con algún detalle los contenidos principales de ``El Poder''\footnote{Bertrand de Jouvenel: EL PODER, Madrid, Ed. Nacional, 1974, segunda edición.}. Esta obra tiene como contenido principal la lucha entre el poder y la libertad individual, que se disputan el predominio del espacio político. La conclusión es pesimista para la libertad individual. Según Bertrand de Jouvenel hay dos tipos de libertad: la libertad-participación, que es la posibilidad que tiene el ciudadano de participar en los órganos del poder y de contribuir a tomar decisiones, y la libertad-resistencia, que es la posibilidad de reservarse una zona de actuación al margen de la intervención estatal. Este último tipo de libertad es el que de Jouvenel valora más porque lo considera una auténtica manifestación de la libertad política.

Los hombres se clasifican -según de Jouvenel- en securitarios, que son la mayoría que busca seguridad antes que nada y está dispuesta a pagarla con libertad, y libertarios, que son una minoría, los pocos que conquistan y defienden su autonomía y asumen los riesgos de su libertad.

Esa libertad es frágil. Requiere muchas condiciones que rara vez se dan juntas: una minoría respaldada por una masa; una élite dotada de alto sentido moral: autodisciplina, función social asumida y reconocida; un cierto equilibrio de fortunas, que haga tolerable la situación de los inferiores. Los hombres libres son aristócratas. Los hombres comunes no son libres.

La democracia -sostiene de Jouvenel- no es respetuosa de las libertades individuales. Tiende a invadir el terreno de las libertades con el respaldo del apoyo popular. En la sociedad contemporánea no hay verdadera libertad: no hay élites libertarias; sólo hay una aristocracia sin honor que rehuye el riesgo y la responsabilidad. El poder ha crecido de un modo indiscriminado en todas las sociedades modernas, cualquiera sea su régimen político.

Bertrand de Jouvenel pretende ser objetivo, y sin duda es sincero, pero sus análisis están impregnados de juicios de valor muy subjetivos. Tiene una abierta simpatía por los regímenes aristocráticos, en los que una minoría, apoyada por la masa, limita el crecimiento del poder. Simpatiza con la libertad individual, entendida como señorío inmediato sobre sus actos (los comportamientos del ``viejo aristócrata'') y desprecia en forma mal disimulada al burgués que lo reemplazó.

Bertrand de Jouvenel casi no le da importancia a la libertad-participación, obsesionado como está por el señorío inmediato del hombre sobre sí mismo; y pasa por alto que las libertades-participación son la condición básica para el mantenimiento de las libertades-resistencia, salvo para una ínfima minoría de personas. Su planteo es anti-comunitario. A nivel general de la sociedad, las libertades-resistencia han de realizarse (si vamos a considerar viable esa posibilidad) sin mengua para la sobrevivencia y bienestar de los grupos. El individuo común, encuadrado en el mejor de los casos en organizaciones productivas, puede ser algo más libre si se atenúan los controles que pesan sobre él y se incrementa su participación, responsabilidad e iniciativa. Pero de Jouvenel considera que la máxima posibilidad de incrementar la libertad está en la automación de los procesos productivos y el aumento del tiempo libre. Esa libertad sería individual y no comunitaria, y en la actual organización conduciría, no al ocio fecundo sino al envilecimiento del desempleo. Creemos que la principal crítica que puede hacerse a Bertrand de Jouvenel es que su condición de liberal elitista lo lleva a considerar como ideal sólo al modelo de la sociedad aristocrática, sin preguntarse si existirán o no otras posibilidades de realizar la libertad-resistencia de un modo más igualitario.

Luis García San Miguel, al prologar la edición castellana de ``El Poder'' plantea en este sentido la posibilidad de adoptar un modelo de ``sociedad autogestionada'' , que mantiene al Estado constreñido a un rol mínimo indispensable porque las empresas y las organizaciones intermedias de la sociedad disponen de amplia autonomía frente al Estado y son controladas por los que trabajan en ellas en un régimen de democracia directa.

En un modelo así, las competencias estatales quedarían reducidas al mínimo necesario para mantener la cohesión del conjunto social, mientras la mayoría de las funciones sociales serían desempeñadas por la sociedad misma. Una sociedad así -sostiene García San Miguel- podría realizar una buena combinación dialéctica del ideal socialista de la IGUALDAD, el ideal democrático de la PARTICIPACIÓN, el ideal liberal de la LIBERTAD INDIVIDUAL y del ideal anarquista de la REDUCCIÓN DEL PODER ESTATAL al mínimo indispensable.

Curiosamente, Bertrand de Jouvenel escribía estas cosas cuando la marcha del mundo parecía orientarse, según la visión predominante en aquellos años, hacia formas socializantes, de incremento de la intervención estatal, no sólo en los países del ``socialismo real'' sino en los países democráticos occidentales, los países del ``welfare state'', de la ``svolta a sinistra'' etc.

Nada parecía anunciar concretamente, en aquellos años, la emergencia del neo-conservadorismo o neo-liberalismo, que hemos vivido recientemente, con sus demandas de ``Estado mínimo'', libertad a la iniciativa individual de la opresiva protección estatal, privatización de los servicios públicos, etc.; corrientes que cambiaron el mapa político de Occidente y derrumbaron los regímenes del socialismo real. Esas corrientes son formas políticas, prácticas e ideológicas, evidentemente afines en muchos aspectos al pensamiento de Bertrand de Jouvenel, aunque cabe mencionar que su aristocrático individualismo tiene un sesgo de nobleza que no se confunde con el crudo pragmatismo crematístico que hoy cunde por doquier.

\hypertarget{cuarta-parte}{%
\section*{Cuarta parte}\label{cuarta-parte}}
\addcontentsline{toc}{section}{Cuarta parte}

\hypertarget{enfoques-metodoluxf3gicos-usuales}{%
\subsection*{Enfoques metodológicos usuales}\label{enfoques-metodoluxf3gicos-usuales}}
\addcontentsline{toc}{subsection}{Enfoques metodológicos usuales}

La extensión temporal (no menos de 2500 años) y espacial (desde China hasta América, pasando por Europa) de la producción politológica normativa, torna imposible todo intento de sistematización detallada del tema metodológico, que aparece además en este caso notablemente ``personalizado'' en cada autor. Es posible, en cambio, dar algunas ideas o pautas generales sobre los criterios metodológicos más frecuentes.

Ya dijimos que las teorías políticas normativas se ubican en un ámbito de fuerte vocación filosófica, en un área intermedia entre la Ciencia Política y la Filosofía Política. De la primera conservan el fuerte impulso de ``entender'' y de ``comprender'' la realidad basándose en ella misma, vale decir, en el contenido empírico de las observaciones. De la segunda conservan la vocación de conceptualización omniabarcativa y de evaluación axiológica en términos perdurables.

En el terreno puramente metodológico, esa doble vertiente también se hace sentir. Hay observación sistemática y acumulación y procesamiento de datos empíricos, y hay también análisis racional y deductivo. Cabe destacar el frecuente uso del método filosófico dialéctico, en las distintas formas en que fue empleado por Platón, Aristóteles, Santo Tomás, Hegel, Marx\ldots{}

También es digno de destacar el frecuente empleo del método histórico. La Historia es una gran ``proveedora de materiales'' para la Ciencia Política en general, y para las teorías normativas en particular. No es casual que muchos teóricos de esta corriente sean eminentes historiadores, o al menos personas de reconocida versación histórica, y que el principal aporte de esta corriente al ``corpus'' politológico esté en la Historia de las Ideas Políticas.

En los escritos de autores normativos es frecuente el empleo de analogías y metáforas. Este recurso tiene un interesante valor pedagógico, pero el excesivo empleo del método analógico, y sobre todo el impulso de llevar la analogía más lejos de lo prudente, es indudablemente riesgoso desde el punto de vista epistemológico, y constituye quizás uno de los puntos más débiles de estas teorías.

Es frecuente en esta corriente teórica el uso de un tratamiento metodológico similar al utilizado en Derecho, Terapéutica y Educación, es decir, en ciencias prácticas, que parten del planteo de problemas individualizados para tratar de resolverlos apelando a principios generales y a antecedentes (como la jurisprudencia).

Los autores normativistas suelen ser partidarios de estudios casuísticos y de monografías prescriptivas. Algunos emplean el método tópico, que parte de la consideración de problemas particulares, evaluados con criterios de comprensión, para remontarse a la enunciación de principios o ideas generales.

Frecuentemente, los autores normativistas contemporáneos recurren a la teoría política clásica (Aristóteles, sobre todo) en busca de fundamentación para sus conclusiones actuales. Son, por otra parte, partidarios de la ``política pura'' y se oponen por lo general a todo reduccionismo de la política a otras variables (clases, modos de producción, factores geográficos, etc.).

Por último, cabe mencionar dos preocupaciones frecuentes en estos autores: por una parte, el valor de la Ciencia Política como fuente de educación política, les hace incluir en sus presentaciones diversas variantes del método pedagógico. Por otra parte, la conciencia del valor de la Ciencia Política para la administración de los bienes públicos los lleva con frecuencia a descuidar otros temas, como el de la participación política y el de la movilización social, lo que favorece la adopción de un pragmatismo metodológico.

\hypertarget{Lasteoruxedasempuxedricoanaluxedticas}{%
\chapter{Las teorías empírico-analíticas}\label{Lasteoruxedasempuxedricoanaluxedticas}}

Primera parte:

\begin{itemize}
\tightlist
\item
  Rasgos generales: El positivismo, el empirismo y sus derivados.
\item
  El objeto y el método.
\item
  Problemas actuales.
\end{itemize}

Segunda parte:

\begin{itemize}
\tightlist
\item
  Behaviorismo, estructural-funcionalismo y enfoque sistémico.
\item
  El enfoque comparatista: Descripción de los enfoques.
\item
  Síntesis de obras teóricas de estas corrientes.
\end{itemize}

Tercera parte:

\begin{itemize}
\tightlist
\item
  Las explicaciones de base psicológica individual: La Psicología del estímulo-respuesta.
\item
  La psicología de la Gestalt.
\item
  La teoría del campo.
\item
  El freudismo ortodoxo.
\item
  El neofreudismo.
\end{itemize}

Cuarta parte:

\begin{itemize}
\tightlist
\item
  El formalismo.
\item
  La teoría de los juegos.
\item
  La teoría de la información y la cibernética.
\item
  Modelos y simulaciones.
\end{itemize}

Quinta parte:

\begin{itemize}
\tightlist
\item
  Enfoques metodológicos usuales: Puntos en común.
\item
  Particularidades metodológicas.
\item
  Reflexiones sobre el lenguaje y la elaboración conceptual.
\end{itemize}

\hypertarget{primera-parte-2}{%
\section*{Primera parte}\label{primera-parte-2}}
\addcontentsline{toc}{section}{Primera parte}

\hypertarget{rasgos-generales-1}{%
\subsection*{Rasgos generales}\label{rasgos-generales-1}}
\addcontentsline{toc}{subsection}{Rasgos generales}

Las teorías empírico-analíticas también suelen ser llamadas ``teorías deductivo-empíricas'' o ``empírico-general-inductivas''. Se basan en distintas variedades de la lógica científica neo-positivista. No hay en ella un acuerdo completo sobre los alcances posibles de una ``teoría'' fuera de su carácter sistemático: que permita describir, explicar y predecir sucesos mediante deducciones formales no contradictorias. Algunos autores, como Talcott Parsons, sostienen la posibilidad y conveniencia de construir teorías generales. Otros, como Robert Merton, sólo consideran viables (al menos, por ahora) las teorías de alcance medio.

En años recientes hemos visto una notable declinación de las pretensiones predictivas de las teorías: muchos autores actuales prefieren limitarse a describir y explicar, dejando al futuro en las brumas de su misterio. Vemos en ésto una influencia de esa ``cultura de la incertidumbre'' que caracteriza al posmodernismo y también una consecuencia de esa lección de modestia que entrañan tantos hechos recientes que nadie previó con suficiente anticipación, desde la derrota de los EE.UU. en Vietnam y de la URSS en Afganistán, la caída del Sha de Irán y la emergencia de fundamentalismos religiosos, hasta la caída del muro de Berlín y de los ``socialismos reales'' en la Europa del Este.

Para comenzar, recordemos brevemente qué significa la orientación científica neo-positivista. La base fue dada por el positivismo del siglo XIX, al que inevitablemente se asocia el nombre de Auguste Comte. El POSITIVISMO puede ser sintéticamente expresado en los siguientes enunciados: - el único objeto del conocimiento es lo dado (``positum'') en la experiencia; - no hay otra realidad que los hechos y las relaciones entre hechos; - no hay que buscar respuesta al qué, porqué y para qué de las cosas, sino únicamente al cómo; - no tiene validez alguna la metafísica, ni el conocimiento a priori, ni la intuición de lo inteligible; - se rechaza todo ``sistema'' filosófico; - la filosofía es sólo el conjunto ordenado de los datos que suministran las ciencias.

Por su parte, el EMPIRISMO (Hume) considera que la única fuente del conocimiento es la experiencia. Recusa todo innatismo: el hombre sólo elabora un conocimiento después de haber estado en contacto con la realidad sensible, y lo hace con elementos que ella le aporta.

El EMPÍRICO-CRITICISMO (Avenarius) fundamenta en la crítica sistemática de la experiencia pura la posibilidad de eliminar los planteamientos de tipo metafísico y los apriori del conocimiento, para lograr una representación neutral del mundo.

EMPIRISMO CIENTÍFICO es ante todo el nombre de una característica metodológica propia de todas las corrientes científicas derivadas o afines al positivismo lógico, que se proponen la unificación de la ciencia. Este enfoque se centra en el concepto de VERIFICABILIDAD, básico para la aceptación de una proposición en cualquier campo del saber. Recordemos que el principio de verificabilidad (Ayer) consiste en ``saber qué observaciones conducirían bajo ciertas condiciones a aceptar una proposición como verdadera o rechazarla como falsa''.

Entre las principales características del NEOPOSITIVISMO podemos mencionar las siguientes: - el único conocimiento digno de tal nombre es el que las ciencias empíricas tienen de sus objetos; - la filosofía no es un saber sobre cosas, sino una actividad crítica del conocimiento positivo y del lenguaje en que éste se formula; - tiene gran importancia la verificación formal (lógica) y el análisis del lenguaje.

El NEOPOSITIVISMO CRITICO (Popper) sostiene que nunca es posible verificar la verdad de un enunciado inductivo por vía empírica; lo que sí puede hacerse es intentar falsarlo: mientras no se lo logre, mientras la afirmación se mantenga en pié, se la acepta como verdadera.

Este conjunto de rasgos que acabamos de repasar ha sido anotado aquí más que nada para dar cuenta del ``ambiente intelectual'' en el que se han desarrollado las teorías empírico-analíticas.

La óptica neopositivista, tal como ha sido definida por Karl Popper, parte de considerar que nuestra ignorancia es muy grande. La ciencia nace en ese contexto, al plantear problemas. Para que haya problemas tiene que haber desconocimiento, pero al mismo tiempo ``no es posible reconocer los problemas sin un cierto grado de conocimiento''\footnote{Klaus von Beyme: TEORIAS POLITICAS CONTEMPORANEAS, Madrid, Instituto de Estudios Políticos, 1977.}.

La tesis principal de Popper es que el METODO, tanto en las ciencias naturales como en las sociales, consiste esencialmente en experimentar y criticar soluciones a los problemas. No hay verificación alguna posible: los ensayos de solución son criticados, o sea se intenta refutarlos y se los acepta mientras se mantienen en pié; en caso contrario se los reemplaza por otros. Esta actitud básica -llamada falsacionismo- no es aceptada por todos los científicos empírico-analíticos, que plantean la objeción de que resulta muy frustrante construir una ciencia en permanente derrumbe. También plantean el problema que presentan las teorías generales, que muchas veces escapan a la falsación empírica, no por ser verdaderas sino por su elevado nivel de abstracción.

El neopositivismo crítico de Popper sostiene que no hay una materia específica como especialidad de cada ciencia, sino que ``cada disciplina es un conglomerado estructurado de problemas''. Según el planteo de los neopositivistas, la demarcación de los límites entre ciencias está aún por resolver. Se advierte, desde luego, una mayor flexibilización de los límites (por ejemplo, entre Sociología, Psicología, Economía y Ciencia Política) y el correspondiente auge de los estudios interdisciplinarios.

La objetividad de la ciencia, según el planteo neopositivista, no depende de la objetividad individual de cada científico sino del hábito generalizado de ofrecer las teorías a la crítica abierta del mundo científico. Por otra parte, frente al relativismo y al historicismo, el neopositivismo crítico reivindica la noción de VERDAD, no sólo en el sentido de verdad histórica sino también en sentido absoluto, en su aspecto lógico-formal. ``Las leyes de la Lógica rigen independientemente de la época histórica'', dice von Beyme.

El neopositivismo crítico evalúa las teorías con un criterio pragmático: una teoría es más válida que otra si es más eficaz, si sus conceptos son más aplicables a la investigación empírica y sobre todo si es técnicamente aplicable en el ámbito social.

El neopositivismo crítico ha sido a su vez criticado. Se ha dicho, por ejemplo, que sus teorías extraen de la realidad sólo aquellos datos que avalan las hipótesis previamente proyectadas; y que ``la exposición, el pronóstico y la proyección de la teoría positivista'' se convierten ``en correa de trasmisión del conocimiento científico y tecnológico en el mundo de artículos de consumo en la civilización industrial'' (von Beyme).

En lo que específicamente se refiere a la Ciencia Política, se dice que la orientación neopositivista degrada a la Ciencia Política a la condición de una ``simple ciencia auxiliar de la administración racional'', vinculada sobre todo con los procesos de toma de decisión. Por el contrario, otros teóricos, como Lehmbruch, reivindican el valor de la orientación neopositivista en Ciencia Política, porque produce una clarificación crítica que suprime prejuicios, y porque permite la ``formulación de pronósticos en forma de hipótesis condicionales que se convierten en el fundamento de una tecnología social prospectiva'', dice von Beyme.

El neopositivismo, en todas sus variantes, siempre ha sostenido la importancia de evitar que la actividad científica se mezcle con la política práctica. Los principales problemas que enfrenta actualmente el neopositivismo crítico son los siguientes: 1) Cómo separar claramente los juicios científicos sobre el ser de los fenómenos, de los juicios normativos sobre el deber ser de los mismos, especialmente en las tareas de asesoramiento político.

\begin{enumerate}
\def\labelenumi{\arabic{enumi})}
\setcounter{enumi}{1}
\item
  Cómo colmar el abismo que separa ``la pura teoría científica'' del ``empirismo descriptivo de la labor científica cotidiana''. En otros términos, cómo cubrir la distancia entre la aspiración a una teoría general omnicomprensiva (que es considerada ``utópica'' por algunos autores, como por ejemplo, Robert Merton) de las ``teorías de alcance medio'', que son las únicas consideradas como realizables actualmente.
\item
  Cómo hacer más operativos los conceptos de la teoría en temas concretos, vinculados con la realidad política; y cómo comunicar más adecuadamente los resultados obtenidos.
\item
  Cómo evitar que el ``consenso científico elitista'' establezca teorías-doctrinas, o sea teorías dominantes, que hagan más difícil su propio cuestionamiento o revisión crítica por vía de la falsación, esgrimiendo tácitamente un anticientífico ``principio de autoridad''.
\item
  Cómo establecer una separación no esquemática y útil entre teoría e ideología, sobre todo teniendo en cuenta que ese vínculo es riesgoso pero a la vez fecundo en interesantes hipótesis, y que las ideologías y utopías suelen no carecer de contenidos empíricos y de observaciones descriptivas.
\end{enumerate}

Bajo el título general de ``teorías empírico-analíticas'' vamos a presentar con cierto detalle las siguientes corrientes teóricas: - Behaviorismo o conductismo; - Estructural-funcionalismo; - Enfoque sistémico; - Enfoque comparatista; - Explicaciones de base psicológica: estímulo/respuesta; gestalt; teoría del campo; dinámica de grupos; freudismo ortodoxo; neofreudismo.

\begin{itemize}
\tightlist
\item
  Formalismo: teoría de los juegos; teoría de la información y la cibernética; modelos y simulaciones.
\end{itemize}

En pocas palabras, se trata -más allá de lo discutible que puedan resultar algunas inclusiones- de dar un panorama lo más completo posible de las corrientes teóricas de raíz empírica vigentes en los países occidentales.

\hypertarget{segunda-parte-2}{%
\section*{Segunda parte}\label{segunda-parte-2}}
\addcontentsline{toc}{section}{Segunda parte}

\hypertarget{behaviorismo-estructural-funcionalismo-y-enfoque-sistuxe9mico}{%
\subsection*{Behaviorismo, estructural-funcionalismo y enfoque Sistémico}\label{behaviorismo-estructural-funcionalismo-y-enfoque-sistuxe9mico}}
\addcontentsline{toc}{subsection}{Behaviorismo, estructural-funcionalismo y enfoque Sistémico}

\hypertarget{el-enfoque-comparatista}{%
\subsubsection*{El enfoque comparatista}\label{el-enfoque-comparatista}}
\addcontentsline{toc}{subsubsection}{El enfoque comparatista}

Estos enfoques guardan entre sí estrechas relaciones de continuidad y de conflicto y hasta expresan, con frecuencia, momentos evolutivos o facetas en la labor de los mismos autores. Con todas sus complejas variantes configuran el esquema conceptual y metodológico predominante en la Ciencia Política actual, si bien ya afectado por la crisis de paradigma a que se enfrentan las ciencias sociales en los últimos años, y que a nuestro entender alcanza a todos los enfoques conocidos.

Aún así, en medio de muchos problemas no resueltos, quedan en pié sus innegables virtudes: flexibilidad, elevada abstracción, capacidad para operar con fenómenos micro, meso, macro y mega-políticos, además de capacidad para incorporar explicaciones provenientes de otros enfoques teóricos, etc.

\hypertarget{el-behaviorismo}{%
\subsubsection*{El behaviorismo}\label{el-behaviorismo}}
\addcontentsline{toc}{subsubsection}{El behaviorismo}

Este vocablo es de orígen anglosajón:``behavior'' = comportamiento. Es una corriente o escuela científica, originada en los EE.UU. y luego relativamente difundida en Europa y en el resto del mundo. Postula el estudio rigurosamente empírico del hombre, mediante la observación directa de su comportamiento, entendido -al decir de Skinner\footnote{B. F. Skinner: SCIENCE AND HUMAN BEHAVIOR, New York, Free Press, 1953.}- como ``una característica primaria de las cosas vivas'' que actúa como ``variable dependiente'' respecto de las ``condiciones externas, de las cuales el comportamiento es una función''.

Estas ``relaciones causa-efecto en el comportamiento son las leyes de una ciencia\ldots expresadas en términos cuantitativos'', dice Skinner. La máxima aspiración del behaviorismo es equiparar a las ciencias del hombre con las ciencias de la naturaleza, en las que el sujeto y el objeto de la investigación no se confunden entre sí.

En la aparición del behaviorismo en la Ciencia Política puede reconocerse la influencia de psicólogos como E.L. Thorndike y J.P. Watson. Sus manifestaciones explícitas más tempranas pueden hallarse en Charles Merriam y su ``Escuela de Chicago'', de la que surgieron, antes de la segunda guerra mundial, algunos científicos políticos sobresalientes como Gabriel Almond, Harold Lasswell, Herbert Simon y David Truman.

Charles Merriam (1874-1953) nació en Iowa. Se doctoró en Columbia, y luego en Leyes por la Universidad de Colorado. Fue profesor de Ciencia Política en la Universidad de Chicago desde 1911. Entre sus numerosos libros cabe citar: ``The American Party System'' (1922); ``News Aspects of Politics'' (1925); ``The Making of Citizens'' (1931); ``Political Power'' (1934); ``What is Democracy'' (1941) y ``Systematic Politics'' (1945).

El enfoque behaviorista apareció como una propuesta renovadora frente a la por entonces predominante escuela legalista o institucionalista, que ya era cuestionada por muchos investigadores debido a su desinterés o falta de capacidad para explicar los numerosísimos fenómenos políticos no-institucionalizados pero de innegable interés y trascendencia.

Quizás el principal hito de esta transición pueda ubicarse en un trabajo de Charles Merriam titulado ``The Present State of the Study of Politics'' (1921), que marca el pasaje desde el punto de vista institucional, de raigambre jurídica, hacia el punto de vista comportamental, de raigambre socio-psicológica, en el estudio de la política: el objeto a estudiar sería, en adelante, ``el comportamiento de individuos y grupos que actúan políticamente''. En forma congruente se produjo, en el plano metodológico, el pasaje desde el uso casi exclusivo de documentos de archivo, históricos, hacia el empleo de la observación, mediante técnicas psico-sociológicas como el sondeo, la encuesta o la entrevista.

El concepto central del behaviorismo es, desde luego, el de ``conducta política''. Apareció por primera vez en el título de un libro en 1928\footnote{F. Kent: POLITICAL BEHAVIOR, THE HERETOFORE UNWRITTEN LAWS, CUSTOMS AND PRINCIPLES OF POLITICS AND PRACTICE IN THE UNITES STATES (1928).}. Un factor promocional del behaviorismo fueron los nuevos problemas prácticos que tuvo que encarar la administración federal de los EE.UU. para su propia racionalización y para llevar adelante los programas de ayuda técnica y económica característicos del ``new deal''.

El predominio del behaviorismo en la Ciencia Política norteamericana se alcanzó en la inmediata posguerra de la segunda guerra mundial. En 1945 se creó el ``Commitee on Political Behavior'' en el seno del ``Social Science Research Council''. En 1950, un behaviorista, Peter Odegard, alcanzó la presidencia de la ``American Political Science Association''. En la década de los sesenta, siete de los diez politólogos americanos más famosos eran declaradamente behavioristas: V.O. Key, D. Truman, R. Dahl, H. Lasswell, H. Simon, G. Almond y D. Easton.

En la práctica, ``behaviorismo'' es un nombre genérico para una gran cantidad de enfoques bastante heterogéneos. Una expresión humorística lo compara con un paraguas bien grande, que ofrece cobijo temporal a un grupo dispar, cuyo único punto en común es su descontento respecto de la Ciencia Política tradicional\ldots No obstante, tienen una cantidad de rasgos y creeencias similares\footnote{Klaus von Beyme, op. cit., pg. 137 y ss.}: - que la Ciencia Política puede ofrecer explicaciones y predicciones en forma semejante a las ciencias naturales (aunque más bien del tipo ``biología'' que del tipo ``físico-química''); y ofrecer también análisis sistemáticos elaborados en base a teorías a experimentar; - que el límite del estudio científico de la política está en los fenómenos observables; - que las instituciones son ``conductas sociales estancadas'' y que el efecto político de las instituciones no puede analizarse por el estudio de la conducta en las instituciones; - que los datos deben ser cuantificados al máximo posible; - que en la opción investigación pura-investigación aplicada se debe elegir decididamente ésta última, apuntando a la solución de problemas políticos concretos y a la innovación de los programas de acción política; - que la valoración no debe ser considerada como parte de la actividad científica: no se puede demostrar científicamente la veracidad o falsedad de los valores; - que la Ciencia Política debe ser interdisciplinaria. Algunos teóricos han llevado esta posición al extremo de negar su carácter de ciencia autónoma y tienden a subsumirla en la Sociología.

El behaviorismo politológico originario utilizaba un paradigma proveniente de la Psicología: S-R (estímulo-respuesta). Al combinarse el behaviorismo con la Teoría de los Sistemas, como por ejemplo ocurrió en la obra de David Easton, se adoptó un paradigma más complejo: S-O-R (estímulo-organismo-respuesta) y se empezaron a tomar en consideración aspectos subjetivos tales como sentimientos y motivaciones, y finalmente la cultura.

El behaviorismo, desde sus orígenes, evolucionó en muchos aspectos, y si bien los primeros estudios partían del individuo como unidad de análisis, los posteriores (sin descuidar al individuo) emplean también conceptos como rol, grupo, institución, organización, cultura, sistema.

El behaviorismo ha sido criticado, sobre todo, por algunas excesivas pretensiones suyas respecto del alcance de sus esquemas explicativos, pero conserva siempre su valor como método descriptivo, en especial en todo lo referente a las interacciones que el sujeto en estudio mantiene con el medio que lo rodea. Las evoluciones posteriores a su aparición lo vincularon con el gestaltismo y con el enfoque sistémico; o configuraron un ``neo-behaviorismo'' como el de E. Tolman, al que nos referiremos ahora porque es un enfoque de raíz psicológica que tiene mucho interés para la Ciencia Política.

Edward C. Tolman, en su obra ``Purposive behavior in animals and men'' (1934) planteó un concepto de ORGANISMO como ente que persigue fines y procura evitar consecuencias negativas para sí o para sus propios fines. Su behaviorismo finalista (``purposive behaviorism'') afirma que los organismos tienen la capacidad de trazar ``mapas cognitivos'' que resumen su experiencia y que pueden ser usados para perseguir o eludir algunos objetivos.

El organismo, según Tolman, interpreta sus percepciones en forma de un complejo total (``Gestalt'') de experiencia, que incluye sus recuerdos y que produce un conjunto de espectativas sobre los medios a usar para conseguir determinados fines. Estas ideas fueron trasladadas desde el ámbito psicológico individual al ámbito social y político y usados en la explicación de los procesos teleológicos del aprendizaje político.

\hypertarget{el-estructuralismo-y-el-funcionalismo}{%
\subsection*{El Estructuralismo y el funcionalismo}\label{el-estructuralismo-y-el-funcionalismo}}
\addcontentsline{toc}{subsection}{El Estructuralismo y el funcionalismo}

El Estructuralismo es una compleja corriente de pensamiento, de orígen europeo, a cuya génesis se suele asociar los nombres de Alfred Reginald Radcliffe-Brown (1881-1955) y de Claude Lévi-Strauss (n.~1908 ). En términos muy generales, lo primero que cabe decir es que el estructuralismo no se reduce a la utilización de la noción de ESTRUCTURA, harto difundida en las ciencias sociales por parte de todos los enfoques teóricos.

Radcliffe-Brown fue un sociólogo y etnólogo inglés, profesor en Chicago y en Oxford, que investigó con un método comparativo los ``principios estructurales'' de las relaciones humanas. También se lo asocia con los orígenes del Funcionalismo, por lo que luego lo volveremos a nombrar. En tiempos recientes se le ha criticado por encontrar en sus desarrollos cierta confusión entre modelo y realidad, así como por cierta reducción de la noción de estructura a una mera articulación de elementos empíricos.

Claude Lévi-Strauss, antropólogo francés (en realidad, de orígen belga), profesor del ``College de France'', es autor, entre muchas otras obras, de ``Anthropologie Structurelle'', cuya lección vamos a seguir para tratar de aclarar qué es el estructuralismo.

Según Lévi-Strauss, el orígen del análisis estructural está en la ``revolución lingüística'' (Saussure-Troubetskoy), que más allá de una transferencia de métodos de investigación desde el campo del lenguaje hacia el campo de la sociedad, llegó a afirmar que todos los fenómenos sociales -incluso los políticos, por supuesto- son ``también'' fenómenos lingüísticos.

En el enfoque de Lévi-Strauss, no se trata de aplicar una hermenéutica que devele ``el sentido oculto del texto explícito'' sino de ver a los ``fenómenos de sentido'' como manifestaciones de un juego estructural cuya explicación hay que buscar en un nivel distinto del empíricamente percibido. En palabras más simples, no es cuestión de buscar un código que ``traduzca'' lo que un elemento significa y explique cuál es su sentido más allá de su apariencia externa, sino de comprender que ese sentido es conferido por un ``juego estructural'', vale decir, por las relaciones del elemento con otros en el interior de una estructura, y por los factores definidores de tales relaciones.

Dice Lévi-Strauss que ``la revolución fonológica consiste en el descubrimiento de que el sentido resulta siempre de la combinación de elementos que no son de por sí significantes. En mi perspectiva, el sentido no es nunca un fenómeno primario''. En definitiva, el orden estructural, productor de sentido, es una ``sintaxis''. En principio nosotros la conocemos sólo por sus productos, sus efectos; y la definimos luego por las relaciones que vinculan entre sí a los elementos y les confieren un ``valor de posición'' similar al que adquieren los fonemas de la Lingüística en el contexto de una frase.

Esos elementos y sus relaciones -dice Althusser en ``Pour Marx''- determinan ``los lugares y las funciones desempeñadas por los seres y los objetos reales. Los verdaderos sujetos de la investigación no son, entonces, los ocupantes de esos lugares o los funcionarios de tales funciones, sino los definidores y distribuidores de esos lugares y funciones''. Esas relaciones, por ser tales, ``no se pueden pensar como sujetos'' y son ``irreductibles a toda intersubjetividad antropológica''.

Esta visión tiene profundas consecuencias en cuanto al modo de investigar lo social y particularmente lo político, ya que pone el acento en el caracter significante y productor de sentido de los vínculos relacionales y de sus valores posicionales emergentes, orientando en definitiva la investigación hacia los factores ``definidores y distribuidores'' de tales relaciones, superando así el clásico enfoque centrado en las designaciones y roles formales de los entes institucionales, o en los desempeños personales.

El estructuralismo -tal como Lévi-Strauss lo entiende- no acepta que pueda realizarse una integración totalizadora de los diversos niveles estructurales de una realidad compleja, ni procediendo por homología estructural ni definiendo una estructura como causa y a las otras como efecto. En ésto el estructuralismo difiere notablemente del marxismo clásico. Dice Lévi-Strauss, por ejemplo, que ``toda cultura puede ser considerada como un conjunto de sistemas simbólicos\ldots pero los diferentes sistemas de símbolos cuyo conjunto constituye la cultura son irreductibles entre sí''.

En las ciencias sociales, el concepto de estructura puede ser entendido de dos modos diferentes pero complementarios. En un sentido amplio, una estructura es el sistema abarcativo que contiene a los casos particulares; es la ``regla de variabilidad'' de esa pluralidad de conjuntos que surgen como variantes de su combinatoria. En un sentido estricto, las estructuras no pertenecen al orden de la realidad empírica: son pautas ``inventadas'' a partir de ella para cumplir, como los modelos, la función de hacerla inteligible.

En sentido estricto, el estructuralismo define, pues, a la estructura como una construcción racional del pensamiento, y reprocha por consiguiente al funcionalismo su concepción ``realista'' de la función y su idea de que toda la sociedad converge en ella. Lévi-Strauss consideraba que el funcionalismo es ``una forma primaria del estructuralismo'' y agregaba: ``decir que una sociedad funciona es una perogrullada, pero decir que todo, en una sociedad, funciona es un absurdo''\ldots{}

El Funcionalismo es una corriente de pensamiento cuyo orígen es europeo y cuyo desarrollo tuvo lugar principalmente en los EE.UU. Su hipótesis fundamental puede resumirse en el siguiente enunciado: Las actividades parciales de los elementos contribuyen a la actividad total del sistema del que forman parte.

A los comienzos del funcionalismo suele asociarse, en forma implícita, el nombre de Emile Durkheim, y en forma ya explícita, el de Bronislaw Malinowski. Emile Durkheim (1858-1917) es considerado ``el padre de la Sociología francesa''. Es autor de numerosas obras, entre las que cabe citar: ``De la División del Trabajo Social'', ``El Suicidio'', ``Las Formas Elementales de la Vida Religiosa'' y ``Las Reglas del Método Sociológico''\footnote{E. Durkheim: DE LA DIVISION DEL TRABAJO SOCIAL Bs. As., Schapire, 1967.
  EL SUICIDIO Bs. As., Schapire, 1965.
  LAS FORMAS ELEMENTALES DE LA VIDA RELIGIOSA Bs. As., Schapire, 1968.}.

La actitud metodológica de Durkheim partía de una exigencia de objetividad, expresada en el tratamiento de los hechos sociales ``como cosas'' (no en el sentido de cosificarlos sino de ``observarlos desde afuera''). Durkheim consideraba que una comprensión de los fenómenos sólo podía derivar de su tratamiento objetivo. A tal fin, el sociólogo debe investigar en primer lugar la CAUSA del fenómeno y en segundo lugar su FUNCION, pero Durkheim aclaraba muy bien que ``hacer ver para qué es útil un hecho no es explicar cómo ha nacido ni cómo es lo que es'', con lo que formulaba una acertada crítica anticipada al futuro funcionalismo.

Durkheim nunca separó sus inquietudes teóricas de sus intenciones reformadoras respecto de la sociedad, para atender las cuales propugnaba un diagnóstico que discrimine lo normal y lo patológico en los fenómenos sociales, vale decir, que permita al sociólogo reconocer los males sociales y decir cómo sanarlos.

La obra de Durkheim, aparte de su proto-funcionalismo, tiene mucho interés para la Ciencia Política, en la que se detectan muchas trazas de su influencia. Ya en su primera obra, ``De la División del Trabajo Social'' (1893), planteaba una original tipología de las sociedades, distinguiendo entre las ``sociedades de solidaridad mecánica'' (cuya cohesión interna se basa en la fuerza de la conciencia colectiva, en la participación intensa de los individuos en una misma sacralidad social) y las ``sociedades de solidaridad orgánica'' (en las que los hombres, por obra de la división del trabajo social se constituyen en individualidades diferenciadas, que cumplen tareas específicas en las que realizan su vocación personal).

En estas sociedades, a diferencia de las anteriores, tienen mucha importancia las diferencias individuales; en ellas la cohesión interna es producto de la complementación de funciones y de un nuevo tipo de representaciones y creencias desarrolladas en torno al concepto de ``persona humana''.La transición desde un tipo de sociedad mecánica a un tipo de sociedad orgánica se produce por causas del tipo ``tamaño y complejidad'' : aumento del volúmen de la sociedad, aumento de la densidad material y aumento de la ``densidad moral'', o sea de la intensidad de los intercambios y de las comunicaciones.

En esta obra de Durkheim que estamos comentando, quizás la parte más pertinente a la Ciencia Política sea su desarrollo del concepto de ANOMIA, entendida como incapacidad social de integración de los individuos a causa de un debilitamiento de la conciencia colectiva. Se trata, en principio, del mal que sufre una sociedad en su conjunto por la carencia o falta de vigencia real de una normativa moral y jurídica que le permita organizar su dinámica interna; es una ruptura de la solidaridad social, una crisis de la sociedad tomada como totalidad. Estas reflexiones de Durkheim continúan en su obra sobre ``El Suicidio'', en la que el concepto de anomia interviene en la definición de una tipología de los suicidas, cuando Durkheim distingue el suicidio altruísta, el egoísta y el anómico. En esta última obra, Durkheim desarrolla otro aspecto de la anomia: la relación del individuo con las normas de su sociedad. Durkheim hace notar el carácter infinito, vertiginoso y angustiante del deseo del hombre librado a sí mismo, cuando se rompe la relación entre el actor social y el orden simbólico de su sociedad. Esa angustia desaparece cuando la sociedad tiene fuerza suficiente para someterlo a sus normas, pero reaparece cuando disminuye la fuerza de esos instrumentos integradores de la sociedad. Esa es la anomia. Durkheim trató sobre todo de establecer una relación entre la anomia y el modo de organización de la sociedad, planteando el problema de la relación entre los sistemas de valores y las estructuras socio-económicas, en función del ritmo de cambio de estas últimas. Desde el punto de vista politológico es evidente la importancia de estos fenómenos, en relación con la creación de condiciones de orden social y consenso cívico, así como en lo referente a la captación de voluntades individuales para el logro de metas colectivas y la asignación autorizada de valores.

Finalmente, otro trabajo de Durkheim digno de mención desde el punto de vista politológico, y que ha motivados muchos análisis, comentarios críticos y hasta polémicas, es su estudio sobre ``Las Formas Elementales de la Vida Religiosa'', que ha influído mucho en análisis posteriores sobre las relaciones entre Política y Religión, tema que tiene en verdad gran importancia y sobre el que hay pocas investigaciones profundas.

En esta obra, Durkheim sostiene que ``bajo la apariencia de lo sagrado, lo que los hombres adoran, sin saberlo, es la Sociedad\ldots{}'', y describe a continuación los procesos mediante los cuales los grupos producen, en épocas de gran exaltación social, los dioses que necesitan; y cómo los transfiguran luego. Como ejemplo paradigmático de este proceso, Durkheim menciona lo ocurrido en los primeros años de la Revolución Francesa, momento en el que ``\ldots bajo el influjo del entusiasmo general, unas cosas puramente laicas por naturaleza fueron transformadas por la opinión pública en cosas sagradas, como la Patria, la Libertad, la Razón. Una religión que tenía su dogma, sus símbolos, sus altares y sus fiestas tendió a establecerse por sí sola. El culto a la Razón y al Ser Supremo intentó aportar una especie de satisfacción oficial a estas aspiraciones espontáneas''.

Bronislaw Malinowski (1884-1942) fue un sociólogo y etnólogo, polaco de nacimiento, que realizó la mayor parte de su labor intelectual en Inglaterra. Fue profesor en la ``London School of Economics and Political Science'' y autor de numerosas obras, entre las que cabe citar ``Crimen y Costumbre en la Sociedad Salvaje'', ``Moeurs et Coutumes des Melanesiens'' y sobre todo ``A Scientific Theory of Culture''\footnote{Bronislaw Malinowski: A SCIENTIFIC THEORY OF CULTURE, New Caroline University Press, 1944.}, donde está resumida su ``teoría funcionalista radical''.

Malinowski solía apodarse a sí mismo ``el jefe del funcionalismo''. Utilizaba el término FUNCION con dos significados diferentes (lo que originó no pocas confusiones posteriores): - como conexión permanente entre los elementos integrantes de una realidad social dada, con carácter regulador y dador de significado; - como relación positiva entre las necesidades primarias de los hombres y los sistemas sociales.

Esta segunda acepción entraña un cierto reduccionismo de la cultura a la necesidad, que es bastante recurrente en el pensamiento de Malinowski y que le ha sido muy criticado posteriormente.

Malinowski partió en sus investigaciones de la consideración de las necesidades fundamentales o básicas de la naturaleza humana, y estudió las diversas formas en que se manifiestan y satisfacen en las diversas culturas. Según Malinowski, la vida social es producto de la urgencia que sienten los individuos de cubrir ciertas ``necesidades fundamentales'', tales como alimentación, seguridad, vinculación, etc. Su ``análisis funcional'' parte del supuesto de que cada costumbre, cada idea, etc., cumple una función vital para los individuos, en cuanto a la satisfacción de sus necesidades, en el particular contexto cultural de cada uno.

Esa vida social-cultural tiende a expresarse en ``instituciones sociales''. Para Malinowski, cada institución tiene su ``mapa'' , vinculado a las representaciones y creencias del grupo social. Ese mapa abarca la definición, estructura y finalidad del grupo institucionalizado, y las reglas que el grupo debe obedecer. Toda institución tiene, pues, normas, actividades propias , personal y aparato material. Entraña, por otra parte, una ``función'', porque está destinada en última instancia, a satisfacer una necesidad.

La difusión del enfoque funcionalista, después de la segunda guerra mundial, fue el más notable cambio de orientación conceptual en la historia reciente de las ciencias del hombre. Especialmente en la década de los cincuenta se produjo una gran eclosión de obras funcionalistas, primero en Sociología y Antropología, luego en Psicología (particularmente en vinculación con la llamada ``dinámica de grupos'') y finalmente en Ciencia Política. En la década de los sesenta, el funcionalismo era el modo de investigación predominante en Ciencia Política, considerado por muchos como ``el mejor enfoque posible para el desarrollo de la teoría''\footnote{Eugène J. Meehan: PENSAMIENTO POLITICO CONTEMPORANEO, Madrid, Revista de Occidente, 1973.}.

Funcionalismo y función son en realidad términos bastante ambiguos. Según Ernest Nagel\footnote{Ernest Nagel: THE STRUCTURE OF SCIENCE: PROBLEMS IN THE LO- GIC OF SCIENTIFIC EXPLANATION, Harcourt,Brace and World Inc., 1961.}, FUNCION tiene por lo menos seis significados distintos, cada uno de los cuales tiene implicaciones específicas para la investigación: 1) Enunciado de la interdependencia de dos variables; 2) Conjunto de procesos dentro de un sistema; 3) Uso corriente de un objeto; 4) Procesos internos de mantenimiento vital de los organismos; 5) Consecuencias que un elemento de un sistema tiene para el sis- tema como totalidad; 6) Contribución de un elemento de un sistema para el mantenimien- to de éste en un estado determinado.

Esta sexta acepción es la que con más frecuencia utilizan los funcionalistas en el campo de las ciencias sociales. Conviene aclarar que es un error usar el término función como sinónimo de ``efecto''. Por otra parte, no hay una definición ``correcta'' de función. Hay que especificar en cada caso qué acepción se está utilizando, para no invalidar el razonamiento u oscurecer el contexto de la discusión.

La forma típica de una explicación funcionalista es el establecimiento de la relación existente entre un fenómeno dado (generalmente,una forma reiterada de comportamiento social) y el sistema dentro del cual se produce dicho fenómeno. Como mínimo, una explicación funcional requiere la existencia de un fenómeno a investigar, un sistema dentro del cual se produce el fenómeno, y la determinación de las consecuencias del fenómeno para el sistema. En este esquema se ve claramente la relación que, a poco andar, se estableció entre el enfoque funcionalista y el sistémico.

Las explicaciones funcionales tienen habitualmente forma causal o factorial. Muy rara vez se ha logrado por esta vía una explicación ``completa'', que incluya todas las consecuencias del fenómeno para el sistema. Por otra parte, es imprescindible definir cuidadosamente el sistema que va a ser analizado. Ahora bien, en general los sistemas (especialmente en ciencias sociales) se definen en forma analítica, no empírica. Esto quiere decir que los sistemas no vienen ``dados por la naturaleza'' sino que son delimitados en función de los propósitos del investigador. Esto, como es obvio, abre las puertas a un riesgo muy grande de forzar los hechos para que se amolden a las intenciones. Hay un límite a la arbitrariedad en la construcción de sistemas conceptuales: hay que conservar correspondencias claras entre el modelo y los aspectos concretos de la realidad en estudio, pero no hay reglas fijas y uniformes para no cruzar esa frontera.

Dentro del funcionalismo hay una gran variedad de criterios sobre aspectos básicos del enfoque: elección de fenómenos, amplitud de sistemas de base, precisión en la definición de relaciones. Algunos enfoques son predominantemente sociológicos; otros, psicológicos; algunos son teleológicos y otros no. También hay diferencias muy importantes acerca del modo de construir teorías y del papel de la teoría en la explicación de fenómenos específicos. Esto se aprecia claramente, como veremos enseguida, al comparar las obras de Robert Merton y de Talcott Parsons.

Inicialmente, el funcionalismo derivó de una analogía orgánica. El enfoque organicista es muy evidente en la obra de Malinowski y de Radcliffe-Brown, y aún hoy nutre la obra de muchos sociólogos funcionalistas. Malinowski, como ya vimos, se inclina a definir las funciones en términos de necesidades fundamentales de todos los seres humanos, necesidades que, en última instancia, tienen una raíz orgánica. Ahora bien, la unidad fundamental de análisis sociológico en Malinowski es la noción de ``institución social'', mientras que Radcliffe-Brown está más interesado en las ``funciones vitales'' de la sociedad, y toma como norma la ``vida social ordenada'', norma que sólo puede mantenerse si todos los miembros de la sociedad comparten ciertos sentimientos básicos comunes.

Para expresar esa idea, Radcliffe-Brown empleaba el término ``consensus'',de larga trayectoria posterior en Ciencia Política, como expresión de un acuerdo o afinidad entre los miembros de una sociedad, acuerdo referido a valores culturales, a normas y a la desiderabilidad de las metas sociales así como a las reglas básicas del juego para obtenerlas. Se trata, en definitiva, de un vínculo de solidaridad social que reduce la necesidad de recurrir a la fuerza para resolver conflictos y crear orden y aumenta la eficiencia global del sistema al no desviar hacia conflictos internos energías que pueden aplicarse a los fines propios del sistema.

Radcliffe-Brown, que fue también un precursor del enfoque comparatista en las ciencias sociales, se interesó mucho por los mecanismos y procesos de transmisión entre las generaciones de los sentimientos sintetizados en el ``consensus'': los llamados procesos de socialización, o, en lenguaje antropológico, endoculturación. Esta preocupación ocupó también un lugar central en la obra de Talcott Parsons, y es cada vez más frecuente en el pensamiento político contemporáneo, especialmente desde el reciente auge de los enfoques ``culturalistas''.

Aunque el estructuralismo y el funcionalismo tuvieron orígenes distintos y mantuvieron en sus comienzos polémicas teóricas,terminaron por converger en su desarrollo posterior, vinculándose estrechamente también con el enfoque sistémico, a tal punto que hoy, cuando se habla de ``funcionalismo en sentido amplio'' se está haciendo alusión a un enfoque de síntesis: estructural-funcionalista-sistémico. Esta convergencia ya se advierte claramente en la obra de Talcott Parsons.

La actitud metodológica típica del estructuralismo consiste en preguntarse cómo es el objeto estudiado, analizando de qué manera están dispuestas las diferentes partes del conjunto. Analíticamente, una ESTRUCTURA es una representación mental de la disposición de las partes de un todo. La actitud metodológica típica del funcionalismo consiste en preguntarse qué hace el objeto, o sea cuál es la función que cumple para el sistema del que forma parte.

Fácilmente puede entenderse que estructuralismo y funcionalismo son dos caras de la misma moneda, ya que el estudio de la estructura lleva a considerar las funciones de los diferentes elementos, y el estudio de las funciones (lo que cada una de las partes hace con respecto al todo) no puede dejar de considerar la estructura. Por otro lado, ese todo es visto como un sistema, del que los elementos considerados son subsistemas. Ellos pueden ser tomados, a su vez, como sistemas de otros subsistemas menores, según el nivel de resolución analítica que se adopte. Así puede entenderse, pensamos, esa confluencia de enfoques en la síntesis estructural-funcionalista-sistémica que fue mencionada párrafos atrás.

\hypertarget{el-enfoque-sistuxe9mico}{%
\subsection*{El enfoque sistémico}\label{el-enfoque-sistuxe9mico}}
\addcontentsline{toc}{subsection}{El enfoque sistémico}

Antes de describir en forma sintética la obra de autores que pueden ser considerados paradigmáticos de los enfoques aquí mencionados, vamos a hacer una serie de consideraciones generales sobre el enfoque sistémico, en el que parecen converger o complementarse el estructuralismo y el funcionalismo desde hace varias décadas.

No es un secreto para nadie que el concepto de SISTEMA ha invadido todos los campos de la ciencia y penetrado en el pensamiento, los medios de comunicación de masas y hasta en el habla popular. Aparece como un aporte nuevo frente a fenómenos que hasta ahora habían sido estudiados como ``mecanismos'' (por el estructuralismo) o como ``cajas negras'' (por el funcionalismo).Este nuevo enfoque irrumpe con fuerza no solo en el campo tecnológico y físico-biológico sino también en el ámbito psico-social, e inclusive, por cierto, en su dimensión política\footnote{Ludwig von Bertalanffy: TEORIA GENERAL DE LOS SISTEMAS, México, FCE, 1981.}.

Qué hay que entender por SISTEMA? Digamos de entrada que no es algo simple, evidente o trivial. Por una parte hay realidades (una galaxia, un animal, una célula, un átomo) que son sistemas reales: entidades que la observación percibe, o que se pueden inferir a partir de ella y que existen por sí mismas, con independencia de cualquier observador. Por otra parte, hay sistemas puramente conceptuales, como los que habitan el campo de la Lógica y de las Matemáticas, sistemas que pueden ser considerados como ``construcciones puramente formales'' o simbólicas. Finalmente, están también los llamados ``sistemas abstraídos'', que constituyen el grueso del cuerpo de todas las ciencias naturales y humanas que trabajan con sistemas. Son sistemas conceptuales correspondientes a hechos reales. Un ecosistema, un sistema social, un sistema político, corresponden a hechos reales, pero evidentemente no se trata de objetos de percepción directa sino de construcciones conceptuales, de abstracciones (de modelos, en definitiva) que son elaborados y tienen valor y utilidad en la medida en que guardan correspondencia con aspectos o hechos de la realidad que a cada ciencia interesan, aunque sea, desde luego, en forma abstracta y simplificada.

En el campo de la Ciencia Política, por ejemplo, el concepto de SISTEMA POLITICO fue elaborado como un modelo teórico, es decir, como una abstracción de la realidad política que se quiere explicar, para lo cual se la simplifica, reduciéndola a sus rasgos considerados fundamentales (elementos básicos y relaciones entre esos elementos) con el fin de hacerla inteligible.Por otra parte,no es el único modelo posible. Dentro del panorama teórico global, está ubicado en uno de los tipos de modelos existentes, denominado ``modelos de integración y de orden'', en contraposición a los denominados ``modelos de conflicto''. Esto no sólo tiene implicaciones teóricas sino también ideológicas y cosmovisionales, como veremos más adelante al estudiar el tema con mayor detalle\footnote{Ver al respecto los Cap. 6 y 7 de este libro.}.

El enfoque sistémico se ha trasladado al campo de las ciencias del hombre desde otros campos del conocimiento, como la Biología y la Ingeniería. No es, en realidad, un enfoque absolutamente nuevo y original. Ya en la obra de antiguos pensadores, desde Nicolás de Cusa, Paracelso, Hobbes, Leibniz hasta Marx y Engels, encontramos ocasionales referencias a la existencia de ``sistemas'', en los que existe interdependencia entre los elementos componentes. Ya vimos también los antecedentes más directos contenidos en la obra de Durkheim y de Malinowski. Pero la sistematización teórica más amplia y rigurosa del enfoque sistémico, de la que derivan todas las aplicaciones modernas conocidas en nuestro campo, es la desarrollada en la década de los años treinta por Ludwig von Bertalanffy, bajo el nombre de ``Teoría General de los Sistemas'', formulación hecha con pretensiones de validez general, omnicientífica\footnote{Ludwig von Bertalanffy: TEORIA GENERAL DE LOS SISTEMAS, México, FCE, 1981.}.

La Teoría General de los Sistemas eligió el término SISTEMA para identificar un concepto propio, con el que expresa toda una concepción del mundo, súmamente ambiciosa. ``Su objeto central es la formulación y derivación de aquellos principios que son válidos para todo sistema en general'' -dice von Bertalanffy- y añade ``\ldots la elaboración de la teoría sistémica general probará ser un paso fundamental para la unificación de la ciencia\ldots{}''. Se trata, en definitiva, de una concepción científica con fuerte vocación holística, cuyo concepto central (sistema) es considerado también por algunos críticos como vago, difuso y metafísico.

Para von Bertalanffy, a partir de su planteo la Ciencia queda dividida en dos grandes parcelas: - Las ciencias que se ocupan de los hechos causales, regidos por el segundo principio de la Termodinámica, cuya lógica válida es la teoría de las probabilidades; - Las ciencias que se ocupan de los todos organizados o sistemas, en los que existe entropía negativa y cuya lógica válida es la Teoría General de los Sistemas.

Von Bertalanffy anota que el concepto de SISTEMA se utiliza corrientemente con diversos significados, que básicamente pueden resumirse en tres: 1) En algunas ocasiones es un término vacío, equivalente a ``forma'', ambiguo y casi sin contenido. Aporta sólo una vaga idea de organización, a veces simplemente reiterativa porque ya está contenida en los términos que lo acompañan, como cuando se dice ``sistema de gobierno'' que equivale a ``gobierno'' o ``sistema de partidos'' que equivale a ``partidos''.

\begin{enumerate}
\def\labelenumi{\arabic{enumi})}
\setcounter{enumi}{1}
\item
  En otras ocasiones expresa una relación entre variables y contiene por lo tanto una idea de estructura, asi sea mínima.
\item
  Finalmente, la palabra ``sistema'' es también empleada con intenciones teóricas, como concepto definido y preciso, dentro de un marco lógico claramente estructurado. En el marco de su teoría sistémica general, von Bertalanffy define al sistema como ``un conjunto de elementos en interacción''.
\end{enumerate}

Como ejemplos de la difusión del enfoque sistémico en distintos ámbitos científicos, podemos mencionar los siguientes: 1) TEORIA DE LOS SISTEMAS VIVOS: Desarrollada en Biología para dar explicación al fenómeno de la vida orgánica, y dentro de ella, en particular, de la vida animal.

\begin{enumerate}
\def\labelenumi{\arabic{enumi})}
\setcounter{enumi}{1}
\item
  TEORIA SISTEMICA GENERAL: Nació inspirada principalmente en la anterior. Trata de superar las limitaciones tradicionales de la ciencia, especialmente su falta de unidad, su dispersión y sus trabas comunicacionales, mediante conceptos propios, de validez general, y de proposiciones aplicables a cualquier campo científico.
\item
  TEORIAS SISTEMICAS ESPECIALES: En cada caso es la adaptación de la teoría general a las distintas ciencias particulares en las que se aplica.
\item
  ANALISIS DE SISTEMAS: Denominación que se utiliza para las aplicaciones de la Teoría General de los Sistemas en el campo de la Ingeniería.
\item
  ENFOQUE SISTEMICO EN CIENCIA POLITICA: Es más adecuado usar este nombre de ``enfoque'' (o de ``aproximación teórica'', para conservar el sentido dinámico de la expresión inglesa ``approach'') porque más que una teoría totalmente estructurada y de aceptación general es un esquema o referencia teórica con el cual poder aproximarse a la realidad política para investigarla\footnote{Eugène J. Meehan, op. cit.}.
\end{enumerate}

La teoría sistémica trae consigo una importante novedad, que es la incorporación de una nueva dimensión: el tiempo. Esta dimensión temporal está implícita en el concepto mismo de sistema, que es un auténtico ``acumulador de tiempo'', que permite plantear en nuevos términos la relación entre los estudios científicos diacrónicos y sincrónicos.

Si aceptamos el concepto de SISTEMA, en sentido amplio, como ``un todo cuyas partes están interrelacionadas'',veremos que dichos sistemas -quizás con otros nombres- han sido estudiados por el hombre desde hace mucho tiempo. El hombre, en su afán de conocer, trató siempre de abordar la naturaleza y la sociedad con una idea de totalidad. Esa actitud holística, en realidad fue abandonada recién en el siglo XVIII.

La contínua acumulación de conocimientos científicos puso al conjunto creciente de nociones fuera del alcance de la mente individual y obligó a organizar el conocimiento de forma cada vez más fragmentaria y especializada. Esto permitió la profundización y la aceleración del proceso científico, pero a la vez significó una notable pérdida de visión totalizadora: ambos resultados fueron la ambigua consecuencia de la especialización.

Ya en nuestro siglo, y más específicamente desde la década de los cuarenta, se plantearon muchos esfuerzos de investigación orientados al estudio de fenómenos que sólo pueden explicarse si el objeto que se investiga es tomado como totalidad. La época indicada es la que en realidad marca, tras los trabajos precursores (y preparatorios) de von Bertalanffy, el comienzo de la ``era de los sistemas'', mientras que la etapa precedente del desarrollo científico más bien merece el título de ``era de la máquina'', con algunas excepciones precursoras en el ámbito de la Biología.

En efecto, ya desde la década de los `20, el término 'sistema' había comenzado a usarse con una significación científica precisa, en publicaciones biológicas. Los biólogos enfrentaban por entonces un grave problema: explicar el fenómeno de la vida, que excede el marco positivista-mecanicista imperante por entonces, pero sin recurrir a apelaciones metafísicas para establecer claros límites y diferenciaciones entre el mundo de los seres vivos y el de la materia inerte. Luego de muchos estudios y propuestas, la Teoría de los Sistemas Vivos aportó una nueva solución al problema de explicar la animación de los seres vivos. Fueron revisados entonces los principios cuantitativos de la lógica mecanicista, evidenciándose la necesidad de tratar a los seres vivos como unidades y no como agregados.

La noción de UNIDAD tuvo y tiene una grandísima importancia en la consolidación del enfoque sistémico. Las características que diferencian a las unidades de los simples agregados son las siguientes: 1) Una unidad posee límites claramente distinguibles, que la separan del ambiente exterior, y eventualmente la vinculan selectivamente con él. Las fuerzas y procesos internos quedan separados por esa frontera de sus homólogos externos.

\begin{enumerate}
\def\labelenumi{\arabic{enumi})}
\setcounter{enumi}{1}
\item
  Como mínimo, una de las dimensiones de la unidad es distinta de la agregación de las dimensiones homólogas de las partes.
\item
  La descripción de la unidad no consiste meramente en la sumatoria de las descripciones de sus elementos componentes.
\item
  Cada uno de sus elementos está en relación con todos y cada uno de los demás y con la unidad misma. La unidad está a su vez en relación con el medio ambiente en el que está inmersa.
\end{enumerate}

Respecto de este último punto, conviene precisar que hay varias clases de relaciones en una unidad: 1) Relaciones de los elementos componentes entre sí.

\begin{enumerate}
\def\labelenumi{\arabic{enumi})}
\setcounter{enumi}{1}
\item
  Relaciones entre los elementos y la unidad como un todo.
\item
  Relaciones de la unidad con su medio ambiente: a) Insumos y exumos: la unidad toma insumos del ambiente y le devuelve exumos.
\end{enumerate}

\begin{enumerate}
\def\labelenumi{\alph{enumi})}
\setcounter{enumi}{1}
\item
  Procesos de adaptación al stress: toda alteración en el me- dio ambiente es una amenaza, que es fuente de stress, al que el sistema debe adaptarse para sobrevivir.
\item
  Procesos de mantenimiento de los límites.
\end{enumerate}

Según algunos autores, como Karl Deutsch por ejemplo, lo que ``circula'' en todos esos procesos y relaciones no es en el fondo otra cosa que información, vale decir, relaciones pautadas entre eventos, tengan o no contenido material o energético.

La vital necesidad de adaptación al stress nos muestra que todo sistema tiende a la homeostasis, concepto vinculado al mantenimiento dinámico del equilibrio. De allí surge el concepto de REGULADOR: el equilibrio es mantenido por medio de ajustes contínuos de la trayectoria de los procesos, siempre propensos a salirse de control. El mecanismo básico en general opera así: ante una alteración en el ambiente, los reguladores envían señales a centros receptores-efectores, que activan mecanismos aliviadores (como el termostato, o la válvula de Watt). Un regulador es un centro transformador encargado de recoger información incomprensible para el organismo (``ruido'') y de convertirla en información, esto es, en pautas reconocibles y comprensibles, que permitan encarar cursos de acción adaptativa.

La vida, como todos los demás procesos del Universo, tiende a un estado de máxima entropía (desorden) según el segundo principio de la Termodinámica. Esa entropía se produce tanto en los sistemas cerrados (conjuntos de cosas inanimadas) como en los sistemas abiertos (sistemas biológicos, psicológicos y sociales) pero en estos últimos la entropía tiende a disminuir e incluso a hacerse momentáneamente negativa por la posibilidad de mantener intercambios metabólicos con el medio, vale decir, de importar materiales y energía del exterior para transformarlos en la propia sustancia, y de eliminar hacia el ambiente los desechos de la propia actividad. Por ese motivo, los seres vivos y los grupos sociales tienen fuertes tendencias ``neguentrópicas'' (de entropía negativa, o sea que tienden hacia estados de complejidad y orden crecientes), posibilidad de la que carecen las cosas inanimadas.

Un sistema abierto se caracteriza, pues, por sus fuertes tendencias anti-entrópicas y porque sus partes actúan en forma intensamente interdependiente. La unidad mantiene relaciones metabólicas con su medio y, a lo largo del tiempo, su existencia atraviesa una serie de estados, cuyo conjunto se denomina ``actuación'' del sistema. Un aspecto de singular importancia de los sistemas abiertos es que, más allá de plano puramente biológico de la vida orgánica, cuando entramos en el terreno de lo psicológico y lo social, no puede hablarse con propiedad de una ``tendencia hacia la homeóstasis'', hacia el equilibrio, sino, como von Bertalanffy lo señala con agudeza, más bien de una tendencia al ``mantenimiento de desequilibrios'' : la homeóstasis no explica las sublimes creaciones ni las execrables violencias de los hombres, y el modelo de ``organismo reactivo'' explica mal los comportamientos humanos, para los que es más adecuado un modelo de ``organismo activo'' que ``\ldots en un sentido muy concreto, crea su universo'', dice von Bertalanffy\footnote{Ludwig von Bertalanffy: TEORIA GENERAL DE LOS SISTEMAS, México, FCE, 1981.}.

La existencia de un sistema abierto implica la presencia de información ordenadora y de energías que trabajan en contra del segundo principio de la Termodinámica. La Teoría Sistémica General estudia con especial interés tales fuerzas. Los ejemplos típicos de sistemas abiertos son los seres vivos, las estructuras psicológicas y los sistemas sociales, uno de los cuales es el sistema político.

Los criterios básicos de trabajo del enfoque sistémico para el estudio del sistema político son los siguientes: 1) La teoría sistémica trabaja únicamente con leyes estadísticas macroscópicas, que son perceptibles sólo cuando se observan todos los elementos del sistema simultáneamente.

\begin{enumerate}
\def\labelenumi{\arabic{enumi})}
\setcounter{enumi}{1}
\item
  Los grupos humanos más abarcativos (sociedades o sistemas sociales globales) no tienen funciones específicas, fuera de su propio mantenimiento y consolidación.
\item
  Vistos desde cierto nivel de abstracción, sociedad y sistema político son homomórficos (forma similar) pero no homofílicos (similar origen). Dentro del enfoque sistémico, en diversos autores hay variedad de posturas sobre la ubicación del sistema político respecto del sistema social, pero en general la lectura de textos sistémicos muestra una concepción de la vida política como fenómeno ordenador, dotado por tanto de cierta primacía.
\end{enumerate}

La teoría sistémica ha experimentado algunas adaptaciones en su aplicación al campo de la Ciencia Política. En ella, el concepto de SISTEMA opera como una estructura mental heurística para intentar la explicación de dos tipos fundamentales de fenómenos: los que se relacionan con el mantenimiento de los sistemas políticos en un estado determinado; y los que se relacionan con los cambios que se producen en ellos, ya sean adaptativos (cambios en el sistema) o disruptivos (cambios de sistema).

Hay dos características que distinguen al sistema político de todo otro sistema social: su universalidad y su condición de árbitro final de la vida social. La primera se refiere a que sólo el sistema político abarca a todos los individuos que forman la sociedad; otras organizaciones, como las religiosas, laborales, culturales, etc., abarcan solamente a una parte del total de la población. La segunda característica se refiere a que el sistema político posee una condición de árbitro final de los conflictos sociales (por su monopolio del poder coercitivo legítimo). Debido a que ocupa el nivel más alto en la jerarquía de las autoridades, tiene la potestad de fijar límites a la coacción que pueden ejercer sobre sus integrantes los sistemas que están por debajo de él. Tiene sobre todo la potestad de imponer la vigencia de una regla política fundamental: la resolución pacífica de los conflictos, con posibilidad de mediar en ellos, aún por medio del empleo legítimo de la fuerza pública, en caso contrario.

La fuente de legitimidad del sistema político como autoridad final es su aceptación explícita como tal por parte de quienes componen la sociedad. Los hombres son seres muy sociables, y generan o aceptan la existencia de grupos o asociaciones con finalidades variadas y específicas, pero su compromiso con el sistema político es más amplio y profundo, y le conceden mayor poder coercitivo en sus vidas. Por el mismo motivo, la ruptura de vínculos y la rebelión al sistema político es más radicalizada y generalmente no se expresa en el alejamiento, como en otros casos, sino en diversas formas de confrontación abierta.

\hypertarget{suxedntesis-de-obras-teuxf3ricas-principales-de-estas-corrientes}{%
\subsection*{Síntesis de obras teóricas principales de estas corrientes}\label{suxedntesis-de-obras-teuxf3ricas-principales-de-estas-corrientes}}
\addcontentsline{toc}{subsection}{Síntesis de obras teóricas principales de estas corrientes}

Hay dos formas básicas de análisis funcional\footnote{Esta expresión, tomada de E. Meehan, en su sentido amplio alude al enfoque de convergencia estructural-funcional-sistémico.}, que se distinguen por sus objetivos y sus estrategias. Vamos a verlas primero en la investigación sociológica general y luego en el campo de la Ciencia Política. Una de esas formas, cuyo paradigma sociológico es la obra de Robert Merton, se concentra en fenómenos específicos y busca explicaciones limitadas en su alcance, estrechamente relacionadas con los hechos concretos de la vida social. Busca formular lo que el mismo Merton denomina ``teorías de alcance medio''. La otra forma básica intenta el desarrollo de una ``teoría general de la sociedad'', o sea de un conjunto omnicomprensivo de categorías que puedan usarse para explicar cualquier conjunto de fenómenos dentro del campo abarcado por la Sociología. El ejemplo clásico de ese funcionalismo generalista\footnote{Idem nota anterior.} es la obra de Talcott Parsons. Parsons es un constructor de sistemas, un gran teórico, mientras que Merton tiene una aguda conciencia de la necesidad de mantenerse en estrecho contacto con los hechos.

En la comparación de la obra de estos dos hombres se ve claramente una disyuntiva de validez muy amplia, dentro del campo de las teorías empírico-analíticas: o se construye un sistema teórico general, muy abarcativo pero, por eso mismo, de tán elevado nivel de abstracción que se tiende a perder contacto con los hechos empíricos, o se mantiene la proximidad con los hechos pero perdiendo área de cobertura teórica y visión de conjunto del campo abarcado.

Merton utiliza el funcionalismo como un instrumento de explicación, o sea como un recurso metodológico para explicar los hechos; Parsons procura sobre todo desarrollar categorías y relaciones utilizables para clasificar y ordenar datos generales y armar modelos descriptivos de amplios conjuntos.

\hypertarget{robert-king-merton}{%
\subsection*{Robert King Merton}\label{robert-king-merton}}
\addcontentsline{toc}{subsection}{Robert King Merton}

Robert K. Merton (n.~1910 ),es un sociólogo norteamericano, animador del ``Bureau of Applied Social Research'' de la Universidad de Columbia. Su funcionalismo presenta rasgos peculiares y no es reductible al de Malinowski. Para Merton, ``la orientación central del funcionalismo se expresa en la práctica de interpretar los datos mediante la determinación de las consecuencias que los mismos tienen para las estructuras más amplias de las que proceden'' . Entre sus principales obras cabe citar a ``Teoría social y Estructura Social'', de donde se ha extraído la cita precedente, ``Elementos de teoría y de método sociológico'', ``La Sociología hoy: problemas y perspectivas'' y ``Selección de Lecturas sobre la Burocracia''\footnote{Robert Merton: SOCIAL THEORY AND SOCIAL STRUCTURE, Free Press, 1949.
  Hay versión en español: TEORIA Y ESTRUCTURA SOCIALES, México, FCE, 1964.
  Robert Merton: ELEMENTS DE THEORIE ET DE METHODE SOCIOLOGI- QUE, Paris, Plon, 1965.
  Robert Merton et al.: SOCIOLOGY TODAY, PROBLEMS AND PROSPECTS New York, Basic Books, 1960.
  Robert Merton et al.: READER IN BUREAUCRACY New York, The Free Press, 1952.} Merton parte de una analogía orgánica y se apoya mucho en principios biológicos, pero le añade gran número de conceptos esenciales para el desarrollo amplio de las posibilidades del método funcionalista. Muchos de sus aportes constituyen una respuesta superadora de las críticas que se formularon a los planteos iniciales del funcionalismo absoluto.

Merton distingue claramente entre los elementos funcionales y disfuncionales de un sistema y reconoce la posible existencia de elementos redundantes. Refiere la función social a consecuencias objetivas observables y no a actitudes subjetivas. Distingue entre las funciones manifiestas, que son consecuencias objetivas que contribuyen al ajuste del sistema y son reconocidas y queridas por los miembros integrantes del mismo, y las funciones latentes, que los miembros del sistema no reconocen ni quieren como propias. Evita y aclara la confusión entre motivación consciente y consecuencias objetivas de los hechos, y presta especial atención a los efectos laterales de las acciones. Afirma el principio del ``balance positivo'' de las consecuencias funcionales de las formas culturales persistentes; y el principio de las ``alternativas funcionales'': cualquier función puede ser cumplida por varias vías alternativas. Finalmente, por razones empíricas rechaza algunos postulados originarios del funcionalismo, referidos a la unidad funcional, el funcionalismo universal y la imprescindibilidad funcional.

El enfoque que Merton hace del análisis funcional fue expuesto por él en un ``paradigma'' de once puntos. Es una especie de guía metodológica-pedagógica, que preparó para sus alumnos y que presenta un gran interés para la investigación en ciencias sociales, incluída la Ciencia Política, por su orientación fuertemente empírica y su preocupación por la precisión: 1) Elementos a los que se atribuyen funciones: - Descripción pura.

\begin{itemize}
\item
  Alternativas desechadas.
\item
  Sentido de la actividad para los miembros del grupo.
\item
  Motivos de los actores.
\item
  Regularidades de comportamiento.
\end{itemize}

\begin{enumerate}
\def\labelenumi{\arabic{enumi})}
\setcounter{enumi}{1}
\item
  Diferenciación entre los motivos de los participantes y las actitudes y creencias.
\item
  Consecuencias objetivas de los fenómenos: - Consecuencias funcionales. \textbar{} Manifiestas - Consecuencias disfuncionales. \textbar{} o - Consecuencias no funcionales. \textbar{} Latentes.
\end{enumerate}

\begin{itemize}
\tightlist
\item
  Balance favorable del conjunto de consecuencias.
\end{itemize}

\begin{enumerate}
\def\labelenumi{\arabic{enumi})}
\setcounter{enumi}{3}
\item
  Los sistemas sociales son plurales: a qué unidad sirve la función.
\item
  Exigencias funcionales: Condiciones esenciales para el mante- nimiento o estabilidad del sistema.
\item
  Mecanismos de realización de las funciones.
\item
  Alternativas o equivalentes funcionales.
\item
  Contexto estructural: estrecha relación entre estructura y función.
\item
  Dinámica y principios de cambio.
\item
  Problemas de verificación.
\item
  Implicaciones ideológicas del análisis: esclarecer la propia parcialidad.
\end{enumerate}

En este paradigma del análisis funcional cabe destacar algunas características principales: la importancia asignada al trabajo de campo y a la investigación concreta; la conceptualización estrechamente ligada a la observación, y la oposición a los intentos de formular teorías generales. Es una guía metodológica que intenta guiar hacia la formulación de planteos claros y de fácil comprobación empírica. En realidad, más que una ``teoría funcional'', lo que ofrece es un método de investigación riguroso, exigente, que no promete nada de antemano ni tiene los atractivos que suelen ofrecer las visiones sinópticas y las grandes síntesis totalizadoras, lindantes con el ensayismo filosófico. Quizás por ello ha tenido pocos seguidores, especialmente en el campo de la Ciencia Política. Por nuestra parte, queremos destacar el grandísimo interés de los planteos metodológicos de Merton, no sólo como guía para la investigación política empírica sobre fenómenos políticos circunscriptos y concretos, sino también como inspiración para los trabajos profesionales de análisis político, especialmente en el sector de los ``análisis de situaciones''.

Para concluir esta semblanza sobre la obra de Robert Merton, vamos a mencionar algunos conceptos suyos de especial interés politológico: Los trabajos de Merton sobre el hiperconformismo que engendra la disciplina burocrática, paralelos a los trabajos de Mayo sobre el factor humano en las empresas, pusieron en evidencia los límites del modelo burocrático racionalista y la importancia de las disfunciones que aparecen en él. En general, actualmente se considera que las trabas burocráticas son disfuncionales, al menos desde el punto de vista de sus ``clientes'', aunque, como bien lo hace notar Michel Crozier\footnote{Michel Crozier: LE PHENOMENE BUREAUCRATIQUE Paris, Seuil, 1964.}, las prácticas burocráticas, aunque no sean funcionales para sus usuarios, sí lo son para sus miembros, ya que los sustraen de la arbitrariedad y de la inseguridad. Desde el punto de vista del sistema político, también puede verse cierta funcionalidad en dichas trabas, que operan como ``portillos sistémicos'' reguladores del flujo de las demandas sociales dirigidas al sistema político, evitando el exceso que provocaría el ``stress'' del sistema, sin negar explícitamente el derecho a formular demandas.

Al analizar la relación del individuo con los valores de su sociedad y con los medios de que dispone para realizarlos, Merton muestra que el conflicto o contradicción entre valores y medios es fuente de desviaciones de las conductas individuales. Por ejemplo, la sociedad norteamericana exalta el éxito económico como una virtud, pero no resuelve claramente el caso en que los medios empleados para enriquecerse no responden al mismo sistema de valores.

Por último, citaremos una reflexión de Merton sobre la relación entre el comportamiento individual y los valores sociales. Merton dice que ``\ldots debido precisamente a que el comportamiento de los individuos está modelado por los valores fundamentales de la sociedad, se puede hablar de una masa de hombres como de una sociedad. Sin un fondo de valores que sean comunes a un grupo de individuos (la conciencia colectiva de Durkheim?) puede haber relaciones sociales, intercambios desordenados entre los hombres, pero no sociedad''.

\hypertarget{talcott-parsons}{%
\subsection*{Talcott Parsons}\label{talcott-parsons}}
\addcontentsline{toc}{subsection}{Talcott Parsons}

Talcott Parsons (n.~1902 ) sociólogo norteamericano, profesor titular de Sociología en Harvard desde 1944. Entre sus \textbf{obras principales cabe citar los siguientes libros}: ``The Structure of Social Action'' (1937), ``Essays in Sociological Theory'' (1949), ``The Social System'' (1951), ``Structure and Process in modern societies (1960),''Sociological Theory and Modern Society" (1967), y ``American Society: Perspectives, Problems, Methods'' (1968)\footnote{Talcott Parsons: EL SISTEMA SOCIAL, Madrid, Rev.~de Occidente, 1976.
  ENSAYOS DE TEORIA SOCIOLOGICA Bs. As., Paidos, 1970.
  EL SISTEMA DE LAS SOCIEDADES MODERNAS México, Trillas, 1974.}.

Parsons es una figura muy polémica dentro del campo del pensamiento social. Se discute mucho el sentido de varios aspectos de su obra. No se expresa con claridad; a decir verdad, es enredado y confuso; su sistema no está empíricamente fundamentado (aunque pretende estar referido al mundo empírico) y no está, por lo tanto, realmente abierto a la convalidación por otros investigadores. El \textbf{paradigma de Parsons} es un sistema inacabado, aún abierto a continuas revisiones\footnote{David Easton: THE POLITICAL SYSTEM New York, Alfred A. Knopf, Inc.; 1953 A FRAMEWORK FOR POLITICAL ANALYSIS Prentice-Hall, Inc., 1965 A SYSTEMS ANALYSIS OF POLITICAL LIFE John Wiley and Sons, Inc., 1965 (Hay versión en castellano: ESQUEMA PARA EL ANALISIS POLITICO Bs. As., Amorrortu, 1969) VARIETIES OF POLITICAL THEORY New Jersey, Prentice Hall, 1966 (Hay versión en castellano: ENFOQUES SOBRE TEORIA POLITICA Bs. As., Amorrortu, 1969) CHILDREN IN THE POLITICAL SYSTEM New York, Mc Graw-Hill, 1969.}.

Para dar una primera idea, podemos decir que Parsons ha hecho aportes polémicos pero valiosos a la teoría sociológica, desde un punto de vista estructural-funcionalista, privilegiando los aspectos estáticos de la realidad social respecto de los aspectos dinámicos, de cambio y de conflicto. En una visión más profunda, el pensamiento de Parsons es complejo, nada fácil de aferrar en una síntesis. Cabe recordar en su descargo que la realidad a la que refiere sus trabajos es en sí misma súmamente compleja.

Puede decirse, por ejemplo, que Parsons intenta combinar el positivismo decimonónico de Wilfredo Pareto, la perspectiva histórica de Max Weber y el subjetivismo e idealismo filosóficos de los historiadores y sociólogos alemanes de los siglos XIX y XX, para crear un modelo o ``tipo ideal'' de sociedad humana, que sirva de base a un sistema explicativo general, de carácter axiomático o deductivo. En ésto, Parsons se muestra inmune a la influencia de los modernos planteos epistemológicos, según los cuales su objetivo es inalcanzable.

Es claramente reconocible en la \textbf{obra de Parsons} la influencia de su temprana afición a la mecánica newtoniana, de la que tomó muchas analogías, metáforas y ejemplos. En nuestros tiempos, la mecánica newtoniana ya no es considerada como una forma ideal y ni siquiera adecuada de explicación científica, pero Parsons persiste en ese camino en muchos aspectos de su obra.

En los libros de sociología de Parsons es perceptible la influencia de Hegel, en el que se inspira para solucionar el problema que plantea en la dinámica sociológica la libertad individual. En el dilema determinismo-voluntarismo opta por este último, pero luego lo vacía prácticamente de contenido al definir a la libertad como ``conducta acorde con las necesidades colectivas''. En el planteo hegeliano, la libertad se logra por interiorización de las normas orientadas hacia las exigencias de la colectividad. Esa síntesis hegeliana concuerda con la noción parsoniana de ``acto social''.

Parsons también tiene una gran deuda con Hobbes. El sistema parsoniano está fuertemente orientado hacia el orden y la estabilidad. Parsons supone que los instrumentos primarios para mantener el orden son las estructuras normativas interiorizadas, producidas por la sociedad y asimiladas por los individuos. Está obligado, pues, a aceptar que todo cambio, todo conflicto, es perturbador y disfuncional. En ello se basa la acusación de mantener una velada colusión con la ideología conservadora, que con frecuencia se ha hecho a su sistema científico.

Parsons procura ubicar cuáles son los elementos de la sociedad que contribuyen al mantenimiento del orden, y concentra su atención en ellos. Hay en esta actitud una evidente parcialidad, cuya consecuencia es la disolución del individuo en un conjunto de ``relaciones con otros''. Aquí resulta claramente visible el paralelismo entre Parsons y Hobbes.

Parsons ha evidenciado siempre gran interés por el estudio de la Economía, de la que provienen muchos de sus paralelismos conceptuales. El ``acto social'' de Parsons presenta gran similitud con una transacción económica: el ``actor'' recuerda fuertemente al conjunto de demandas de una unidad económica en un mercado libre\ldots Parsons concibe a la interacción humana como un calco de la interacción económica, y la estabilidad social es prácticamente una trasposición al plano general de la sociedad de la estabilidad económica.

La parte principal de la estructura conceptual de Parsons proviene de Pareto y de Weber. Su originalidad no reside en los conceptos sino en la manera de seleccionarlos y de combinarlos. De Weber tomó, entre otros, el concepto de ``Verstehen'', entendido como ``definición de la situación según la percepción subjetiva del actor''; y la idea de ``conducta social'' como ``orientación recíproca de los individuos y los grupos''. Estos elementos, tomados en el contexto del indeterminismo weberiano, proporcionaron a Parsons la definición de un concepto clave: el de ``acción social significativa'': una interacción entre dos o más personas, que incluye la motivación o intención de todas las partes intervinientes y no es una simple acción refleja.

Parsons tomó también, como ya dijimos, muchos conceptos de Pareto, en primer lugar, la noción central de ``sistema'' entendido como ``conjunto de elementos funcionalmente interdependientes''. También proviene de Pareto la noción de sociedad como ``instrumento de adaptación social'' y la finalidad que le asigna a la investigación social: ``la construcción de una teoría funcional general que explique la estabilidad del sistema social''.

La ``deuda'' de Parsons con Pareto abarca también varios otros conceptos: - Los ``requisitos funcionales'', o sea las condiciones necesarias para que el funcionamiento social tenga continuidad; - los ``residuos'', o sea las fuerzas subyacentes a la conducta; esos ``sentimientos interiorizados de valoración'' que son la base de la explicación de la estabilidad; - el interés prioritario por la acción irracional antes que por la racional; - la diferenciación entre utilidad individual y utilidad social; - la precupación marcadamente prioritaria por el orden y la estabilidad, considerados como emergentes de una combinación de mecanismos sociales y de sentimientos interiorizados por los individuos; - la atención preferente que se le dedica al proceso de socialización, ubicado principalmente en la familia.\footnote{Eugène J. Meehan: PENSAMIENTO POLITICO CONTEMPORANEO Madrid, Rev.~de Occidente, 1973.} Parsons afirma haber sido muy influído por Freud, pero de la lectura de sus textos surge la impresión de que la interpretación parsoniana de Freud es muy forzada, y más afín con los desarrollos de la Psicología del Ego realizados, tras la muerte de Freud, por psicólogos como Anna Freud y Erik Erikson. Conceptualmente, Parsons está más cerca de Karen Horney y de Harry Stack Sullivan que de Freud. Parsons utiliza mucha terminología freudiana, pero la ubica en un contexto significativo diferente del que Freud utilizaba.

El objetivo original de Parsons era, como ya vimos, la formulación de una ``teoría general de la sociedad''. Luego de su adscripción al Funcionalismo, y en un lapso de diez años, Parsons planteó dos intentos de formulación de su teoría, diferentes pero al mismo tiempo muy relacionados entre sí.

Su primera formulación parte del individuo, del ``actor individual'', ubicado en una situación concreta e interactuando con los elementos que la integran. Su libro ``Toward a General Theory of Action'' (1951) fue fruto de este primer planteo, que resultó poco satisfactorio para la crítica especializada y hasta para él mismo. En esta obra se perciben claramente los condicionantes, sobre el intento de labor científica, de los trasfondos cosmovisionales e ideológicos de la cultura y del ambiente social en el que opera un investigador.

De hecho, en su segunda formulación invirtió el enfoque y definió a los elementos del sistema social en función de la estructura global de la sociedad. Redujo notablemente la importancia que le asignaba antes a los factores psicológicos individuales e incrementó la gravitación de los factores estructurales y funcionales. Prestó menos atención a los ``valores internalizados'' y más a los ``valores institucionalizados''. Esta nueva construcción fue esbozada por primera vez en ``Working Papers in the Theory of Action'' (1953), y se completó luego en obras como ``Family, Socialization and Interaction Process'' (1955) y ``Economy and Society'' (1956).\footnote{David Easton: THE POLITICAL SYSTEM New York, Alfred A. Knopf, Inc.; 1953 A FRAMEWORK FOR POLITICAL ANALYSIS Prentice-Hall, Inc., 1965 A SYSTEMS ANALYSIS OF POLITICAL LIFE John Wiley and Sons, Inc., 1965 (Hay versión en castellano: ESQUEMA PARA EL ANALISIS POLITICO Bs. As., Amorrortu, 1969) VARIETIES OF POLITICAL THEORY New Jersey, Prentice Hall, 1966 (Hay versión en castellano: ENFOQUES SOBRE TEORIA POLITICA Bs. As., Amorrortu, 1969) CHILDREN IN THE POLITICAL SYSTEM New York, Mc Graw-Hill, 1969.} En el campo de la Ciencia Política ha tenido mucha más influencia esta segunda formulación, por lo que la vamos a ver con un poco más de detalle. Desde nuestro punto de vista, son especialmente interesantes sus ideas sobre las estratificaciones sociales. Como todas las teorías funcionalistas, la de Parsons considera que las estratificaciones sociales responden a necesidades sociales. Son sistemas jerárquicos fundados sobre los valores máximos de cada sociedad. Esos valores están relacionados con la ``acción social'', vale decir, con la ``actividad intencional que despliegan los individuos dentro del marco de las instituciones''. En síntesis, Parsons define a la estratificación social como ``la clasificación diferencial de los individuos que componen un sistema social dado, y su calificación de superiores o inferiores los unos en relación con los otros, según valores importantes para la sociedad''.

Al definirla como ``clasificación diferencial de los individuos .. según valores importantes'', Parsons parece suponer que es siempre la posesión por los individuos de determinados valores socialmente estimados lo que los ubica en determinadas posiciones en la estratificación social. Descuida, a nuestro criterio, el rol de las organizaciones intermedias de la sociedad, desde la familia hasta diversos grupos, partidos y corporaciones, que pueden llegar a tener poder suficiente como para ubicar a sus integrantes en determinadas posiciones sociales aunque individualmente no posean los valores correspondientes, e incluso sin que posean ningún valor relevante\ldots{}

En la óptica de Parsons, la estratificación social es consecuencia directa de la acción social y al mismo tiempo, su medio de manifestación. Dice Parsons que la división del trabajo social produce una diversificación de actividades; no todas son juzgadas igualmente importantes: en función de su sistema de valores, cada sociedad determina para sí una jerarquía de actividades. Nuevamente aquí encontramos algo que señalar: ésto puede haber sido correcto en antiguos tiempos, de relativo aislamiento de las comunidades sociales, pero en la medida en que se intensifica la interacción internacional, se incrementa el rol del ``efecto-demostración'' de unas sociedades sobre otras; y también el de la ``influencia'' de las sociedades más poderosas sobre las más débiles, por las interacciones asimétricas que se establecen, hasta llegar a los extremos de la a-culturación y la dependencia cultural.

Según Parsons, los criterios de evaluación que conducen en definitiva a una determinada estratificación social, se basan en tres elementos: las cualidades, las realizaciones y lo adquirido: - las cualidades son posesiones personales de cada individuo, que están ubicadas fuera de toda circunstancia especial externa (por ejemplo, inteligencia, nobleza, talento, etc.); - las realizaciones son producto de la actividad del individuo en relación con los demás (por ejemplo, el prestigio, el ascendiente, etc.); - lo adquirido es la posesión de objetos o bienes (como la fortuna material, las propiedades, etc.) o de certificaciones de talentos o aptitudes (diplomas, reconocimientos, premios).

Estos criterios de evaluación se aplican según las indicaciones del sistema de valores de cada sociedad. Parsons sostiene que dicho sistema está integrado por cuatro tipos de valores, todos necesarios para el buen funcionamiento de la sociedad, aunque cada sociedad arma su propio esquema de prioridades para estos valores: 1) Universalismo: Se trata de la capacidad de adaptación, que corresponde a la necesidad de toda sociedad de ajustarse a sus condiciones objetivas de existencia. Se relaciona con la racionalidad (en sentido weberiano) y con la eficiencia técnica, o sea con el uso de medios adecuados para alcanzar determinados fines, a costos adecuados.

\begin{enumerate}
\def\labelenumi{\arabic{enumi})}
\setcounter{enumi}{1}
\item
  Definición de objetivos: Toda sociedad se propone alcanzar ciertas metas colectivas, y trata de que esas metas prevalezcan sobre los intereses individuales o sectoriales. La definición de esos objetivos es la configuración de la finalidad social, y eventualmente la satisfacción del objetivo logrado. Se relaciona, por lo tanto, con las normas de realización.
\item
  Integración: La solidaridad social es un valor primordial. Las acciones sociales son evaluadas según la medida en que favorezcan o impidan la integración de los individuos en la sociedad, y su mutua solidaridad.
\item
  Mantenimiento del modelo: Cada sociedad tiene un modelo cultural propio, con sus propias estructuras y normas, y tiende a conservarlo. En este aspecto, el valor supremo es el tradicionalismo.
\end{enumerate}

Parsons no explica porqué una sociedad tiene una determinada jerarquía de estratos sociales, o porqué en una sociedad predomina un tipo de valores y no otro. Sólo invita a constatarlo, lo que se hace\ldots observando cómo es la jerarquía social establecida, que es precisamente lo que se quería explicar\ldots{}

Si bien el modelo básico de Parsons es de equilibrio, y por consiguiente estático y hasta de inspiración conservadora, hay que reconocer que Parsons relativizó este enfoque al afirmar taxativamente que el ``estado de equilibrio'' es un estado teórico: ningún sistema social real está verdaderamente en equilibrio estático, salvo como ``estado hacia el cual tiende''. Se trata, pues, de un concepto-límite, que marca el sentido final de las oscilaciones re-equilibradoras de los sistemas sociales, cuyo equilibrio verdadero sería entonces dinámico.

Para Parsons, el principal elemento equilibrante,o re-equilibrante del sistema social es el CONTROL SOCIAL, o sea el conjunto de los procesos por medio de los cuales una sociedad impone su dominio sobre los individuos y mantiene su cohesión. Lo opuesto al control social es la DESVIACION, que es la transgresión a las normas del grupo.

La ACCION SOCIAL, en el sistema parsoniano, queda definida por cinco dimensiones o formas de la sociabilidad: - especificidad o generalidad; - afectividad o neutralidad afectiva; - universalismo o particularismo; - cualitatividad; - orientación hacia el individuo o hacia la colectividad.

Sobre la acción social gravitan los VALORES que la gobiernan, el STATUS SOCIAL de sus actores u sus ROLES SOCIALES.

En conclusión, el sistema social concebido por la óptica estructural-funcionalista de Parsons es un conjunto abstracto, simplificado y coherente, que no toma en cuenta la presencia de instituciones o usos sociales capaces de producir consecuencias contradictorias con el modelo vigente. En este sentido es una concepción que puede ser tildada de irreal, ya que no explica satisfactoriamente la presencia evidente de contradicciones internas en los sistemas sociales reales.

Por otra parte, y en forma coherente con lo anterior, el estructural-funcionalismo de Parsons descuida el estudio del dinamismo social. No tiene en cuenta, por ejemplo, los efectos de las estratificaciones sociales sobre el devenir de las sociedades. En este sentido se contenta, bastante superficialmente, con encontrar una relación de armonía o correspondencia entre la estratificación social y las estructuras del sistema social, lo que en la práctica equivale a legitimarlas en cualquiera de sus formas, minimizando las consecuencias de los conflictos que producen los desequilibrios sociales crecientes y la acentuada desigualdad en la posesión y disfrute de los bienes sociales, especialmente cuando no están respaldados por contraprestaciones individuales y grupales de valor equivalente. En alguna forma, las concepciones básicas de Parsons recuerdan al ``optimismo metafísico'' de filósofos como Leibniz y otros racionalistas del siglo XVIII, que llegaron a pensar que vivimos en el mejor de los mundos\ldots posibles.

Es bastante evidente que el estructural-funcionalismo parsoniano ofrece una visión de la sociedad más ``racional'' que las ofrecidas por las teorías basadas en modelos de conflicto, pero a un precio muy alto en cuanto a la correspondencia entre el modelo teórico y la realidad presuntamente representada; quizás por eso mismo no explica satisfactoriamente cómo funcionan esas sociedades cuando sus procesos históricos tienden a desbordar los marcos ``racionales'' en que las teorías pretenden encerrarlas\ldots Aún aceptando que todas las teorías, en última instancia, son incompletas e insatisfactorias, creemos que con justa razón se ha dicho que el estructural-funcionalismo parsoniano explica bien cómo las sociedades perduran, pero no explica cómo cambian\ldots{}

Esta y las anteriores críticas a Parsons no deben ser interpretadas como intentos de negar todo valor a una teoría que, como bien dice Helio Jaguaribe\footnote{Helio Jaguaribe: SOCIEDAD, CAMBIO Y SISTEMA POLITICO Bs. As., Paidos, 1972.}, es ``el intento más amplio que se hizo hasta ahora, para ubicar a la sociedad en un marco analítico general de realidad''. A nuestro criterio, el principal valor de la obra de Parsons no se encuentra en sus concepciones de detalle, siempre susceptibles de crítica y de polémica, sino en su intento de construir una visión general (indudablemente perfectible) de una realidad muy compleja; y especialmente en ``su reconocimiento de la necesidad de entender a la sociedad como un todo estructurado, que presenta relaciones típicas con su medio extrasocial'', reconocimiento que lo llevó a ``superar el esquema weberiano de acción social, orientado a la comprensión de los fenómenos intrasocietales pero no a la ubicación de la sociedad en un marco general de realidad''.\footnote{Helio Jaguaribe: SOCIEDAD, CAMBIO Y SISTEMA POLITICO Bs. As., Paidos, 1972.} A ésto responde el esquema propuesto por Parsons, que considera tres planos de la realidad: el transhumano (la deidad o el lugar analítico de las preocupaciones esenciales del hombre); el humano (compuesto por cuatro sistemas analíticamente distintos: cultural, social, de personalidad y de organismo humano) y el infrahumano, que es el ambiente físico-orgánico del hombre. Los cuatro sistemas del plano humano cumplen las cuatro funciones que todo sistema social debe atender para sobrevivir: mantenimiento de pautas, integración, logro de objetivos y adaptación, que es como decir la institucionalización cultural, la comunidad societal, la función política y la función económica.

Los cuatro subsistemas del sistema social mantienen entre sí constantes intercambios de sus productos -objetos de valor tales como creencias-símbolos, actores-roles-status, órdenes y mercancías- intercambios regidos por un principio de congruencia, en el que cada subistema recibe de los demás algunos de los elementos que necesita para su propio funcionamiento. Ese modelo general ha tenido y tiene indiscutible valor e influencia en el campo de las ciencias sociales, pese a las objeciones de detalle que pueden hacérsele A fin de mostrar qué tipos de estructuras conceptuales pueden construirse dentro de las posibilidades del enfoque estructural-funcionalista-sistémico, en el campo de la Ciencia Política, vamos a exponer a continuación tres ejemplos representativos de esta línea de pensamiento, de indudable repercusión en la teoría política contemporánea. Dos de ellos reconocen una fuerte filiación intelectual proveniente de Parsons y su teoría sociológica: - la teoría del sistema político de David Easton; - el esquema llamado ``de las siete variables'', de Gabriel Almond.

El tercero está más bien enrolado en la corriente de Robert Merton y su enfoque sobre las ``teorías de alcance medio'': - el ``análisis funcional'' de los problemas internacionales, de Morton Kaplan.

\hypertarget{david-easton-y-su-teoruxeda-del-sistema-poluxedtico}{%
\subsection*{David Easton y su teoría del sistema político}\label{david-easton-y-su-teoruxeda-del-sistema-poluxedtico}}
\addcontentsline{toc}{subsection}{David Easton y su teoría del sistema político}

En el campo de la Ciencia Política, el planteo teórico estructural-funcionalista más coherente y sistemático es el de David Easton. La obra de Easton guarda notables similitudes con la de Talcott Parsons, que harían pensar en una filiación intelectual directa, pero no se debe olvidar que una parte de sus fuentes son otras: se trata de la ya mencionada Teoría General de los Sistemas, desarrollada en la Universidad de Michigan con gran influencia de la Biología y de las Matemáticas, y a la que está directamente vinculado el nombre de Ludwig von Bertalanffy.

En forma similar a la de Parsons, Easton busca construir una ``teoría general'' o al menos un esquema general unificado que permita un análisis uniforme y comparable de la vida política en sus múltiples manifestaciones. Easton, al igual que Parsons, se interesa principalmente por la estabilidad y el orden, por los mecanismos que posibilitan la ``persistencia'' de los sistemas políticos en un mundo de cambios y tensiones. Easton tiene una idea muy similar a la de Parsons en lo que se refiere al concepto y función de la teoría. Las principales diferencias son más bien formales y literarias: Parsons es oscuro y de difícil lectura; Easton tiene un estilo claro, directo, fácil de comprender y, por lo tanto, de criticar\ldots{}\footnote{Eugène J. Meehan, op. cit.} En 1953, Easton publicó ``The Political System'', obra en la que hace una revisión crítica del ``estado de la teoría'' politológica e intenta desarrollar un enfoque funcional integral del estudio de la política. Prácticamente todas sus ideas básicas están contenidas (si bien en forma introductoria) en esa obra, que causó un fuerte impacto en el ambiente científico de la especialidad. Easton continuó desarrollando sus ideas en sus obras posteriores, entre las que cabe mencionar ``A framework for Political Analysis'' (1965), ``A Systems Analysis of Political Life'' (1965), ``Varieties of Political Theory'' (1966), ``Children in the Political System'' (1969), entre otras.\footnote{David Easton: THE POLITICAL SYSTEM New York, Alfred A. Knopf, Inc.; 1953 A FRAMEWORK FOR POLITICAL ANALYSIS Prentice-Hall, Inc., 1965 A SYSTEMS ANALYSIS OF POLITICAL LIFE John Wiley and Sons, Inc., 1965 (Hay versión en castellano: ESQUEMA PARA EL ANALISIS POLITICO Bs. As., Amorrortu, 1969) VARIETIES OF POLITICAL THEORY New Jersey, Prentice Hall, 1966 (Hay versión en castellano: ENFOQUES SOBRE TEORIA POLITICA Bs. As., Amorrortu, 1969) CHILDREN IN THE POLITICAL SYSTEM New York, Mc Graw-Hill, 1969.} En su primera obra, ``The Political System'', dejando a un lado los capítulos históricos y de repaso del ``estado de la teoría'', Easton centra su atención en dos aspectos principales: - la búsqueda de una definición de POLITICA que distinga analíticamente de una manera efectiva la actividad política de la que no lo es; - la búsqueda de un modo de combinar el concepto de equilibrio con el de sistema.

La definición de POLITICA se presta a muchas polémicas. Para Easton, POLITICA es todo lo que se refiere a ``la distribución autoritaria de valores'', definición en la que la palabra ``autoritaria'' significa que los miembros de la sociedad aceptan que esa distribución de valores es vinculante. Por otra parte, hay que tener en cuenta que la palabra española ``autoritario'' no traduce exactamente el sentido que en inglés tiene la voz ``authoritative'' , que significa tanto ``autoritario'' como ``autorizado''. De todos modos, es una definición bastante decepcionante, que no permite diferenciar claramente el accionar de un Gobierno del de la Comisión Directiva de un club de fútbol. Por su parte, el tratamiento de los conceptos de SISTEMA y de EQUILIBRIO es muy breve en esta primera obra, en la que es bien notorio que los conceptos planteados provienen de la Ciencia Económica. Esboza allí algunas consideraciones sobre los principios de interdependencia y de unidad funcional,pero sin llegar a desarrollar plenamente la estructura del análisis de sistemas.\footnote{Eugène J. Meehan, op. cit.} En las obras posteriores ya mencionadas se fue completando el cuadro de su vasto planteo teórico pero sin ir, en general, más allá de una fase introductoria. De todos modos, y con todas sus carencias, Easton ha producido uno de los pocos intentos serios y sistemáticos de fundamentar el empleo del Análisis de Sistemas en el campo de la Ciencia Política, y de proporcionar por esa vía una teoría general de la política.

El objetivo general de su trabajo es ambicioso: ``\ldots desarrollar un conjunto lógicamente integrado de categorías, con acusada trascendencia empírica, que haga posible el análisis de la vida política como sistema de comportamiento'', dice Easton, quien se interesa particularmente por un aspecto del sistema de comportamiento:``\ldots los procesos básicos mediante los que el sistema político\ldots puede persistir y mantenerse, tanto en un mundo estable como en un mundo en cambio''. Este enfoque, que prioriza la estabilidad, lo emparenta notoriamente con Parsons.

Los elementos básicos de la estructura teórica de Easton son simples, y sus relaciones son pocas y directas. Es muy probable que esta economía o simplicidad de su modelo contribuya a su atractivo teórico. Hay un sistema (el SISTEMA POLITICO) que opera en un ENTORNO (el ambiente intra y extrasocietal); hay insumos ( las DEMANDAS y los APOYOS) y exumos (las DECISIONES y ACCIONES de las autoridades); hay una REALIMENTACION (o ``feedback'') que mantiene informado al sistema de los resultados de su accionar, y hay un LAZO (o ``loop'') que conecta a las autoridades del sistema político con los miembros del sistema social.

La unidad básica del análisis es la INTERACCION, que surge de la conducta de los miembros del sistema cuando actúan como tales. El SISTEMA es definido por el investigador de acuerdo con sus objetivos. Easton no acepta la idea de que existan ``sistemas naturales'': para él, un sistema es un recurso metodológico, pese a lo cual considera a la vida política como ``un conjunto de interacciones que mantiene su propia frontera y está inserto y rodeado por otros sistemas sociales a cuya influencia está expuesto de modo constante''. Aquí se hace muy evidente la similitud biológica de su modelo y es difícil aceptar que no se refiere a un ``sistema natural''.\footnote{Eugène J. Meehan, op. cit.} El sistema político de Easton trabaja y se mueve según el modelo ``insumo-conversión-exumo-retroalimentación'', que en su formulación general originaria proviene de la Teoría General de los Sistemas, sostiene Domenico Fisichella\footnote{Domenico Fisichella: LINEAMENTI DI SCIENZA POLITICA CONCETTI, PROBLEMI, TEORIE Roma, NIS, 1990.}.``Por''insumo" se entiende la fase en la cual el sistema es sometido a estímulos; por ``conversión'' se entiende el conjunto de los procesos internos durante y mediante los cuales el sistema elabora las respuestas; por ``exumo'' se entiende la fase de emisión de las respuestas hacia el ambiente, y finalmente por ``retroalimentación'' se entiende el conjunto de los efectos de retorno, y por lo tanto de las modificaciones, que las respuestas del sistema producen sobre los estímulos a los cuales él está a su vez sometido".

Si bien los insumos en general se originan en el ambiente, como producto de las múltiples y variadas pulsiones que en él operan (espectativas, intereses, preferencias, ideologías, etc. que hacen plantear necesidades configuradas de determinada forma) también hay, según sostiene Easton, una categoría de influencias que surgen en el interior del mismo sistema político, como consecuencia de la acción de unos actores del sistema político sobre otros: las denomina intra-inmisiones (``withinputs''). Easton considera que deben ser analizadas según la mecánica general de los insumos, y aunque este autor no desarrolla en forma completa esta noción, es notorio que su importancia no puede ser pasada por alto en la comprensión de la dinámica del sistema político, y en particular en el análisis de los procesos de toma de decisiones.

Con las demandas se solicita a las instituciones políticas del sistema para que actúen realizando ``asignaciones autoritarias de valores'' cuando tal asignación no ha podido lograrse por medio de acuerdos privados; por medio de los apoyos se otorga confianza y consenso a esas instituciones. Según Easton, ese apoyo puede aplicarse a diversos niveles: la comunidad política, el régimen político, las autoridades.

Ante esos insumos, el sistema político tiene que realizar una ``conversión'': debe impedir que la sobrecarga de demandas insatisfechas cree tensiones insolubles y que el apoyo se desilusione de una manera insanable. El sistema debe producir exumos ``que estén en condiciones de satisfacer las demandas de al menos una parte de los miembros y de mantener el apoyo de la mayor parte de ellos'', dice Easton. Esos exumos repercuten sobre el comportamiento posterior del sistema.

En este aspecto, nos parece muy importante mencionar una observación del prof. Fisichella\footnote{Domenico Fisichella: LINEAMENTI DI SCIENZA POLITICA CONCETTI, PROBLEMI, TEORIE Roma, NIS, 1990.} quien hace notar que el flujo de insumos y exumos políticos no es pasivo, como el de una instalación hidráulica, ya que ``un sistema político es un sistema que se asigna previamente fines\ldots está constituído por sujetos capaces de anticipar, juzgar y actuar\ldots estos sujetos pueden tratar de corregir aquellos disturbios que podrían presumiblemente causar tensiones'', por lo cual ``las demandas y el apoyo pueden ser modelados según los objetivos y deseos de los miembros del sistema, en los límites de los conocimientos, de los recursos y de las preferencias disponibles''.

Easton reconoce que las ``distribuciones autoritarias de valores'' ocurren un poco en todas partes, en todo tipo de organizaciones, y que hace falta otro criterio más satisfactorio para definir lo político. En su terminología más reciente habla de ``sistemas parapolíticos'', que son sólo sistemas menores, mientras que el ``sistema político societario'' (expresión que verosímilmente puede interpretarse como una manera de aludir al Estado sin nombrarlo) abarca un ámbito más grande, tiene mayores poderes y evidencia una capacidad especial para movilizar recursos y apoyos. Desde entonces, el concepto ``sistema político'' queda reservado para las interacciones importantes, que se refieren a las asignaciones autoritarias de valores dentro de la sociedad tomada en su totalidad. De allí a decir que el sistema político es el Estado hay un solo paso, y en tal caso, el planteo no resulta novedoso. De hecho, tras varias décadas de virtual exclusión del vocabulario politológico, el concepto de Estado ha vuelto a ingresar recientemente en el uso corriente de los politólogos, como el mismo Easton tuvo la honestidad intelectual de reconocerlo.

En este aspecto, un aporte interesante es la propuesta de Domenico Fisichella\footnote{Domenico Fisichella: LINEAMENTI DI SCIENZA POLITICA CONCETTI, PROBLEMI, TEORIE Roma, NIS, 1990.}, quien sugiere considerar al Estado, con sus tres poderes, más la burocracia y otras instituciones públicas, como un subsistema del sistema político, que interactúa con los otros subsistemas: partidario, sindical. etc.

Todo intento de crear un esquema teórico con pretensiones omniabarcativas plantea arduos problemas conceptuales. Easton los enfrenta, si bien con poco éxito a nuestro entender. Por ejemplo,adoptando primero una perspectiva parsoniana, define al ``sistema'' como ``un conjunto de interacciones'', pero luego sus textos muestran expresiones tales como ``\ldots los miembros de un sistema tienen oportunidad de\ldots{}''o ``..un sistema político ha logrado mantenerse\ldots{}'', expresiones en las que parece referirse a conjuntos de personas y no a ``conjuntos de interacciones''.

Asímismo, es frecuente encontrar en sus obras definiciones relacionadas en forma circular, que vuelven al punto de partida. Por ejemplo, define a la POLITICA como distribución autoritaria de valores para una sociedad. Luego define a la PERSISTENCIA de la política como el mantenimiento de esa capacidad distribuidora, y a las TENSIONES como actividades que amenzan dicha capacidad. Esas tensiones son encuadradas por Easton en términos de ciertas ``variables esenciales'' que finalmente resultan ser\ldots la capacidad de una sociedad para distribuir valores entre los miembros de una sociedad y asegurar la aceptación de éstos! En realidad, para Easton como para Parsons, una teoría es más un esquema conceptual general que una explicación de relaciones empíricas. De ello resulta una estructura abstracta, bastante cuestionable desde el punto de vista lógico-formal y poco útil para la investigación empírica, pero, por otra parte, muy valiosa como visualización por medio de un modelo simplificado del conjunto complejo de la vida política y sobre todo de las relaciones entre sistema político y sociedad, o si se quiere, entre Estado y Sociedad. En un gráfico muy simple se lo puede representar así: Entorno ------------------ (Insumos)Demandas \textbar{} \textbar{} Decisiones o (Exumos) -------------\textgreater{} Sistema \textbar{} Políticas Apoyos ---------------\textgreater{} -------------\textgreater{} Político \textbar{} \textbar{} (Conversión) \textbar{} \textbar{} \textbar{} ------------------ \textbar{} Entorno \textbar{} (Retroalimentación) \textbar{} Entorno -------------------------- Por otra parte, Easton también formula observaciones de gran interés cuando se aparta de su esquema teórico y analiza la política occidental.

\hypertarget{gabriel-a.-almond-y-su-teoruxeda-funcional-de-la-comunidad-politica-tambien-llamada-de-las-siete-variables}{%
\subsection*{Gabriel A. Almond y su Teoría Funcional de la Comunidad Politica, tambien llamada ``de las siete variables''}\label{gabriel-a.-almond-y-su-teoruxeda-funcional-de-la-comunidad-politica-tambien-llamada-de-las-siete-variables}}
\addcontentsline{toc}{subsection}{Gabriel A. Almond y su Teoría Funcional de la Comunidad Politica, tambien llamada ``de las siete variables''}

Probablemente sea Gabriel Almond y su escuela quienes han profundizado más el estudio de la política según los esquemas del estructural-funcionalismo, hasta desembocar, como veremos luego, en la perspectiva de la política comparada. Los serios y sistematicos esfuerzos de Almond apuntaron en primer lugar a formular una ``teoría funcional de la comunidad política'' que especifique sus elementos constitutivos básicos y permita lograr ``formulaciones estadísticas y quizás matemáticas''.\footnote{Morton A.Kaplan: SYSTEM AND PROCESS IN INTERNATIONAL POLITICS New York, Wiley, 1957.} Lo que en realidad obtuvo fue un modelo general, un esquema clasificatorio muy interesante, que puede usarse para ordenar y hacer comparables las observaciones de fenómenos políticos de diferentes sistemas: de allí su valor para los desarrollos de la Política Comparada, aunque sea al precio de tener que usar categorías amplias y abstractas, un tanto lejanas del nivel empírico de los fenómenos.

El otro gran aporte de Almond y su escuela fue la inclusión del concepto de CULTURA en estudios originariamente behaviorísticos puros (no debemos olvidar que Gabriel Almond proviene de la ``Escuela de Chicago'' de Charles Merriam), hecho que tuvo un gran impacto en la evolución de la Ciencia Política americana y europea occidental, a tal punto que muchos estudiosos del tema hablan de un antes y un después de la publicación de CIVIC CULTURE (1963). En la nota\footnote{Eugène J. Meehan, op. cit.} damos una lista de las principales obras de Gabriel Almond.

Yendo ya al tema específico de este apartado, diremos que para Almond el sistema político es ``\ldots aquel sistema de interacciones, existente en todas las sociedades independientes, que realiza las funciones de integración y de adaptación (tanto internamente como en relación con otras sociedades) mediante el empleo, o la amenaza de empleo, de una coacción física más o menos legítima''. En esta ecléctica definición, Almond combina, como puede verse, la definición de Estado de Weber, las ideas de Easton sobre la distribución autoritaria de valores y el criterio de Parsons sobre la función social del subsistema político.\footnote{Véase, por ejemplo, Domenico Fisichella: LINEAMENTI DI SCIENZA POLITICA - CONCETTI, PROBLEMI, TEORIE, Roma, NIS, 1990.} Ese sistema es considerado por Almond como característico de las ``sociedades independientes'', oscura expresión que, al parecer, en función del contexto, se refiere a las naciones-estado.

Una de las ventajas que presenta el enfoque estructural-funcionalista-sistémico es que puede trabajar en diversos niveles de análisis. Así, en un primer nivel muy general, Almond considera el funcionamiento del sistema político como ``unidad'' dentro de su ambiente. En ese nivel, se habla de CAPACIDAD del sistema para describir ``la prestación global del sistema en su ambiente'', dato importante para establecer la viabilidad del sistema y sus posibilidades de cambio y desarrollo.

Almond considera la existencia de cinco ``capacidades'': -CAPACIDAD EXTRACTIVA: es la capacidad de procurarse recursos materiales y humanos del ambiente nacional e internacional; -CAPACIDAD REGULATORIA: es el ejercicio del control sobre el comportamiento de los individuos y los grupos, mediante la coerción legítima; -CAPACIDAD DISTRIBUTIVA: es la asignación de bienes, servicios, honores, posiciones y oportunidades de varios tipos, a individuos y a grupos; -CAPACIDAD SIMBOLICA: consiste en la producción de exumos simbólicos eficaces dirigidos al sistema social y al ambiente internacional (afirmaciones de valor, declaraciones de programas o intenciones, ostentaciones de banderas, paradas militares, etc.) -CAPACIDAD RECEPTIVA: es la sensibilidad a los estímulos externos, que permite responder a conjuntos de presiones, internas y/o externas.

En un segundo nivel de análisis encontramos un conjunto de funciones o variables, que son desempeñadas por todos los sistemas políticos (más específicamente por sus estructuras internas) pero no del mismo modo. Son siete en total; cuatro están vinculadas a los insumos y tres a los exumos. De allí viene el nombre de ``teoría de las siete variables'' con que se conocen estos desarrollos. Almond afirma haber determinado esas funciones ``formulando una serie de preguntas basadas en las actividades claramente políticas que existen en los complejos sistemas occidentales'': Funciones del insumo: 1. SOCIALIZACION Y RECLUTAMIENTO POLITICO: es el proceso de asimilación por los individuos de las pautas de su cultura política; el proceso por el cual las culturas políticas son conservadas o cambiadas; y el proceso por medio del cual los roles de los sistemas políticos son cubiertos.

\begin{enumerate}
\def\labelenumi{\arabic{enumi}.}
\setcounter{enumi}{1}
\tightlist
\item
  ARTICULACION DE LOS INTERESES: es el proceso a través del cual los individuos y los grupos formulan demandas a las estructuras decisionales políticas. En esta función actúan cuatro tipos de estructuras: - grupos de intereses institucionalizados (particulares); - grupos no asociacionales (étnicos, religiosos, etc.); - grupos de intereses anónimos (masas); - grupos asociacionales de intereses (sindicatos).
\end{enumerate}

El estilo de actuación de estos grupos puede ser específico o difuso, general o particular, instrumental o afectivo. La estructura y estilo de la articulación de intereses tiene la función de definir los límites entre el sistema político y la sociedad 3. AGREGACION DE LOS INTERESES: es la función de conversión de las demandas en opciones políticas alternativas, mediante la elaboración de plataformas y organizaciones políticas. La agregación de intereses puede lograrse mediante la formulación de propuestas generales que combinen los intereses y mediante el reclutamiento de personas comprometidas en una orientación política determinada. La agregación de intereses puede ser realizada por cualquier institución social, pero su instrumento principal y específico son los partidos políticos.

\begin{enumerate}
\def\labelenumi{\arabic{enumi}.}
\setcounter{enumi}{3}
\tightlist
\item
  COMUNICACION: es la función mediante la cual se trasmiten mensajes e informaciones; es el medio por el cual se realizan las demás funciones, tanto en el sistema político como en la sociedad.
\end{enumerate}

Funciones del exumo: 1. ELABORACION DE NORMAS 2. APLICACION DE NORMAS 3. ADMINISTRACION JUDICIAL DE NORMAS En estos tres casos, es claramente visible la equivalencia de estas ``funciones del exumo'' del sistema político con la ``división tripartita de poderes'' de la teoría constitucionalista clásica.

No podemos dejar este comentario sobre la obra de Almond y su escuela sin hacer alguna referencia a sus aportes al estudio de la cultura política, que complementan magníficamente sus trabajos teóricos sobre los sistemas políticos. Su contribución a la Política Comparada será tratada por separado (ver pg.116). Almond y Powell definen a la cultura política como ``el conjunto de actitudes y orientaciones de los miembros de un sistema político en relación con la política'', incluyendo en ella ``la percepción que los miembros del sistema tienen de los otros individuos y de sí mismos en cuanto actores políticos\ldots{}''.

Almond, en ``Civic Culture'', distingue tres variedades fundamentales: parroquial o comunal, de súbditos, y de participantes. ``Cada tipo de cultura política tiene una estructura política que le es afín: la congruencia máxima se encuentra entre una estructura política tradicional y una cultura política parroquial; una estructura política centralizada autoritaria y una cultura de súbditos; una estructura política democrática y una cultura participante''.\footnote{Ludwig von Bertalanffy: TEORIA GENERAL DE LOS SISTEMAS- FUNDAMENTOS, DESARROLLO, APLICACIONES, México, FCE, 1981.} La cultura política interesa mucho al estudio de los sistemas políticos, sobre todo por tres aspectos principales: el problema de la legitimidad y de las reglas del juego político; el problema de la estabilidad política; y el problema del estilo de la toma de decisiones.

Si bien puede decirse que Gabriel Almond no ha producido una verdadera teoría, en el sentido formal del término, sí ha producido un esquema ordenador de mucho valor y utilidad, que por otra parte ejemplifica bien dos tendencias básicas de la Ciencia Política actual: - la búsqueda de un esquema teórico general ordenador, más que de explicaciones aisladas sobre hechos individuales; - el interés por los estudios de política comparada.-\footnote{Morton A.Kaplan: SYSTEM AND PROCESS IN INTERNATIONAL POLITICS New York, Wiley, 1957.} Gabriel Almond y J. Coleman: THE POLITICS OF THE DEVELOPING AREAS Princeton University Press, 1960\footnote{Eugène J. Meehan, op. cit.} Gabriel Almond y G. Powell: COMPARATIVE POLITICS.

A DEVELOPMENTAL APPROACH Boston,Little Brown and Co., 1966 Hay versión en castellano: POLITICA COMPARADA Bs. As., Paidos, 1972 Gabriel Almond y S. Verba: CIVIC CULTURE Princeton University Press, 1963 THE CIVIC CULTURE REVISITED Boston, Little Brown, 1980 Gabriel Almond: POLITICAL DEVELOPMENT.

ESSAYS IN HEURISTIC THEORY Boston, Little Brown and Co., 1970 Gabriel Almond, Flanagan y Mundt: CRISIS, CHOISE AND CHANGE Boston, Little Brown, 1973\footnote{Véase, por ejemplo, Domenico Fisichella: LINEAMENTI DI SCIENZA POLITICA - CONCETTI, PROBLEMI, TEORIE, Roma, NIS, 1990.} Eugène J. Meehan: PENSAMIENTO POLITICO CONTEMPORANEO Madrid, Revista de Occidente, 1973\footnote{Ludwig von Bertalanffy: TEORIA GENERAL DE LOS SISTEMAS- FUNDAMENTOS, DESARROLLO, APLICACIONES, México, FCE, 1981.} Domenico Fisichella: LINEAMENTI DI SCIENZA POLITICA CONCETTI, PROBLEMI, TEORIE Roma, NIS, 1990 Morton Kaplan y su teoría funcionalista de las relaciones internacionales.

David Easton está teóricamente ubicado muy cerca de Talcott Parsons aunque difiera en su lenguaje y en algunos conceptos. Almond está en rasgos generales en la misma línea, pero no es un funcionalista ortodoxo. Ambos son más organicistas que mecanicistas. El autor cuya obra vamos a comentar ahora es muy diferente. Los trabajos de Morton Kaplan sobre Relaciones Internacionales\footnote{Morton A.Kaplan: SYSTEM AND PROCESS IN INTERNATIONAL POLITICS New York, Wiley, 1957.} evidencian un planteo funcionalista mecanicista, muy formal y preciso, con influencias provenientes del campo de la Ingeniería. Su objetivo es realizar un análisis factorial y construir modelos rigurosos, rehuyendo el enunciado de amplias generalizaciones, lo que lo acerca mucho a Robert Merton y sus ideas sobre ``las teorías de alcance medio''.

En Ingeniería de Sistemas, un SISTEMA es definido como ``un conjunto de variables que puede ser considerado como una entidad definida, destacada sobre un trasfondo dado''. Kaplan usa ese concepto en su definición: ``SISTEMA es un conjunto de variables relacionadas de tal modo que, en contraste con su entorno, las relaciones internas de las variables entre sí y las relaciones externas del conjunto con combinaciones de variables exteriores están caracterizadas por regularidades de comportamiento que pueden ser descriptas''.

Los sistemas son descriptos por Kaplan en términos de ``estados''. Un ``estado'' es una especificación completa (o lo más completa posible) de los valores de las variables del sistema. El comportamiento del sistema se establece con referencia a cambios en las variables (exumos). Los cambios en el entorno que influyen en el sistema son insumos. Los insumos que producen cambios fundamentales en el sistema son denominados ``funciones de escalón''.

El exumo (``output'') de un sistema puede servir de insumo(``input'') a otro. El flujo de emparejamiento puede ser unidireccional o bidireccional. En este último caso hay una realimentación, positiva o negativa.

Los dos ``estados del sistema'' más importantes son el equilibrio y la estabilidad. Un sistema está en equilibrio cuando las variables se mantienen dentro de ciertos límites de variación, durante un período determinado, pese a los cambios que se operen en los exumos. Una observación importante de Kaplan es que un sistema puede ser equilibrado pero inestable o estable pero desequilibrado. El equilibrio, a su vez, puede ser dinámico o estático.

Como es obvio, los sistemas políticos son dinámicos. Kaplan sostiene que en el caso de los sistemas políticos lo importante es saber si los cambios que se producen son reversibles o no. El efecto que los cambios de equilibrio tienen sobre diversas especies de sistemas está gobernado por algunos principios generales: 1) Un sistema equilibrado permanecerá así a menos que se lo perturbe; 2) Un sistema estable pasará a un nuevo estado de equilibrio o desaparecerá por completo si es perturbado con suficiente fuerza; 3) Si la perturbación desaparece y el sistema es incapaz de retornar a su anterior estado, se ha operado un ``cambio de sistema''.

Para el estudio sistémico de las relaciones internacionales, Kaplan sugiere utilizar cinco variables: 1) Las reglas esenciales del sistema: las relaciones entre sus elementos y las funciones de los mismos; 2) Las reglas de transformación: la relación de las reglas esenciales con valores paramétricos determinados; 3) Variables clasificatorias de los actores: características estructurales de los actores o elementos del sistema; 4) Variables de capacidad: referidas, por ejemplo, a la capacidad de un actor dado para realizar determinadas acciones en situaciones específicas; 5) Variables de información: estimaciones sobre la capacidad de los actores y sus aspiraciones.

La impresión de simplicidad que da este corto listado de cinco variables es engañosa: cada una es en realidad una macrovariable que resume muchas variables propias de los subsistemas.

La definición que da Kaplan del sistema político se basa en el concepto de SOBERANIA. ``El sistema político moderno se caracteriza por el hecho de que sus reglas especifican el ámbito de jurisdicción de todas las restantes unidades de decisión y establecen métodos para resolver los conflictos de jurisdicción''. Kaplan considera que la existencia de un gobierno es un síntoma empírico inequívoco de la existencia de un sistema político. Para Kaplan, la POLITICA es la competencia para asumir papeles con funciones de decisión, para escoger entre objetos políticos alternativos o para cambiar las reglas esenciales de los sistemas políticos.

Kaplan establece una distinción muy interesante entre los sistemas políticos ``a dominancia de sistema'' (o sea con predominio del gobierno central) y los sistemas ``a dominancia de subsistemas'' (o sea con predominio de las autonomías regionales).

Por último, Kaplan intenta formular un conjunto de reglas básicas para que los actores decidan entre alternativas, en el transcurso de la interacción, para tratar de alcanzar lo que en la Teoría de los Juegos se denomina ``una estrategia victoriosa'': 1) Actuar para incrementar la capacidad propia, pero preferir la negociación a la lucha; 2) Luchar antes que dejar escapar una oportunidad para incrementar la capacidad propia; 3) Dejar de luchar antes que eliminar a un actor principal, un actor necesario para mantener el equilibrio de poder; 4) Actuar para oponerse a una coalición o a un actor singular que tiendan a asumir una situación de predominio frente al resto del sistema; 5) Actuar para obligar a los actores a que acepten principios organizativos supranacionales; 6) Permitir que los actores nacionales esenciales que han sido vencidos o forzados en algun sentido vuelvan a entrar en el sistema como interlocutores válidos; o actuar a fin de que se incorporen a la categoría de actores esenciales otros que no tenían ese rango. Tratar a todos los actores esenciales como interlocutores válidos\footnote{Eugène J. Meehan, op. cit.}.

\hypertarget{evaluaciuxf3n-cruxedtica-de-la-teoruxeda-sistuxe9mica-poluxedtica}{%
\subsection*{Evaluación crítica de la teoría sistémica política}\label{evaluaciuxf3n-cruxedtica-de-la-teoruxeda-sistuxe9mica-poluxedtica}}
\addcontentsline{toc}{subsection}{Evaluación crítica de la teoría sistémica política}

En el enfoque sistémico convergen todos los esfuerzos intelectuales que configuran las corrientes teóricas que hemos venido describiendo. Al evaluarlo críticamente, pues, en cierto modo evaluamos a todo el grupo.

Si bien la teoría sistémica política no se reduce solamente al trabajo de David Easton, puesto que otros investigadores importantes, como S. Beer, M. Kaplan, H. Spiro, K. Deutsch, G. Almond, etc., también hicieron valiosas aportaciones, es innegable que Easton ha sido el más influyente, el más conocido y, probablemente, el más representativo del grupo, razones por las que vamos a tomarlo como ejemplo paradigmático para esta evaluación.

Quizás esté de más aclarar que de ninguna manera pretendemos erigirnos en jueces de tan importantes y originales estudiosos. Criticar aquí no significa considerarse por encima de ellos, ni tan siquiera a su altura, sino simplemente tomar nota de lo que el tiempo y la evolución posterior de la ciencia mostraron, sin negar el valor de originalidad, de innovación y hasta de audacia intelectual que cada uno de ellos en su momento significó. Pero, como herederos de esos aportes, también tenemos que preguntarnos sobre la utilidad y limitaciones actuales de esas teorías, que son nuestras herramientas para trabajar la dura roca de la realidad\ldots hasta que logremos fabricarnos otras mejores\ldots{}

D. Easton permanece, en lo fundamental, muy cerca de la Teoría Sistémica General. A pesar de que su motivación principal es ``posibilitar la investigación politológica empírica'', uno de los puntos más débiles de su teoría se relaciona justamente con este punto: su difícil, y a veces imposible, operacionalización.

Dos deficiencias recurrentes pueden encontrarse en este aspecto en todo su trabajo: 1) Las aplicaciones empíricas de su marco teórico son dejadas por él para ser realizadas en el futuro y, según se infiere, por terceros ajenos a él, que se animen a hacerlo ante las posibilidades que dicho marco teórico ofrece; 2) Formula muchas definiciones, pero ninguna de ellas resulta plenamente operativa. Easton no indica cómo hacerlas operativas en el momento de realizar una investigación empírica.

Hasta ahora nadie ha logrado hacer una verdadera ``investigación científica empírica sistémica'' de orientación cuantitativa, en el campo político. Esto es debido a que son muchos los impedimentos que deben enfrentar estos intentos, entre los que cabe mencionar: 1) Es prácticamente imposible ``medir'' toda la información, el ruido y la información errónea existentes, no sólo en el sistema en sí sino también en su medio ambiente y en las complejas interacciones entre ambos; 2) La estrategia habitual de todo investigador empírico consiste en orientar su trabajo de investigación sobre un sector de la realidad bien definido y concreto, aislando para su análisis unas pocas variables. La teoría sistémica política pretende abarcar y tomar en consideración todos los aspectos involucrados, lo que evidencia una aspiración holística muy sana como ideal teórico, pero en el plano concreto de la investigación empírica ésto complica mucho las cosas; 3) Si se toman algunos casos concretos, cómo podemos, por ejemplo, investigar empíricamente y con espíritu cuantitativo un insumo? Es un insumo lo que cualquiera desea del Gobierno? Easton contesta que insumos son sólo aquellos deseos que son demandas, es decir, que pueden ejercer cierta presión sobre el Gobierno. Conceptualmente ésto está muy bien, pero qué es esa presión? Cuál es su umbral? Tomando otro caso, se sabe que algunos insumos no proceden del ``entorno societario'' sino del interior del sistema político mismo, como consecuencia de la actuación de algunos subsistemas o actores sobre otros. En cierto sentido, no serían insumos pero actúan como tales en cuanto presionan para conseguir la producción de algún exumo. Easton habla de ``intra-inmisiones'' (``withinputs''), concepto muy importante, que plantea pero que no desarrolla, sobre todo en el aspecto de la relación de esas intra-inmisiones con los insumos societarios propiamente dichos; 4) En el caso de los exumos, el camino de la investigación empírica tampoco queda claro en la teoría de Easton. Por ejemplo, cuando el sistema político produce una ley (supongamos, educación para todos, o derecho a una vivienda digna) pero no hay dinero para ponerla en ejecución, o el dinero que hay es malgastado o malversado, es ésto un exumo o no? Desde el punto de vista de la efectivización material de una política, desde luego que no lo es, pero desde el punto de vista de la generación de espectativas (para ganar tiempo) y de su posterior eventual frustración, sí lo es. Recién en los últimos años, el desarrollo de ese nuevo campo de estudios politológicos que se denomina ``análisis de las políticas públicas'' ha iluminado el tema del estudio de los exumos del sistema político como productos de complejas interacciones entre actores societales y estatales.

\begin{enumerate}
\def\labelenumi{\arabic{enumi})}
\setcounter{enumi}{4}
\item
  En el caso de la ``sobrecarga del sistema'', la teoría sistémica nos dice que la sobrecarga depende de la capacidad del sistema para convertir los insumos en exumos, y que sólo cuando dicha capacidad es superada puede hablarse de sobrecarga. Esto es muy lógico como modelo teórico, pero no nos permite saberlo antes que se produzca. Lo podríamos saber a posteriori, cuando (en términos de política práctica) ya es demasiado tarde. En realidad, ni siquiera en el momento del hundimiento de un régimen político tendríamos la seguridad de que el fenómeno se debe a una sobrecarga, porque podría deberse a otras causas.
\item
  Con respecto al apoyo y a la falta de apoyo, el enfoque sistémico acertadamente reconoce la necesidad de apoyo para todo régimen político, aunque sea autoritario, pero si se quiere hacer una investigación empírica de orientación cuantitativa (que es la aspiración del enfoque teórico que estamos comentando) surgen algunas preguntas: Cómo establecer la cantidad de apoyo que necesita un régimen político? Cómo determinar el punto crítico en la provisión de apoyo? Cómo diferenciar el apoyo espontáneo de la sociedad del promovido desde el régimen? Cómo cuantificar factores como el consenso, los hábitos de obediencia, la coacción o la amenaza de emplearla? 7) Con respecto al ``stress'' y a su resolución, la teoría dice que un sistema reduce el stress mediante cambios. Según la intensidad del stress el cambio puede ser adaptativo (modificaciones dentro del mismo sistema) o disruptivo (disolución del sistema). Es la cantidad de stress en relación con la capacidad del sistema lo que determina qué tipo de cambio se va a producir. Esto implica la existencia de un punto crítico de ruptura del sistema. Cómo determinarlo cuantitativamente? Como ese punto depende de la capacidad del sistema, nuevamente aparece el problema que ya vimos anteriormente: sólo se lo puede determinar a posteriori, o sea cuando ya es demasiado tarde\ldots{}
\item
  Con respecto a los procesos de cambio, cuya consideración protagónica es un grande y novedoso aporte del enfoque sistémico respecto de sus antecesores estructural-funcionalistas, Easton afirma la existencia de dos tipos de cambio sistémico no disruptivo: el ``cambio en un sistema'' (que implica cambios adaptativos en el sistema, o sea respuestas adaptativas ante estímulos corrientes del ambiente); y el ``cambio de sistema'' (que designa transformaciones profundas, cambios estructurales, de desarrollo o involución permanentes). Por otra parte, según Easton, el ``cambio disruptivo'', provoca ``la disolución del sistema, con caída de su neguentropía a cero''.
\end{enumerate}

El tema del cambio, adaptativo o disruptivo, ha sido uno de los aspectos de la teoría sistémica más atractivos para los politólogos americanos y europeos posteriores, y ha suscitado también varios interrogantes. Hasta qué punto puede hablarse del cambio de sistema como un cambio no disruptivo? Se puede, por ejemplo, cambiar de régimen político pero no de sistema? Detrás de estas preguntas está latente el interrogante sobre el verdadero ``nivel de profundidad'' de los cambios políticos, especialmente de las reformas (que pueden ser casi-cosméticas pero también pueden llevar por acumulación a modificaciones estructurales políticamente profundas, como ocurrió, por ejemplo, en Suecia) y de las revoluciones (que son definidas tradicionalmente como ``cambios rápidos, violentos y profundos'', pero que en general no son tán profundos como para modificar aspectos centrales de la identidad nacional de los pueblos que las viven: los franceses siguieron siendo franceses y los rusos, rusos, después de sus revoluciones, e incluso retomaron posteriormente aspectos momentáneamente dejados de lado).

Las lecciones recientes de la historia parecieran indicar que hay niveles culturales de mayor nivel de profundidad que los cambios políticos estructurales, lo que replantea, entre otras cosas, el tema de la primacía de la política.

Los puntos expuestos intentan mostrar el conjunto de dificultades que ha enfrentado la investigación empírica sistémica\footnote{Eugéne J. Meehan, PENSAMIENTO POLITICO CONTEMPORANEO, Madrid, Revista de Occidente, 1973.}. Todos los enfoques sistémicos o sistémico-funcionales presentan similares características, porque son construcciones abstractas muy elaboradas de la realidad política, alejadas del plano empírico, indudablemente útiles para visualizar estructuras globales en su funcionamiento general, pero no tanto para acercarse empíricamente a la explicación de fenómenos puntuales.

Quizás el principal error inicial de estos teóricos fue mantener la impostación cuantitativista que caracterizaba la investigación empírica desde el behaviorismo, mientras que el desarrollo científico posterior mostró la conveniencia de afinar primero el instrumental de los conceptos y los sistemas de conceptos (el uso representacional del lenguaje) antes de intentar un tratamiento cuantitativo de las variables y sus relaciones\footnote{Giovanni Sartori: LA POLITICA - LOGICA Y METODO EN LAS CIENCIAS SOCIALES, México, FCE, 1984.}. De hecho, hoy en día, por obra de muchos autores americanos y europeos, el enfoque sistémico ofrece un elaboradísimo arsenal conceptual, que conserva los grandes esquemas de la teoría pero permite aproximarse mucho más a los fenómenos puntuales, que caracterizan al grueso de la investigación empírica\footnote{Véase, por ejemplo, Domenico Fisichella: LINEAMENTI DI SCIENZA POLITICA - CONCETTI, PROBLEMI, TEORIE, Roma, NIS, 1990.}.

En general, el análisis sistémico parece poner énfasis, bajo una nueva forma, en la necesidad del equilibrio. Tiende a concentrarse más en la explicación de los mecanismos de autorregulación y de preservación de los sistemas políticos que en los procesos de conflicto interno, de contradicciones y de choque de fuerzas contrapuestas internas y externas, que originan el dinamismo de tales sistemas. Tienden a destacar la estabilidad y la permanencia como valores subyacentes al quehacer teórico, a diferencia de otros enfoques que, por el contrario, prefieren describir los conflictos que todo sistema social (y particularmente el sistema político) presenta dentro de su estructura y en su entorno ambiental. En nuestra opinión son dos lecturas indudablemente complementarias pero que en la polémica teórica se presentan como contrapuestas.

Creemos que en este énfasis puesto en el orden, el equilibrio, la autorregulación y la preservación de estados (que está tán cerca de la ideología del statu-quo y del ``optimismo metafísico'' que hace creer que vivimos en el mejor de los mundos\ldots posibles) hay una manifestación de ese fenómeno al que ya hemos aludido varias veces: el condicionamiento de la labor teórica por los marcos cosmovisionales e ideológicos de los autores y de su momento-lugar históricos. En este caso, el fenómeno es particularmente visible porque los ha llevado con frecuencia a pasar por alto algunas posibilidades abiertas por su propio marco teórico general, o sea la Teoría General de los Sistemas. Veamos ésto con mayor detalle.

El enfoque sistémico político parece basarse en que todo sistema perdura por obra de un equilibrio dinámico y por un proceso homeostático. Este concepto proviene de la Teoría General de los Sistemas pero no es válido en todos los casos, y mucho menos en los que trascienden lo puramente biológico, como es el plano psicológico y el social. En realidad, según la teoría del ``organismo como sistema abierto'' (von Bertalanffy), su actuación no es ``un mantenimiento o restauración del equilibrio'' sino, por el contrario, ``un mantenimiento de desequilibrios''.\footnote{Ludwig von Bertalanffy: TEORIA GENERAL DE LOS SISTEMAS- FUNDAMENTOS, DESARROLLO, APLICACIONES, México, FCE, 1981.} El esquema homeostático es especialmente inapropiado como principio explicativo para las actividades humanas que van más allá de la satisfacción de las necesidades primarias de la sobrevivencia, desde el desarrollo de las técnicas hasta el arte, la filosofía y la religión. Qué tienen que ver con el ajuste homeostático o la sobrevivencia, la evolución de la escultura griega, la pintura italiana del Renacimiento o la música barroca alemana? se pregunta von Bertalanffy.

Para describir, explicar y comprender las manifestaciones de la vida humana que trascienden las necesidades primarias (y que en realidad abarcan la mayor parte del quehacer propiamente humano) hay que adoptar, como dice von Bertalanffy, ``un nuevo modelo o imágen del hombre'': es el modelo del hombre como ``sistema activo de personalidad'', que hace hincapié en el lado creador de los seres humanos, en la importancia de las diferencias individuales y en el valor de la solidaridad consciente, o sea en aspectos que no son meramente utilitarios y están más allá de los valores biológicos de subsistencia y superviviencia, comprensibles en términos de homeostasis y de equilibrio elemental.

En contraste con el modelo del organismo reactivo -prosigue von Bertalanffy- expresado por el esquema estímulo-respuesta, es preferible (y más próximo a la realidad) considerar al organismo psico-físico como un sistema primariamente activo. Es acaso ``homeostático'' el hombre de negocios que lleva adelante su frenética actividad? La Humanidad, inventando superbombas, intenta satisfacer acaso necesidades biológicas? se pregunta nuestro autor. Por nuestra parte diremos que, a veces, hemos tenido la impresión de que autores que dicen basarse en la Teoría General de los Sistemas se han saltado la lectura del capítulo que habla sobre los sistemas abiertos\ldots{}

Las dificultades presentadas aquí no invalidan, desde luego, el enfoque sistémico, y sus posibilidades están muy lejos de estar agotadas, pero es bueno tener conciencia de sus limitaciones para intentar superarlas, como viene haciéndose en las últimas décadas con cierto éxito, especialmente al tener en cuenta dos aspectos: por una parte, que el sistema político de las sociedades y el comportamiento político de los hombres están estrechamente relacionados con el sistema de valores sociales fundamentales; y por otra, que la investigación empírica no sólo requiere macromodelos que tracen en grandes rasgos ``visiones de conjunto'' del mundo político sino también conceptos capaces de aferrar aspectos particulares de la realidad. La crisis de paradigma del enfoque estructural-funcionalista-sistémico viene superándose, a nuestro entender con bastante éxito, por medio de la adición de planteos culturalistas y del desarrollo de conceptos y sistemas de conceptos de corto y medio alcance, que hacen más asequible el estudio empírico de fenómenos específicos y los estudios de política comparada.

\hypertarget{el-enfoque-comparatista-o-de-la-poluxedtica-comparada}{%
\subsection*{El enfoque comparatista, o de la política comparada}\label{el-enfoque-comparatista-o-de-la-poluxedtica-comparada}}
\addcontentsline{toc}{subsection}{El enfoque comparatista, o de la política comparada}

Vamos a comenzar el tratamiento de este tema diciendo que, en lo esencial, este enfoque no es ninguna novedad. En Ciencia Política siempre se hicieron comparaciones, desde Aristóteles e incluso antes. Siempre la comparación ha sido fuente de conocimientos y ratificación de juicios y evaluaciones.

Esa tradición remota ha llegado hasta nuestros días y se ha ampliado y consolidado. Es notoria en muchas obras clásicas de la teoría política normativa, desde Santo Tomás a Maquiavelo y a Montesquieu. En las primeras décadas de nuestro siglo también abundaron las obras comparativas de instituciones políticas y jurídicas.

Fue en la década de los cincuenta cuando se produjo una verdadera revolución intelectual en el campo de la Política Comparada. A ella vamos a referirnos con mayor detalle porque allí se origina lo que hoy entendemos por enfoque comparatista.

En la década de los cincuenta, la Política Comparada anterior a la segunda guerra mundial fue objeto de muchas críticas\footnote{G.A. Almond y G.B. Powell: POLITICA COMPARADA, Bs. As., Paidós, 1972.}: Se la acusó de parroquialismo, porque sus estudios se limitaban al mundo anglosajón y europeo continental. Se sostuvo que su enfoque era meramente configurativo y formalista, por cuanto centraba su interés en el estudio comparado de las instituciones y las normas legales. Finalmente, se la acusó de falta de interés por los regímenes que no responden al modelo democrático occidental, al punto de llegar a sostener el carácter ``patológico'' de los regímenes totalitarios.

Según Almond y Powell\footnote{G.A. Almond y G.B. Powell: POLITICA COMPARADA, Bs. As., Paidós, 1972.}, en la década de los cuarenta ocurrieron tres procesos que precipitaron la revolución de la Política Comparada: 1) La explosión nacionalista, expresada en la emergencia de nuevos estados en Medio Oriente, Africa y Asia; 2) La ampliación del poder internacional de los EE.UU. en las áreas ex-coloniales y semi-coloniales anteriormente dependientes de potencias europeas; 3) La aparición del comunismo y de los regímenes del ``socialismo real'' como competidores por la hegemonía mundial.

El nuevo panorama internacional creó nuevas necesidades para la Ciencia Política norteamericana. En Europa, por motivos algo diferentes, ocurrió lo mismo.

H. Eckstein y D. Apter\footnote{H. Eckstein y D. Apter: COMPARATIVE POLITICS. A READER, New York, 1963.} hicieron un aporte complementario muy interesante. Ellos mencionaron como factor detonante externo, el advenimiento a la escena internacional de países con estructura política atípica respecto del modelo constitucional-pluralista de los países occidentales.

También mencionaron la existencia de factores internos a la disciplina: como consecuencia del enfoque etnocéntrico y formalista vigente hasta ese momento se encontraron con que debían afrontar las nuevas necesidades explicativas munidos sólo de conceptos eurocéntricos y de fachada jurídico-institucional; incapaces, por lo tanto, de penetrar la realidad política informal, que suele ser la verdaderamente significativa, y de captar la realidad de sistemas construídos sobre otras bases culturales.

Otro factor que señalan, bastante paradojal, es el exceso de datos informativos, provocado por la expansión de la misma investigación empírica, que aumenta la necesidad de contar con esquemas clasificatorios adecuados, so pena de que junto con la información aumente la confusión.

La ``revolución intelectual'' de la Política Comparada en los años cincuenta se propuso como objetivos\footnote{G.A. Almond y G.B. Powell: POLITICA COMPARADA, Bs. As., Paidós, 1972.}: 1) Adoptar un plan de trabajo más amplio, que escape del parroquialismo y del etnocentrismo; 2) Asumir un mayor realismo, abandonando el formalismo legalista y analizando prioritariamente las estructuras y procesos involucrados en el quehacer político concreto; 3) Buscar una mayor precisión, por la vía del empleo de estadísticas, análisis de factores y correlaciones, encuestas, análisis cuantitativos y modelos matemáticos; 4) Construir un nuevo orden intelectual, estructurado con nuevos conceptos y relaciones ``capaces de viajar'' entre sistemas nacionales diferentes.

Estos objetivos se relacionan, a su vez, con una nueva visión de la comunidad mundial de estados nacionales, a la que ya no se ve como un conjunto de entidades aisladas en un contexto de anarquía parcialmente neutralizada por tenues relaciones inter-partes, sino como un sistema en sí misma, con intensas interacciones entre todos sus elementos componentes. También se relacionan con una apreciación más clara de la influencia y el impacto que ese sistema internacional tiene en la estructura y procesos de la política interna de cada estado nacional.

La propuesta de Almond y Powell es denominada por ellos ``enfoque funcional de la política comparada''\footnote{G.A. Almond y G.B. Powell: POLITICA COMPARADA, Bs. As., Paidós, 1972.} y plantea un conjunto de relaciones intra e intersistémicas, relaciones de interdepedencia (no necesariamente de armonía, aclaran). Estas relaciones se expresan en funciones, según un esquema que parte de la antigua teoría de la división de poderes, del siglo XVIII, pero actualizada y puesta al día según un esquema que presentaremos a continuación. Se supone que dicho esquema puede ser utilizado en estudios comparativos entre sistemas diferentes, a partir de una idea básica: que estas funciones siempre se realizan, aunque varíe la forma de realizarlas por los diversos sistemas: FUNCIONES \textbar{} MANTENIMIENTO DEL SISTEMA \textbar{} \textbar{} ADAPTACION \textbar{} SOCIALIZACION \textbar{} \textbar{} RECLUTAMIENTO \textbar{} \textbar{} CONVERSION \textbar{} ARTICULACION DE INTERESES \textbar{} \textbar{} COMBINACION DE INTERESES \textbar{} \textbar{} COMUNICACION \textbar{} \textbar{} LEGISLACION \textbar{} \textbar{} APLICACION \textbar{} \textbar{} ADJUDICACION \textbar{} \textbar{} INTERACCION CON EL CONTEXTO \textbar{} INTERNO \textbar{} \textbar{} INTERNACIONAL \textbar{} CAPACIDAD DEL SISTEMA Este ``enfoque funcional'' se completa con una propuesta de clasificación de los sistemas políticos (según su grado de diferenciación estructural y de secularización cultural) de acuerdo al siguiente esquema\footnote{G.A. Almond y G.B. Powell: POLITICA COMPARADA, Bs. As., Paidós, 1972.}: I. SISTEMAS PRIMITIVOS (estructuras políticas intermitentes) A. Bandas primitivas (bergdama) B. Sistemas segmentarios (nuer) C. Sistemas piramidales (ashanti) II. SISTEMAS TRADICIONALES (estructuras políticas diferenciadas) A. Sistemas patrimoniales (uagadugu) B. Burocracias centralizadas (Incas, Etiopía) C. Sistemas políticos feudales (Francia siglo XII) III. SISTEMAS MODERNOS (infraestructuras políticas diferenciadas) A. Ciudades-estado secularizadas (Atenas) Diferenciación limitada B. Sistemas modernos movilizados Elevada diferenciación y secularización 1. Sistemas democráticos Autonomía de los subsistemas-cultura de participación a) Elevada autonomía de los subsistemas (Gran Bretaña) b) Limitada autonomía de los subsistemas(IV Rep.Fcesa) c) Escasa autonomía de los subsistemas (México) 2. Sistemas autoritarios Control de los subsistemas-cultura de súbdito a) Totalitarismo radical (URSS) b) Totalitarismo conservador (Alemania nazi) c) Autoritarismo conservador (España franquista) d) Autoritarismo modernizante (Brasil) C. Sistemas modernos premovilizados Limitada diferenciación y secularización 1. Autoritarismo premovilizado (Ghana) 2. Democracia premovilizada (Nigeria) La obra citada termina con un esbozo de una teoría del desarrollo político, construída en base a tres variables interrelacionadas: la diferenciación estructural, la autonomía de los subsistemas y la secularización (ver Cap. 10).

Desde la aparición de los trabajos de Almond y Powell sobre la materia, el enfoque de Política Comparada\footnote{Ver el artículo ``Política Comparada'' de G. Urbani, en el DICCIONARIO DE POLITICA, de Bobbio y Matteucci (comp.), México, Siglo XXI, 1986.} puede ser considerado bajo dos aspectos complementarios: como campo y como método. En alguna medida, se trata de responder a dos preguntas clásicas: qué cosa comparar? y cómo comparar? El enfoque comparativo como campo es el conjunto de las observaciones y estudios realizados por los politólogos sobre fenómenos similares en muchos países (o por extensión, en diferentes regiones de un mismo país). Abarca desde la simple compilación de ``inventarios paralelos'' de datos relativos a dos o más países, hasta el establecimiento de ámbitos de validez de las generalizaciones referidas a conjuntos de fenómenos políticos, sobre la base de efectuar comparaciones entre países o entre regiones de los mismos con diferencias de régimen político.

La comparación como campo puede significar: 1) Una investigación no viciada por prejuicios etnocéntricos; 2) Una confrontación analítica de las instituciones políticas de diversos países o regiones, y especialmente de sus estructuras constitucionales; 3) Una comparación de las funciones desempeñadas por las distintas estructuras políticas en los distintos países. Este es el sentido más cercano a los planteos de Almond y Powell.

El enfoque comparativo como método significa la utilización de un método de control -la comparación- en la verificación o falsación empíricas de las hipótesis, generalizaciones o teorías. Se trata, en definitiva, de un procedimiento de confrontación empírica de los conceptos.

La comparación como método es, pues, un aporte a la controlabilidad empírica de los fenómenos políticos. En ciencias sociales hay cuatro procedimientos básicos de control: experimental, estadístico, comparativo e histórico. El método comparativo es el procedimiento al que la Ciencia Política puede más fácilmente recurrir.

La experimentación sería lo ideal pero no es casi nunca posible, y no sólo por motivos éticos: solo podría darse en muy pocos casos, en los que las variables resulten manipulables y las condiciones generales estén bajo control. El procedimiento estadístico también es poco aplicable, por la frecuente falta de cuantificación de las variables y la escasez de casos analizables. El método histórico, finalmente, es congruente con la investigación politológica en su momento ``nomotético'' o de generalización, pero no lo es en su momento ``idiográfico'' o de restauración de la individualidad del hecho. El procedimiento comparativo queda, pues, como el más adecuado, en la mayoría de los casos, para la Ciencia Política.

La importancia del enfoque comparativo en Ciencia Política está ampliamente reconocida en la literatura sobre la materia. Ya en 1954, S.E. Finner\footnote{S.E. Finner: ``Metodo, Ambito e Fini dello Studio Comparato dei Sistemi Politici'', en STUDI POLITICI, 1, III, 1954.} decía: ``La Ciencia Política debiera ser sobre todo comparada, mientras los otros tipos de análisis debieran tener un rol secundario\ldots{}''.

En 1967, Giovanni Sartori\footnote{Giovanni Sartori: ``La Scienza Politica'' en IL POLITICO, 4, XXXII, 1967.} afirmaba que ``\ldots la esencia de la Ciencia Política parece reconducirnos a la política comparada\ldots{}'' y más adelante agregaba ``\ldots podemos ser acusados\ldots de insistir mucho sobre la comparación, sobre el método comparado\ldots{}'' En un libro reciente\footnote{L. Morlino (comp.): GUIDE AGLI STUDI DI SCIENZE SOCIALI IN ITALIA - SCIENZA POLITICA, Torino, Edizioni della Fondazione Giovanni Agnelli, 1989.} compilado por L. Morlino, el citado autor se pregunta el porqué de esta insistencia en la comparación, y responde: ``\ldots porque la comparación parece el modo más coherente de hacer Ciencia Política según los cánones prefijados: - proceder por hipótesis y verificación; de donde la gran importancia de la elaboración teórica, pero también la del control empírico; - aprovechar la mejor oportunidad, si no la única, de explicación por la existencia de varios casos; - aprovechar la mejor oportunidad de mostrar la aplicabilidad del análisis en Ciencia Política''.

Y más adelante, agrega: ``Como he sugerido recién, y ahora insisto explícitamente, la comparación (o sea el conocimiento del fenómeno estudiado en países diversos, o de muchos fenómenos similares en el mismo país) es, habitualmente, un modo particularmente útil e importante de alcanzar una mejor comprensión-explicación del fenómeno mismo: entiendo, aún del fenómeno singular en su especificidad''.

Pero también aclara y limita el alcance de su juicio: ``\ldots de esta afirmación a aquella otra, extrema, para la cual `no hay Ciencia Política si no es comparada' hay una notable distancia, que yo no estaría dispuesto a recorrer completamente. De todos modos, existe una importante cantidad de investigaciones sobre fenómenos políticos aislados, que no pueden ser ignorados ni descartados demasiado fácilmente''.

En el ámbito de los estudios políticos comparados se presentan con cierta frecuencia algunos inconvenientes metodológicos, casi siempre vinculados con el problema de querer ``comparar lo incomparable'', consciente o inconscientemente. G. Urbani\footnote{G. Urbani: LA POLITICA COMPARATA, Bolonia, 1972.} ennumera algunas pautas de procedimiento para allanar esos inconvenientes: - comenzar con una buena clasificación, para asegurar el orden y la homogeneidad de los fenómenos; -usar conceptos ``capaces de viajar'' (aplicables en distintos países) y a la vez buenos colectores de hechos. Se trata de lograr un buen equilibrio entre requerimientos opuestos: generalidad y relevancia empírica. Se trata de evitar conceptos tán amplios que sean inespecíficos, o tán específicos que impidan comparar países diferentes; - tomar muy en cuenta la incidencia de los contextos socio-políticos de los países sobre los fenómenos comparados; - usar del modo más racional y productivo todas las técnicas de investigación conocidas.

Las posibilidades de aplicación y las perspectivas futuras de la Política Comparada son muy grandes: es un enfoque que aumenta el grado de validez de los conocimientos y la oferta de nuevas hipótesis significativas. Por otra parte, es un modo de pensar que aumenta las posibilidades de aprender de los demás, y disminuye los riesgos de experimentar a ciegas.

Prácticamente todos los campos especializados de la Ciencia Política son susceptibles de tratamiento comparativo: sistema político, partidos y sistemas de partidos, grupos de presión, técnicas decisionales, parlamentos, procesos judiciales, cultura política, socialización política, etc.

Por supuesto, es un enfoque que no está exento de objeciones\footnote{R.T. Holt y E. Turner: THE METHODOLOGY OF COMPARATIVE RESEARCH, Nerw York, 1970.}, que van desde la inmadurez metodológica, que no puede negarse pero que sólo se puede superar por medio de la aplicación, hasta el consabido argumento de que toda comparación es vana porque los fenómenos son irrepetibles.

A ésto puede contestarse que la ``unicidad'' de un fenómeno sólo puede comprobarse de manera seria\ldots por comparación, y que ésta, las más de las veces, revela la existencia en el fenómeno de aspectos irreductiblemente propios (que efectivamente no son comparables) y de aspectos comunes, respecto de los cuales la comparación hace posible un mejor conocimiento y ubicación en relación con otros fenómenos similares.

\hypertarget{tercera-parte-2}{%
\section*{Tercera parte}\label{tercera-parte-2}}
\addcontentsline{toc}{section}{Tercera parte}

\hypertarget{las-explicaciones-de-base-psicoluxf3gica-individual}{%
\subsection*{Las explicaciones de base psicológica individual}\label{las-explicaciones-de-base-psicoluxf3gica-individual}}
\addcontentsline{toc}{subsection}{Las explicaciones de base psicológica individual}

Con frecuencia, el modo de pensar occidental moderno presenta un condicionamiento ideológico-cultural que le hace sentir un interés predominante (y hasta excluyente) por las raíces y facetas psicológicas individuales de la conducta social. En consecuencia, abundan las descripciones y explicaciones de base psicológica individual respecto de hechos económicos, sociales y políticos, tanto en el lenguaje de la conversación corriente como en los lenguajes científico y filosófico.

Ese rasgo ideológico suele denominarse ``individualismo''. Tiende a ver a la sociedad como simple suma de entidades individuales atomizadas, y atribuye importancia dominante a la iniciativa individual como fuente del dinamismo social, y a las relaciones interpersonales. Aportan también a ese contexto ideológico-cultural el racionalismo, el empirismo, el materialismo, el biologicismo y el mecanicismo.

En esa óptica, parecería imposible estudiar la sociedad o la política sin apelar a la Psicología, e indudablemente es verdad que ésta es una fuente insoslayable de conocimientos para comprender lo social. En ese sentido, el aporte de la Psicología a la Ciencia Política es muy interesante y valioso, pero no excluyente ni suficiente por sí mismo. La interacción social produce ``algo más'' que la simple suma de las entidades psicológicas actuantes, de modo que la Psicología (y en particular, sus enfoques ``individualistas'') no basta. Esto no significa negar su importancia, indudablemente grande aún en una valoración crítica.

Sin embargo, observamos un hecho curioso: los científicos sociales hacen poco uso de las teorías psicológicas sistematizadas y apelan con frecuencia a nociones psicológicas ``de sentido común'', parcializadas, simplificadas, a veces ingenuas e incluso ambiguas. Es innegable, por otra parte, que el panorama de la teoría psicológica está muy lejos de ser claro. Existen no menos de cinco corrientes teóricas (doce, según algunos autores) y son poco compatibles y hasta contradictorias entre sí.

De todos modos, en el campo de la Ciencia Política, entendida en sentido amplio, es frecuente que se tomen conceptos o enfoques psicológicos (muchas veces separados de su contexto originario) para explicar fenómenos políticos. Esto es particularmente frecuente en la llamada ``corriente de la crítica social''.

Las teorías psicológicas más conocidas y usadas en estudios políticos son: - la Psicología del Estímulo-Respuesta; - la Psicología de la Gestalt; - la Teoría del Campo (``Field Theory''); - el Freudismo ortodoxo; - el Neofreudismo (versión sociologizada del psicoanálisis).

Estos enfoques no aparecen hoy como ``tipos puros''. La Psicología -como otras ciencias de nuestro tiempo- tiende a combinar elementos de diverso orígen y a hacerse más ecléctica.\footnote{Eugène J. Meehan: PENSAMIENTO POLITICO CONTEMPORANEO, Madrid, Revista de Occidente, 1973.}

\hypertarget{la-psicologuxeda-del-estuxedmulo-respuesta}{%
\subsection*{La Psicología del Estímulo-Respuesta}\label{la-psicologuxeda-del-estuxedmulo-respuesta}}
\addcontentsline{toc}{subsection}{La Psicología del Estímulo-Respuesta}

Dentro de la gran corriente del conductismo, que invoca la tradición empírico-asociacionista, de raíz cultural anglosajona, a la que puede vincularse los nombres de J. Stuart Mill y Ernst Mach, la Psicología del Estímulo-Respuesta comienza con los estudios sobre condicionamiento de Iván Petrovich Pavlov (1849-1936), a partir de 1901, que culminan con su ``teoría del reflejo condicionado''. En esos primeros años del siglo también se dedicó mucha atención a la naturaleza de la memoria y a los procesos de aprendizaje, tanto en animales como en seres humanos, como por ejemplo los trabajos de E.L. Thorndike (1874-1949) y Robert M. Yerkes (nacido en 1876). Pero la obra de este enfoque que alcanzó mayor importancia e influencia en el ámbito de las ciencias humanas fue la de John Watson.

John Watson (1878-1958) fue un psicólogo experimental norteamericano, profesor en Chicago y luego en Baltimore, fundador del behaviorismo, conductismo o psicología del comportamiento, corriente surgida como una reacción contra el uso de la introspección, habitual en la psicología experimental tradicional, y como un intento de liberar a la Psicología de las limitaciones del mentalismo y del instintivismo.

Watson consideraba que una investigación científica sólo puede fundarse en el estudio de hechos observables: un estímulo que produce una respuesta. Los hechos de conciencia, en su opinión, no pueden ser objeto de un estudio científico objetivo.

El conductismo es radicalmente empírico, orientado al experimento y la cuantificación, eminentemente práctico en su finalidad última y vinculado al proceso educativo, al punto de plantear toda una teoría del aprendizaje. Su objetivo es explicar la conducta del organismo en términos de un estímulo observable (S) y de una respuesta observable (R). Busca establecer la frecuencia con que S y R se relacionan en la experiencia del organismo, y el lapso de tiempo que transcurre entre S y R. No niega la subjetividad pero la ignora a los fines de la investigación científica. Por otra parte, postula un cierto reduccionismo fisiológico.

La relación S-R es insuficiente para explicar la compleja conducta del organismo. Pronto se le añadieron otras nociones, como la de IMPULSO (``Drive''), que es un ímpetu innato, muy parecido a la vieja noción de instinto. También se introdujo el concepto de RETRIBUCION, que es el efecto que tiene sobre el organismo que realiza la acción su propia conducta.

El conductismo, pues, conceptualiza la conducta sobre la base de la idea de que existe un organismo sujeto a estímulos y capaz de dar respuestas. El organismo en sí no es objeto de ningún postulado, y toda conducta es considerada como efecto de algún condicionamiento externo.

Algunos investigadores posteriores consideraron necesario postular, al menos, algunas capacidades interiores del organismo, evolucionando desde el simple esquema originario S-R a un esquema S-O-R (estímulo-organismo-respuesta) que caracteriza al neoconductismo.

En realidad, la Psicología conductista tiene poco que decir sobre las relaciones sociales, pese a que la conducta humana es indudablemente social, aprendida y no innata. Los conductistas se han centrado, un tanto artificiosamente, en el estudio del acto de aprendizaje aislado, del individuo aislado. Con esas limitaciones, no es extraño que las aplicaciones en el campo de las ciencias sociales, incluída la Ciencia Política, sean escasas. Citaremos algunos ejemplos: Clark Hull había anticipado su intención de escribir un libro sobre las interacciones entre organismos, pero falleció antes de poder terminarlo. En su obra ``A Behavior System''\footnote{ Clark L. Hull:A BEHAVIOR SYSTEM, Yale University Press, 1952.}, estructurada de un modo rígidamente deductivo, en base a 17 postulados de los que deduce 133 teoremas, muchos a su vez con corolarios, solo uno de esos teoremas se refiere a la interacción social, y su utilidad en el campo de la Ciencia Política es por lo menos discutible: ``Para ser repetida de modo sostenido, toda interacción social voluntaria ha de producir un refuerzo sustancial de la actividad de cada una de las partes''.

Edward C. Tolman planteó un modelo psicológico afín con el enfoque estructural-funcionalista de Talcott Parsons, referente a la teoría social\footnote{Edward C. Tolman: ``A Psychological Model'' en Talcott Parsons y Edward Shils (eds.): TOWARD A GENERAL THEORY OF ACTION, Harvard University Press, 1962.}. Presenta cierto interés pero se trata de un modelo más heurístico que explicativo.

Algunos psicólogos sociales han intentado adoptar un enfoque puramente conductista, con resultados en general decepcionantes. Tal es el caso, por ejemplo, de John Dollard y Neal E. Miller, autores de una teoría de las relaciones interpersonales basada en postular que el individuo posee una mínima capacidad para aprender y retener nociones relacionales, claves-guía de la conducta motivada por estímulos externos. En su formulación, la conducta es motivada por estímulos, orientada por claves relacionales y produce respuestas retributivas que reducen el impulso y reconducen al equilibrio u homeóstasis\footnote{John Dollard y Neal E. Miller: PERSONALITY AND PSYCHOTHERAPY: AN ANALISYS IN TERMS OF LEARNING, THINKING AND CULTURE, McGraw-Hill Book Co., 1950.}. Esta teoría ha sido muy criticada por su notoria insuficiencia para explicar la conducta humana.

Distinto es el caso de la obra de Carl I. Hovland y su equipo de investigadores de Yale, que constituye la más sistemática exposición de conocimientos sobre técnicas inductoras de cambios de actitud, tema de obvio interés politológico\^{}\{Ver entre otros, Carl I. Hovland et al.: COMMUNICATION AND PERSUASION: PSYCHOLOGICAL STUDIES OF OPINION CHANGE, Yale University Press, 1953.\}.

También presenta cierto interés politológico la obra de George C. Homans\footnote{George C. Homans: SOCIAL BEHAVIOR: ITS ELEMENTARY FORMS, Harcourt, Brace and World Inc., 1961.} sobre los comportamientos grupales, que aparecen vinculados con: - retribuciones en el pasado; - frecuencia de las retribuciones; - calidad de las retribuciones; - satisfacción con el tratamiento social; - beneficios decrecientes de las relaciones interpersonales.

Uno de los pocos estudios políticos que se basan explícitamente en la psicología conductista es ``Political Participation'' de Lester W. Milbrath\footnote{Lester W. Milbrath: POLITICAL PARTICIPATION, Rand Mc Nally and Co., 1965.}.

Milbrath estudia la participación o implicación en la política, definida operacionalmente por medio de acciones tales como votar, discutir de política, portar emblemas, hacer peticiones, hacer propaganda, aportar dinero a fondos electorales, buscar cargos políticos, etc.

Usa el concepto de PREDISPOSICIONES (quizás, otro modo de designar al ``impulso''). Considera que el REFUERZO (o retribución) es la causa del vigor de las predisposiciones políticas, de las creencias y actitudes que llevan a acciones de participación política. El circuito de una actuación política continuada se monta, pues, según Milbrath, como la consecuencia hedonista de una sucesión de gratificaciones.

Milbrath ignora la posibilidad de que la acción política pueda aparecer sin estímulo externo: por el contrario, considera necesario que la acción política vaya precedida de un estímulo importante. Deja sin explicar algo fundamental: qué es lo que hace que determinado estímulo (S) sea ``político'', o tenga repercusiones en forma de conductas políticas. Debemos conformarnos con la mención de un factor subjetivo: la actitud individual.

Como crítica general al enfoque metodológico conductista, puede decirse que es muy limitada su aplicación al estudio del hombre. La parte más sustancial de su contenido queda fuera de su rígido esquema S-R. Por otra parte, los hombres viven en un medio muy complejo y resulta casi imposible definir qué S produce determinada R. No toma en cuenta las interacciones entre diversos S y diversos R, y el rol -sin duda relevante- de los condicionamientos culturales y de las espectativas.

Por otra parte, no puede explicar la ACCION CREADORA. Trata de hacerlo apelando a explicaciones basadas en mecanismos de imitación presente y diferida, pero es evidente la existencia de situaciones en las que los hombres actúan de modo tal que es imposible que hayan aprendido su conducta por imitación.

\hypertarget{la-psicologuxeda-de-la-gestalt-la-teoruxeda-del-campo-y-la-dinuxe1mica-de-grupos}{%
\subsection*{La Psicología de la Gestalt, la Teoría del Campo y la Dinámica de Grupos}\label{la-psicologuxeda-de-la-gestalt-la-teoruxeda-del-campo-y-la-dinuxe1mica-de-grupos}}
\addcontentsline{toc}{subsection}{La Psicología de la Gestalt, la Teoría del Campo y la Dinámica de Grupos}

La ``Gestaltpsychologie'', o Gestaltismo, o Psicología de la Forma, es una teoría psicológica sobre la percepción, que se opone al ``asociacionismo'' de la psicología clásica, o sea esa doctrina según la cual el principio general del desarrollo de la vida mental es la asociación de ciertos estados de conciencia elementales, lo cual llevaba a plantear la investigación psicológica, por vía del estudio analítico, como un ``desmenuzamiento'' del psiquismo. Frente a esa concepción ``asociacionista'', el Gestaltismo plantea un enfoque netamente holístico.

El orígen del Gestaltismo es alemán. En 1891, Ehrenfels hizo las primeras descripciones de inspiración gestáltica. Helmholtz, Mering, Wertheimer, Köhler\footnote{Wolfgang Köhler: PSYCHOLOGIE DE LA FORME, Ed. Gallimard, Col. ``Idées''.}, Koffka y Lewin lo desarrollaron en Alemania y luego en los EE.UU., tras su forzada emigración. Guillaume lo introdujo en Francia. Con el tiempo, alcanzó difusión mundial, y gran influencia en las ciencias sociales, así como en la Estética y en la Crítica del Arte.

Su punto de partida es la experiencia humana consciente, el aspecto interno o subjetivo de la conducta humana. Rechaza al positivismo, por considerarlo inapropiado para el estudio de la conducta humana y recurre a la tradición filosófica fenomenológica: a Kant, Dilthey y sobre todo a Edmund Husserl.

El Gestaltismo se basa, pues, en una reflexión fenomenológica sobre ``lo vivido'' y afirma que, en la percepción humana, la totalidad es vivida antes que las partes que la forman, y que el valor de cada parte depende de su participación en el conjunto. La ``Gestalt'' (o ``forma'') es justamente el modo en que las partes se encuentran dispuestas en el todo.

El Gestaltismo procura desarrollar estudios significativos sobre la conducta humana. Podemos sintetizar sus criterios básicos en los siguientes enunciados: - hay que considerar al hombre como una entidad indivisible; - no hay que descomponer analíticamente la conducta ni el psi- quismo; - la acción del cerebro desarrolla un complejo campo de interre- laciones en contínuo fluir; - el hombre percibe su entorno en forma de unidades complejas e integradas, o sea como ``gestalt'' o formas totalizadas, con pau- tas estructuradas y organizadas; - la formación de esas estructuras depende de factores tales como - la similitud de los elementos presentes; - la proximidad, contigüidad, etc.; - la dirección: orígen, trayectoria, destino; - la percepción está regida por dos leyes: - la ``ley de cierre'', según la cual el observador humano tiende a cerrar o ``completar'' las pautas parciales o fragmentadas; - la ``ley de concisión'', según la cual el hombre tiende a estructurar sus percepciones según la forma más simple y ``mejor''; - el enfoque gestaltista es hedonista y teleológico, y concede mucha importancia a las operaciones de integración y reorgani- zación de la experiencia (``insight''); - hay tres tipos básicos de aprendizaje: - mediante condicionamiento; - mediante ensayo y error; - mediante la reagrupación de la experiencia en una rela- ción de medio a fin (``insight''). Este tercer tipo de a- prendizaje es objeto privilegiado de estudio por parte del gestaltismo.

La Psicología de la Gestalt ha tenido gran influencia en los estudios de Psicología Social; ha sido en cambio poco utilizada por los politólogos. Al final de este apartado analizaremos los casos más conocidos y los enfoque más prometedores.

La TEORIA DEL CAMPO (``Field Theory'') es principalmente obra de uno de los creadores de la ``Gestaltpsychologie'', Kurt Lewin (1890-1947), psicólogo alemán emigrado a los EE.UU. cuando se produjo el advenimiento del nazismo. Fue profesor en Berlin y luego en varias universidades norteamericanas.

Según sus propias palabras, ``\ldots difícilmente cabe llamar teoría a la teoría del campo\ldots más exacto es denominarla método\ldots un método para analizar las relaciones causales y erigir construcciones científicas''\footnote{Kurt Lewin: FIELD THEORY IN SOCIAL SCIENCE, Dorwin Cartwrigth (Harper and Bros.), 1951.}.

Kurt Lewin parte de un enfoque gestaltista ortodoxo, tomando en consideración la situación total del individuo, su ``espacio vital'', que es psicológico, cercado por el entorno físico (con el que interactúa) y definido en términos de presente. Para armar su modelo (con un sentido más descriptivo y heurístico que explicativo) toma muchas ideas y elementos del lenguaje de la Geometría Topológica y del Análisis Vectorial, pero no los combina en una estructura matemática formal, sino que los usa libremente, de acuerdo a sus necesidades.

Lewin creó la noción de CAMPO PSICOLOGICO para explicar la interacción de las fuerzas que emanan del sujeto y las influencias sociales. El campo psicológico es una ``totalidad dinámica'' que manifiesta el estado relacional de una persona con su entorno social en un momento determinado. Incluye percepciones y motivaciones. Cada situación combina influencias que generan estados de tensión, los que provocan nuevos comportamientos, en procura de nuevos estados de equilibrio.

El campo psicológico es, pues, un asiento de fuerzas y tensiones que se forman, se modifican y se reequilibran contínuamente. Un hombre dinámico (por ejemplo, el líder de un grupo) puede, con sus propias fuerzas, reorganizar las influencias sociales de su campo psicológico. Otros hombres, más pasivos, pueden evidenciar tendencias adaptativas a las tensiones, en diversas modalidades (positivas o negativas) tales como el aprendizaje, la adaptación y la frustración.

Otra noción importante de Kurt Lewin (de indudable interés sociológico y politológico) es la noción de NIVEL DE ASPIRACION, o sea la posición futura que un hombre se siente capaz de alcanzar cuando va a emprender una nueva actividad. Resultan muy interesantes sus observaciones sobre las modificaciones que sufren esas aspiraciones sobre la marcha, a medida que se experimentan triunfos y fracasos, según las diversas configuraciones psicológicas.

El espacio vital del hombre está dividido en ``regiones'', que son áreas situacionales diferenciadas, que van emergiendo al nivel de la conciencia a medida que el hombre se desarrolla. Esas regiones psíquicas están vinculadas y a la vez separadas entre sí por fronteras, que eventualmente pueden convertirse en barreras.

Sobre ese modelo topológico, más bien estático, K. Lewin introduce el dinamismo psicológico por medio de ``vectores'' que indican los movimientos de aproximación o alejamiento de la persona, de acuerdo a las valencias (positivas o negativas) de esas regiones.

La personalidad, en este modelo, es un ``sistema de regiones''; su diferenciación individual se explica en términos de cambios de región, de fuerza de vectores, de situación de fronteras, etc.

El dinamismo psíquico busca el equilibrio, la reducción de tensiones. Las tensiones incitan a abrir vías a través de las regiones, hacia objetivos determinados. El logro de un objetivo produce equilibrio. Si el objetivo no se alcanza, el desequilibrio persiste hasta que aparece otra tensión, que abra otro curso de acción. Esos objetivos pueden ser perseguidos de manera realista o irreal. Los esfuerzos frustrados pueden llevar a las personas a hundirse en la depresión o a huir hacia lo fantástico, en un desplazamiento psíquico ``sustitutivo'' o ``imaginario''.

El esquema conceptual de Kurt Lewin es muy complejo, rico y fecundo en sugerencias. Puede ser criticado porque define el campo psicológico en términos de presente, ignorando o pasando por alto la historia del individuo, a diferencia del freudismo y otras corrientes; y porque está pobremente desarrollada su explicación del proceso de aprendizaje, del que Lewin en realidad se ocupó poco. De todos modos, es muy amplio el abanico de sugerencias que ofrece, no solo a la Psicología y a la Psicología Social sino también a la Sociología y a la Ciencia Política. Al final de este apartado, pasaremos revista a algunas aplicaciones.

La DINAMICA DE GRUPOS, en su orígen también está vinculada al nombre de Kurt Lewin, quien fundó, en vísperas de la segunda guerra mundial, el ``Research Center for Group Dynamics'' en el ``Massachusetts Institute of Technology''. En un sentido amplio, se designa con ese nombre a un conjunto de trabajos de diversos autores, referidos a los grupos pequeños, considerados como resultantes de la interacción de fuerzas múltiples y cambiantes, a las que se procura identificar, describir y, en lo posible, medir. La dinámica de grupos vincula muy estrechamente la investigación pura y la aplicada.

Lewin considera que el grupo es una totalidad estructurada, cuyas propiedades son diferentes a la suma de las propiedades de las partes. El grupo y el entorno que lo rodea configuran un campo dinámico. Ese dinamismo, su estabilidad y modificaciones, pueden explicarse por el juego de las fuerzas psicosociales, tales como la presión de las normas sociales, la resistencia de las barreras psicológicas, la prosecución de objetivos, etc. Este modelo se presta para una representación gráfica vectorial, susceptible, a su vez, de ser operada matemáticamente.

A partir de la obra pionera de Lewin y su grupo en este terreno, la Dinámica de Grupos ha tendido a hacerse cada vez más ecléctica, y en las obras más recientes sobre el tema, junto al gestaltismo originario pueden discernirse influencias del conductismo, del psicoanálisis y del neofreudismo.

D. Cartwrigth y A. Zander\footnote{Dorwin Cartwrigth y Alvin Zander: GROUP DYNAMICS: RESEARCH AND THEORY, Ed. Harper and Row, 1962.} agrupan en cinco áreas los estudios hechos sobre dinámica de grupos: - cohesión del grupo; - presiones y criterios del grupo; - motivos individuales y finalidades del grupo; - dirección (``leadership'') y logros del grupo (``performance''); -propiedades estructurales de los grupos.

Otros temas que aparecen en la bibliografía especializada\footnote{Ver, por ejemplo, Morton Deutsch y Robert M. Krauss: THEORIES IN SOCIAL PSYCHOLOGY, Basic Books, Inc., 1965.} son: - los campos de fuerza (``power fields''); - los conflictos internos del grupo; - las comunicaciones intra e intergrupales.

Muchos estudios sobre este tema tratan de establecer la interrelación de algunos factores y elementos componentes de los grupos, tales como: - estratificación social y cohesión grupal; - dirección autoritaria y uniformidad grupal; - efectos de las interrupciones sobre la actividad del grupo; - posición del dirigente, ambiente grupal y comunicación.

La simple lectura de este temario da una idea clara de sus contenidos y también de las afinidades y sugerencias que presenta para las ciencias sociales en general y para la Ciencia Política en particular. En este último caso, la dinámica de grupos se ha mostrado especialmente útil para el análisis de la estructura y dinámica de los comités y otros grupos decisorios, vale decir, en estudios de micropolítica, más que en estudios de nivel macropolítico.

Ahora vamos a pasar revista a algunas aplicaciones politológicas de estos enfoques (Gestalt, Teoría del Campo y Dinámica de Grupos) y a algunas sugerencias que provienen de ellos, y que a nuestro entender son muy fecundas.

De las experiencias de K. Lewin, una de las más importantes (y también de las más citadas) es la referente al ambiente psicológico, o sea el clima afectivo y normativo que impera en un grupo humano, y que influye fuertemente en el comportamiento de los integrantes del grupo y en los logros o fracasos del mismo.

En la experiencia en cuestión, Lewin, Lippitt y White sometieron a diversos grupos de jóvenes a tres ambientes psicológicos sucesivos: autoritario (pautas rígidas, objetivos prefijados, jerarquía y órdenes); democrático (pluralismo, confrontación, reglas básicas del juego); y liberal (tipo ``laissez-faire'').

La mejor integración grupal y los mejores logros se alcanzan en un clima democrático, en el que la interacción humana aumenta la eficacia y el sentido de la responsabilidad de cada uno. El clima autoritario frustra el deseo de libertad y cohíbe la responsabilidad individual. El clima liberal produce malestar por falta de orientaciones y límites. En estos dos últimos casos, paradojalmente se producen a nivel de las conductas individuales los mismos resultados: agresividad e indiferencia hacia los fines grupales.

Otra experiencia importante se refiere a los valores propios del grupo, que operan como factor mediatizante (conjuntamente con las predisposiciones psíquicas) en la percepción de mensajes provenientes de los medios de comunicación de masas u otras fuentes.

Los mensajes llegan a nosotros ``tamizados'' por los valores colectivos de nuestro grupo de pertenencia. En la medida en que valoramos nuestra pertenencia al grupo, nos sentimos obligados a adoptar sus valores. Todo mensaje acorde con ellos tiene buena acogida, y si es contrario, encuentra una fuerte oposición.

Kurt Lewin solía decir, en ese sentido, que es más fácil hacer cambiar de opinión a un grupo que a un individuo. Ahora bien, al parecer la única técnica adecuada para lograr ésto es la ``discusión en grupo'', que está en las antípodas de las técnicas de difusión masiva\ldots{} Con respecto a éstas, un importante corolario que deriva de las experiencias antedichas es que el mensaje difundido por los medios de comunicación de masas tiene por efecto reforzar las opiniones preexistentes, más bien que hacer aceptar nuevas opiniones. Otra consecuencia es que se puede lograr una acción más precisa y eficaz del individuo si se logra clarificar y reforzar su pertenencia al grupo.

Volvamos al tema del cambio de opinión en los grupos, porque allí encontramos otra cuestión de relevante importancia para la Ciencia Política: la modificación de los hábitos colectivos. Sabemos que las tentativas de modificar hábitos sociales arraigados despierta en general grandes resistencias. Esto es un problema fundamental en el cambiante mundo moderno, que muchas veces requiere una flexibilidad mayor de la que, al parecer, están dispuestos a adoptar individualmente los hombres.

Los experimentos de Lewin y su grupo mostraron la superioridad del procedimiento de la discusión en grupo y de las decisiones tomadas en común para lograr cambios de hábitos sociales. La razón de ello estriba en que la discusión libre compromete a los individuos en una interacción social, de tal manera que la inseguridad producida por el cambio es atenuada por el sentimiento de pertenencia al grupo. El cambio individual de actitud es facilitado si se piensa que, en realidad, es el propio grupo el que está cambiando.

Según el enfoque de Lewin, un grupo (antes de que se intente un cambio) puede ser definido como un ``estado casi estacionario''; un equilibrio, en definitiva, de fuerzas psicosociales. Para superar la resistencia inicial y producir un cambio, los pasos a dar son: - ``descristalizar'' los hábitos colectivos mediante la libre discusión; - promover nuevas normas mediante la decisión del grupo; - consolidar esas normas mediante la instauración de una organización adecuada.

De ese modo, las técnicas de la dinámica de grupos facilita el cambio de hábitos sociales de un modo que es políticamente muy importante: por consenso y con mínima coacción.

Otro enfoque muy interesante para los estudios politológicos es la llamada ``teoría de la disonancia cognoscitiva'' de Leo Festinger\footnote{Leo Festinger: A THEORY OF COGNITIVE DISSONANCE, Row, Peterson and Co., 1957.}. En síntesis, Festinger sostiene que el hombre normal tiene un estado interno que revela un grado elevado de coherencia. Sus ideas, representaciones, creencias y actos son bastante coherentes, homogéneos, equilibrados, consonantes. Si esa equilibración interna se rompe por algún motivo, el hombre experimenta un malestar que lo mueve a actuar en alguna forma para restaurarla.

Un hombre, por ejemplo, puede verse obligado a hacer un acto o una declaración contrarios a sus valores; o percibe una contradicción entre sus ideas y sentimientos personales y la representación que se hace de la opinión predominante en su grupo de pertenencia o en su sociedad, etc. Se crea entonces una ``disonancia cognoscitiva'' que es fuente de un malestar interior porque lesiona la anterior coherencia. En tales casos, el hombre, para reducir la disonancia, puede modificar su opinión hasta llegar a estar de acuerdo con los demás; o percibir la opinión de los demás como menos contraria a la suya de lo que es en realidad; o rechazar toda información contraria a su opinión; o interpretar esa información de una manera más acorde con su opinión; o disgustarse con la persona que disiente de su criterio, etc. En definitiva, la ``disonancia'' o incompatibilidad entre cogniciones distintas del individuo lo impulsan a realizar acciones orientadas a reducir la disonancia.

Esta idea (a la que parece exagerado llamar ``teoría'') contiene sugerencias interesantes para los estudios politológicos, en particular para explicar la difusión y aceptación de contenidos ideológicos, habida cuenta de la escasa consistencia que en general presentan las ideologías desde el punto de vista lógico-formal. La mayoría de las personas busca adaptarse a la opinión consagrada por el grupo, o experimenta diversas distorsiones en su percepción, de modo que un mismo contenido ideológico puede ser aceptado por muy diversas personas en función de sus diferentes representaciones del mismo.

La ``teoría del poder social'' de John R.P. French\footnote{John R.P. French Jr.~: ``A Formal Theory of Social Power'' en Cartwright y Zander: GROUPS DINAMICS: RESEARCH AND THEORY, ed.~Harper and Row, 1962.} es un esfuerzo por aplicar la teoría del campo en Ciencia Política. Es interesante desde un punto de vista heurístico pero adolece de algunas indefiniciones conceptuales y de una limitada operacionalidad. Está desarrollada como una estructura deductiva que abarca: - las relaciones de poder dentro de un grupo; - las formas de comunicación dentro de un grupo; - otras relaciones internas.

Consta de tres axiomas y de algunos teoremas: Axioma 1: En cualquier discrepancia de opiniones entre A y B, la potencia de la fuerza resultante que un inductor A puede ejercer sobre un inducido B para hacerle aceptar la opinión de A es proporcional a la potencia de las bases del poder de A sobre B.

French define el poder como ``el máximo de fuerza que A puede ejercer sobre B menos la máxima resistencia que B puede oponer a A.''Base de poder" es la relación duradera entre A y B que permite el surgimiento del poder.

Axioma 2: La potencia de la fuerza que un inductor A ejerce sobre un inducido B para hacerle aceptar la opinión de A es proporcional a la magnitud de la discrepancia entre ambas opiniones.

Axioma 3: En una unidad (que en su lenguaje es el tiempo necesario para que todos los miembros del grupo sometidos a influencia cambien sus opiniones hasta llegar al punto de equilibrio de todas las fuerzas actuantes al comienzo de la unidad) cada una de las personas sometidas a influencia cambiarán su opinión hasta alcanzar el punto de equilibrio en el que la fuerza resultante es cero.

Ejemplos típicos de teoremas son: Teorema 1: En una estructura de poder perfectamente conectada, y para todas las posibles estructuras de opinión inicial, las opiniones de todos los miembros alcanzarán un equilibrio común igual a la media aritmética de las opiniones iniciales de todos los miembros, y esta opinión final se logrará dentro de una unidad.

Teorema 2: En un grupo conectado débilmente, los miembros no lograrán un acuerdo salvo en el caso de que existan condiciones especiales en la distribución de las opiniones iniciales.

Como vemos, es un conjunto de ideas interesantes pero conceptualmente bastante imprecisas pese a su enunciado formal (``potencia de la fuerza'', ``discrepancia entre opiniones'') y muy difíciles de cuantificar, aunque sea estimativamente y, por supuesto, de operacionalizar. No obstante, sin pretender verificar o falsear la exactitud matemática de los enunciados, lo cierto es que son válidos como ``enunciados de tendencia'' a los fines del análisis de hechos reales: es cierto que el poder es la resultante de una interacción, que la discrepancia debilita el poder, que los miembros del grupo cambian sus opiniones hasta encontrar un nuevo equilibrio, que el vínculo grupal fuerte favorece ese proceso y que uno débil lo perjudica, etc.

La Teoría de la Organización\footnote{Ver Robert Golembiewski: BEHAVIOR AND ORGANIZATION: ORGANIZATION AND METHODS AND THE SMALL GROUP, Rand McNally and Co., 1962. También James G. March y Herbert A. Simon: ORGANIZATIONS, John Wiley and Sons, 1962. Entre las obras más recientes ver, por ejemplo, Stephen Robbins: COMPORTAMIENTO ORGANIZACIONAL, Prentice Hill, México, 1987.} ha sido la principal beneficiaria de los estudios de la Teoría del Campo y de la Dinámica de Grupos. Otros temas sociológicos y politológicos donde suelen aplicarse son las actitudes políticas, los fenómenos de formación y cambio de hábitos y de opiniones sociales y políticas, así como en algunos estudios sobre desarrollo y subdesarrollo.

Sidney Verba\footnote{Sidney Verba: SMALL GROUPS AND POLITICAL BEHAVIOR, Princeton University Press, 1961. También Gabriel Almond y Sidney Verba: THE CIVIC CULTURE, Princeton Univesity Press, 1963.} ha destacado con agudeza la importancia politológica de la llamada ``hipótesis de participación'', o sea ese principio según el cual la efectividad de los cambios importantes en la conducta de los integrantes de grupos pequeños requiere la participación de los miembros en el proceso de adopción de la decisión de cambio.

En la misma obra, Verba hace un excelente análisis de las posibilidades y limitaciones que tiene la aplicación de las ``teorías del pequeño grupo'' en Ciencia Política. En síntesis dice que no es una panacea pero que puede resultar muy útil si se la emplea con inteligencia. En especial, hay que ser muy prudente en la extrapolación de conclusiones obtenidas en las condiciones cuasi-experimentales del pequeño grupo, a grandes grupos (sociedades globales, por ejemplo) en un nivel de observación empírica.

\hypertarget{el-freudismo-ortodoxo-psicoanuxe1lisis-freudiano}{%
\subsection*{El Freudismo ortodoxo (Psicoanálisis freudiano)}\label{el-freudismo-ortodoxo-psicoanuxe1lisis-freudiano}}
\addcontentsline{toc}{subsection}{El Freudismo ortodoxo (Psicoanálisis freudiano)}

En las ciencias sociales y en los juicios normativos sobre hechos sociales es muy frecuente encontrar explicaciones basadas en la Teoría Psicoanalítica, o mejor, en la obra de Sigmund Freud (1856-1939) psiquíatra austríaco nacido en Freiberg y muerto en Londres, fundador del Psicoanálisis. Tanto este autor como su obra son universalmente conocidos aunque con frecuencia mal comprendidos. El freudismo es, con el marxismo, uno de las dos corrientes intelectuales surgidas en los siglos XIX y XX que han alcanzado máxima difusión e influencia, y motivado también las más grandes controversias.

El impacto del freudismo sobre el pensamiento contemporáneo es muy grande. En su momento, Freud revolucionó la Psicología, pero hoy pareciera incluso tener más influencia en las ciencias sociales y en las humanidades que en el propio campo psicológico.

La Teoría Psicoanalítica es el componente principal y hasta cierto punto fundacional de ese conjunto de teorías denominado por Bleuer ``psicología de las profundidades'' o ``psicología profunda''. Su objetivo, inspirado en razones teóricas, curativas y existenciales, es traer a la conciencia aquellas partes de la psiquis del ser que le son habitualmente desconocidas. Esa toma de conciencia del inconsciente es esencial en el enfermo para su curación, y en el sano para acceder a la totalidad de su ser. En realidad, INCONSCIENTE es sólo una palabra; no es una entidad, ni una sustancia ni un lugar: es una hipótesis de trabajo (que no tiene las resonancias ideológico-filosófico-religiosas de palabras tales como espíritu o alma) que permite nombrar lo que en la psicología humana no puede ser captado directamente por la conciencia.

En general, el pensamiento de los grandes creadores suele ser presentado en forma desvinculada de sus fuentes, como si fueran grandes torres aisladas. Pero así como no puede entenderse a Marx sin pasar (como mínimo) por Hegel, Feuerbach y David Ricardo, tampoco puede entenderse bien a Freud sin tener alguna idea del trabajo preparatorio que hizo el pensamiento occidental para acceder a esa ``psicología de las profundidades'', desde las ``representaciones inconscientes'' de Leibnitz, el ``inconsciente'' de Herbart, los ``sueños'' de Carus, hasta la ``filosofía del inconsciente'' de von Hartmann; y desde los remotos atisbos de Paracelso, pasando por los trabajos de Mesmer sobre ``magnetismo animal'' (con todas sus tergiversaciones) y P. Janet, hasta la Escuela de la Salpêtrière, con Charcot y la Escuela de Nancy, con Liébault y Bernheim, y sus trabajos sobre hipnotismo y sugestión.

De los tres últimos mencionados, Freud fue discípulo directo. ``Allí fue -escribirá más tarde- donde recibí las más fuertes impresiones relativas a la posibilidad de fuertes procesos que, sin embargo, permanecen ocultos a la conciencia de los hombres''.

El término PSICOANALISIS fue acuñado por Freud en base a ciertas analogías entre el trabajo del terapeuta y el del químico. Apareció por primera vez en publicaciones del año 1896 y fue definido por Freud desde tres puntos de vista, diferentes pero concatenados: - como procedimiento de investigación de procesos mentales que serían prácticamente inaccesibles de otro modo; - como método para el tratamiento de trastornos neuróticos; - como conjunto de concepciones psicológicas que van formando una nueva disciplina científica.

Saldría completamente fuera de los límites y de la intención de este trabajo una descripción completa del vastísimo campo psicoanalítico. Aquí no va a interesar especialmente una parte del tercer punto de vista: los lineamientos generales del Psicoanálisis como teoría científica de la psicología individual profunda; y sobre todo sus repercusiones en el modo de entender lo social.

Como características generales del Psicoanálisis freudiano podemos mencionar las siguientes: - se basa en una visión del hombre predominantemente biológica (organicista, materialista); - piensa los procesos en términos evolucionistas darwinianos, o más exactamente lamarckianos, ya que Freud creía que los caracteres adquiridos pueden trasmitirse por vía genética; - su enfoque básico es instintual e individualista; - es determinista y considera que en la investigación toda acción humana tiene relevancia y significación: que se debe deducir a partir de lo que se manifiesta en la conciencia lo que hay debajo de su superficie. Afirma, pues, la existencia de una relación determinista entre la acción manifiesta y la motivación inconsciente; - en su teoría, Freud generalizó los resultados de una prolongada introspección, conjuntamente con las observaciones provenientes de una larga tarea clínica, propia y de otros.

Su esquema o modelo básico de la psicología humana se basa en la afirmación de la existencia de una energía impulsora, innata en el hombre, muy semejante al ``élan vital'' de Bergson, a la que llamó LIBIDO (deseo, apetito, aspiración). Luego de 1923, también la denominó ID, y también EROS\footnote{A partir de 1923, Freud considera la existencia de dos impulsos instintivos: EROS (instinto de vida) y THANATOS (instinto de muerte).}.

Todos los procesos mentales (excepto la recepción de estímulos externos) derivan de la interacción de esas fuerzas instintivas, que son de orígen orgánico. Son características de la libido: - está gobernada por el ``principio del placer''; -es indiferente a la moralidad; - es indiferente s su propia seguridad; - recibe su placer del acto de la descarga, sin intermediación del ego.

Los INSTINTOS BASICOS (conservación, preservación, etc.) son sistemas de dirección de los impulsos libidinales. Su estructura está superpuesta al id y su función es imprimir direccionalidad y sentido a las energías libidinales,que originariamente no lo tienen.

El EGO se desarrolla en el ser humano aproximadamente a partir de los seis meses de edad. Es una estructura mediadora entre el puro impulso del id y la realidad del entorno externo. Está gobernado por el ``principio de realidad''.

El SUPEREGO es una instancia de la personalidad, cuya función es equiparable a la de un juez o censor del ego. La conciencia moral, la auto-observación, la formación de ideales, son algunas de sus manifestaciones. Según la ortodoxia freudiana, el superego es heredero del complejo de Edipo, producido por interiorización de las exigencias y prohibiciones familiares.

El CUERPO, en este esquema, puede ser visto como un receptáculo dividido en áreas de diferente valor erógeno, y conectado con el entorno de forma poco precisa. Las ``zonas erógenas'' (oral, anal y genital) son utilizadas por Freud en su teoría de la personalidad y del desarrollo del carácter.

Esta ``visión topográfica'' o esquema básico del aparato psíquico puede representarse gráficamente del siguiente modo: ENTORNO ----------------------------------------------------------------- CUERPO \^{} GENITAL ANAL ORAL \textbar{} EGO \textless---\textgreater{} SUPEREGO placer \textbar{} ------ v dolor CONSCIENTE ----------------------------------------------------------------- SUBCONSCIENTE \^{} \^{} \^{} INSTINTOS \textbar{} \textbar{} \textbar{} (energía libidinal INCONSCIENTE LIBIDO o ID dirigida) Cómo ``funciona'' este modelo? Una síntesis de la dinámica freudiana puede presentarse del siguiente modo: Los impulsos del id o libido proporcionan la energía propulsora de todo el sistema. La estructura de los instintos transforma esa energía pura en energía libidinal dirigida. En la ortodoxia freudiana, las cuestiones más importantes surgen de la relación entre el id, el ego y el superego. Los neofreudianos, en cambio, enfatizan más la importancia de las relaciones entre el ego y el entorno.

El id está totalmente inmerso en el inconsciente. Allí se originan todas las tendencias e impulsos. Sólo la interpretación psicoanalítica puede determinar el sentido profundo de la conducta humana, que se origina en este plano. Afirma, por ejemplo, que los sueños siempre tienen un significado: siempre son la satisfacción de un deseo reprimido en la vigilia, pero su contenido real aparece siempre disfrazado y oculto y debe ser interpretado.

Los impulsos o deseos que brotan del id son vitalmente suficientes para el recién nacido, pero el hombre necesita vivir en sociedad, y para ello ha de acomodar su conducta a los deseos ajenos, so pena de ser destruído. Freud reedita así la vieja tesis de Hobbes.

El ego, que funciona de acuerdo con el principio de realidad, es el encargado de reprimir los impulsos. Aunque Freud escribió como si el ego fuera un elemento concreto, es más lógico considerarlo como una función del aparato psíquico. Quizás la más importante contribución de Freud a la psicología fue la identificación de muchas funciones del ego, a las que designó con expresiones que frecuentemente utiliza hoy el pensamiento ilustrado y hasta la conversación corriente: - Represión: es impedir que un impulso entre en la conciencia; - Racionalización: es un intento de explicación coherente, lógica, moral, de un acto o hecho cuyos motivos verdaderos no se perciben; - Proyección: es expulsar de sí y localizar en otro (persona o cosa) algo que no se reconoce o que se rechaza de sí mismo; - Introyección: es hacer pasar, en forma fantasmática, de ``afuera'' a ``adentro'' objetos o cualidades propias de los mismos. Se relaciona también con la identificación del yo con otra persona o con alguna de sus cualidades; - Regresión: Dentro de un proceso psíquico, es ir hacia atrás en la secuencia de los estadios del desarrollo psíquico; - Formación reactiva: Es una actitud o hábito de sentido opuesto a un deseo reprimido, constituído como reacción contra éste; - Desplazamiento: Es el traspaso de la actitud de interés, de un objeto a otro.

Otros conceptos también usuales en el lenguaje freudiano y difundidos luego con mayor o menor exactitud en el lenguaje corriente son: sentimiento de culpabilidad, frustración, angustia, mecanismos de defensa, etc.

La teoría freudiana sobre el desarrollo de la personalidad se concentra en el estudio de los primeros cinco años de la vida. En ese período se atraviesan tres estadios, marcados por la principal fuente de placer para el individuo en cada uno de ellos: oral, anal y genital. En el estadio oral, el placer viene principalmente de comer, de ``incorporar'' cosas al cuerpo, y ese esquema se aplica a toda la relación con el mundo. El dolor y el temor son originados por la ausencia del factor protector primordial y fuente nutricia: la madre. El bebé se comporta enteramente según el ``principio del placer'', buscando un estado cenestésico. El estadio anal comienza con la educación del control de esfínteres (hacia los dos años de edad) que es también la primera confrontación con el ``principio de realidad''. Según la forma de educarlo, el niño puede hacerse ``expulsivo'' (cruel y destructivo); ``retentivo'' (mezquino y miserable) o productivo y creador (si la madre estimula positivamente sus esfuerzos). El estadio fálico es la fase siguiente de la organización infantil de la libido, caracterizada por la unificación de las pulsiones bajo la primacía de los órganos genitales. Corresponde a la culminación y declinación del complejo de Edipo (atracción sexual hacia el padre de sexo opuesto y odio por el del mismo sexo).

En el varón, el temor a la autoridad paterna y a ser castigado con la castración producen una represión del deseo sexual de la madre y una identificación con el padre. El desarrollo de la niña no es simétrico: ella ama a su padre (complejo de Electra) pero cuando descubre que ella no tiene pene comienza a envidiar a los varones y evoluciona en dirección a una actitud ambivalente hacia su padre, objeto de amor-envidia al mismo tiempo.

Ha sido necesaria esta resumida y seguramente incompleta exposición de las ideas básicas de Freud sobre la psicología individual para entender su pensamiento social. Hemos visto que el hombre, en la concepción de Freud, aparece como un ser aislado y solitario, llevado por pulsiones y deseos heredados hacia actividades muy difícilmente compatibles con una convivencia social estable y organizada. De allí el espíritu ``hobbesiano'' de sus ideas sobre la vida social y el rol central asignado por él a la represión en la génesis de cualquier orden social productor de cultura.

A partir de 1913, Freud escribió obras importantes sobre temas sociales. En ellas emplea la misma orientación y los mismos conceptos básicos desarrollados en sus obras sobre psicología individual. Presta preferente atención a la génesis de lo social (tabúes, totems, mitos, creencias religiosas). Con frecuencia emplea datos antropológicos que ya eran anticuados en su época, y principios genéticos que hoy resultan francamente insostenibles. La concepción freudiana de la sociedad es aristocrática, autoritaria, pesimista respecto de la naturaleza humana, y sus implicaciones políticas prácticas son radicalmente conservadoras. Como ya dijimos, su pensamiento es de neto corte hobbesiano.

Freud reconoce, por supuesto, que el hombre necesita de la sociedad para sobrevivir, y que esa necesidad lo obliga a aceptar limitaciones a sus deseos, pero destaca que se somete de mala gana, bajo constantes amenazas y presiones. Para Freud, todo individuo es, en el fondo, un enemigo de la civilización. La civilización se construye sobre la represión del hombre: una civilización no represiva es considerada por él como totalmente imposible.

Freud es individualista; manifiesta un gran rechazo por el hombre-masa. Puede encontrarse en él un anticipo de la idea del inconsciente colectivo, que luego desarrolló K.Jung y que es, indudablemente, un elemento importante en la descripción psicoanalítica de la génesis de la conducta de las masas.

En cuestiones internacionales, Freud emplea los mismos enfoques. El hombre está naturalmente impulsado desde sus instintos a agredir y dominar a los demás, y lo mismo ocurre con las naciones. La base de la sociedad y de la vigencia del derecho es la unión de los débiles en contra de los fuertes; en definitiva, la imposición de un poder colectivo sobre todos.

En su idea de la naturaleza o condición humana, Freud difiere completamente de Marx. Por ello siempre nos ha llamado la atención la combinación que importantes corrientes del pensamiento contemporáneo, desde la Escuela de Frankfurt hasta algunos representantes de la Crítica Social, han hecho de la obra de ambos pensadores. Para Freud, por ejemplo, la agresividad humana es anterior al surgimiento de la propiedad privada, de modo que la abolición de ésta no modificará sustancialmente la conducta humana. Los hombres, para Freud, no luchan por un motivo en especial, sino porque tienen que hacerlo; porque está en su naturaleza, como consecuencia de un impulso instintivo. Los actuales grupos y sociedades humanos son manifestaciones contemporáneas del comportamiento de horda. En ellos se ha desarrollado, como en los individuos, un super-ego, que obliga a respetar ciertos límites y mantener una conducta considerada socialmente adecuada.

Según nuestro criterio, son cuatro las principales obras de Freud referentes a lo social: - TOTEM Y TABU (``Totem und Tabu''-1913);\footnote{Sigmund Freud: OBRAS COMPLETAS, Madrid, Ed. Biblioteca Nueva, 1973, tomo II pág. 1745.} - PSICOLOGIA DE LAS MASAS Y ANALISIS DEL YO (``Massenpsychologie und ich-analyse''-1921);\footnote{Sigmund Freud: op. cit., tomo III, pág. 2563.} - EL PORVENIR DE UNA ILUSION (``Die Zurunft einer Illusion''-1927);\footnote{Sigmund Freud: op. cit., tomo III, pág. 2961.} - EL MALESTAR EN LA CULTURA (``Das Unbehagen in der Kultur''-1930);\footnote{ Sigmund Freud: op. cit., tomo III, pág. 3017.}.

``Totem y Tabu'' es la primera tentativa que hizo Freud para aplicar el punto de vista psicoanalítico a problemas de psicología social. Como él mismo dice, el tema de los tabúes está exhaustivamente tratado en esta obra, mientras que la investigación del totemismo está apenas esbozada. ``Se trata de un libro que estudia el orígen de la religión y la moral\ldots{}'' dice Freud en el Prólogo de la edición hebrea. La obra reúne cuatro ensayos que fueron originalmente publicados en forma separada: I - EL HORROR AL INCESTO, II - EL TABU Y LA AMBIVALENCIA DE LOS SENTIMIENTOS, III - ANIMISMO, MAGIA Y OMNIPOTENCIA DE LAS IDEAS, IV - EL RETORNO INFANTIL AL TOTEMISMO.

En ``El horror al incesto'' describe, en base a un abundante material etnográfico aportado sobre todo por Frazer\footnote{Frazer: TOTEMISM AND EXOGAMY - 1910. También: Andrew Lang: THE SECRET OF THE TOTEM, 1905.}, las particularidades del totemismo como modo primitivo de organización de los grupos humanos. Un totem ``\ldots es un animal comestible\ldots más raramente una planta o una fuerza natural..'' vinculado con el grupo humano de un modo especial: es considerado como el antepasado del clan y también como su espíritu bienhechor y protector. Los integrantes del grupo totémico no pueden matar a dicho animal, ni comerlo, ni aprovecharlo de ninguna otra forma, bajo pena de muerte.

Otra consecuencia es la ``consanguinidad totémica'' de los integrantes del grupo, de donde deriva una exigencia de exogamia: los miembros del mismo clan totémico no deben casarse entre sí. Ese ``horror al incesto'', que se presenta como el primer valor de una naciente moral social, va aún más allá: las tribus se dividen en dos ``fratrias'' (clases matrimoniales) y éstas a su vez en dos o más subclases, todas exogámicas entre sí, de modo que se restringen mucho las posibilidades de elección matrimonial. Esas restricciones van también acompañadas por reglas de trato social que refuerzan el ``horror al incesto'': prohibición de trato familiar y hasta de dirigir la palabra, a parientes cercanos del otro sexo: madre, hermanas, cuñadas, suegra, etc.

En ``El Tabú y la ambivalencia de los sentimientos'', Freud sostiene que si se estudia el tabú con óptica psicoanalítica se encuentran muchas similitudes con las ``neurosis obsesivas'' de los hombres ``civilizados'', con su característica ambivalencia de deseos y contradeseos. En este ensayo, quizás la parte más interesante para la Teoría Política sean las reflexiones sobre temas tales como: - La conducta para con los enemigos: reconciliación con el enemigo muerto; restricciones a observar; actos de expiación o purificación del matador; prácticas ceremoniales.

\begin{itemize}
\item
  El tabú de los soberanos: el súbdito debe protegerse de ellos porque son portadores de una energía (``maná'') que puede ser peligrosa, y a la vez debe amarlos y protegerlos: aquí aparece nuevamente el tema de la ambivalencia, que es clásico en los estudios politológicos sobre el poder.
\item
  El tabú de los muertos: es debido a la contaminación o impureza derivada del contacto con los muertos, de donde deriva, por ejemplo, la prohibición de pronunciar su nombre y la necesidad de celebrar ritos propiciatorios, etc.
\end{itemize}

El ensayo titulado ``Animismo, magia y omnipotencia de las ideas'' comienza con una interesante reflexión de Freud sobre cómo veía él sus propios aportes a las ciencias del hombre: ``\ldots no aspiran sino a estimular a los especialistas y a sugerirles ideas que puedan utilizar en sus investigaciones\ldots{}'', amplitud de criterio que no siempre es tenida en cuenta por los actuales seguidores del freudismo\ldots{}

La idea básica de este ensayo es que, en la construcción de sistemas cosmovisionales (animismo, magia, religión) los hombres no se vieron impulsados sólo por ``\ldots una pura curiosidad intelectual, por la sóla ansia de saber. La necesidad práctica de someter al mundo debió de participar, indudablemente, en esos esfuerzos''.

Las concepciones del mundo, según Freud, evolucionaron a través de fases: animista (la omnipotencia está en el hombre); religiosa (la omnipotencia es transferida a los dioses) y científica (que pretende abandonar la ``omnipotencia de las ideas'' pero dejando rastros de ella ``en nuestra confianza en el poder de la inteligencia humana''); algo comparable (aunque no igual) a la ``ley de los tres estados'' que según Comte habían atravesado las sociedades occidentales en su evolución histórica: teológico, metafísico y positivo; y que es retomado luego por Erik Kahler en su ``Historia Universal del Hombre'', cuando plantea las diversas actitudes que puede asumir el ser humano cuando cobra conciencia de su humanidad y se visualiza a sí mismo como un ente ``separado'' de la Naturaleza: la magia, la religión y la ciencia serían en este caso, las grandes fases de la re-vinculación del ser individual-social con el Todo.\footnote{Erik Kahler: HISTORIA UNIVERSAL DEL HOMBRE, México, FCE.} El cuarto ensayo, titulado ``El Retorno Infantil al Totemismo, es un intento de explicar el orígen de la religión como fundamento de la vida social, en base al concepto de totem, aunque Freud aclara que''no puede retraerse a una sola fuente un fenómeno tán complicado como la religión".

Los dos tabúes (o sea, prohibiciones) fundamentales del totemismo ``con los cuales se inicia la moral humana'' son la muerte del totem y el incesto. Freud esboza aquí su famosa explicación mítico-histórica (basada en algunas observaciones de Darwin) sobre esos orígenes: los hijos, que aman y odian al padre, que los protege pero los excluye del comercio con las mujeres, finalmente se dejan llevar por su odio, lo matan y lo comen, para asimilar mágicamente su fuerza; luego prima nuevamente el amor, experimentan culpa y ``lo que el padre había impedido anteriormente\ldots se lo prohibieron luego los hijos a sí mismos\ldots{}''.

``Totem y Tabú'' es, tal vez, la obra de Freud donde más claramente se manifiesta su reduccionismo psíquico individual de la vida social, reduccionismo que potencia el rol del psiquismo inconsciente. En esta obra, Freud intenta explicar todas las costumbres primitivas en función de represiones de la libido o de neurosis obsesivas. Desconoce, en este sentido, el rol de la acción social propiamente dicha. Por otra parte, adoptando una clásica postura ``eurocéntrica'', Freud considera a las sociedades primitivas como ``embrionarias'', sin reconocerles una estructura autónoma, con funciones y objetivos propios.

En ``Psicología de las Masas y Análisis del Yo'' (1921) Freud comienza su desarrollo cuestionando la oposición entre psicología individual y psicología social o colectiva, porque ``\ldots en la vida anímica individual aparece siempre integrado\ldots el otro'', pero reconoce claramente las diferencias que se dan entre los fenómenos ``narcisistas'' o ``autísticos'', los fenómenos de interacción social entre dos o pocas personas, y los fenómenos de influencia simultánea de gran número de personas, o sea la psicología de las masas.

Freud parte de las observaciones de Gustave Le Bon\footnote{Gustave Le Bon: (PSYCHOLOGIE DES FOULES, Paris, Alcan, 1921) PSICOLOGIA DE LAS MULTITUDES, Buenos Aires, Albatros, 1978.} sobre la aparición en la multitud de un ``alma colectiva'' que obra de manera completamente distinta a los individuos que la componen. Se trata de un ``ser provisional'' en el que emerge ``lo inconsciente social'' y se borran ``las adquisiciones individuales''.

Aparece allí ``un sentimiento de potencia invencible'', que hace más fácil ``ceder a los instintos'', lo que se ve favorecido por el carácter ``anónimo e irresponsable'' de la multitud. La supresión de las represiones permite la manifestación, no de caracteres nuevos -sostiene Freud- sino de elementos ya existentes en el inconsciente individual.

En la multitud aparecen fenómenos de gran interés, como el ``contagio mental'' efecto de la ``sugestionabilidad'', a veces semejante a la ``fascinación del hipnotizado''. En la multitud hay una tendencia a pasar inmediatamente a la acción. ``La multitud es impulsiva, versátil e irritable''; es omnipotente, influenciable, crédula, extremista. Es autoritaria e intolerante, conservadora y reacia a las novedades, y altamente sensible al poder mágico de las palabras.

Freud recuerda palabras de MacDougall, para quien el fenómeno más importante de la formación de la masa es la exaltación de la emotividad, y considera que ``el nivel de la vida psíquica de la multitud'' puede ser elevado por medio de una organización adecuada. Freud dice que ésto ``\ldots equivale a crear en la masa las facultades características del individuo\ldots{}'' La explicación psicológica de la modificación psíquica ocasionada al individuo por la masa se encuentra para Freud en ``la influencia sugestiva de la masa'' que es condición necesaria para que se manifieste ``el prestigio del caudillo''. Detrás de esa sugestión, Freud postula la existencia en la masa de ``lazos afectivos'', manifestación del Eros ``\ldots que mantiene la cohesión de todo lo existente\ldots{}'' No resulta entonces extraño que los regímenes políticos totalitarios, basados en la movilización incesante de las masas, siempre se hayan visualizado a sí mismos como ``orgánicos''.

La multitud, dice Freud, necesita de un jefe, pero para que éste pueda dominarla ``es preciso que el mismo posea ciertas cualidades: una gran convicción, una voluntad potente e imperiosa, prestigio''; cualidades que produzcan ``una especie de fascinación''.

Este notable trabajo de Freud impresiona como una descripción hecha por anticipado de los fenómenos políticos de movilización de multitudes y emergencia de conductores carismáticos que años después surgirían en Alemania e Italia e implantarían regímenes totalitarios, una de cuyas víctimas ideológicas sería precisamente el movimiento psicoanalítico orientado por Freud\ldots{} Aún hoy son pertinentes sus aportes para la explicación de los fenómenos políticos movimientistas y de ciertos procesos de sugestión y de construcción de liderazgos cuasi-artificiales, producidos por los modernos medios de comunicación de masas.

En el resto del ensayo, Freud analiza algunos temas especiales. En primer lugar, el caso de la Iglesia y del Ejército, a los que considera ``\ldots masas artificiales, esto es, masas sobre las que actúa una coerción exterior encaminada a preservarlas de la disolución y a evitar modificaciones de su estructura''. En esas masas artificiales ``\ldots el individuo se halla doblemente ligado\ldots al jefe (Cristo o el General) y\ldots a los restantes individuos de la colectividad''. También analiza la actuación de las masas con y sin conductor (que en algunos casos puede ser sustituído por una idea o una abstracción); la ausencia en las masas de esa ``normal hostilidad que aparece en todo vínculo estrecho, aún amoroso''; el fenómeno de la identificación como vínculo de enlace recíproco entre los integrantes de la masa; el ``efecto hipnótico del enamoramiento colectivo'' que hace de la masa una experiencia ``de carácter místico''; la masa vista como una resurrección moderna de la horda primitiva, etc.

Finalmente, analiza la neurosis como patología que ``hace asocial al individuo,extrayéndolo de las formaciones colectivas habituales''. La neurosis es para Freud ``un factor disgregador de multitudes'', e inversamente, sostiene que en una ``\ldots enérgica tendencia a la formación colectiva se atenúan las neurosis\ldots{}'' En este libro, pues, Freud parece anticipar, como ya dijimos, las intensas experiencias políticas de masas que sacudirían Europa pocos años después.

``El Porvenir de una Ilusión'' (1927) es un libro en el que Freud desarrolla a fondo sus ideas sobre aspectos básicos de la sociedad humana. Considera que cultura y civilización son sinónimos y que muestran dos aspectos básicos de la problemática huamana: el dominio de la naturaleza y la regulación de las relaciones humanas.

La cultura ha de ser defendida contra los individuos, que se rebelan contra ella a causa de los sacrificios que les impone la vida en común, pese a ser conscientes de que la necesitan para sobrevivir. Toda civilización -sostiene Freud- se basa en la coerción y en la renuncia a los instintos. La civilización es algo ``impuesto a una mayoría contraria a ella por una minoría que supo apoderarse de los medios de poder y coerción''.

Las prohibiciones culturales más antiguas se refieren a deseos instintivos como el incesto, el canibalismo y el homicidio. Sólo el canibalismo está completamente dominado. Los otros deseos aún se hacen sentir ``detrás de la prohibición'' y el homicidio se practica e incluso se ordena en nombre de altos valores, en determinadas circunstancias.

Freud reconoce, sin embargo, que existe cierto ``progreso anímico'' de la humanidad, que consiste en ``la transformación paulatina de la coerción externa en coerción interna\ldots por la acción del superego'', pero también anota que ``\ldots una multitud de individuos no obedecen a las prohibiciones\ldots más que bajo la presión de la coerción externa''.

Freud considera ``comprensible'' que cuando la satisfacción de algunas pocas personas tiene por base la opresión de muchas otras (lo cual ``sucede en todas las civilizaciones actuales'') los oprimidos sean hostiles a la civilización que sostienen con su trabajo pero de la cual no disfrutan.

Cuáles son las compensaciones que pueden obtenerse ante tánta opresión? Una es la participación en los ideales de la propia civilización -ideales forjados como secuela de los primeros logros de ésta- los cuales procuran satisfacciones ``de naturaleza narcisista'' y generalmente se convierten en ``motivos de discordia'' entre las naciones. De ese orgullo y satisfacción participan también ``las clases \ldots oprimidas\ldots en cuanto al derecho de despreciar a los que no pertenecen a su civilización'', lo cual ``les compensa de las limitaciones que la misma les impone a ellos''.

Otra compensación es el Arte, de impacto socialmente menos extenso, porque es ``inasequible a las masas, absorbidas por el trabajo agotador y poco preparadas por la educación''. El Arte ofrece ``satisfacciones sustitutivas compensadoras'' e ``intensifica los sentimientos de identificación'' contribuyendo también ``a la satisfacción narcisista''.

Freud analiza a continuación el orígen y función de lo que llama ``el elemento más importante del inventario psíquico de una civilización\ldots sus representaciones religiosas\ldots o, con otras palabras\ldots sus ilusiones''. ``Ilusión'' es, en el lenguaje freudiano, ``una creencia cuando aparece engendrada por el impulso a la satisfacción de un deseo'' sin prejuzgar si es o no verdad en sí misma. Las creencias religiosas son, según Freud, ``realizaciones de los deseos más antiguos, intensos y apremiantes de la Humanidad''.

Freud dice que ``la función capital de la cultura es defendernos contra la naturaleza'', pero todos sabemos que la naturaleza no está totalmente dominada: la tierra que tiembla, el agua que inunda, la tempestad que destruye, las enfermedades, el doloroso enigma de la muerte, provocan angustia y temor. Por otra parte, la imperfecta civilización en la que vivimos nos acarrea también sufrimientos.

La cultura nos defiende, en un primer paso, humanizando a la naturaleza. No convierte a las fuerzas naturales en simples seres humanos sino en dioses paternales, ``conforme a un prototipo infantil y filogénico''. Esos ``dioses'' tienen una triple función: - conjurar los terrores que inspira la naturaleza; - conciliar al hombre con el destino y la muerte; - compensar al hombre por las privaciones que la civilización le impone.

Con el tiempo, se acentúa la importancia de esta tercera función: - compensar los daños ocasionados por la civilización; - precaver los sufrimientos que los hombres se causan entre sí; - velar por el cumplimiento de los preceptos culturales.

Surge entonces un acervo de representaciones que protege a los hombres contra la naturaleza, el destino y los daños sociales. La vida en este mundo sirve a un fin más alto; el objetivo de esa superación es la parte espiritual del hombre; lo que sucede en el mundo es conducido (aunque sea difícil de comprender) por una inteligencia superior hacia el bien; la muerte no es un fin sino un tránsito hacia una evolución superior. La Sabiduría, la Bondad y la Justicia son los atributos del ``Unico Ser Divino'' en el cual ``nuestras civilizaciones han condensado el politeísmo de épocas anteriores''. Freud hace, evidentemente, un alegato en favor de un fundamento puramente racional de los preceptos culturales, pero se interrumpe por un repentino escrúpulo: ``\ldots los motivos puramente racionales pueden aún muy poco contra las pasiones del hombre\ldots{}'', dice.

Finalmente, al analizar si conviene o no al hombre y a la sociedad perder esas ``ilusiones'', Freud se pronuncia decididamente en favor de su conservación" ``No extrañará -dice-\ldots que me declare partidario de la conservación del sistema religioso como base de la educación y de la vida colectiva. Se trata de una cuestión práctica y no del valor de realidad del sistema''.

En ``El Malestar en la Cultura'' (1930) Freud prosigue la línea de pensamiento sobre la vida social iniciada en sus obras anteriores, abordando en este caso problemas morales y religiosos vinculados con el individuo y la sociedad. Su punto de partida es una observación de su amigo Romain Rolland sobre la ``sensación de eternidad'' o ``sentimiento oceánico'' que sería la fuente última de la religiosidad. Freud se confiesa ajeno a tales sentimientos, pero aclara que en ``El Porvenir de una Ilusión'' no pretendió ocuparse de ``las fuentes más profundas del sentido religioso'' sino de ``lo que el hombre común concibe como su religión'', con sus explicaciones integrales y su solícita Providencia; en definitiva se ocupó de la vigencia y rol social de la religión.

Se plantea luego la cuestión del objeto que tendría la vida humana (``sólo la religión puede responder al interrogante sobre la finalidad de la vida'', dice) y la abandona luego para encarar otra más modesta: Qué esperan los hombres de la vida? Se responde que aspiran a la felicidad, a ser felices: en primer término a experimentar fuertes placeres, pro luego con frecuencia se conforman con no sufrir, con escapar a la desgracia.

Al hombre le resulta muy difícil llegar a ser feliz, por varios motivos: la supremacía amenazante de la naturaleza, la caducidad de nuestro cuerpo y la insuficiencia o precariedad de nuestros métodos para regular las relaciones humanas en la familia, en el Estado y en la sociedad. De estos tres motivos, los dos primeros son más soportables porque son ineluctables, pero el tercer motivo es el más difícil de aceptar: porqué las instituciones creadas por nosotros mismos han dado tán malos resultados? Esa frustración desemboca en una ``extraña actitud de hostilidad contra la cultura'' pese a ser ella obra nuestra y necesaria para nuestra supervivencia. El hombre -dice Freud- ``cae en la neurosis porque no logra soportar el grado de frustración que le impone la sociedad en aras de sus ideales de cultura''. Esto se expresa, por ejemplo, en una nostalgia de la vida primitiva, erróneamente visualizada como ``simple, modesta y feliz''.

Intenta luego Freud hacer un análisis de los factores a los que ``debe su orígen la evolución de la cultura, cómo surgió y qué determinó su derrotero ulterior'' y sus dificultades. La familia primitiva (originada en la permanencia de la pulsión sexual y en la prolongada indefensión de la prole) evolucionó hacia las ``alianzas fraternas'' de la vida social posterior, como ya había explicado Freud en ``Totem y Tabú''. Ahora bien: la vida social es frustrante porque el hombre no es ``una criatura tierna y necesitada de amor'' sino un ser bastante agresivo, violento y cruel. Esas tendencias agresivas ``son el factor que perturba nuestra relación con los semejantes'', dice Freud.

``Los comunistas -añade a continuación- creen haber descubierto el camino hacia la redención del mal..'': la abolición de la propiedad privada. ``No me concierne la crítica económica del sistema comunista\ldots pero\ldots puedo reconocer como vana ilusión su hipótesis psicológica''\ldots{}``el instinto agresivo no es una consecuencia de la propiedad sino que regía\ldots en épocas primitivas\ldots{}'' cuando la propiedad privada no existía. Señalamos nuevamente la clara disyunción planteada entre freudismo y marxismo, a través de esta crítica a un aspecto básico de la concepción antropológica marxista.

Como ya dijimos, nos llaman mucho la atención las frecuentes combinaciones posteriores entre estas dos concepciones, que si bien tienen algunos elementos en común (materialismo y determinismo, por ejemplo) tienen también muy marcadas diferencias, de las cuales quizás la principal sea la orientación general de una y otra línea de pensamiento: mientras el marxismo es un claro exponente del ``encantamiento de la modernidad'', el freudismo se anticipó a los tiempos por venir, en su ``desencanto de la modernidad'', propio de nuestros tiempos post-modernos\ldots{}

En definitiva, concluye Freud, ``\ldots si con toda justificación reprochamos al actual estado de nuestra cultura cuán insuficientemente realiza nuestra pretensión de un sistema de vida que nos haga felices\ldots quizás convenga que nos familiaricemos también con la idea de que existen dificultades inherentes a la esencia misma de la cultura, e inaccesibles a cualquier intento de reforma''.

La tendencia agresiva ``\ldots constituye el mayor obstáculo con que tropieza la cultura''. Esta última procura coartar la agresividad del individuo ``\ldots haciéndolo vigilar por una instancia alojada en su interior, como una guarnición militar en una ciudad conquistada''. Se trata del super-ego, cuya tensión con el ego produce ``el sentimiento de culpabilidad'' que se manifiesta como ``necesidad de castigo'', tema al que le dedica un amplio desarrollo.

``A mi juicio -termina diciendo Freud- el destino de la especie humana será decidido por la circunstancia de si -y hasta qué punto- el desarrollo cultural logrará hacer frente a las perturbaciones de la vida colectiva emanadas del instinto de agresión y de autodestrucción''. Y nosotros -más de sesenta años después- podemos terminar este resumen con las mismas palabras con que Freud termina su ensayo: ``Mas, quién podría augurar el desenlace final?''.

El esquema freudiano no es adecuado para formular explicaciones formalmente rigurosas de los fenómenos políticos y sociales, y de hecho se lo ha utilizado poco, aunque es incuestionable la profundidad y agudeza de muchas de sus observaciones y reflexiones. La muy citada ``aplicación'' que hizo Harold Lasswell\footnote{Harold D. Lasswell: PSICHOPATOLOGY AND POLITICS, Viking Press Inc., 1962.} es, en realidad, un intento de aplicar el método, no la teoría en su conjunto.

Las críticas a la teoría freudiana ortodoxa son muy conocidas: su carácter de sistema cerrado, su organización de ``escuela'', con la consiguiente intolerancia teórica, su vaguedad conceptual, su falta de definición empírica, su oscilación incierta entre el uso simbólico y concreto de los vocablos. Pero indudablemente es una poderosa vertiente nutricia del pensamiento contemporáneo, como lo testimonian, por ejemplo, la ``Escuela de Frankfurt'' y la corriente de ``Crítica Social'', que veremos luego.

El freudismo tiene valor heurístico, capacidad de sugerencia, de apertura y de ampliación de líneas de investigación. Creemos que al leerlo, todos sentimos la estimulación de un pensamiento poderoso, que se atreve a nombrar a las cosas de modos nuevos, que nos atrae y repele a la vez, que nos presenta al hombre y a la vida bajo aspectos que con frecuencia nos chocan, pero en los que también percibimos duras verdades y ominosos anticipos del drama contemporáneo. Estemos o no de acuerdo con sus teorías, hay en la historia de la ciencia contemporánea un antes y un después de Freud, que a nuestro juicio está marcado por la incorporación sin cuestionamientos de la dimensión psicológica profunda -lo emocional, lo irracional, lo inconsciente- en todos los estudios de lo humano.

\hypertarget{el-neo-freudismo-o-psicoanuxe1lisis-socializado}{%
\subsection*{El neo-freudismo (o psicoanálisis socializado)}\label{el-neo-freudismo-o-psicoanuxe1lisis-socializado}}
\addcontentsline{toc}{subsection}{El neo-freudismo (o psicoanálisis socializado)}

Una característica a nuestro entender negativa de la ``Escuela de Viena'', fundada por Freud y sus primeros seguidores, fue justamente su carácter de ``escuela'', es decir, de cultivo grupal de un conocimiento proveniente de una fuente ``paternal'' y cerrado a todo cuestionamiento y revisión externa, algo opuesto al carácter abierto de la ciencia empírica corriente, que no reconoce padre ni principio de autoridad alguno. Cuando una corriente teórica se estructura como ``escuela'', por elevados que sean los motivos preservadores aducidos para ello, fatalmente se configura una ``ortodoxia'' interna y una ``heterodoxia'' nutrida por todos aquellos cuyo pensamiento se aparta de la orientación originaria. Esto ocurrió con la ``escuela'' de Freud.

En 1911, Alfred Adler (1870-1937), médico oftalmólogo vienés, que de su especialidad había pasado a la psicoterapia, y había sido uno de los primeros discípulos de Freud, fundó su propia escuela, rompiendo con Freud y su círculo de Viena, con lo que culminaron los enfrentamientos que venían sosteniendo desde 1905 por discrepancias sobre el rol de la sexualidad en la conducta humana. Mientras Freud le asignaba un papel dinámico central, Adler ubicaba en ese lugar a ``la voluntad de poder'', compensación de la inferioridad constitucional inicial del hombre. Esa voluntad de poder, en el lenguaje propio de Adler, debe entenderse más en el sentido de ``dominio de sí'' que de ``dominio de los otros''.

Adler era socialista; no compartía, por lo tanto, la visión profundamente pesimista y hobbesiana que Freud tenía del hombre. Adler postuló la existencia de una tendencia social innata, una propensión a atender el interés general tanto como el personal.

Otros psicoanalistas, también discípulos de Freud en sus comienzos, como Carl Jung, Otto Rank y Wilhem Stekel, siguieron el ejemplo de Adler, separándose de la ortodoxia freudiana; pero indudablemente fue la obra de Adler la más fecunda en el campo del pensamiento político sistemático, en particular por medio de la llamada corriente o escuela de la ``crítica social'', en la que se destacan los nombres de Karen Horney, Erich Fromm y Harry Stack Sullivan. También puede discernirse su influencia en autores importantes de otras corrientes, como Norman Brown, Herbert Marcuse y Theodore Adorno.

La escuela de la ``crítica social'' es un foco de convergencia de poderosas corrientes del pensamiento contemporáneo, que van desde el freudismo al marxismo, desde la Antropología Cultural hasta la Teoría del Campo y la Semántica, para enfrentar desde una actitud severamente crítica las realidades de la sociedad y la cultura contemporáneas, especialmente su versión anglosajona y específicamente norteamericana, tomada como modelo dominante y difundido mundialmente.

\hypertarget{karen-horney}{%
\subsection*{Karen Horney}\label{karen-horney}}
\addcontentsline{toc}{subsection}{Karen Horney}

Karen Horney (1885-1952), psicoanalista nacida en Alemania, desarrolló prácticamente toda su labor en los EE.UU. Inspiró su revisión del freudismo en las ideas de Adler, con un sesgo aún más radical, aunque algunos de los seguidores de éste llegaron a acusarla de plagio.También se percibe en sus obras la influencia de antropólogos de la corriente culturalista norteamericana, como Margaret Mead, Ruth Benedict y Edward Sapir; y de la crítica dialéctica del marxismo. En 1941 rompió sus vínculos con la ortodoxa ``New York Psychoanalytic Society'' y fundó la ``Association for the Advancement of Psychoanalysis''.

Entre sus principales obras cabe citar ``New Ways in Psychoanalysis'' (1939), ``Self-Analysis'' (1942) y ``Our Inner Conflicts: a Constructive Theory of Neurosis'' (1945) entre otras. Del freudismo, Karen Horney tomó la motivación inconsciente, el determinismo psíquico, la importancia de los sueños y los mecanismos de defensa del ego. Rechazó, en cambio, la teoría del instinto: la reemplazó por la consideración teórica de las influencias del ambiente; también redujo la importancia de la motivación sexual, a la que consideró producto y no causa de la ansiedad. Otros aspectos (Edipo, Libido, Thanatos) fueron considerados consecuencias de las relaciones culturales o interpersonales.

Karen Horney fue una psiquíatra clínica, preocupada por problemas terapéuticos de índole, si se quiere, individual, pero su obra tiene partes de gran interés para las ciencias sociales en general y para la Ciencia Política en particular: - los objetivos primordiales del hombre son la seguridad y la sa-tisfacción (en ese orden); - el hombre busca el poder para autoprotegerse; - la ansiedad surge de las relaciones humanas, con sus componentes de miedo, inseguridad, soledad, hostilidad; no surge de impulsos genéticos; - la ansiedad es el orígen de la neurosis; - la ansiedad es inducida por la cultura; - la moderna cultura norteamericana es un perfecto caldo de cul-tivo de la ansiedad, y por consiguiente de la neurosis.

Karen Horney señala cinco rasgos de la cultura norteamericana contemporánea que son los factores principales de la ansiedad: - la competencia, con la inseguridad y la hostilidad que implica; - las desigualdades en el acceso a la cultura y a los demás bie-nes de la vida; - el miedo a la reprobación, agravado por el fomento de la culpabilidad; - la falta de fundamento de las relaciones humanas; -las contradicciones entre los valores culturales y las realida-des de la vida social.

Karen Horney sugirió en diversas formas que una combinación de crítica social marxista y de análisis freudiano de la neurosis puede forjar un arma coherente de crítica social, que promueva modificaciones culturales para reducir la ansiedad producida en la sociedad Karen Horney: LA PERSONALIDAD NEUROTICA DE NUESTRO TIEMPO, Bs. As., Paidós, 1963.

\hypertarget{erich-fromm}{%
\subsection*{Erich Fromm}\label{erich-fromm}}
\addcontentsline{toc}{subsection}{Erich Fromm}

Erich Fromm (n.~1900 ), psicoanalista nacido en Alemania y afincado en los EE.UU., procura reinterpretar el psicoanálisis freudiano mediante la utilización de esquemas sociológicos e históricos tomados de Marx, cuya influencia es muy marcada en él; así como la de Max Weber, en su lectura de las relaciones entre capitalismo y protestantismo. Entre las principales obras de Erich Fromm cabe citar ``Scape for Freedom'' (1941) ``Man for Himself: an Inquiry into the Psychology of Ethics'' (1947), ``The Sane Society'' (1955) y ``Marx's Concept of Man'' (1961) entre otras\footnote{Erich Fromm: EL MIEDO A LA LIBERTAD, Bs. As., Paidos, 1971. Ver también HUMANISMO SOCIALISTA, Bs. As., Paidos, 1971.}.

Erich Fromm no se dedicó a la psiquiatría clínica sino a la crítica social, campo en el que alcanzó gran difusión e influencia como autor muy leído y comentado, sobre todo entre la juventud de la posguerra. Como crítico social utiliza concepciones psicoanalíticas, sobre todo adlerianas, para describir los males sociales, pero los explica en términos histórico-dialécticos esencialmente marxistas.

El interés de Erich Fromm se centra en las condiciones del medio en el que se forma la persona. Considera que el carácter es producto del ambiente, y que éste es configurado por la dinámica del sistema de producción. A diferencia de Karen Horney, Erich Fromm enfatiza más la importancia del modo de producción que la de las relaciones personales. Fromm concibe al hombre esencialmente como un productor, cuya actividad laboral define en lo esencial el sentido (o falta de sentido) de su vida. En ese contexto, Fromm considera que existen cinco necesidades humanas fundamentales: - la relación con los demás; - trascender la naturaleza; crear; - tener arraigo en un tiempo y lugar; - tener estabilidad; - tener un cuadro de orientación.

Para Fromm, el hombre occidental moderno es un solitario, enajenado de su trabajo, incapaz de mantener relaciones humanas fructíferas, insatisfecho, neurótico e infeliz. La causa de esa triste situación es el capitalismo, que exige rasgos de carácter incompatibles con las necesidades humanas: conformismo, competencia, formalidad, puntualidad, morigeración, control, racionalización, inserción en estructuras de escala sobrehumana. En esas condiciones, la vida es estéril e insatisfactoria, enajenante, proclive a la sumisión masoquista o a la dominación sádica, al conformismo o al poder.

Es el Amor -sostiene Fromm- la única fuente de seguridad interior y de relaciones humanas sólidas; de una adecuada consideración de sí mismo y de los demás. El Amor es dador de sentido a la existencia.

El valor y la repercusión de Erich Fromm como crítico social han sido grandes. No ocurre lo mismo en el plano teórico, por la debilidad conceptual y metodológica de su construcción intelectual.

\hypertarget{harry-stack-sullivan}{%
\subsection*{Harry Stack Sullivan}\label{harry-stack-sullivan}}
\addcontentsline{toc}{subsection}{Harry Stack Sullivan}

Harry Stack Sullivan fue un psiquíatra clínico nacido y educado en los EE.UU., que experimentó la influencia del freudismo pero la trascendió, creando sus propios conceptos. Su obra es la estructura conceptual no freudiana más importante de que dispongamos hoy en Psiquiatría. Entre sus obras cabe citar a ``The Interpersonal Theory of Psychiatry'' (1953) y ``The Fusion of Psychiatry and Social Science'' (1964)\footnote{Harry Stack Sullivan: THE INTERPERSONAL THEORY OF PSYCHIATRY, W.W. Norton and Co.~Inc., 1953; THE FUSION OF PSYCHIATRY AND SOCIAL SCIENCE, W.W. Norton and Co., 1964.}. Como puede advertirse desde el título de esta última obra, Sullivan fue también un notable psicólogo social, que intentó tender un amplio puente entre la Psicología y las ciencias sociales. Esta parte de su obra tiene un especial interés para la Ciencia Política.La filiación intelectual de su obra reconoce la influencia del empirismo lógico de Percy Bridgman, de la obra de antropólogos y sociólogos como George Mead, Ruth Benedict y Bronislaw Malinowski. Tomó también elementos de la Teoría del Campo, de Kurt Lewin y nociones de semántica y comunicación de Edward Sapir.

Harry Stack Sullivan es cauto, empírico, riguroso. Concibe al individuo como un ``centro'' de un conjunto de interacciones. Esas interacciones de algún modo ``son'' la persona, organismo en desarrollo, de cambiantes necesidades, que atraviesa a lo largo de su vida siete etapas de crecimiento: infancia, niñez, juvenil, preadolescente, adolescente, adolescente tardía y adulta.

Sullivan tiene del amor una idea similar a la de Fromm. Para él, las relaciones amorosas son la principal fuente de satisfacción de la vida humana. El organismo -dice Sullivan- es un homeóstato que se mueve entre dos polos de tensión absoluta: el terror y la euforia. La conducta humana está orientada a aliviar las tensiones que tienden a llevar al hombre hacia esos extremos. Hay otro generador de conductas: la ansiedad. Para Sullivan, las necesidades son específicas y pueden ser satisfechas; la ansiedad, en cambio, es general y no puede ser satisfecha del mismo modo. La seguridad es lo que nos permite relajar la ansiedad, y no se puede obtener seguridad sin ayuda del medio exterior. La satisfacción -o sea el cumplimiento de las necesidades- y la seguridad -o sea el alivio de la ansiedad- son los dos principales objetivos de los seres humanos.Una persona adulta, con una personalidad plenamente integrada, desarrolla formas de conducta que satisfacen sus necesidades internas y las exigencias externas de la sociedad. Nadie puede buscar su propia satisfacción -sostiene Sullivan- sin atender las consecuencias de sus acciones para los demás.Quizás sea ésta una de sus aportaciones más valiosas a una visión madura y equilibrada de la vida individual y social.

Tanto Sullivan como todos los autores antes mencionados han tenido mucha influencia en el pensamiento social y político contemporáneo, como expresión de una necesaria interdisciplina en el arduo proceso de conocer las dimensiones interactuantes de la dinámica existencial humana, psicológica, social, económica y política. Hay grandes áreas que la Ciencia Política no puede explorar sin apoyo de la Psicología. De ahí el interés de este repaso que acabamos de hacer.

\hypertarget{cuarta-parte-1}{%
\section*{Cuarta parte}\label{cuarta-parte-1}}
\addcontentsline{toc}{section}{Cuarta parte}

\hypertarget{el-formalismo---la-teoruxeda-de-los-juegos---la-teoruxeda-de-la-informaciuxf3n-y-la-cibernuxe9tica---modelos-y-simulaciones}{%
\subsection*{El Formalismo - La Teoría de los Juegos - La Teoría de la Información y la Cibernética - Modelos y simulaciones}\label{el-formalismo---la-teoruxeda-de-los-juegos---la-teoruxeda-de-la-informaciuxf3n-y-la-cibernuxe9tica---modelos-y-simulaciones}}
\addcontentsline{toc}{subsection}{El Formalismo - La Teoría de los Juegos - La Teoría de la Información y la Cibernética - Modelos y simulaciones}

El Formalismo es una interesante dirección en el desarrollo del pensamiento político contemporáneo. Se manifiesta en el creciente empleo, para el estudio de la política, de modelos formales, simulaciones y estructuras matemáticas. Es un procedimiento y un lenguaje que va más allá del simple empleo del cálculo de probabilidades y de la estadística para manejar datos políticos.\footnote{Eugène J. Meehan: PENSAMIENTO POLITICO CONTEMPORANEO, Madrid, Revista de Occidente, 1973.} El Formalismo incluye cuatro tipos de actividad: 1) La creación de modelos (lógicos, matemáticos o informales) utilizables en estudios políticos; 2) La aplicación de dichos modelos al estudio de fenómenos políticos; 3) El análisis de los problemas metodológicos y prácticos que plantean las actividades ya mencionadas; 4) El empleo de conceptos tomados del formalismo pero usados fuera de su contexto originario.

El Formalismo es una orientación bastante reciente en el campo de los estudios políticos. La experiencia de los últimos treinta años no ha visto ni el florecimiento que pronosticaron sus partidarios iniciales ni la extinción que presagiaron sus críticos, sino que lo han ubicado como una herramienta metodológica y analítica entre otras.

Quedan en pié, evidentemente, algunos obstáculos para un mayor desarrollo: el principal a nuestro juicio es que hay algunos aspectos de la política (su dimensión simbólica, mítica y emocional) que no se prestan ni se prestarán nunca a un manejo matemático formal: éste es irreductible; pero hay otros cuya consistencia dependerá del sentido que tomen los futuros desarrollos: hay pocos politólogos con suficiente preparación y vocación lógica y matemática; hay pocas ramas de las matemáticas que resulten útiles para los estudios políticos; la complejidad de los fenómenos políticos, fuertemente interactivos y significantes, exceden frecuentemente las posibilidades representativas de los modelos matemáticos disponibles; y sobre todo el hecho de que el entusiasmo por el formalismo procede fundamentalmente de la Economía y está muy impregnado de su estilo, que es próximo pero no exactamente coincidente con el de la Política. De hecho, algunas de las más significativas aplicaciones de modelos matemáticos al estudio de la política han sido hechas por economistas, como es el caso de ``An Economic Theory of Democracy'' de Anthony Downs\footnote{Anthony Downs: AN ECONOMIC THEORY OF DEMOCRACY, Harper and Row Publishers, 1957.}, o por matemáticos, como Herbert A. Simon y su obra ``Models of Man: Social and Rational''\footnote{Herbert A. Simon: MODELS OF MAN: SOCIAL AND RATIONAL, John Wiley and Sons Inc., 1957.}.

Vamos a repasar ahora algunos detalles de las cuatro actividades recién mencionadas.

\hypertarget{la-creaciuxf3n-de-modelos-aplicables}{%
\subsection*{La creación de modelos aplicables}\label{la-creaciuxf3n-de-modelos-aplicables}}
\addcontentsline{toc}{subsection}{La creación de modelos aplicables}

En principio, la creación de modelos lógicos o matemáticos es tarea de lógicos o matemáticos, pero si esos modelos han de ser útiles para el estudio de la política, sus axiomas han de tener relevancia para los datos empíricos que maneja el politólogo. Cómo coordinar ambas especialidades? No es fácil que buenos matemáticos se intereses por los problemas específicos del politólogo, que suelen resultarle poco atractivos desde el punto de vista formal. El politólogo ha de plantear sus propios modelos, pero aquí aparece el problema de la escasa predisposición y formación matemática que suele caracterizar a los científicos sociales en general.

Hay muchas obras de gran valor como orientación para el politólogo interesado en las posibilidades que ofrece la aplicación de las matemáticas en su campo. Entre ellas cabe citar: F. Massarik y P. Ratoosh: MATHEMATICAL EXPLORATIONS IN BEHAVIORAL SCIENCE, Irwin Inc.~y Dorsey Press, 1965; K.Arrow, S. Karlin y P. Suppes: MATHEMATICAL METHODS IN THE SOCIAL SCIENCES, Stanford University Press, 1960; J. Kemeny y J.L. Snell: MATHEMATICAL MODELS IN THE SOCIAL SCIENCES, Ginn and Co., 1962; J.L. Bernd: MATHEMATICAL APPLICATIONS IN POLITICAL SCIENCE, Arnold Foundation, 1966; Hayward Alker: EL USO DE LA MATEMATICA EN EL ANALISIS POLITICO, Ed. Amorrortu.

Del mismo modo que el físico, el politólogo que tenga interés en este campo ha de poder emplear las matemáticas sin ser matemático él mismo, lo que plantea un interesante problema en el terreno de las currículas universitarias para la formación profesional.

\hypertarget{aplicaciuxf3n-de-modelos-a-fenuxf3menos-concretos}{%
\subsection*{Aplicación de modelos a fenómenos concretos}\label{aplicaciuxf3n-de-modelos-a-fenuxf3menos-concretos}}
\addcontentsline{toc}{subsection}{Aplicación de modelos a fenómenos concretos}

La utilidad de los modelos formales para la investigación científica depende de sus propiedades, de la naturaleza de los datos de que se dispone y de la finalidad de la investigación.

Un modelo formal es un conjunto de elementos cuidadosamente definidos y de reglas para manejarlos. Comprende un conjunto de axiomas y todos los postulados de teoremas que pueden deducirse de esos axiomas siguiendo los cánones de la lógica formal.

La Lógica y las Matemáticas son técnicas para manejar las interrelaciones entre ``cosas'' especificadas que se comportan de un modo particular; las matemáticas y la lógica realizan un razonamiento abstracto acerca de las interacciones entre símbolos especificados, estando también total y cuidadosamente especificadas las reglas de la interacción. El producto es un modelo representativo de los resultados de esas interacciones. Estos modelos formales en sí mismos no tienen nada que ver con la realidad empírica: son construcciones racionales derivadas de axiomas.

El objeto de la investigación política puede ser la explicación de fenómenos políticos, su valoración o ambas cosas. El valor del formalismo ha de juzgarse según su contribución a dichas finalidades. Los modelos formales no reemplazan a la teoría. La utilización de modelos puede ser un instrumento o recurso para facilitar la explicación o la evaluación, pero los modelos en sí mismos no explican ni evalúan. Son una analogía, una aproximación cuyo valor metodológico sólo puede ser establecido con referencia a casos concretos.

Qué se logra reduciendo las cuestiones empíricas a términos lógico-formales? Los resultados varían según los casos: clarificar el problema, obtener una cuasi-comprobación de los supuestos, cuando no es posible la experimentación (por ejemplo, usando simulaciones en el análisis de las relaciones internacionales); disponer de un instrumento para intentar la formulación de predicciones.

Los aspectos negativos, o mejor dicho, los riesgos que involucra el empleo de modelos formales son: darle más importancia a los símbolos que a lo que representan, atender más a la lógica de las relaciones formales que a las interacciones reales; caer en una hipersimplificación de los procesos; forzar los hechos para que se acomoden al modelo; olvidar que los modelos no son teorías. No hay reglas para evaluar la utilidad de los modelos formales en la investigación política. Es un problema de buen juicio y experiencia. Los modelos no tienen valor en sí mismos, excepto para los lógicos y los matemáticos.

\hypertarget{anuxe1lisis-de-los-problemas-metodoluxf3gicos-y-pruxe1cticos-del-formalismo}{%
\subsection*{Análisis de los problemas metodológicos y prácticos del formalismo}\label{anuxe1lisis-de-los-problemas-metodoluxf3gicos-y-pruxe1cticos-del-formalismo}}
\addcontentsline{toc}{subsection}{Análisis de los problemas metodológicos y prácticos del formalismo}

Ya hemos visto que, si bien son escasos, hay algunos intentos de aplicar modelos matemáticos a la descripción-explicación de fenómenos políticos. En cambio, abundan los trabajos referidos a los problemas metodológicos y prácticos del formalismo. Algunos se ocupan del tema en general, otros tratan de algún tipo determinado de modelos, como la Teoría de los Juegos, por ejemplo.

Como orientación bibliográfica general sobre este tema, cabe citar las siguientes obras: J. Charlesworth: MATHEMATICS AND THE SOCIAL SCIENCES, American Academy of Political and Social Science, 1963; Harold Lasswell et al.: THE POLICY SCIENCES: RECENT DEVELOPMENTS IN SCOPE AND METHOD, Stanford University Press, 1951; Carl J. Friedrich: NOMOS VII: RATIONAL DECISION, Atherton Press, 1964.

En lo que se refiere específicamente al campo de la Ciencia Política, cabe citar muy especialmente a Karl Deutsch: LOS NERVIOS DEL GOBIERNO, México, FCE, 1985, al cual nos vamos a referir ampliamente más adelante (Cap. 5).

\hypertarget{el-empleo-de-conceptos-tomados-del-formalismo}{%
\subsection*{El empleo de conceptos tomados del formalismo}\label{el-empleo-de-conceptos-tomados-del-formalismo}}
\addcontentsline{toc}{subsection}{El empleo de conceptos tomados del formalismo}

En la investigación politológica suelen también emplearse esquemas conceptuales derivados del formalismo pero que no son en sí mismos sistemas formales. Es muy difícil evaluar esos ``modelos informales''. Su valor depende de su utilidad para la investigación, aunque también puede cuestionarse su validez científica. Tienen, a nuestro criterio, las mismas fragilidades que las analogías, de cierto valor didáctico y heurístico pero de poca consistencia para una descripción-explicación de base empírica.

Volviendo ahora al tema general del Formalismo, diremos que, dentro de esta corriente teórica, el modelo más empleado es la Teoría de los Juegos. Más raramente se utiliza la Teoría de la Comunicación y la Cibernética. Una tercera clase de modelos, frecuentemente utilizada en ciertos campos específicos como el análisis de la política exterior o de interacciones políticas internas en curso, son las Simulaciones, que pueden ser diseñadas para su representación mediante actores vivos (``simuladores'') o mediante computadoras.

\hypertarget{la-teoruxeda-de-los-juegos}{%
\subsection*{La Teoría de los Juegos}\label{la-teoruxeda-de-los-juegos}}
\addcontentsline{toc}{subsection}{La Teoría de los Juegos}

John von Neumann y Oskar Morgenstern publicaron en 1944 un libro titulado ``Theory of Games and Economic Behavior'', que puede ser considerado el orígen de la Teoría de los Juegos\footnote{Eugène J. Meehan, op. cit. Ver también Karl Deutsch: LOS NERVIOS DEL GOBIERNO, México, FCE, 1985.}. En esta obra, los autores mencionados presentaron nuevos enfoques sobre el estudio de las decisiones económicas, políticas y sociales, y más en general, sobre las estrategias para la toma de decisiones.

Este nuevo enfoque se basa en la existencia de notables similitudes entre las situaciones sociales habituales y algunos juegos normados. Estas similitudes -sostiene la teoría- no son accidentales. Los hombres encontramos más interesantes aquellos juegos que evocan prácticas sociales o que permiten representar experiencias sociales bajo la forma simbólica e ``inofensiva'' de un juego: jugar al ajedrez en lugar de hacer la guerra, jugar al póker en lugar de engañar a los demás en la política o en la vida económica\ldots o como forma de entrenamiento para hacer la guerra o el engaño\ldots{}

Las similitudes entre los juegos y la vida real se producen -según la teoría- sobre todo en tres aspectos: - la existencia de recompensas y castigos a los jugadores, relacionadas con la racionalidad de sus decisiones; - la dependencia de dichos premios y castigos respecto de la in-teracción de las decisiones de los jugadores; - el estado de incertidumbre e información incompleta en que los jugadores deben tomar sus decisiones.

El paralelismo del juego con la acción política práctica es muy claro. En la vida política, como en el juego, es fundamental: - reconocer el propio interés y actuar en forma adecuada para lo-grarlo; - tomar adecuadamente en cuenta las probables acciones de los ad-versarios y de los aliados; - actuar con prudencia en condiciones de incertidumbre y conoci-miento parcial de los hechos.

Esta teoría parte de la afirmación del valor de los juegos para analizar comportamientos políticos, y sobre esa base analiza prototipos simplificados de juegos como el ajedrez o el póker; calcula las probabilidades de triunfo de cada jugador en cada mano y determina las condiciones para constituir coaliciones ventajosas, evaluando las estrategias alternativas que aumenten las probabilidades de éxito.

En estos juegos, las decisiones se toman en condiciones de incertidumbre. En el póker no conocemos la mano del adversario ni las cartas que vienen en el mazo. En el ajedrez, ignoramos la estrategia del adversario. De manera similar, en política nacional e internacional las decisiones son tomadas, y las coaliciones son hechas y rehechas en condiciones de información incompleta sobre el presente y de incertidumbre respecto del futuro.

La Teoría de los Juegos ha promovido un nuevo modo de pensar en ciencias sociales, que busca llegar a formulaciones ``conceptualmente cuantificables'',expresables por medio de exactas representaciones matemáticas, lo que obliga a una mayor precisión en la definición de los términos y las operaciones practicables para probar o medir cada concepto. Aquí aparece para nosotros la primera gran duda: si esa precisión que se logre en el juego-modelo de representación de una interacción política se corresponde o no con una precisión semejante en la realidad misma; en otras palabras, si no estaremos forzando demasiado a la realidad para que entre en un estrecho molde rígido de valores y relaciones cuantificadas, de lo que resulte una caricatura más que una representación.Dicho ésto, aceptamos también que las caricaturas suelen representar y hasta enfatizar con acierto los rasgos dominantes de la realidad\ldots pero su valor científico es escaso.

La Teoría de los Juegos afirma, en el ámbito de cada juego, el llamado ``supuesto de transitividad'': si un caballo vale más que una sota y un rey más que un caballo, un rey vale más que una sota. Ahora bien, en la realidad biológica, psicológica y social muchas veces este supuesto no se aplica, y se dan con frecuencia situaciones ``no-transitivas'' o ``en rizo'': A come a B, B come a C pero C come a A. Especialmente en política son muy frecuentes estos casos, que suelen ser usados como modos de contrabalancear poderes: el Parlamento puede destituir al Primer Ministro, pero el Primer Ministro puede disolver al Parlamento y convocar a nuevas elecciones; los votantes pueden derrotar al Parlamento anterior pero el Parlamento puede postergar las elecciones, etc.

De modo semejante cabe analizar en forma crítica las ideas de la Teoría de los Juegos sobre la ``transitividad'' del sistema de decisiones políticas. La idea de que cada sistema político debe tener una sola instancia final de decisión a veces corresponde a la realidad y muchas otras veces no. Es frecuente, por ejemplo, que los subsistemas estén dotados de autonomía -vale decir, de autoconducción y autocontrol- y que no sean, por lo tanto, completamente transitivos. También es muy frecuente -casi general, diría- que las decisiones sean producto de complejos procesos de interacción entre los elementos del sistema, aunque luego aparezca uno de ellos como promulgador formal de la decisión adoptada.

Lo que sí nos parece realmente muy valioso es la formulación de la Teoría de los Juegos sobre el tema de las soluciones o ``salidas'' de las situaciones. Quienes tenemos experiencia en análisis y evaluación de problemas y proyectos sabemos que nunca hay una sola solución para cada situación, aunque en general suele pensarse que, para cada conjunto de condiciones dadas, hay una solución mejor que cualquier otra. La Teoría de los Juegos, desde los tiempos de Neumann y Morgenstern, va bastante más allá: afirma que las soluciones no son únicas, que siempre hay múltiples soluciones para cada situación y que acaso haya más de una ``solución mejor que todas'', aunque lógicamente, la cantidad de soluciones estables es siempre necesariamente limitada en cada caso.

Por otra parte, la Teoría de los Juegos ayuda a poner en evidencia las diferencias que existen entre las estrategias que objetivamente tienen más probabilidades de éxito y las estrategias que son subjetivamente preferidas en función de los hábitos, deseos y necesidades del jugador; y concentra decididamente su interés en las primeras. Es una contribución nada desdeñable a la ``racionalidad'' de las soluciones.

Actualmente, la Teoría de los Juegos presenta restricciones que reducen sus posibilidades de aplicación en el campo de los problemas políticos. Cabe preguntarse, después de tántos años, si esas restricciones podrán superarse. Hasta ahora, la parte más desarrollada de la Teoría de los Juegos es la del juego de dos personas y suma cero, que es la parte menos útil para la Ciencia Política, donde el grueso de los problemas se dan en el contexto de un juego de varias personas y suma variable, ya que aún en el caso de la confrontación entre dos superpotencias dentro de un sistema bipolar, es muy gravitante la presencia y actuación de los demás actores internacionales, aliados o adversarios de uno u otro.

La Teoría de los Juegos es estática: supone que no se producen cambios en las características de funcionamiento de los elementos intervinientes mientras dure el juego, ni tampoco cambios en las reglas. Esto la aparta bastante de los procesos políticos reales, sobre todo en análisis de procesos de larga duración. Otra expresión de su estatismo radica en que ha resultado idónea para construir modelos de representación de procesos de distribución de recursos disponibles, no así para procesos que incluyen la creación de nuevos recursos. Los problemas relacionados con el crecimiento y la innovación quedan fuera de sus posibilidades. Von Neumann y Morgenstern reconocen el carácter estático de su teoría, pero consideran que su desarrollo es necesario para el posterior planteo de cualquier teoría dinámica.

En efecto, un desarrollo posterior intentó el análisis de procesos dinámicos mediante secuencias de juegos, en las cuales el resultado del primer juego determina la naturaleza del juego siguiente, y así sucesivamente. Quizás sea posible, por este camino, elaborar hasta ahora sólo se ha desarrollado con vigor la teoría estática de los juegos, por lo que es muy probable que la mayoría de los investigadores no otorguen una adecuada consideración a los factores dinámicos. Esto quizás no perjudique al póker, pero puede causar mucho daño a la lectura de la política interna o de las relaciones internacionales.

Otra dificultad emerge del tratamiento dado a los valores. La Teoría de los Juegos supone que los valores son definidos desde afuera, que no cambian y que son independientes de los resultados del juego. En realidad, al tomar decisiones políticas hacemos mucho más que jugar un juego: se trata de expresar los propios valores y al mismo tiempo, de sobrevivir como grupo. Casi todas las culturas creen que sus valores son compatibles con su sobrevivencia a largo plazo: tal creencia no siempre resulta verdadera, como lo ilustran con elocuencia numerosos casos a lo largo de toda la historia, desde los antiguos espartanos y los cátaros medievales hasta los caballeros del Sud esclavista norteamericano en el siglo XIX y los nazis en el siglo XX.

La limitación de fondo, para decirlo con mayor precisión y claridad, estriba en que la Teoría de los Juegos valora a las ``piezas'' en función de las reglas del juego. Cuando esas ``piezas'' son seres humanos nos encontramos con una seria objeción a la teoría: los seres humanos no derivan su valor de ninguna de sus actividades; son unidades irrepetibles, de valor intrínseco y propósitos múltiples.

Respecto de la estrategia frente a los riesgos de perder y las posibilidades de ganar, la Teoría de los Juegos recomienda la llamada ``minimax'', que consiste en tratar de perder lo menos posible aún a riesgo de que la ganancia sea también mínima. Se trata, evidentemente, de una estrategia defensiva y poco audaz, que suele inspirar desagrado a los verdaderos jugadores. Por ese lado, la Teoría de los Juegos no ha resuelto el problema de la toma de decisiones, ya que, aunque el minimax puede ser defendido como el comportamiento más racional, es bastante evidente que ese tipo de condiciones rara vez se da en la vida real.

Un buen ejemplo de lo cuestionable que resulta la Teoría de los Juegos cuando se pretende usarla como instrumento metodológico de la toma de decisiones, es el libro de Morton Kaplan ``System and Process in International Politics'' (1957)\footnote{Morton Kaplan: SYSTEM AND PROCESS IN INTERNATIONAL POLITICS, John Wiley and Sons, 1957, especialmente los cap. XI y XII.}. Kaplan, si bien es consciente de las insuficiencias de la Teoría de los Juegos, la considera ``el mejor instrumento de que se dispone para el análisis de los problemas de estrategia'' y que su empleo ``incrementará verosímilmente las probabilidades de éxito de una política''. Estas son pretensiones bastante excesivas y objetables, especialmente en lo que se refiere a las aplicaciones prácticas de la teoría analizadas en el capítulo ``Estrategia y Arte del Estadista'', en forma de un análisis puramente teórico, muy alejado del ámbito de la adopción real de decisiones.

La Teoría de los Juegos, en su planteo original, supone que toda la información pertinente para el juego está disponible y que su empleo puede hacerse sin limitación de tiempo o de costo. Estos supuestos resultan poco realistas en política.

Pese a las críticas que puedan hacerse, el interés de los teóricos por la Teoría de los Juegos se justifica por la clarificación conceptual que por esa vía se ha logrado en varios campos importantes de la investigación social: - la teoría de la negociación; - los estudios sobre conflictos internos y externos; - los estudios sobre relaciones de poder.

La Teoría de los Juegos, en su versión original, se presta muy bien para el análisis de juegos de suma cero, en los que cualquier ganancia de uno de los participantes significa una pérdida para el otro. Es apropiada, por lo tanto, para situaciones de antagonismo despiadado de intereses, como es, por ejemplo, un duelo. En ese sentido, es de temer que su uso desprevenido lleve a proyectar sobre la realidad las características del juego, y a ver en todo conflicto una confrontación irreductible de intereses que sólo puede resolverse por el aniquilamiento de uno de los adversarios, sin considerar las posibilidades de compatibilización transaccional por negociación ni la mutua necesidad de la presencia del otro, que son las situaciones realmente típicas de la vida política real.

Un enfoque más avanzado, más refinado, consiste en considerar que los dos contrincantes tienen intereses antagónicos y a la vez intereses en común, como dos superpotencias, en un sistema bipolar, que mantienen complejas relaciones de conflicto y colaboración. El estudio de este tipo de situaciones es característico de la obra de Thomas C. Schelling, especialmente ``The Strategy of Conflict''\footnote{Eugène J. Meehan, op. cit.}.

El estudio de Schelling sobre el conflicto muestra la utilidad de la Teoría de los Juegos como instrumento de clarificación conceptual. El trabajo de Schelling es una contribución importante, tanto a la Teoría de los Juegos en sí como a la demostración de su utilidad para la Ciencia Política. Schelling deja de prestar atención a los juegos de puro conflicto (que son los que en general apasionan a los matemáticos) y centra su interés en los juegos llamados ``de regateo'' o ``de motivos mezclados''; vale decir, aquellos en los que se combina el conflicto con la mutua dependencia; mucho más semejantes, por lo tanto, a las situaciones que se producen en la realidad política.

El trabajo de Schelling intenta, en forma muy brillante, hacer un análisis racional de la política internacional, basada en la amenaza como mecanismo de disuasión. Schelling sostiene que las amenazas sólo tienen sentido entre actores que tienen importantes intereses en común. No es precisamente útil en el caso de la ``hostilidad pura'' y de los ``intereses absolutamente contrapuestos'' sino justamente cuando los ``intereses mezclados'' producen esas complejas relaciones de colaboración y de conflicto a que aludimos páginas atrás.

Las amenazas -sostiene Schelling- son efectivas en función de su intensidad y de su credibilidad. Cuando la ejecución de tales amenazas implica un alto precio para el que las formula, o para el entorno global de ambos contendientes -como es el caso de la amenaza de emplear armas atómicas- el problema radica en cómo tornar verosímiles tales amenazas. En este sentido -dice Schelling- quizás resulten ventajosas para la negociación la torpeza, la temeridad, la ineptitud para prever el propio daño, así como el hecho de crear situaciones que tiendan a escapar del propio control.

Ciertas pautas de comportamiento de los niños, los presidiarios y los dementes recluídos en manicomios, así como ciertas técnicas de los secuestradores y los chantajistas -dice Schelling- pueden ofrecer lecciones valiosas para el manejo de la política exterior.

Realmente, es estremecedor pensar que durante años el destino del mundo estuvo en manos de gente nutrida con tales enfoques. Moral aparte, el límite para el empleo de estas técnicas está en que no funcionan en una relación prolongada, a lo largo de muchos años, con encuentros intensos y repetidos. Un desplante ocasional, un arrebato momentáneo, pueden sorprender alguna vez al adversario y permitir la obtención momentánea de alguna ventaja; pero no produciría el mismo efecto una sucesión de desplantes, que más bien ocasionaría la propia ruina. En este sentido, el brillante trabajo de Schelling debe ser interpretado más bien como una exploración intelectual de las posibilidades-límite de técnicas que sería ingenuo, y probablemente muy perjudicial, pretender aplicar en forma directa y, sobre todo, repetida.

En la realidad de la vida personal, los juegos tienen un límite, un término: los niños recogen las bolitas o la pelota y se van, cada uno a su casa. La política internacional, por el contrario, es un ``juego interminable'': los beneficios obtenidos en un momento dado difícilmente se mantengan o se repitan en otros momentos; ambos contendientes aprenden con el juego y mejoran su estrategia, etc.

Un factor decisivo en lo que respecta a la eficacia de las amenazas -que Schelling no menciona y que Deutsch destaca mucho- es la probabilidad de que el comportamiento que la amenaza intenta inhibir ocurra de todos modos: la necesidad y la motivación, intensamente sentidas, pueden llevar a no creer en amenazas, a no tomarlas en cuenta e incluso a reaccionar mediante conductas de violencia ``preventiva''. Cuando Schelling analiza los motivos del comportamiento en política exterior, distingue entre comportamientos inspirados en la racionalidad y comportamientos motivados por el despecho. Pero justamente, hay que tener en cuenta, como hace Deutsch, que las frustraciones repetidas aumentan la probabilidad de respuestas irracionales o despechadas. El temor o la tensión no siempre inhiben la conducta: también pueden producir reacciones agresivas.

En base a los supuestos de la Teoría de los Juegos se construyó, en tiempos de la ``guerra fría'', una ``teoría de la disuasión'', que proponía, por ejemplo, frustrar al adversario mediante un gran temor y luego confiar en su serena racionalidad para nuestra propia supervivencia. El análisis de los supuestos de la ``teoría de la disuasión'' revela una mezcla de la tradicional Teoría de los Juegos con elementos de la ideología nacionalista tradicional. Una manifestación concreta de ésto puede encontrarse en la estrategia del ``equilibrio del terror'', cuyos puntos de partida son los siguientes: - las aptitudes de los adversarios se mantienen estables en el tiempo; - las consecuencias de posibles cambios tecnológicos o económicos son desdeñables; - es muy baja la probabilidad de que se produzca una guerra acci- dental o inducida; - es mínima la probabilidad de que se produzca el comportamiento que se procura inhibir mediante amenazas verosímiles; - es despreciable el rol del interés nacional vital del adversa- rio y puede confiarse en su capacidad de actuar racionalmente aún mientras recibe amenazas intensas y verosímiles.

Esta estrategia supone, de un modo tácito o subyacente, la existencia de una asimetría o diferencia oculta en la manera de ser de los norteamericanos respecto de otros pueblos, como los rusos o los chinos. Se supone, por ejemplo, que las amenazas humillantes intimidan a los rusos e irritan a los norteamericanos, lo que condujo la conflictiva relación varias veces al borde de situaciones que hubieran hecho víctima del conflicto a toda la humanidad.

Una ``teoría de la disuasión'', para ser eficaz, tiene que ser útil, no para un encuentro aislado o para una breve crisis, sino para una larga interacción conflictiva. La teoría que acabamos de describir, dominante durante la década de los cincuenta y principios de los sesenta, presenta notables carencias y una gran debilidad en sus fundamentos intelectuales (para no hablar de los morales) y fue, en efecto, reemplazada por otras, especialmente por la teoría de la ``coexistencia pacífica'', que fue posible cuando ambos bandos reconocieron que la existencia del adversario era un hecho histórico duradero.

La teoría de la coexistencia pacífica planteó un complejo sistema de relaciones de colaboración y conflicto entre las dos superpotencias. Por un lado, conservó el ``equilibrio del terror'', basado en la ``capacidad del segundo golpe'' (quien es atacado por sorpresa y con éxito, aún en esas condiciones conserva la capacidad de aniquilar al agresor, lo que hace racionalmente impensable la agresión directa). Por otro lado, desarrolló una serie de relaciones de colaboración (venta de trigo subsidiado, intercambio tecnológico, colaboración espacial) así como de acción conjunta frente a algunos conflictos en el resto del mundo. Esas relaciones incluyeron también cierta prescindencia en los conflictos planteados en las áreas de influencia exclusiva de cada uno de ellas, excepto la posibilidad de formular declaraciones declamatorias de fuerte contenido ideológico y de ayudar clandestinamente y por vías indirectas, con armas, dinero e información, a los insurrectos de cada bando.

La reciente crisis económica y política del bloque socialista, especialmente de la URSS; la virtual y frágil hegemonía de los EE.UU., respaldada sólo por su potencia militar, sin apoyo de otras fuentes de poder (económico, tecnológico, cultural); la emergencia de otros centros de poder en el mundo (Comunidad Europea, Japón, China); son todos factores que están cambiando rápidamente el escenario internacional, que evoluciona desde un esquema bipolar, a través de una transitoria fase monopolar hacia un esquema probablemente tripolar. Es obvio que las estrategias basadas en una Teoría de los Juegos simple, de dos contrincantes, que pudo ser apta en algunos momentos del pasado para ese mundo bipolar que emergía de los acuerdos de Yalta, ya resultan notoriamente insuficientes.

En el libro de William R. Riker ``Theory of Political Coalitions''\footnote{Eugène J. Meehan, op. cit.} encontramos un modo muy formal de utilizar la Teoría de los Juegos: adoptar un modelo formal y contrastar las conclusiones derivadas de él con los datos empíricos, para obtener generalizaciones aplicables en otros estudios. Al mismo tiempo, presenta interesantes novedades en el uso de la teoría: el modelo formal adoptado por Riker es un juego de N jugadores y de suma cero; los jugadores son racionales, tienen información perfecta y pueden realizar pactos (coaliciones) entre sí, pero Riker amplía la noción de racionalidad al caso en que los jugadores, en lugar del ya comentado ``minimax'', opten por estrategias más audaces, que lleven a una posibilidad de mayor ganancia\ldots o de mayor pérdida.

El principal objetivo del trabajo de Riker es mostrar algunos de los principios que rigen la formación de coaliciones en el seno de los grupos: - el principio del ``tamaño''; - el principio ``estratégico''; - el principio del ``desequilibrio''.

El principio del tamaño sostiene que los participantes de un grupo sólo forman coaliciones del tamaño que creen necesario para asegurar su triunfo, y no mayores. Entraña afirmar que no hay un impulso integrador superior a la necesidad de asegurar el triunfo individual de los participantes. Este principio se complementa con el llamado ``efecto información'', según el cual cuanto menor es la información disponible, tanto mayor es el número de coaliciones que se busca formar y que exceden el tamaño mínimo. Es algo así como una aplicación de aquel principio general de que a mayor incertidumbre, mayores resguardos.

El principio estratégico o ``de la ventaja estratégica'' sostiene que, si en un estadio cualquiera del juego, unas protocoaliciones pueden formar una coalición mínimamente vencedora, tendrán una ventaja estratégica, que consiste en que pueden llegar a un acuerdo sobre el modo más ventajoso de distribuir las ganancias. Entre jugadores racionales, esta ventaja garantiza a quienes ocupen esa posición en el penúltimo estadio del juego, que pertenecerán en el último estadio a la coalición vencedora.

El principio del desequilibrio es la consecuencia del logro de la ventaja estratégica. Este principio destruye la suposición de que una política racional es estable bajo cualquier circunstancia. No hay ningún sistema de equilibrio de poderes que garantice la estabilidad. Las fuentes del desequilibrio son los cambios en la relación de poder entre los elementos del sistema, debidos a factores endógenos y exógenos y a las pretensiones acrecentadas de quien se perfila como vencedor.

\hypertarget{la-teoruxeda-de-la-informaciuxf3n-y-la-cibernuxe9tica}{%
\subsection*{La teoría de la información y la cibernética}\label{la-teoruxeda-de-la-informaciuxf3n-y-la-cibernuxe9tica}}
\addcontentsline{toc}{subsection}{La teoría de la información y la cibernética}

La influencia de la Teoría de la Información en el campo de las ciencias sociales ha sido indirecta y conceptual, pero así y todo, importante. Hay muy pocos ejemplos de su aplicación empírica directa en la investigación social, pero sí muchos rastros de su influencia.

La Teoría de la Información fue desarrollada en forma separada por Claude E. Shannon y Norbert Wiener. El objetivo de Wiener era ``separar un símbolo de un fondo que contiene muchas señales''. Shannon, por su parte, se interesaba por ``el problema de codificar eficazmente los mensajes y trasmitirlos con un mínimo de error y a la mayor velocidad posible por canales con ruido''.

El tema es similar (pero mucho más simple) que un clásico problema político: cómo puede conseguir el gobernante del país A que el gobernante del país B comprenda claramente el sentido y las intenciones de sus declaraciones. Los planteos de Shannon resuelven el problema de trasmitir información pero no el de trasmitir conocimiento, ésto es, ``significados en un contexto'', cuyo soporte informativo ya no es la letra o la palabra sino la frase. La enorme complejidad de este problema sugiere que probablemente la Teoría de la Comunicación tendrá sólo un impacto conceptual indirecto en las ciencias sociales.

En 1948, Wiener publicó ``Cybernetics''\footnote{Norbert Wiener: CYBERNETICS, John Wiley and Sons, 1948.} y dos años después ``The Human Use of Human Beings: Cybernetics and Society''\footnote{Norbert Wiener: THE HUMAN USE OF HUMAN BEINGS: CYBERNETICS AND SOCIETY, Doubleday and Co.~Inc., 1950.}. Estas obras fueron escritas con fines de divulgación, pero dieron comienzo a la llamada ``teoría cibernética'', que tuvo varios seguidores, entre los que cabe citar a W. Ross Ashby, con su obra ``An Introduction to Cybernetics''\footnote{W. Ross Ashby: AN INTRODUCTION TO CYBERNETICS, J. Wiley and Sons, 1956.}, y en el campo específico de la Ciencia Política, a Karl W. Deutsch, con obras como la ya citada ``The Nerves of Government: Models of Political Communication and Control''\footnote{ Karl Deutsch: LOS NERVIOS DEL GOBIERNO, México, FCE, 1985.} y ``Politics and Government''\footnote{Karl Deutsch: POLITICA Y GOBIERNO, México, FCE, 1976.}.

Wiener derivó el término ``cibernética'' de la palabra griega ``kibernetes'' que designa al que comanda una nave, al piloto, expresión de donde derivan también palabras como ``gobierno'' y ``gobernante''. La semejanza entre las tareas del dirigente político y las del piloto de un barco fueron reconocidas desde antiguo, al menos desde los tiempos de Platón.

Dice Karl Deutsch en ``Política y Gobierno'' que la política ``se ocupa primordialmente del gobierno, es decir, de la dirección y autodirección de las grandes comunidades humanas''; y enfatiza la analogía entre gobernar y pilotear, haciendo notar que ``el timonel de un barco debe tener información acerca de muchas cosas:\ldots dónde se encuentra el timón,\ldots dónde se encuentra él mismo en relación con\ldots su barco y lo que tiene que hacer para seguir controlándolo\ldots{} debe saber dónde se encuentra su barco, dónde se está moviendo y de que clase de barco se trata\ldots debe saber dónde se encuentra el medio ambiente importante para el barco -arrecifes, bancos de arena, aguas bajas, corrientes y canales de navegación- y dónde se encuentra su barco en relación con todas estas cosas. Por último, debe saber adónde quiere ir. Debe tener alguna idea de su meta, propósito o camino preferido y saber\ldots si el curso presente de su barco lo está aproximando o alejando de su objetivo''.

``Algo muy similar -prosigue diciendo Karl Deutsch- constituye el proceso de gobierno. Cualquiera que dirija los asuntos de un país -o de cualquier organización o comunidad grande- debe saber cómo permanecer en el control; cuál es la naturaleza básica y el estado actual del país u organización que está controlando; cuáles son los límites y oportunidades existentes en el medio al que debe enfrentarse y cuáles los resultados que desea obtener. Combinando estas cuatro clases de conocimientos y actuando en consecuencia, se tiene la esencia del arte del gobierno''.

Según Wiener y Ashby, el tema esencial de la cibernética es la regulación y control de todo tipo de máquinas y, por extensión, de todos los sistemas dinámicos y sus procesos. Es un campo tán amplio que no ha sido formalizado en su totalidad sino sólo parcialmente, con limitaciones y dificultades.

La unidad formal básica de la cibernética es la ``transformación'': un operador, actuando sobre un operando, produce un cambio denominado ``transición''. Una transformación es un conjunto de transiciones producidas por un mismo operador. Se considera sólo un tipo de transformación, denominado ``cerrado'', porque no contiene ningún elemento nuevo, lo que quiere decir que sólo produce aquellos efectos ya contenidos en los operandos.

Cuando se cumplen tales condiciones, el comportamiento de una máquina queda inequívocamente determinado. Resulta claro que con tales restricciones, esta herramienta crea serios problemas a los científicos sociales: en su campo no quedan claros cuáles son todos los efectos de un determinado operador, las transformaciones son con frecuencia ``abiertas'' ya que aparecen elementos nuevos, etc.

En su forma primaria y elemental, la cibernética es, pues, un método de análisis de las propiedades de ciertos sistemas llamados ``máquinas determinadas''. El mismo método puede usarse, si bien con menor precisión, para analizar sistemas de comportamiento no determinado ni aleatorio sino probabilístico o tendencial, o sea comportamientos que pueden describirse estadísticamente, porque presentan ciertas pautas de regularidad o predictibilidad. Esto ya tiene mayor interés para las ciencias sociales en general y para la Ciencia Política en particular, porque ese es el tipo de comportamiento más frecuente en su campo.

En el estudio de las ``máquinas determinadas'', la regulación y el control se definen en términos formales. Para aplicar el método formal de la cibernética es necesario conocer las variables esenciales del sistema y los estados necesarios para asegurar su existencia continuada. En los sistemas probabilísticos, en cambio, las transformaciones se convierten en ``procesos estocásticos''\footnote{Estocástico o conjetural: En Estadística, dícese de la relación que existe entre dos variables tales que, sin ser ninguna de ellas función de la otra, tampoco son independientes.} cuyas secuencias de estados se conocen con el nombre de ``cadenas de Markov'', que tienen algunos aspectos similares a los de las máquinas determinadas y otros bastante diferentes, como una consecuencia del carácter tendencial de su comportamiento.

En general, la Ciencia Política no ha experimentado una influencia directa importante de parte de la Cibernética ni de la Teoría de la Información. Un buen ejemplo de aplicación parcial de estas teorías es el ya citado libro de Karl W. Deutsch ``Los Nervios del Gobierno'' (1963). Deutsch utiliza muchos conceptos proveniente de la Cibernética y de la Teoría de la Información (como retroalimentación, entropía, canal, etc.) pero combinándolos con el uso corriente de los términos, con lo que pierden mucha de su precisión. Deutsch ha combinado conceptos y relaciones provenientes de Wiener con otros debidos a Warren S. McCulloch, especialista en Electrónica Neurológica, del M.I.T., y también usa nociones del sociólogo Talcott Parsons. Configura así una obra que es, a la vez muy sugerente y exasperante. Como vamos a ver bastante en detalle esta obra más adelante (ver Cap. 5) ahora daremos solamente algunas indicaciones generales.

Deutsch suscita muchas preguntas interesantes sobre las funciones del Estado, o mejor dicho, del Gobierno, pero ayuda poco a contestarlas: Qué factores influyen sobre el Gobierno al adoptar decisiones? Qué divergencia hay entre una demanda y la respuesta que el Poder le da? Qué nivel de eficacia tiene el Gobierno para prever problemas y tomar medidas preventivas? Qué nivel puede alcanzar? Haber suscitado tales preguntas quizás sea mérito de la Cibernética, pero no parece contribuir mucho a su respuesta. A nuestro criterio, la parte de esta obra más fecunda en sugerencias para la Ciencia Política, y también para la política práctica, son sus conclusiones sobre las relaciones existentes entre Política, Desarrollo y Aprendizaje Social.

Según Deutsch hay tres factores fundamentales para la perduración de toda sociedad o cultura: el desarrollo, la adaptabilidad y la capacidad de aprendizaje: El desarrollo es un incremento en la existencia y articulación de elementos para los fines propios del sistema, y abarca varias dimensiones: los recursos humanos (la población), el desarrollo económico, la disponibilidad de recursos materiales y humanos (las reservas operativas del sistema), el incremento de la autonomía (autodeterminación), la capacidad de cambiar sus propias pautas de organización y comunicación y la capacidad de cambiar de objetivos.

La adaptabilidad se expresa en el modo flexible de asumir las nuevas tensiones o desafíos originados en el ambiente, y se relaciona con la capacidad de aprendizaje. Ésta, a su vez, se manifiesta, por una parte, en la capacidad de orientarse hacia la búsqueda de nuevos objetivos; y por otra parte, en la capacidad de realizar modificaciones estructurales profundas en la propia organización para desarrollar funciones nuevas.

En el enfoque de Deutsch, la Política es imprescindible para alcanzar los objetivos señalados anteriormente. ``Si definimos al sector básico de la política -dice Deutsch- como el de todas las decisiones respaldadas por alguna combinación de probabilidades significativas de asentimiento voluntario y de coacción, la política se convierte en el método por excelencia que permite asegurar el tratamiento preferencial de los mensajes y las órdenes, y la redistribución de los recursos humanos y materiales; y aparece entonces como un instrumento fundamental para retardar o acelerar el aprendizaje social y la innovación, funciones para las cuales se la ha empleado en el pasado. La política ha sido empleada para aumentar la rigidez de los sistemas sociales ya semipetrificados y para acelerar los procesos de cambio en curso''.

Deutsch, finalmente, afirma que es una característica de la política y de los sistemas políticos de Occidente el hecho de haber desarrollado diversas técnicas cuya función o misión es, según la apreciación de este autor, acelerar la innovación y el aprendizaje social. Entre dichas ``técnicas institucionalizadas'' se destacan la regla de la mayoría, la protección de las minorías y la institucionalización del disenso. En ese contexto -dice Deutsch- la Política ``es una técnica para formular y llevar a la práctica las decisiones'' la cual, por esa razón, ``no es un fin en sí misma'' sino ``un instrumento esencial del aprendizaje social'': un ``instrumento de supervivencia y desarrollo'' más que de destrucción.

\hypertarget{modelos-y-simulaciones}{%
\subsection*{Modelos y Simulaciones}\label{modelos-y-simulaciones}}
\addcontentsline{toc}{subsection}{Modelos y Simulaciones}

No todos los modelos formales que se ha intentado emplear en Ciencia Política son derivados directamente de las Matemáticas. Hay también modelos que se construyen de modo semejante a los modelos de un avión o una presa hidráulica que los ingenieros construyen para estudiar su comportamiento en ciertos aspectos y bajo determinadas condiciones. Son las llamadas ``simulaciones''.

Para que puedan apreciarse bien sus rasgos característicos, vamos a analizar en forma comparativa dos casos destacados. El primero es un ejemplo clásico de ``modelo'' construído al modo de los economistas: se trata de la propuesta de Anthony Downs contenida en su obra ``An Economic Theory of Democracy''\footnote{Anthony Downs: AN ECONOMIC THEORY OF DEMOCRACY, New York, Harper and Row, 1957.}. El segundo es el planteo de Harold Guetzkow y su equipo, desarrollado en su obra ``Simulation in International Relations: Development for Research and Training''\footnote{Harold Guetzkow et al.: SIMULATION IN INTERNATIONAL RELATIONS: DEVELOPMENT FOR RESEARCH AND TRAINING, Prentice-Hall Inc., 1963.} y en algunos escritos posteriores de esa misma línea.

El modelo de política democrática de Anthony Downs, de típica inspiración economicista, es un modelo de dos elementos (partidos y votantes) que explora los efectos de la incertidumbre y del costo de la información sobre el comportamiento político. Su postulado fundamental es la ``racionalidad'': por medio de ella, el ``homo aeconomicus'' es introducido por Downs en la política, pero en forma limitada sólo a la elección de los medios para alcanzar los fines perseguidos. Los fines en sí mismos no son objeto de ninguna valoración racional. ``Racionalidad'', según Downs, significa que el actor siempre puede decidir entre alternativas ordenadas según su preferencia. El actor siempre elige la alternativa de más alta preferencia y la decisión, en circunstancias idénticas, es siempre la misma.

Los actores (partidos y votantes) se mueven en un entorno ``democrático'', lo que significa -según Downs- que el poder es ejercido por un partido o coalición elegido por votación popular y sometido a elecciones periódicas, cuya periodicidad no puede ser modificada por el que gobierna. Pueden votar todos los que reúnen requisitos legales mínimos y cada votante tiene un voto. El partido mayoritario gobierna hasta la próxima votación; los partidos minoritarios no buscan el poder por medios ilegales y el partido gobernante no los limita en modo alguno mientras actúen dentro de la ley; finalmente, en cada elección compiten dos o más partidos.

En el modelo de Downs los actores son racionales y egoístas; los partidos buscan el poder, y si ya lo ocupan, la reelección; los candidatos aspiran a los cargos para disfrutar de ellos; el votante calcula cómo satisfacer mejor sus fines particulares, y el Gobierno busca votos. En estas condiciones, los programas de acción política son una consecuencia accidental de la lucha entre individuos movidos por sus propias ambiciones. Sólo falta -comentamos nosotros- alguna alusión a la ``mano invisible'' para completar el paralelo con el mercado libre de Adam Smith.

Según Downs, existe un ``plan maestro'' gubernamental permanente, sólo modificado por ``alteraciones marginales'' y no por cambios fundamentales. En una situación de plena información, el Gobierno ha de actuar según la opinión de la mayoría en cada cuestión, lo que le asegurará el triunfo siempre que haya un consenso intenso, apasionado, y no solamente de opinión. Si no hay plena información (que es el caso más frecuente) hay incertidumbre y ésta tiene importantes consecuencias para el modelo: la incertidumbre hace posible la persuasión y produce competencia para ganar influencia por medio de su empleo. Los partidos elaboran ideologías destinadas a ``persuadir'' al elector de manera emocional, para captar su voto. Por otra parte, el ``costo de la información'' influye sobre la población de diversos modos: algunos se ven privados de votar, otros se ven motivados a votar y otros son inducidos a la abstención.

En base a su modelo, Downs formula dos axiomas: - En una democracia, los partidos planean su política con la intención de obtener el máximo de votos; - Todos los ciudadanos buscan incrementar al máximo sus beneficios.

De esos axiomas extrae algunas proposiciones. Muchas de ellas son ``lugares comunes'' (lo que no es poco, tratándose de un modelo formal) y otras tienen un especial interés: - Los partidos se ponen de acuerdo sobre aquellas cuestiones que suscitan un enérgico consenso de los ciudadanos; - Los gobiernos de coalición son menos eficaces que los gobiernos de un sólo partido; - Los gobiernos democráticos tienden a redistribuir la riqueza; - Muchos votantes no están bien informados acerca de las cuestiones sobre las que votan; - Para la mayoría de los ciudadanos,el incentivo para votar es pequeño; - Los gobiernos democráticos favorecen más a los productores que a los consumidores; - Los partidos elegidos para gobernar tienden a realizar lo más que pueden las promesas que han hecho; - En los países habitualmente gobernados por coaliciones, los votantes no ven a las elecciones como verdaderos mecanismos de selección de gobernantes.

El modelo de Downs demuestra el valor heurístico de estos procedimientos de investigación. Sus proposiciones son inferencias deductivas, pero al mismo tiempo son afirmaciones empíricamente verificables. Es cierto que la prueba empírica solamente convalida o falsea a los resultados y no al modelo del cual proceden, y que por lo tanto el modelo de Downs es más predictivo que explicativo; pero, como bien dice Milton Friedman, los modelos se prueban más por la exactitud de sus predicciones que por la veracidad de sus supuestos.

Un modelo es una estructura parcialmente isomórfica con una realidad empírica, normalmente más compleja que su modelo. De acuerdo a la naturaleza y extensión de ese isomorfismo será el uso que pueda hacerse del modelo. Generalmente, se busca que un modelo sirva para describir y explicar la realidad de referencia, o sea que ayude a construir su teoría. En ese sentido, el modelo de Downs tiene un alto valor heurístico y predictivo, pero su valor teórico-explicativo es limitado.

Algo similar ocurre con los trabajos de Guetzkow. La Simulación Internacional (INS) propuesta por Guetzkow tiene sus antecedentes en los ``juegos de guerra'' (``war games'') y en los ``juegos de empresa'', e invoca como basamento teórico a la Teoría de la Decisión y a la Dinámica de Grupos. Si bien incluye algunos procedimientos formalizados, el modelo en sí mismo es informal. Su propósito fundamental es heurístico y didáctico, e incluso de entrenamiento y análisis hipotético de situaciones, pero resulta cuestionable, como en el caso de Downs, su valor como instrumento de producción y comprobación de teorías.

El modelo originalmente propuesto por Guetzkow consta de cinco ``unidades nacionales'', gobernada cada una de ellas por un ``decisor'' cuyo objetivo principal es mantenerse en el cargo (otra vez aparece la motivación puramente egoísta) para lo que necesita el apoyo de sus ``mantenedores''. Cada unidad nacional dispone de recursos que puede acrecentar por medio de negociaciones y alianzas. El apoyo al decisor se basa en su gestión respecto de los recursos para consumo interno y seguridad nacional.

La dependencia del decisor respecto de sus mantenedores varía según una escala que marca diferencias entre democracias y totalitarismos. Hay dos sistemas de comunicaciones: un sistema de comunicación directa entre naciones y un ``periódico mundial'' que recoge declaraciones públicas de los actores. De estas comunicaciones están excluídas las formas de propaganda interna. Las naciones pueden comerciar entre sí, ayudarse, firmar acuerdos o hacerse la guerra.

Las naciones vencidas pueden ser ocupadas u obligadas a pagar indemnizaciones y esos recursos aumentan las disponibilidades de los vencedores. A su vez, las unidades nacionales ocupadas pueden organizar revueltas ``de liberación nacional''.

La simulación tiene una duración limitada en el tiempo. Como el objetivo primordial de los decisores es mantenerse en el poder, la política interna tiende a volverse protagónica, en perjuicio de la política internacional, que tiende a convertirse en un resultado casi accidental del choque de intereses egoístas, inspirados en dos valores fundamentales: el propio consumo y la propia seguridad.

El modelo ha sido objeto de muchas críticas, especialmente en cuanto a su pretendido realismo. Los principales defectos que se le encuentran son: que no hay verdadera oposición interna en cada entidad nacional, con sus exigencias de negociación transaccional; que la historia, con su impacto sobre la idiosincracia nacional, tiene poca influencia; que no se toman en cuenta los condicionamientos geográficos; que las comunicaciones están limitadas al nivel explícito, sin ``canales secretos''; que la propaganda está excluída, etc.

Con posterioridad, se ha procurado mejorar el modelo en cuanto a su realismo, acentuando el pluralismo y la interacción en la toma de decisiones de cada entidad nacional, tomando más en cuenta el contexto geográfico e histórico; permitiendo comunicaciones fuera del circuito oficial y expresando con mayor fidelidad los ``estilos nacionales'' de los distintos actores. Por otra parte, el auge de las comunicaciones a nivel planetario, vía satélite-computadora, ha permitido que el escenario de los participantes en estos juegos sea realmente internacional, ampliando al mismo tiempo el número de participantes. Se ha ensanchado también el ámbito de los valores fundamentales, más allá de los ya citados ``consumo'' y ``seguridad'', para hacer lugar a valores provenientes de cosmovisiones no puramente utilitaristas.

Los modelos de tipo INS (Simulación Internacional) tienen dos usos principales: - como instrumento de aprendizaje y entrenamiento activo para estudiantes y agentes de relaciones internacionales; - como campo de comprobación primaria de hipótesis sobre la estructura y funcionamiento del sistema internacional.

Finalmente diremos que muchas de las limitaciones teóricas que hemos señalado a propósito del formalismo no son exclusivas de éste sino rasgos propios del escaso desarrollo actual de la teoría política en general. Como bien dice Eugène J. Meehan\footnote{Eugène J. Meehan, op. cit., pg. 304.}: ``En la mayoría de los casos, la Ciencia Política se ocupa aún hoy fundamentalmente de la descripción más que de la explicación, y la investigación se concentra sobre todo en la búsqueda de conceptos que puedan ser utilizados para la descripción y proporcionen una base adecuada para el establecimiento de generalizaciones útiles. Dicho en otros términos, el''análisis" político es realmente, en la inmensa mayoría de los casos, clasificación y descripción estáticas, más que explicación``; y concluye con algunos comentarios valiosos para la orientación general presente y futura de la disciplina:''En una palabra, la ciencia política tiende a ocuparse de los aspectos estáticos de la política, más que de sus aspectos dinámicos``. Sostiene que hay que plantear interrogantes que muevan a buscar explicaciones y no sólo descripciones, porque''\ldots Las explicaciones atienden a la dinámica, dan cuenta del movimiento y del cambio en el tiempo, exigen la diferenciación (en este sentido no carece de fundamento la pretensión de los comparatistas de que sus investigaciones son fundamentales para el desarrollo de la teoría política). Para que resulten explicables y no solamente suceptibles de ser descritos, los fenómenos políticos han de ser expuestos mediante conceptos que subrayen más sus propiedades dinámicas que sus aspectos estáticos." Resulta interesante constatar que en los veinticinco años transcurridos desde que fueron escritas estas palabras, los progresos reconocidos en la Ciencia Política se han producido principalmente en los aspectos señalados: el enfoque comparado y el estudio del dinamismo político.

\hypertarget{quinta-parte}{%
\section*{Quinta parte}\label{quinta-parte}}
\addcontentsline{toc}{section}{Quinta parte}

\hypertarget{enfoques-metodoluxf3gicos-usuales-1}{%
\subsection*{Enfoques metodológicos usuales}\label{enfoques-metodoluxf3gicos-usuales-1}}
\addcontentsline{toc}{subsection}{Enfoques metodológicos usuales}

Pese a sus aparentemente grandes diferencias, todas estas corrientes teóricas tienen muchos puntos en común desde el punto de vista metodológico. Esos rasgos comunes, justamente, permiten agruparlas en la gran corriente empírico-analítica, dominante en la investigación científica política de los países occidentales.

Ese acervo común suele ser designado como empirismo, o método empírico general. Se lo puede caracterizar: por la recusación de todo innatismo y la afirmación de la experiencia (en cuanto contacto inteligible con la realidad) como única fuente de conocimientos; por la eliminación de todo planteo metafísico explícito y a priori; y por la exigencia de verificabilidad de todas las proposiciones. También puede mencionarse la construcción de sistemas de conocimientos abiertos y opuestos a todo principio de autoridad.

En la ciencia social contemporánea, la posición epistemológica al parecer dominante es el neopositivismo crítico, cuyo enfoque agrega a los rasgos recién enunciados los siguientes: - La falsación de los enunciados inductivos:no hay que demostrar que un enunciado es verdadero; hay que tratar de demostrar que es falso y considerarlo verdadero mientras logre mantenerse en pié; - No interesa tanto la objetividad de cada científico (que de todos modos será siempre relativa y cuestionable) sino el ofrecimiento de las teorías a la crítica abierta del mundo científico; - La evaluación de las teorías debe hacerse con un criterio de economía y de eficacia: la teoría mejor es la más simple y la más eficaz, la más aplicable en la investigación y en el ámbito social.

Dentro de este esquema general, cada una de las corrientes que hemos mencionado en este capítulo se caracteriza por ciertas particularidades metodológicas: El behaviorismo utiliza desde sus comienzos un paradigma o esquema metodológico proveniente de los estudios psicológicos: el esquema S-R (estímulo-respuesta). Sus desarrollos posteriores, ya en convergencia hacia el enfoque sistémico, originaron el esquema S-O-R (estímulo-organismo-respuesta). El behaviorismo también se caracteriza (aunque no en forma exclusiva) por el empleo de la observación sistemática mediante técnicas psico-sociológicas como el sondeo, la encuesta y la entrevista. Actualmente conserva plenamente su valor, sobre todo como método descriptivo. Tiende a elaborar y manejar solamente conceptos operacionales, o sea conceptos reducidos solo a las propiedades observables de la realidad, propiedades definidas por las operaciones que las verifican.

El estructuralismo se reconoce desde el punto de vista metodológico por su pregunta subyacente: Cómo es el objeto? Cuál es la disposición de sus partes? Los investigadores estructuralistas suelen ser muy conscientes de que la estructura es una construcción racional del pensamiento, y suelen reprochar a los funcionalistas su creencia en la sustantividad de la función. Su método, basado también en la observación sistemática, pero combinada frecuentemente con el uso de analogías lingüísticas, apunta a establecer, en el fenómeno que estudia, los vínculos relacionales entre sus partes y los valores posicionales emergentes.

El funcionalismo también es reconocido por su pregunta subyacente: Qué hace el objeto? Cuál es la función que cumple para su sistema? La mayoría de los funcionalistas la formulan de un modo más detallado y explícito: Cuál es la contribución de un elemento de un sistema al mantenimiento de éste en un estado determinado? Metodológicamente, el funcionalismo apunta a la obtención de explicaciones causales y factoriales, y también tiende a elaborar y utilizar conceptos operacionales.

El enfoque sistémico, en su aplicación a las ciencias sociales, se caracteriza por las construcción de sistemas teóricos o ``modelos'' abstraídos de la realidad. Esto quiere decir que dichos modelos guardan una relación de correspondencia con la realidad; que no son propiamente reales ni tampoco puramente formales. El enfoque sistémico también se caracteriza por la incorporación de la noción de ``proceso'', lo que equivale a decir, de la dimensión-tiempo, ya que un sistema es, entre otras cosas, un ``acumulador de tiempo''.

El principal problema metodológico del enfoque sistémico es justamente la construcción de ``sistemas abstraídos'', ya que a diferencia de los sistemas reales (como una galaxia o una célula), los sistemas de las ciencias sociales son creaciones analíticas de la inteligencia, cuya correspondencia con la realidad es lo primero que hay que probar.

El enfoque sistémico es metodológicamente más dinámico que el enfoque funcionalista, ya que apunta a describir y explicar dos órdenes fundamentales de procesos: - el mantenimiento del sistema en un estado determinado; - los cambios que se producen en él, ya sean cambios en el sistema (adaptativos) o cambios de sistema (disruptivos).

El enfoque comparativo es, como ya sugerimos en páginas anteriores, un método de control de nuestras hipótesis, generalizaciones, previsiones o leyes, que se utiliza cuando no pueden realizarse experimentos ni controles estadísticos (lo cual es lo más frecuente en Ciencia Política). El control comparado suele hacerse en términos sincrónicos: confrontamos unidades, procesos o instituciones políticas ``en tiempos equivalentes'': en general en un ``presente'' que nos permita obtener los datos que necesitamos, lo que no puede hacerse con el restante método de control -el histórico- en el que sólo se dispone de los datos que se hayan conservado.

El problema central del método comparativo es determinar con precisión qué cosas son comparables. El planteo que desarrolla Sartori a este respecto\footnote{Giovanni Sartori: LA POLITICA - LOGICA Y METODO EN LAS CIENCIAS SOCIALES, México, FCE, 1984, pg. 261 y ss.} parte de considerar que quien compara no solo busca semejanzas sino también diferencias, y que ambas operaciones son complementarias. En definitiva, propone trabajar por ``género próximo'' (lo que ambos elementos de la comparación tienen en común, lo similar, lo homogéneo) y por ``diferencia específica'' (lo diferente, lo heterogéneo, lo propio de cada uno).

En el campo de las explicaciones políticas de base psicológica individual, encontramos originalidad metodológica solamente en el psicoanálisis freudiano y sus derivados, ya que la psicología del estímulo-respuesta usa métodos behavioristas, y la psicología de la gestalt y del campo usan métodos reductibles en última instancia al estructuralismo.

El psicoanálisis, desde el punto de vista metodológico, es un procedimiento de investigación de procesos mentales que serían prácticamente inaccesibles de otro modo. Se trata del llamado ``método clínico''. Su discusión en profundidad excede completamente los límites de este trabajo, y ciertamente los de nuestra capacidad, por lo que vamos a dar de él sólo un concepto general.

Metodológicamente, en el psicoanálisis freudiano es posible distinguir tres niveles\footnote{J. Laplanche y J.B. Pontalis: DICCIONARIO DE PSICOANALISIS, Barcelona, Labor, 1974.}: 1. Un método de investigación que esencialmente consiste en volver evidente la significación inconsciente de las manifestaciones de un individuo, en base a las ``asociaciones libres'' del mismo suje-to, que garantizan la validez de la interpretación.

\begin{enumerate}
\def\labelenumi{\arabic{enumi}.}
\setcounter{enumi}{1}
\item
  Un método de cura psicoanalítica, basado en la investigación y caracterizado por la interpretación controlada de la resistencia, de la transferencia y del deseo.
\item
  Un conjunto de teorías psicológicas y psicopatológicas que sistematizan los datos aportados por la investigación y el tratamiento. En este tercer nivel es donde se apoyan las concepciones político-sociales y culturales del freudismo.
\end{enumerate}

Se dice que Freud sistematizó los resultados de una prolongada introspección, corroboró esas visiones en sus pacientes y proyectó luego sus resultados en su hobbesiana visión de la sociedad y la política, con fuerza muy sugerente, como hemos visto.

Finalmente, en lo que se refiere al formalismo, desde el punto de vista metodológico el mismo se caracteriza por el empleo de métodos formales, ésto es, derivados de la Lógica formal y de las Matemáticas, los cuales llegan al campo de los estudios políticos principalmente desde el ámbito de la Economía.

Creemos oportuno terminar este resumen sobre los enfoques metodológicos usuales en las teorías empírico-analíticas con algunas reflexiones\footnote{Giovanni Sartori, op. cit.} sobre características de las ciencias sociales que configuran su especial dificultad y son fuentes de problemas metodológicos, particularmente notables en estas teorías.

Veamos en primer término el problema del lenguaje, o, mejor dicho, del ``condicionamiento lingüístico del pensamiento'', que gravita tanto al construir una ciencia como al comunicarla. La ciencia requiere un lenguaje empírico sistematizado (así como la Filosofía requiere un lenguaje especulativo sistematizado). Ahora bien, hay ciencias sociales que han avanzado mucho en esa sistematización del lenguaje (por ejemplo, la Economía): hay un acuerdo y un orden en sus conceptos fundamentales que permite trabajar acumulativamente en la ``ciencia normal'' (según la terminología de Kühn) sobre bases estables. Otras ciencias, entre ellas la Política, na han llegado todavía a ese nivel, y experimentan las dificultades emergentes del ``desorden lingüístico''.

La sistematización del lenguaje entraña una tarea de elaboración conceptual y de acuerdo sobre sus contenidos, que resuelva en medida apreciable dos problemas básicos: la elaboración científica y la comunicación, problemas del proceso cognoscitivo que Sartori plantea aproximadamente así: -- SIGNIFICADOS (que están en la mente) \textbar{} \textbar{} \textbar{} \textbar{} ambigüedad \textbar{} hay pocas palabras para \textbar{} \textbar{} equivocidad \textbar{} muchos contenidos \textbar{} \textbar{} \textbar{} PALABRAS (que los expresan) \textbar{} \textbar{} \textbar{} \textbar{} vaguedad \textbar{} no marcan los límites y/o \textbar{} \textbar{} indetermi- \textbar{} no discriminan los contenidos \textbar{} \textbar{} nación \textbar{} \textbar{} \textbar{} -- REFERENTES (cosas observables que los denotan) Otro aspecto vinculado a ésto es la necesidad de perseverar en una misma lógica para la construcción o la comunicación, y no saltar desde la lógica de la identidad y la no contradicción a la lógica dialéctica y viceversa.

Una característica importante de las ciencias sociales se refiere a la relación causa-efecto. En ellas, a diferencia de las ciencias naturales, la CAUSA es condición necesaria pero no suficiente. Dicho de otro modo: dada la causa C, es sólo probable que se produzca el efecto E, debido al rasgo de indeterminación e imprevisibilidad que, en alguna medida, tiene la conducta humana individual y grupal.

Una referencia similar cabe hacer respecto de la relación primero-después. En las ciencias sociales, el efecto puede preceder en el tiempo a la causa, o sea, ser efecto de la espectativa de un acontecimiento.

En páginas anteriores hemos visto que la aspiración al logro de una mayor precisión se traduce con frecuencia en la adopción de vías metodológicas cuantitativas, matemáticas, etc. Respecto de ésto, cabe destacar la importancia de una reflexión de Sartori\footnote{Giovanni Sartori, op. cit. pg. 62.} según la cual ``\ldots la cuantificación de las ciencias sociales''mide" a lo observado con una medida que no está en ellas, que es una atribución del observador".

Otra reflexión metodológica interesante de Sartori se refiere a la tendencia a considerar ``más importante'' el método que las técnicas de investigación: ``\ldots las dos cosas\ldots son diferentes: necesitamos de ambas, y si una falta, el edificio está manco y amenaza caerse''\footnote{Giovanni Sartori, op. cit. pg. 63.}.

Según la propuesta de Sartori, si se quiere consolidar el status de ``teoría científica'' de las ciencias sociales y colmar el vacío metodológico que las afecta, se debe comenzar por sistematizar el lenguaje, lo cual implica: - la formación de conceptos empíricos, evaluables (o sea, validables, invalidables, modificables) mediante observaciones; - el tratamiento adecuado de los conceptos, ya sea en forma disyuntiva (según la lógica de la clasificación); contínua (según la lógica de la gradación); o como organización jerárquica (según una lógica clasificatoria por género próximo y diferencia específica).

El actual status teórico de las ciencias sociales -entre ellas, la Política- es modesto. No existe UNA teoría de la sociedad o de la política, por defecto de instrumentación lingüística o por carencias metodológicas. Lo que existe -en la etapa actual- son teorías parciales, que se desarrollan en el ámbito de una multiplicidad de aproximaciones. El control de los conocimientos, su comprobación o falsación, se realiza en dos instancias: en la investigación (por control estadístico o por control comparado) y en la práctica, o sea por la confirmación de los hechos sociales.

\hypertarget{Lasteoruxedasempuxedricoanaluxedticas}{%
\chapter{Las teorías empírico-analíticas}\label{Lasteoruxedasempuxedricoanaluxedticas}}

Primera parte:

\begin{itemize}
\tightlist
\item
  Rasgos generales: El positivismo, el empirismo y sus derivados.
\item
  El objeto y el método.
\item
  Problemas actuales.
\end{itemize}

Segunda parte:

\begin{itemize}
\tightlist
\item
  Behaviorismo, estructural-funcionalismo y enfoque sistémico.
\item
  El enfoque comparatista: Descripción de los enfoques.
\item
  Síntesis de obras teóricas de estas corrientes.
\end{itemize}

Tercera parte:

\begin{itemize}
\tightlist
\item
  Las explicaciones de base psicológica individual: La Psicología del estímulo-respuesta.
\item
  La psicología de la Gestalt.
\item
  La teoría del campo.
\item
  El freudismo ortodoxo.
\item
  El neofreudismo.
\end{itemize}

Cuarta parte:

\begin{itemize}
\tightlist
\item
  El formalismo.
\item
  La teoría de los juegos.
\item
  La teoría de la información y la cibernética.
\item
  Modelos y simulaciones.
\end{itemize}

Quinta parte:

\begin{itemize}
\tightlist
\item
  Enfoques metodológicos usuales: Puntos en común.
\item
  Particularidades metodológicas.
\item
  Reflexiones sobre el lenguaje y la elaboración conceptual.
\end{itemize}

\hypertarget{primera-parte-3}{%
\section*{Primera parte}\label{primera-parte-3}}
\addcontentsline{toc}{section}{Primera parte}

\hypertarget{rasgos-generales-2}{%
\subsection*{Rasgos generales}\label{rasgos-generales-2}}
\addcontentsline{toc}{subsection}{Rasgos generales}

\begin{itemize}
\tightlist
\item
  ADORNO, Theodor W. ``LA PERSONALIDAD AUTORITARIA'', Paidós, Bs.As., 1969.
\item
  ALIGHIERI, Dante ``DE LA MONARQUIA'', Losada, Bs.As.
\item
  ALBERIONI et al.~``L'ATTIVISTA DI PARTITO'', Bolonia, 1967.
\item
  ALMOND, G. y COLEMAN ``THE POLITICS OF THE DEVELOPING AREAS'', Princeton
  University Press, 1960.
\item
  ALMOND, G.A. y VERBA, Sidney ``THE CIVIC CULTURE'', Princeton University Press, 1963.
\item
  ALMOND, G.A. y POWELL, G.B. ``POLITICA COMPARADA'', Paidós, Bs.As., 1972.
\item
  ALTHUSSER, Louis ``PARA LEER EL CAPITAL'', Siglo XXI, México, 1969.
\item
  AMIN, Samir ``SOBRE EL DESARROLLO DESIGUAL DE LAS FORMACIONES SOCIALES'',
  Anagrama, Barcelona, 1976.
\item
  AMIN, Samir ``L'ACCUMULATION A L'ECHELLE MONDIALE'', Editions Anthropos, París,
\end{itemize}

\begin{enumerate}
\def\labelenumi{\arabic{enumi}.}
\setcounter{enumi}{1969}
\tightlist
\item
\end{enumerate}

\begin{itemize}
\tightlist
\item
  ANDERSON, Perry ``CONSIDERACIONES SOBRE EL MARXISMO OCCIDENTAL'', Siglo XXI,
  México, 1990.
\item
  APONTE, Antonio ``LA ECONOMIA DE LOS PAISES SOCIALISTAS'', Salvat ed., Barcelona,
\end{itemize}

\begin{enumerate}
\def\labelenumi{\arabic{enumi}.}
\setcounter{enumi}{1972}
\tightlist
\item
\end{enumerate}

\begin{itemize}
\tightlist
\item
  ARENDT, Hannah ``LOS ORIGENES DEL TOTALITARISMO'', Alianza ed., Madrid, 1981-1982,
  2 vol.
\item
  APTER, David E. ``POLITICA DE LA MODERNIZACION'', Paidós, Bs.As., 1972.
\item
  ARISTOTELES ``LA POLITICA'', Editora Nacional, Madrid, 1977.
\item
  ARNOLETTO, Eduardo J. ``APROXIMACION A LA CIENCIA POLITICA'', Artesol ed., Córdoba,
\end{itemize}

\begin{enumerate}
\def\labelenumi{\arabic{enumi}.}
\setcounter{enumi}{1988}
\tightlist
\item
\end{enumerate}

\begin{itemize}
\tightlist
\item
  ARON, Raymond ``EL OPIO DE LOS INTELECTUALES'', Siglo XX, Bs.As., 1968.
\item
  ARON, Raymond ``DEMOCRACIA Y TOTALITARISMO'', Seix, Barcelona, 1971.
\item
  ARON, Raymond ``PAZ Y GUERRA ENTRE LAS NACIONES'', Alianza ed., Madrid, 1984.
\item
  ARON, Raymond ``REPUBLIQUE IMPERIALE: LES ETATS-UNIS DANS LE MONDE (1945-
  1972)'', Calmann-Lévy, París, 1973.
\item
  ``BREVE DICCIONARIO POLITICO'', Ed. Progreso, Moscú, 1983.
\item
  BALANDIER, George ``ANTROPOLOGIA POLITICA'', Península, Barcelona, 1969.- BARAN, Paul y SWEEZY, Paul ``EL CAPITALISMO MONOPOLISTA. UN ENSAYO SOBRE LA
  SOCIEDAD INDUSTRIAL AMERICANA'', Siglo XXI, México, 1968.
\item
  BARRACLOUGH, Geofrey ``UNE INTRODUCTION A L'HISTOIRE CONTEMPORAINE'', Ed.
  Stock, París, 1964.
\item
  BELL, Daniel ``FIN DE LAS IDEOLOGIAS'', Tecnos, Madrid, 1964.
\item
  BELL, Daniel ``THE END OF IDEOLOGY: ON THE EXHAUSTION OF POLITICAL IDEAS IN
  THE FIFTIES'', New York, 1960.
\item
  BENDIX, R. ``NATION-BUILDING AND CITIZENSHIP'', John Wiley, New York, 1964.
\item
  BERTALANFFY, Ludwig von ``TEORIA GENERAL DE LOS SISTEMAS'', FCE, México, 1981.
\item
  BENOIST, Alain de ``DEMOCRATIE: LE PROBLEME'', Le Labyrinthe, París, 1985.
\item
  BEYME, Klaus von ``TEORIAS POLITICAS CONTEMPORANEAS - UNA INTRODUCCION'',
  Instituto de Estudios Políticos, Madrid, 1977.
\item
  BOBBIO, N. et al.~``DICCIONARIO DE POLITICA'', Siglo XXI, México, 1986.
\item
  BOBBIO, Norberto ``SAGGI SULLA SCIENZA POLITICA IN ITALIA'', Bari, 1969.
\item
  BOBBIO, Norberto ``IL FUTURO DELLA DEMOCRAZIA. UNA DIFESA DELLE REGOLE DEL
  GIOCO'', Einaudi ed., Torino, 1984.
\item
  LE BON, Gustave ``PSICOLOGIA DE LAS MULTITUDES'', Ed. Albatros, Bs.As., 1978.
\item
  BRZEZINSKI, Zbigniew ``IDEOLOGIA Y PODER EN LA POLITICA SOVIETICA'', Paidós,
  Bs.As., 1968.
\item
  BRAILLARD, Philippe ``THEORIE DES SYSTEMES ET RELATIONS INTERNA-TIONALES'',
  Ed. Bruylant, Bruselas, 1977.
\item
  BRAILLARD, P. y DE SENARCLENS, P. ``EL IMPERIALISMO'', FCE, México, 1982.
\item
  BRUNSCHWIG, H. ``LE PARTAGE DE L'AFRIQUE NOIRE'', Flammarion, París, 1971.
\item
  CARDOZO, F.H. y FALETTO, E. "DEPENDENCIA Y DESARROLLO EN AMERICA LATINA,
  Siglo XXI, México, 1969.
\item
  CARTWRIGHT, Dorwin y ZANDER, Alvin ``GROUP DYNAMICS: RESEARCH AND THEORY'',
  Ed. Harper and Row, 1962.
\item
  CASSIRER, Ernest ``EL MITO DEL ESTADO'', FCE, México, 1968. CESAREO, Vincenzo et al.
  ``LA CULTURA DELL'ITALIA CONTEMPORANEA. TRASFORMAZIONE DEI MODELLI DI
  COMPORTAMENTO E IDENTITA SOCIALE'', Ed. Fondazione Giovanni Agnelli, Torino, 1990.
\item
  CHATELET, F., DUHAMEL, O. y PISIER, E. ``DICTIONNAIRE DES OEUVRES POLITIQUES'',
  P.U.F., París, 1989.- CHEVALIER, J.J. ``LOS GRANDES TEXTOS POLITICOS - DESDE MAQUIAVELO A
  NUESTROS DIAS'', Aguilar, Madrid, 1979.
\item
  COPLIN, W.D. "INTRODUCTION TO INTERNATIONAL POLITICS. A THEORETICAL
  OVERVIEW, Chicago ,1971.
\item
  CROZIER, Michel ``LE PHENOMENE BUREAUCRATIQUE'', Seuil, París, 1964.
\item
  DENQUIN, Jean-Marie ``SCIENCE POLITIQUE'', P.U.F., París, 1991.
\item
  DEUTSCH, K.W. ``NATIONALISM AND SOCIAL COMMUNICATION. AN INQUIRY INTO
  THE FOUNDATIONS OF NATIONALITY'', M.I.T. Press, Mass., 1953.
\item
  DEUTSCH, K. et al.~``POLITICAL COMMUNITY AND THE NORTH ATLANTIC AREA.
  INTERNATIONAL ORGANIZATION IN THE LIGHT OF HISTORICAL EXPERIENCE'',
  Princeton, 1957.
\item
  DEUTSCH, Karl ``POLITICA Y GOBIERNO'', FCE, México, 1976.
\item
  DEUTSCH, Karl ``LOS NERVIOS DEL GOBIERNO'', FCE, México, 1985.
\item
  DEUTSCH, Morton y KRAUSS, Robert ``THEORIES IN SOCIAL PSYCHOLOGY'', Basic Books,
  Inc., 1965.
\item
  DIAMANT, A. ``THE NATURE OF POLITICAL DEVELOPMENT'' en Finkle y Gable
  ``POLITICAL DEVELOPMENT AND SOCIAL CHANGE'', John Wiley, New York, 1966.
\item
  DJILAS, Milovan ``LA NUEVA CLASE'', Sudamericana, Bs.As., 1965.
\item
  DRAPER, T. ``ABUSE OF POWER'', Secker and Warburg, London, 1966.
\item
  DURKHEIM, E. "DE LA DIVISION DEL TRABAJO SOCIAL, Schapire, Bs.As, 1967.
\item
  ------------ ``EL SUICIDIO'', Schapire, Bs.As., 1965.
\item
  ------------ ``LAS FORMAS ELEMENTALES DE LA VIDA RELIGIOSA'', Schapire, Bs.As., 1968.
\item
  EASTON ,D. y DENNIS, J. ``CHILDREN IN THE POLITICAL SYSTEM'', New York, 1969.
\item
  EASTON, David "ESQUEMA PARA EL ANALISIS POLITICO, Amorrortu, Bs.As., 1969.
\item
  ECKSTEIN, H. y APTER, D. ``COMPARATIVE POLITICS. A READER'', New York, 1963.
\item
  ECKSTEIN, H. ``DIVISION AND COHESION IN DEMOCRACY - A STUDY OF NORWAY'',
  Princeton University Press, 1966.
\item
  EISENSTADT, S.N. ``MODERNIZACION, MOVIMIENTOS DE PROTESTA Y CAMBIO
  SOCIAL'', Amorrortu, Bs.As., 1969.
\item
  EMMANUEL, Arghiri "L'ECHANGE INEGAL. ESSAI SUR LES ANTAGONISMES DANS LES
  RAPPORTS ECONOMIQUES INTERNATIONAUX, Maspero, París, 1969.
\item
  FEJTÖ, François ``LA SOCIAL-DEMOCRATIE QUAND MEME'', Ed. Robert Laffont, París, 1980. - FESTINGER, Leo ``A THEORY OF COGNITIVE DISSONANCE'', Row, Peterson and Co., 1957.
\item
  FRANK, André Gunder ``CAPITALISMO Y SUBDESARROLLO EN AMERICA LATINA'', Siglo
  XXI, México, 1970.
\item
  FRENCH, John R.P. Jr.~``A FORMAL THEORY OF SOCIAL POWER'' en Cartwright y Zander:
  ``GROUPS DINAMICS: RESEARCH AND THEORY'', Ed. Harper and Row, 1962.
\item
  FREUD, Sigmund ``OBRAS COMPLETAS'' , Tomos II y III, Ed. Biblioteca Nueva, Madrid, 1973.
\item
  FRIEDRICH, Carl ``EL HOMBRE Y EL GOBIERNO: UNA TEORIA EMPIRICA DE LA
  POLITICA'', Tecnos, Madrid, 1968.
\item
  FRIEDRICH, C. y BRZEZINSKI, Z. ``DICTADURA TOTALITARIA Y AUTOCRACIA'', Libera,
  Bs.As., 1975.
\item
  FULBRIGHT, J.W. ``THE ARROGANCE OF POWER'', Vintage Books, New York, 1966.
\item
  FURTADO, Celso ``DESARROLLO Y SUBDESARROLLO'', Eudeba, Bs.As., 1965.
\item
  GALLAGER, John y ROBINSON, Ronald ``AFRICA AND THE VICTORIANS. THE OFFICIAL
  MIND OF IMPERIALISM'', Ed. Macmillan, London, 1961.
\item
  GARCIA PELAYO, Manuel ``MITOS Y SIMBOLOS POLITICOS'', Taurus, Madrid, 1964.
\item
  GENIAGE, Jean ``L'EXPANSION COLONIALE DE LA FRANCE SOUS LA IIIe REPUBLIQUE
  (1871-1914)'', Payot, París, 1968.
\item
  GERMANI, Gino ``POLITICA Y SOCIEDAD EN UNA EPOCA DE TRANSICION'', Paidós,
  Bs.As., 1965.
\item
  GOLEMBIEWSKI, Robert ``BEHAVIOR AND ORGANIZATION: ORGANIZATION AND
  METHODS AND THE SMALL GROUP'', Rand McNally and Co., 1962.
\item
  GORI, U., BRUSCHI, A., ATTINA, F. ``RELAZIONI INTERNAZIONALI. METODI E TECNICHE
  DI ANALISI'', Milán, 1974.
\item
  GRAMSCI, Antonio ``NOTAS SOBRE MAQUIAVELO, LA POLITICA Y EL ESTADO'', Juan
  Pablos, México, 1975.
\item
  GURVITCH, Georges ``TRATADO DE SOCIOLOGIA'', Kapeluz, BS.As., 1963.
\item
  ----------------- ``LES CADRES SOCIAUX DE LA CONNAISSANCE'', PUF, París, 1966.
\item
  HABERMAS, Jürgen ``TEORIA Y PRAXIS'', Sur, Bs.As., 1967.
\item
  ---------------- "TEORIA E PRASSI NELLA SOCIETA TECNOLOGICA, Bari, 1969.
\item
  HARGROVE, E.C. ``PRESIDENTIAL LEADERSHIP - PERSONALITY AND POLITICAL
  STYLE'', New York/London, 1966. - HAURIOU, A. ``DERECHO CONSTITUCIONAL E INSTITUCIONES POLITICAS'', Ed. Ariel,
  Barcelona, 1971.
\item
  HEARNSHAW, F.J.C. ``HISTORIA DE LAS IDEAS POLITICAS'', Empresa Letras, Santiago de
  Chile, .
\item
  HILFERDING, Rudolf ``LE CAPITAL FINANCIER. ETUDE SUR LE DEVELOPMENT RECENT
  DU CAPITALISME'', Ed. du Minuit, París, 1970.
\item
  HOBSON, J.A. ``ESTUDIOS DEL IMPERIALISMO'', Alianza, Madrid, 1981.
\item
  HOFFMAN, Stanley ``GULLIVER EMPETRÉ. ESSAI SUR LA POLITIQUE ETRANGERE DES
  ETATS-UNIS'', Seuil, París, 1971.
\item
  HOLT, R.T. y TURNER E. ``THE METHODOLOGY OF COMPARATIVE RESEARCH'', New
  York, 1970.
\item
  HOMANS, George C. ``SOCIAL BEHAVIOR: ITS ELEMENTARY FORMS'', Harcourt, Brace and
  Word Inc., 1961.
\item
  HORKHEIMER, Max y ADORNO, Theodor ``DIALECTICA DEL ILUMINISMO'', Sur, Bs.As.,
\end{itemize}

\begin{enumerate}
\def\labelenumi{\arabic{enumi}.}
\setcounter{enumi}{1968}
\tightlist
\item
\end{enumerate}

\begin{itemize}
\tightlist
\item
  HOVLAND, Car I. et al.~``COMMUNICATION AND PERSUATION: PSYCOLOGICAL STUDIES
  OF OPINION CHANGE'', Yale University Press, 1953.
\item
  HULL, Clark L. ``A BEHAVIOR SYSTEM'', Yale University Press, 1952.
\item
  HUNTINGTON, S.P. y MOORE, C.H. ``AUTHORITARIAN POLITICS IN MODERN SOCIETY'',
  New York, 1970.
\item
  HUNTINGTON, S.P. ``EL ORDEN POLITICO EN LAS SOCIEDADES EN CAMBIO'', Paidós,
  Bs.As., 1972.
\item
  JAGUARIBE H. et al.~``LA DEPENDENCIA POLITICO-ECONOMICA DE AMERICA LATINA'',
  Siglo XXI, México, 1971.
\item
  JAGUARIBE-FURTADO-FALETTO-DITELLA-ESPARTACO-SUNKEL- CARDOSO ``LA
  DOMINACION DE AMERICA LATINA'', Amorrortu, Bs.As., 1972.
\item
  JAGUARIBE, Helio ``SOCIEDAD, CAMBIO Y SISTEMA POLITICO'', Paidós, Bs.As., 1972.
\item
  ---------------- ``DESARROLLO POLITICO - SENTIDO Y CONDICIONES'', Paidós, Bs.As., 1972.
\item
  ---------------- ``AMERICA LATINA - REFORMA O REVOLUCION'', Paidós, Bs.As., 1972.
\item
  ---------------- ``O NOVO CENARIO INTERNACIONAL'', Ed. Guanabara, Río de Janeiro, 1986.
\item
  JALEÉ, Pierre ``L'IMPERIALISME EN 1970'', Maspero, París, 1973.
\item
  JAMES. Emile ``HISTORIA DEL PENSAMIENTO ECONOMICO'', Aguilar, Madrid, 1974. - JOUVENEL, Bertrand de ``EL PODER'', Ed. Nacional, Madrid, 1974.
\item
  2a ed.~KAPLAN, M.A.~``SYSTEM AND PROCES IN INTERNATIONAL POLITICS'', New York,
\end{itemize}

\begin{enumerate}
\def\labelenumi{\arabic{enumi}.}
\setcounter{enumi}{1956}
\tightlist
\item
\end{enumerate}

\begin{itemize}
\tightlist
\item
  KEOHANE, R.O. y NYE, J.S. ``TRANSNATIONAL RELATIONS IN WORLD POLITICS'',
  Harvard University Press, 1972.
\item
  KNOLL, E. y McFADEN, J. ``AMERICAN MILITARISM - 1970'', The Viking Press, New York,
\end{itemize}

\begin{enumerate}
\def\labelenumi{\arabic{enumi}.}
\setcounter{enumi}{1968}
\tightlist
\item
\end{enumerate}

\begin{itemize}
\tightlist
\item
  KORNHAUSER, William ``ASPECTOS POLITICOS DE LA SOCIEDAD DE MASAS'', Amorrortu,
  Bs.As., 1969.
\item
  LAGROYE, Jacques ``SOCIOLOGIE POLITIQUE'', Presses de la F.N. des Sc. Po. \& Dalloz, París,
\end{itemize}

\begin{enumerate}
\def\labelenumi{\arabic{enumi}.}
\setcounter{enumi}{1990}
\tightlist
\item
\end{enumerate}

\begin{itemize}
\tightlist
\item
  LANE, R. ``POLITICAL IDEOLOGY'', New York, 1962.
\item
  LAPLANCHE, J. y PONTALIS, J.B. ``DICCIONARIO DE PSICOANALISIS'', Ed. Labor,
  Barcelona, 1974.
\item
  LASSWELL, Harold D. ``PSYCHOPATHOLOGY AND POLITICS'', Viking Press Inc., 1962.
\item
  LERNER, D. "THE PASSING OF TRADITIONAL SOCIETY. MODERNIZING THE MIDDLE
  EAST, New York, 1958.
\item
  LEWIN, Kurt ``FIELD THEORY IN SOCIAL SCIENCE'', Dorwin Cartwright (Harper and Bros.),
\end{itemize}

\begin{enumerate}
\def\labelenumi{\arabic{enumi}.}
\setcounter{enumi}{1950}
\tightlist
\item
\end{enumerate}

\begin{itemize}
\tightlist
\item
  LIJPHART, A. ``THE POLITICS OF ACCOMODATION. PLURALISM AND DEMOCRACY IN
  THE NETHERLANDS'', Berkeley, Los Angeles, 1968.
\item
  LIPSET, Seymour Martin ``EL HOMBRE POLITICO. LAS BASES SOCIALES DE LA POLITICA'',
  Eudeba ed., Bs.As., 1977.
\item
  LISKA, George ``IMPERIAL AMERICA. THE INTERNATIONAL POLITICS OF PRIMACY'',
  John Hopkins Press, Baltimore, 1967.
\item
  LOCKE, John ``ENSAYO SOBRE EL GOBIERNO CIVIL'', Aguilar, Madrid, 1981.
\item
  LOPEZ, Mario Justo ``INTRODUCCION A LOS ESTUDIOS POLITICOS'', Tomos I y II, Kapeluz,
  Bs.As., 1975.
\item
  LUKACS, György ``EL ASALTO A LA RAZON'', Grijalbo, México, 1976.
\item
  LUXEMBURG, Rosa ``LA ACUMULACION DEL CAPITAL'', Grijalbo, México, 1967.
\item
  MAIMONIDES, M. ``THE GUIDE OF THE PERPLEXED'', University of Chica-go Press, Chicago,
\end{itemize}

\begin{enumerate}
\def\labelenumi{\arabic{enumi}.}
\setcounter{enumi}{1962}
\item
  \begin{itemize}
  \tightlist
  \item
    MANNHEIM, Karl ``IDEOLOGIA Y UTOPIA'', Aguilar, Madrid, 1973.
  \end{itemize}
\end{enumerate}

\begin{itemize}
\tightlist
\item
  MAQUIAVELO, Nicolás ``EL PRINCIPE'', Alianza ed., Madrid, 1981.
\item
  ------------------- ``DISCURSOS SOBRE LA PRIMERA DECADA DE TITO LIVIO'', en ``Obras'',
  Vergara, Barcelona, 1965.
\item
  MARCH, James G. y SIMON, Herbert A. ``ORGANIZATIONS'', John Wiley and Sons, 1962.
\item
  MARX, Karl ``LA IDEOLOGIA ALEMANA'', Grijalbo, México, 1969.
\item
  MARX, Karl ``ELEMENTOS FUNDAMENTALES PARA LA CRITICA DE LA ECONO- MIA
  POLITICA'', Siglo XXI, Madrid, 1972.
\item
  ---------- ``TEORIAS SOBRE LA PLUSVALIA'', FCE, México, 1982.
\item
  MARCUSE, Herbert ``EL FIN DE LA UTOPIA'', Siglo XXI, México, 1968.
\item
  ---------------- ``EL HOMBRE UNIDIMENSIONAL'', Mortiz, México, 1970.
\item
  MEEHAN, E.J. ``PENSAMIENTO POLITICO CONTEMPORANEO'', Rev.~de Occidente, Madrid,
\end{itemize}

\begin{enumerate}
\def\labelenumi{\arabic{enumi}.}
\setcounter{enumi}{1972}
\tightlist
\item
\end{enumerate}

\begin{itemize}
\tightlist
\item
  MERTON, Robert K. ``TEORIA Y ESTRUCTURAS SOCIALES'', FCE, México, 1964.
\item
  MICHELS, Robert ``LOS PARTIDOS POLITICOS'', Amorrortu, Bs.As., 1969
\item
  MILBRATH, Lester W. ``POLITICAL PARTICIPATION'', Rand Mc Nally and Co., 1965.
\item
  MOONEY, Alfredo y ARNOLETTO, Eduardo ``CUESTIONES FUNDAMENTALES DE CIENCIA
  POLITICA'', Alveroni ed., Córdoba, 1993.
\item
  MORGENTHAU, Hans ``POLITICS AMONG NATIONS. THE STRUGGLE FOR POWER AND
  PEACE'', Knopf, New York, 1955.
\item
  MORLINO, L. Comp. ``GUIDE AGLI STUDI DI SCIENZE SOCIALI IN ITALIA - SCIENZA
  POLITICA'', Ed. Fond. G. Agnelli, Torino, 1989.
\item
  MORO, Tomás ``UTOPIA'', Bruguera, Barcelona, 1973.
\item
  ORGANSKI, A.F.K. ``THE STAGES OF POLITICAL DEVELOPMENT'', A. Knopf, New York,
\end{itemize}

\begin{enumerate}
\def\labelenumi{\arabic{enumi}.}
\setcounter{enumi}{1964}
\tightlist
\item
\end{enumerate}

\begin{itemize}
\tightlist
\item
  OSSOWSKI, Stanislaw ``ESTRUCTURA DE CLASE Y CONCIENCIA SOCIAL'', Península,
  Barcelona, 1971.
\item
  PARETO, Vilfredo ``FORMA Y EQUILIBRIO SOCIALES'', Rev.~de Occidente, Madrid, 1966.
\item
  PARETI, Luigi et al.~``HISTORIA DE LA HUMANIDAD - DESARROLLO CULTURAL Y
  CIENTIFICO'', Tomo II (UNESCO), Ed. Sudamericana, Bs.As., 1969.
\item
  PARTRIDGE, P.H. ``CONSENT AND CONSENSUS'', Londres, 1971.
\item
  PARSONS, Talcott ``EL SISTEMA SOCIAL'', Rev.~de Occidente, Madrid, 1976. - ---------------- ``ENSAYOS DE TEORIA SOCIOLOGICA'', Paidós, Bs.As., 1970.
\item
  ---------------- ``EL SISTEMA DE LAS SOCIEDADES MODERNAS'', Trillas México, 1974.
\item
  PITKIN, H. ``THE CONCEPT OF REPRESENTATION'', Berkeley, 1967.
\item
  PINTO, A. ``POLITICA Y DESARROLLO'', Ed. Universitaria, Santiago de Chile, 1972.
\item
  PLATON ``LA REPUBLICA'', UNAM, México, 1971.
\item
  ------ ``LAS LEYES'', Inst. de Est. Políticos, Madrid, 1960.
\item
  ------ ``EL POLITICO'', Inst. de Est. Políticos, Madrid, 1955.
\item
  PUTNAM, R.D. "THE BELIEF OF POLITICIANS: IDEOLOGY, CONFLICT AND
  DEMOCRACY IN BRITAIN AND ITALY, London, 1973.
\item
  PYE, L.W. ``COMMUNICATIONS AND POLITICAL DEVELOPMENT'', Princeton, 1963.
\item
  --------- ``POLITICS, PERSONALITY AND NATION-BUIDING. BURMS'S SEARCH FOR
  IDENTITY'', Yale University Press, New Haven, 1966.
\item
  --------- ``ASPECTS OF POLITICAL DEVELOPMENT'', Little Brown, Boston, 1966.
\item
  PYE, L.W. y VERBA, S. ``POLITICAL CULTURE AND POLITICAL DEVELOP-MENT'',
  Princeton University Press, 1969.
\item
  RIBEIRO, Darsy ``EL DILEMA DE AMERICA LATINA'', Siglo XXI, México, 1971.
\item
  RICHARDSON, Lewis ``ARMS AND INSECURITY'', Quadrangle Press, Chicago, 1960.
\item
  ROBINSON, R. ``THE NON-EUROPEAN FOUNDATIONS OF EUROPEAN IMPERIALISM:
  SKETCH FOR A THEORY OF COLLABORATION'', Longman, London, 1972.
\item
  ROUQUIÉ, Alain ``EXTREMO OCCIDENTE. INTRODUCCION A AMERICA LATINA'', Emecé,
  Bs.As., 1991.
\item
  ROSTOW, Walt ``LAS ETAPAS DEL CRECIMIENTO ECONOMICO'', FCE, México, 1961.
\item
  ROSTOW, W.W. ``POLITICS AND THE STAGES OF GROWTH'', Cambridge, 1971.
\item
  ROSENAU, J.N. ``THE SCIENTIFIC STUDY OF FOREIGN POLICY'', New York, 1971.
\item
  RUNCIMAN, W.G. ``SOCIOLOGY IN ITS PLACE'', Cambridge, 1970.
\item
  RUYER, Raymond ``L'UTOPIE ET LES UTOPIES'', PUF, París, 1950.
\item
  SABINE, G.H. ``HISTORIA DE LA TEORIA POLITICA'', FCE, México, 1984.
\item
  SANTOS, Theodoro dos ``LA NUOVA DIPENDENZA'', Milán, 1971.
\item
  SARTORI, G. ``SISTEMI RAPPRESENTATIVI'' en ``DEMOCRAZIA E DEFINIZIO- NI'', Bolonia,
\end{itemize}

\begin{enumerate}
\def\labelenumi{\arabic{enumi}.}
\setcounter{enumi}{1968}
\tightlist
\item
\end{enumerate}

\begin{itemize}
\tightlist
\item
  ----------- ``LA POLITICA - LOGICA Y METODO EN LAS CIENCIAS SOCIA- LES'', FCE,
  México, 1984. - SCHLESINGER, A.M. ``THE IMPERIAL PRESIDENCY'', Popular Library, New York, 1974.
\item
  SCHUMPETER, Joseph ``CAPITALISMO, SOCIALISMO Y DEMOCRACIA'', Agui- lar, México,
\end{itemize}

\begin{enumerate}
\def\labelenumi{\arabic{enumi}.}
\setcounter{enumi}{1960}
\tightlist
\item
\end{enumerate}

\begin{itemize}
\tightlist
\item
  ------------------ ``IMPERIALISMO Y CLASES SOCIALES'', Tecnos, Ma- drid, 1965.
\item
  SCHMITT, Carl ``LEGALIDAD Y LEGITIMIDAD'', Aguilar, Madrid, . SAINT-SIMON
  ``OEUVRES'', Anthropos, París, 1966.
\item
  SIMON, Herbert A. ``MODELS OF MAN: SOCIAL AND RATIONAL'', Wiley, New York, 1957.
\item
  SKINNER, B.F. ``SCIENCE AND HUMAN BEHAVIOR'', Free Press, New York, 1953.
\item
  SOREL, Jean ``REFLEXIONES SOBRE LA VIOLENCIA'', Alianza, Madrid, 1976.
\item
  SUNKEL, O. y PAZ, P. ``EL DESARROLLO LATINOAMERICANO Y LA TEORIA DEL
  DESARROLLO'', Siglo XXI, México, 1970.
\item
  SWEEZY, Paul ``ÉLITE DE PODER O CLASE DIRIGENTE?'', Jorge Alvarez, Bs.As., . TUCKER,
  Robert ``NATION OR EMPIRE? DEBATE OVER AMERICAN FOREIGN POLICY'', J. Hopkins
  Press, Baltimore, 1968.
\item
  TZU, Sun ``L'ART DE LA GUERRE'', Flammarion, París, 1972.
\item
  URBANI, G. ``LA POLITICA COMPARATA'', Bolonia, 1972.
\item
  VERBA, Sidney ``SMALL GROUPS AND POLITICAL BEHAVIOR'', Princeton University Press,
\end{itemize}

\begin{enumerate}
\def\labelenumi{\arabic{enumi}.}
\setcounter{enumi}{1960}
\tightlist
\item
\end{enumerate}

\begin{itemize}
\tightlist
\item
  VOEGELIN, Eric ``NUEVA CIENCIA DE LA POLITICA'', Rialp, Madrid, 1968.
\item
  VRANICKI, P., SUPEK, R. et al.~``EL SOCIALISMO YUGOESLAVO ACTUAL'', Grijalbo ed.,
  México, 1975.
\item
  WEBER, Max ``EL POLITICO Y EL CIENTIFICO'', Alianza, Madrid, 1967.
\item
  ---------- ``ECONOMIA Y SOCIEDAD'', FCE, México, 1964.
\item
  YARMOLINSKY, Adam ``THE MILITARY ESTABLISHMENT. ITS IMPACT ON AMERICAN
  SOCIETY'', Harper \& Row, New York, 1971.
\item
  ZEITLIN, Irving ``IDEOLOGIA Y TEORIA SOCIOLOGICA'', Amorrortu ed., Bs.As., 1973
\end{itemize}

  \bibliography{book.bib,packages.bib}

\end{document}
